%\section{Geometry and cohomology of K\"ahler manifolds}

\subsection{\textswab{Complex Structure on a vector space}}

Let $\vectorspace$ be a $2m$-dimensional real-valued vector space. A \textit{complex structure} on $\vectorspace$ is an automorphism $J : \vectorspace \rightarrow \vectorspace$ such that $J^2 = - \idop_\vectorspace$. With this structure, $\vectorspace$ is naturally brought into an $m$-dimensional complex-valued vector space by letting 

\begin{equation}
    (\alpha + i\beta) v = \alpha v + \beta J v, \quad \begin{array}{cc}
         v \in \vectorspace  \\
         \alpha, \beta \in \R.
    \end{array}
\end{equation}

In other words, an $m$-dimensional complex-valued vector space can be thought of as a $2m$-dimensional real-value vector space endowed with the complex structure $J = i\id_{\vectorspace}$. 
Hence, this vector space $\vectorspace$ -equipped with the complex structure $J$- has an \textit{adapted basis}

\begin{equation}
    (v_1, \cdots, v_m, Jv_1, \cdots, J v_m), \quad \textnormal{ s.t. } \quad J = \left(
        \begin{array}{cc}
            0 & \idop_\vectorspace  \\
           - \idop_\vectorspace & 0
        \end{array}
    \right).
\end{equation}

\medbreak

An automorphism $\rho: \vectorspace \rightarrow \vectorspace$ preserves a complex structure $J$ on $\vectorspace$ if and only if it commutes with $J$. Hence, these automorphisms form the commutant $\{J\}' \subset \generallineargroup(2m, \R)$ of $J$. It turns out that there is an explicit mapping $\phi$ for the $\{J\}$-commutant of complex-valued $m$-dimensional matrices and the $\{J\}'$-commutant of real-valued $2m$-dimensional matrices. In effect, note that the commutant $\{J\}' \subset \generallineargroup(2m, \R)$ of $J$ is the image of the group $\generallineargroup(m,\C)$ under the monomorphism $\phi$, whose action is given as follows 

\begin{equation}
    \begin{split}
    \phi: \generallineargroup(m,\C) \rightarrow \generallineargroup(2m,\R), \textnormal{ s.t. }
    \begin{array}{cc}
         M \in \generallineargroup(m,\C), \\
         M \overset{\phi}{ \mapsto} 
        \left(
            \begin{array}{cc}
                \Re M & -\Im M  \\
                \Im M & \Re M
            \end{array}
        \right) \in \generallineargroup(2m, \R).
    \end{array}
    \end{split}
\end{equation}|

By invoking \cref{Theorem_Quotient_Group}, one notices that there is an 
explicit one-to-one correspondence between the complex structures 
on a  $2m$-dimensional real-valued vector space $\vectorspace$ and the elements of the quotient
$\frac{\generallineargroup(2m, \R)}{\generallineargroup(m, 
\C)}$\footnote{Here, a short summary of the main topological properties of the real-valued and complex-valued general linear groups is presented.

\begin{enumerate}
    \item The real-valued $\generallineargroup(m, \R)$ is non-compact. Its maximal compact subgroup is the orthogonal group $O(m)$, while the maximal compact subgroup of $\generallineargroup^+(m, \R)$ is the special orthogonal group $SO(m)$. As for $SO(m)$, the group  $\generallineargroup^+(m, \R)$ is not simply connected if $m \neq 1$, but rather has a fundamental group
        
        \begin{equation*}
            \pi_1(SO(m)) = \left\{ \begin{array}{cc}
                 \mathds{Z} \textnormal{ for } m = 2 \\
                \mathds{Z}_2 \textnormal{ for } m > 2 
            \end{array}
            \right..
        \end{equation*}
    
    \item The complex-valued $\generallineargroup(m, \C)$ is a connected space. This follows, in part, since the multiplicative group of complex numbers $\C - \{0\}$ is connected as well. The complex-valued general linear group is not compact however, rather its maximal compact subgroup, $U(m)$, is a compact (group/space). As for $U(m)$, the group manifold $\generallineargroup(m, \C)$ is not simply connected but has a fundamental group $\pi \simeq \mathds{Z}$.
\end{enumerate}
}. \medbreak

\begin{remark}
    \textnormal{First consider the topological space }$X = \generallineargroup(2m,\R)$. \textnormal{Then consider a subspace of it, } $A = \generallineargroup(m,\C)$. \textnormal{The homomorphism } $\phi: \generallineargroup(m,\C) \rightarrow \generallineargroup(2m,\R)$, \textnormal{which in reality is an isomorphism, induces an equivalence relationship } $\sim_{\phi}$ s.t. 
    
    $$
        A \sim_{\phi} B \leftrightarrow A = B, \quad A,B \in \generallineargroup(2m,\R).
    $$
    
   \textnormal{Now, the quotient space } $\frac{\generallineargroup(2m,\R)}{\generallineargroup(m,\C)}$ \textnormal{is, by definition, } $\frac{\generallineargroup(2m,\R)}{\sim_{\phi}}$, \textnormal{given by }
    
    \begin{equation}
        \frac{\generallineargroup(2m,\R)}{\sim_{\phi}} = \{\generallineargroup(2m,\R) - \generallineargroup(m,\C)\} \cup \{0_{\vectorspace}\}.
    \end{equation}
\end{remark}

\bigbreak

A complex structure $J$ on $\vectorspace$ generates a complex structure on the dual space to $\vectorspace$, $\vectorspace^{*}$, as follows 

\begin{equation}
    \langle v, J \omega \rangle = \langle Jv, \omega \rangle, \quad \begin{array}{cc}
         v \in \vectorspace \\
         \omega \in \vectorspace^{*}
    \end{array}.
\end{equation}

\begin{df}
    \textnormal{A scalar product $h: \vectorspace \rightarrow \C$ on a real-valued vector space $\vectorspace$ equipped with a complex structure $J$ is called \textit{Hermitian} if it is $J$-invariant, i.e.}

    \begin{equation}
        h(Jv, Jv') = h(v, v'), \quad v,v' \in \vectorspace.
    \end{equation}

\end{df}

\blanky \bigbreak

From this, it follows immediately that $h(Jv, v) = 0, \blanky \forall v \in \vectorspace$. Moreover, $\vectorspace$ admits an adapted basis, which is orthonormal with respect to this $h$-scalar product. Furthermore, one can also define a skew-symmetric bilinear form, which reads

\begin{equation}\label{Complex_Manifolds_skew_symmetric_form}
    \Omega(v,v') \equiv h(Jv, v'),
\end{equation}

on $\vectorspace$, which is $J$-invariant as well. \smallbreak

One may be interested in defining the so-called \textit{complexification} of a real-valued vector space $\vectorspace$ by considering of a morphism between the rings $\R$ and $\C$, as follows,

\begin{df}
    \textnormal{Let $\vectorspace$ be a real-valued vector space. The \textit{complexification} of $\vectorspace$ is defined as the tensor product of $\vectorspace$ with $\C$, thought of as a two-dimensional real veector space, as follows}
        
        \begin{equation*}
            \vectorspace^{\C} = \C \otimes \vectorspace, \textnormal{ s.t. } \begin{array}{cc}
                 \alpha(v \otimes \beta) = v \otimes (\alpha \beta), \quad v \in \vectorspace, \alpha, \beta \in \C. \\
                 \vectorspace^{\C} \simeq \vectorspace \oplus i \vectorspace \rightarrow v = v_1 \otimes 1 + v_2 \otimes i, \quad v_1, v_2 \in \vectorspace.
            \end{array}
        \end{equation*}
    
    \textnormal{Alternatively, one may use the direct sum as the definition of the complexification $\vectorspace^\C$ of $\vectorspace$ in such a way that , i.e.}
    
    $$
        \vectorspace^\C \equiv \vectorspace \oplus \vectorspace,
    $$    
    \textnormal{ where $\vectorspace^\C$ is imbued with a linear complex structure operator $J$ s.t. 
    $J(v,w) \equiv (-w, v)$. This linear complex structure, thus, encodes the operation "multiplication by $i$" in matrix form. 
    }
\end{df}

\begin{remark}
    \textnormal{From its definition, the complexification $\vectorspace^\C$, the following properties and results hold}
    
    \begin{itemize}
        \item \textnormal{Given a real linear transformation $f: \vectorspace \rightarrow \mathcal{W}$, between two real vector spaces, there is a natural complex linear transformation $f^\C$, the complexification of $f$, $f^\C : \vectorspace^\C \rightarrow \mathcal{W}^\C$ given by}
        
        \begin{equation*}
            f^\C (v\otimes z) = f(v) \otimes z, 
        \end{equation*}
        
        \textnormal{ s.t. } 
        
        \begin{enumerate}
             \item $(\idop_\vectorspace)^\C = \idop_{\vectorspace^\C}$,
             \item $(f \circ g) = f^\C \circ g^\C$,
             \item $(f + g) = f^\C + g^\C$,
             \item $(af)^\C = a f^\C, \quad \in \R$.
        \end{enumerate} 
        
        \textnormal{
        The map $f^\C$ commutes with conjugation, and maps the real subspace of $\vectorspace^\C$ with the real subspace of $\mathcal{W}^\C$. Conversely, a complex linear map $g : \vectorspace^\C \rightarrow \mathcal{W}^\C$ is the complexification of a real linear map if and only if it commutes with conjugation. Hence, it follows that 
        }
        
        $$
            \textnormal{Hom}_{\R}(\vectorspace, \mathcal{W})^\C \simeq \textnormal{Hom}_{\R}(\vectorspace^\C, \mathcal{W}^\C),
        $$
        
        \textnormal{
        where $\textnormal{Hom}_{\R}(\vectorspace, \mathcal{W})$ is the space of all real-valued linear maps from $\vectorspace$ to $\mathcal{W}$.
        }
        
        \item \textnormal{
        The dual $\vectorspace^{*}$ of a real-valued vector space $\vectorspace$ is the space $\vectorspace^{*}$ of all real-valued linear maps from $\vectorspace$ to $\R$. 
        The complexification of $\vectorspace^{*}$ can be naturally be thought of as $\textnormal{Hom}_{\R}(\vectorspace, \C)$. This is,
        }
        
        $$
            (\vectorspace^{*})^\C = {\vectorspace^{*}} \otimes \C \simeq \textnormal{Hom}_{\R}(\vectorspace, \C).
        $$
        
        The isomorphism is given by $(\omega_1 \otimes 1 + \omega_2 \otimes i) \leftrightarrow \omega_1 + i\omega_2, \quad \omega_1, \omega_2 \in \vectorspace^{*}$. 
        
        \item Given a real linear map $\phi : \vectorspace \rightarrow \C$, it may be extended by linearity to yield a complex linear map $\phi: \vectorspace^\C \rightarrow \C$ by letting $\phi(v \otimes z) = z \phi(v)$. 
        
        \item Moreover, the previous extension results in an natural isomorphism between the two following structures
        
        \begin{equation*}
            (\vectorspace^{*})^\C \simeq (\vectorspace^\C)^{*}. 
        \end{equation*}
    \end{itemize}
    
\end{remark}

\blanky \bigbreak

Then, the complexification $\vectorspace$ is a $2m$-dimensional complex space. This gives rise to the following theorem

\begin{theorem} \label{Complex_Manifolds_Theorem_holo_antiholo}
      A complex structure $J$ on $\vectorspace$ is naturally extended to $\vectorspace^\C$ by letting $J \circ i = i \circ J$, allowing $\vectorspace^\C$ to be split into a direct sum of two components

    \begin{equation}         \label{Complex_Manifolds_holo_antiholo_structures}
        \begin{split}
        &\vectorspace^\C = \vectorspace^{1,0} \oplus \vectorspace^{0,1}, \textnormal{ where }\begin{array}{cc}
            \begin{array}{cc}
                 \textnormal{$\vectorspace^{1,0}$ is the complex } \\
                 \textnormal{holomorphic subspace}:
            \end{array}
              \vectorspace^{1,0} = \{v + i J v, \blanky v \in \vectorspace\}  \\
              \begin{array}{cc}
                   \textnormal{$\vectorspace^{0,1}$ is the complex } \\
                   \textnormal{antiholomorphic subspace}:
              \end{array}
            \vectorspace^{0,1} = \{v - i J v, \blanky v \in \vectorspace\}.
        \end{array}
        \end{split}
\end{equation}

These are the eigenspaces of $J$ characterized by the eigenvalues $i$ and $-i$ respectively. Complex conjugation on $\vectorspace^\C$ induces an $\R$-isomorphism $\vectorspace^{1,0} \simeq \vectorspace^{0,1}$. 
\end{theorem}

\begin{proof}
    Since $\vectorspace^{1,0} \cap \vectorspace^{0,1} = \emptyset$, $\textnormal{ the canonical map } \vectorspace^{1,0} \oplus \vectorspace^{0,1} \rightarrow \vectorspace^{\C} \textnormal{ is injective. }$. Furthermore, the previous decomposition is an isomorphism due to the existence of the inverse map
    
    \begin{equation*}
        v \mapsto 
             \frac{1}{2}(v + i Jv) \oplus \frac{1}{2}(v - i Jv).
    \end{equation*}
    
    The second assertion follows from decomposing any vector $v \in \vectorspace^\C$ as $v = x+iy$, with $x,y \in \vectorspace$. Then, 
    
    \begin{equation*}
        \xoverline{v - i J v} = (x - iy + i Jx + Jy) = \xoverline{v} + i J \xoverline{v}.
    \end{equation*}
    
    Hence, complex conjugation interchanges the two factors.

\end{proof}

\begin{remark}
    \textnormal{There is an alternative definition for these (anti)-holomporhic subspaces, as follows}
        
        \begin{equation}         
            \begin{split}
            &\vectorspace^\C = \vectorspace^{1,0} \oplus \vectorspace^{0,1}, \textnormal{ where }\begin{array}{cc} 
                \vectorspace^{1,0} = \{ J(v) = i v, \blanky v \in \vectorspace\}  \\ 
                \vectorspace^{0,1} = \{J(v) = - i v, \blanky v \in \vectorspace\}.
            \end{array}
            \end{split} 
        \end{equation}
    
    \textnormal{
    This definition thus induces the existence of two (almost) complex structures on $\vectorspace^\C$. 
    One is given by $J$ and the other one is given by $i$, which coincide on $\vectorspace^{1,0}$ but differ by a sign on $\vectorspace^{0,1}$. 
    Naturally, both $\vectorspace^{1,0}$ and $\vectorspace^{0,1}$ are complex subspaces of $\vectorspace^\C$ with respect to these complex structures.
    }\textnormal{
    If $\vectorspace^\C$ is taken to be complex vector space with respect to $i$, the $\C$-linear extension of $J$ is the additional structure that gives rise to the direct sum decomposition. If $\vectorspace^{1,0}$ and $\vectorspace^{0,1}$ are considered with the complex structure $i$, then
    } 
    
    \begin{equation*}
        \begin{array}{cc}
             \textnormal{the composition } \vectorspace \subset \vectorspace^\C \rightarrow \vectorspace^{1,0} \textnormal{ is complex linear,}  \\
             \textnormal{the composition } \vectorspace \subset \vectorspace^\C \rightarrow \vectorspace^{0,1} \textnormal{ is complex antilinear.} 
        \end{array}
    \end{equation*}
    
    \textnormal{
    In the previous definition, the one being considered in \cref{Complex_Manifolds_Theorem_holo_antiholo}, $J$ is regarded as an overall complex structure, taking on different eigenvalues for different subcomponents of $\vectorspace^\C.$
    }
    
\end{remark}

 Furthermore, there is the \textit{antilinear complex conjugate morphism}  

\begin{equation}
    v = v_1 + iv_2 \mapsto \bar{v} = v_1 - i v_2, \quad \bm{\omega} \mapsto \bar{\bm{\omega}}, \begin{array}{cc}
         \bm{\omega} \in \vectorspace^{r,s} \\
         \bar{\bm{\omega}} \in \vectorspace^{s,r}
    \end{array} r ,s = 0,1, \quad \textnormal{ and s.t. } \bar{Jv} = J \bar{v}.
\end{equation}

From the previous remarks, it is clear that the complexification $(\vectorspace^{*})^\C$ of the dual $\vectorspace^{*}$ of $\vectorspace$ is the complex dual of $\vectorspace^\C$. Hence, a similar decomposition to the one obtained in \cref{Complex_Manifolds_Theorem_holo_antiholo} holds for the complexification of the dual space $(\vectorspace^{*})^\C$, as follows  

 \begin{theorem}
    Let $\vectorspace$ be a real vector space endowed with an (almost) complex structure $J$. Then, the dual space $\vectorspace^{*} = \textnormal{Hom}_{\R}(\vectorspace, \R)$ has a natural (almost) complex structurer given by $J(f) v = f(J(v))$. This induces a decomposition of the complexification of the dual space $(\vectorspace^{\C})^{*} = (\vectorspace^{*})^\C = \textnormal{Hom}_{\R}(\vectorspace, \C)$. Then, 
 
    \begin{equation} \label{Complex_Manifolds_holo_antiholo_dual_structures}
    \begin{split}
        &(\vectorspace^\C)^{*} = (\vectorspace^{1,0})^{*} \oplus (\vectorspace^{0,1})^{*}, \textnormal{ where }\begin{array}{cc}
             \textnormal{$(\vectorspace^{1,0})^{*}$ is the subspace }: (\vectorspace^{1,0})^{*} = \{\omega - i J \omega, \blanky \omega \in \vectorspace^{*}\}  \\
            \textnormal{$(\vectorspace^{0,1})^{*}$ is the subspace }: (\vectorspace^{0,1})^{*} = \{\omega + i J \omega, \blanky \omega \in \vectorspace^{*}\}.
        \end{array}
    \end{split}
    \end{equation}

    are the annihilators\footnote{
   The annihilator of a vector subspace $\mathcal{S}$ of a vector space $\vectorspace$ is the set $S^{0} \subset \vectorspace^{*}$ of linear functionals s.t. $f(s) = 0, \quad s \in \mathcal{S}$.
    } of ${\vectorspace^{1,0}}$ and ${\vectorspace^{0,1}}$ respectively. They are the eigenspaces of the complex structure $J$ on $(\vectorspace^{*})^\C$ characterized by the eigenvalues $i$ and $-i$, respectively. 
\end{theorem} \bigbreak

\begin{remark}
    \textnormal{
    The previous annihilators may be defined alternatively as
    }
    
     \begin{equation*} 
        \begin{split}
            &(\vectorspace^\C)^{*} = (\vectorspace^{1,0})^{*} \oplus (\vectorspace^{0,1})^{*}, \textnormal{ where }\begin{array}{cc}
                 \textnormal{$(\vectorspace^{1,0})^{*}$ is the subspace }: (\vectorspace^{1,0})^{*} = \{ f(J(v)) = i J(f(v)) , \blanky f(\cdot) \in \vectorspace^{*}\}  \\
                \textnormal{$(\vectorspace^{0,1})^{*}$ is the subspace }: (\vectorspace^{0,1})^{*} = \{ f(J(v)) = - i J(f(v)) , \blanky f(\cdot) \in \vectorspace^{*}\}.
            \end{array}
        \end{split}
    \end{equation*}
    
    \textnormal{
    noting that $(\vectorspace^{0,1})^{*} = \textnormal{Hom}_{\C} \bigg((V, J), \C\bigg)$.
    }
\end{remark}

\bigbreak
\begin{lemma}\label{complex_manifolds_exterior_algebra_decom}
    
    Hence, for a real vector space $\vectorspace$ of dimension $n$, the natural decomposition of its exterior algebra is of the form 
    
    \begin{equation}
        \bigwedge \vectorspace\bigg = \bigoplus_{k=0}^{d} \bigwedge^k \vectorspace.
    \end{equation}
    
    Similarly, $(\bigwedge \vectorspace^\C)^{*}$ denotes the exterior algebra of the complex vector space $\vectorspace^\C$, which decomposes as 
    
    \begin{equation}
        \bigg(\bigwedge \vectorspace^\C\bigg)^{*} = \bigoplus_{k=0}^{d} \bigwedge^k \vectorspace^\C.
    \end{equation}
    
    Furthermore, $(\bigwedge \vectorspace^\C )^{*} = \bigwedge \vectorspace \otimes_{\R} \C$ and $(\bigwedge \vectorspace)^{*}$ is the real subspace of $(\bigwedge \vectorspace^\C)^{*}$ that is left invariant under complex conjugation. 
\end{lemma}

If $\vectorspace$ is endowed with an almost complex structure $J$, then its real dimension $d$ is even, and $\vectorspace^\C$ naturally decomposes as $\vectorspace^\C = \vectorspace^{1,0} \oplus \vectorspace^{0,1}$, each complex vector space having dimension $n$. \bigbreak

From the preceding discussion on the complexification and the dual complexifaction of a real vector space, one can  it is natural to define the complexifications of higher-order exterior products as 

\begin{df}
    
    Let $\vectorspace$ be a real-valued vector space imbued with a complex structure $J$. If $\bigwedge^{p,q} \vectorspace$ is the space of $(p,q)$-real valued tensors, then its complexification can be defined as follows
    
    \begin{equation}
    \bigwedge^{p,q} \vectorspace \equiv \bigwedge^{p} \vectorspace^{1,0} \blanky \otimes_{\C} \blanky  \bigwedge^{q} \vectorspace^{0,1}.
\end{equation}
        
\end{df}

where the exterior products of $\vectorspace^{1,0}$ and $\vectorspace^{0,1}$ are taken as exterior products o complex vectors spaces.
An element $\alpha \in \bigwedge^{p,q} \vectorspace$ is of bidegree $(p, q)$.

\begin{lemma}
    For a real-valued vector space $\vectorspace$, endowed with an almost complex structure $J$, then the following results hold

    \begin{itemize}
        \item $\bigwedge^{p,q} \vectorspace$ is a subspace of $\bigwedge^{p+q} \vectorspace^{\C}$. 
        \item $\bigwedge^{k} \vectorspace^{\C} = \bigoplus_{p+q = k} \bigwedge^{p,q} \vectorspace$
        \item Complex conjugation on $\bigwedge^{*} \vectorspace$ defines a $\C$-antilinear isomorphism

            $$
                \bigwedge^{p,q} \vectorspace \simeq \bigwedge^{q, p} \vectorspace, \textnormal{ and } \xoverline{\bigwedge^{p,q} \vectorspace} = \bigwedge^{q,p} \vectorspace.
            $$

        \item The exterior product is of bidegree (0,0). In other words, the exterior product -thought of as the map - $(\alpha, \beta) \overset{\langle \cdot, \cdot \rangle}{\longmapsto} \langle \alpha, \beta \rangle$ is s.t. that it maps

            \[
               \bigwedge^{p,q} \vectorspace \times \bigwedge^{r,s} \vectorspace \longmapsto \bigwedge^{p+r,q+s} \vectorspace,
            \]

        where the latter must be understood as a subspace. 
    \end{itemize}
\end{lemma}

\begin{proof}

    Let $\{v_i\}_{i=1}^{n} \in \bigwedge^{1,0} \vectorspace = \vectorspace^{1,0}$ and let $\{\omega^i\}_{i=1}^{n} \in \bigwedge^{0,1} \vectorspace = \vectorspace^{0,1}$ be $\C$-basis on both components, whose sum yields thee complex vector space $\C$.
    Then 

    \[
       \{v_{J_1} \otimes \omega_{J_2}\} \in \bigwedge^{p,q} \vectorspace, \textnormal{ with } \begin{array}{cc}
            J_1 = \{i_1 < \cdots < i_p\} \\
            J_2 = \{j_1 < \cdots < j_q\}  
       \end{array} \begin{array}{cc}
            \textnormal{forms a basis } \\
            \textnormal{of $\bigwedge^{p,q} \vectorspace$}.
       \end{array}
    \]

    Hence, this proves the first and second items. 
    An equivalent derivation could have been obtained from using the general fact that any direct sum decomposition $V^\C = W_1 \oplus W_2$ induces a direct sum decomposition 

    $$
        \bigwedge^{k} \vectorspace_\C = \bigoplus_{p+q=k} \bigwedge^{p} W_1 \otimes \bigwedge^{q} W_2.
    $$

    \blanky \medbreak

    On the other hand, note that complex conjugation is multiplicative, i.e. $\xoverline{\omega_1 \land \omega_2} = \xoverline{\omega}_1 \land \xoverline{\omega}_2$. 
    From this fact, the third assertion follows immediately since
    $\xoverline{\vectorspace^{1,0}} = \vectorspace^{0,1}$.
    Finally, the last assertion holds again true for any decomposition $\vectorspace^\C = W_1 \oplus W_2$. 
\end{proof}

Noting that any vector $v \in \vectorspace^\C$ can be decomposed as $x+iy$, with $x,y \in \vectorspace$,  

\[
   \textnormal{then let } \{z_i\}^{i}, z_i = \frac{x_i - i y_i}{2} \in \vectorspace^{1,0} \textnormal{ be a basis of } \vectorspace^{1,0}, \textnormal{ with $x,y \in \vectorspace$. }
\]

In \cref{Complex_Manifolds_Theorem_holo_antiholo}, it was established that $\vectorspace^{1,0}$ was defined as the set of vectors s.t. $J(v) = i v$. 
Since the $z_i$-vectors belong to said structure, this implies that $J(z_i) = i z_i$, from which one finds that $y_i = J(x_i)$ and $x_i = -J(y_i)$. 
Moreover, $\{x_i, y_i\}^{i}$ forms a real basis of $\vectorspace$ and, hence, a basis of its complexification $\vectorspace^\C$.
A natural basis on $\vectorspace^{0,1}$ is given by $\{\bar{z}_i\}, \blanky z_i = \frac{x_i + i y_i}{2}$. \smallbreak

\[  \begin{array}{cc}
     \begin{array}{cc}
         \textnormal{Conversely, if $v \in \vectorspace$, then it  follows immediately that }
    \end{array} \begin{array}{cc}
         \frac{v - i J(v)}{2} \in \vectorspace^{1,0} \\
         \\
         \frac{v + i J(v)}{2} \in \vectorspace^{0,1} \\
    \end{array}. \\
     \textnormal{ Hence, if } \langle x_i, y_i := J(x_i) \rangle \textnormal{ is a basis of $\vectorspace$, then} \begin{array}{cc}
          \{z_i\}^{i}, \blanky z_i = \frac{x_i - iy_i}{2} \textnormal{ is a basis of the complexified vector space $\vectorspace^{1,0}$}  \\
          \\
          \{\bar{z}_i\}^{i}, \blanky \bar{z}_i = \frac{x_i + iy_i}{2} \textnormal{ is a basis of the complexified vector space $\vectorspace^{0,1}$}
     \end{array}
\end{array}
    
\]

Then, one can imbue on $\vectorspace$ a very important element, 

\begin{lemma}
    For any $m \leq \dim_{\C} \vectorspace^{1,0}$, 

    \[
        (-2i)^m \bigwedge_{j = 1}^{m} (z_j \land \bar{z}_j) = \bigwedge_{j=1}^{m} (x_j \land y_j),
    \]

    defines a positive oriented volume form the natural orientation of $\vectorspace$, if $m = \dim_{\C} \vectorspace^{1,0}$.
\end{lemma}

\begin{remark}    
\end{remark}

An analogous result holds for the dual bases. If $\langle x^i, y^i \rangle$ is the basis of $(\vectorspace^\C)^{*}$ -dual to $\langle x_i, y_i\rangle$.
Then $z^i = x^i + i y^i$ and $\xoverline{z}^{i} = \bar{x}^i - i \bar{y}^i$ are the basis of the dual spaces $(\vectorspace^{1,0})^{*}$ and $(\vectorspace^{0,1})^{*}$, dual to $\langle z_i \rangle$ and $\langle \bar{z}_i \rangle$, respectively. 
Then, it follows that 

\[
    \left(\frac{i}{2}\right)^m \bigwedge_{j = 1}^{m} (z^j \land \bar{z}^j) = \bigwedge_{j=1}^{m} (x^j \land y^j).
\]

In this case, note that $ -y^i = J(x^i)$ and $x^i = J(y^i)$. 
Note as well the existence of the natural isomorphism $\bigwedge^{k} \vectorspace^* \simeq (\bigwedge^{k} \vectorspace)^{*}$ given by 

\[
    (\bigwedge_{j=1}^{k} \alpha_j) (\bigwedge_{j=1}^{k} v_j) = \det (\alpha_j(v_j))_{ij}
\]

\blanky \medbreak

With such a structure for the exterior algebras of the complexified vectorspace defined in terms of the exterior algebras of the real-valued vector space, one can define natural projections, 

\begin{df}
    With respect to the direct sum decomposition of the exterior algebras of the real-valued vector space and of its complexification, given in \cref{complex_manifolds_exterior_algebra_decom}, 
    then the following operations immediately arise,

    \begin{equation*}
        \Pi^{k}: \bigwedge^{*} \vectorspace^{\C} \rightarrow \bigwedge^{k} \vectorspace^\C \textnormal{  and  } \Pi^{p,q}: \bigwedge^{*} \vectorspace^\C \rightarrow \bigwedge^{p,q} \vectorspace,
    \end{equation*}
    \begin{equation*}
        {\bf I}: \bigwedge^{*} \vectorspace^{\C} \rightarrow \bigwedge^{*} \vectorspace^{\C},   
    \end{equation*}

    where the fist two operators are natural projections, with the third operation being the linear operator acting on $\bigwedge^{p,q} \vectorspace$ by multiplication by $i^{p,q}$, i.e.

    \[
        {\bf I} = \sum_{p,q} i^{p-q} \Pi^{p,q}.
    \]
    
\end{df}

\begin{remark}
\end{remark}

Note here, that the first projection, $\Pi^{k}$, does not depend on the almost-complex structure $J$, contrary to the ${\Pi^{p,q}}$ and ${\bf I}$ operations. 
Moreover, ${\bf I}$ is the multiplicative extension of the almost complex structure $J$ on ${\bf V}$, but it is not an almost-complex structure by itself. 
Given that $J$ is defined on the real-valued vector space $\vectorspace$, ${\bf I}$ is an endomorphism of the real exterior algebra $\bigwedge^{*} \vectorspace$. 
Note that the corresponding operator on the dual space $\bigwedge \blanky^{*} (\vectorspace^{\C})^{*}$ are also given by $\Pi^k, \Pi^{p,q}, {\bf I}$, respectively. \medbreak

Now, let $(\vectorspace, \langle , \rangle)$ be a finite-dimensional euclidean vector space, with $\vectorspace$ being a real-valued vector space and $\langle, \rangle$ being a positive-definite symmetric bilinear form. 
When complexifying this space, one must require a degree of compatibility between the induced 
almost-complex structure $J$ and the inner product, leading to the following definition \smallbreak

\begin{mdframed}[style=MyFrame]
\begin{df}
    An almost-complex structure $J: \vectorspace \rightarrow \vectorspace$ is \textswab{compatible} with the scalar product $\langle, \rangle$ if 
    $$
        \langle J(v), J(w) \rangle = \langle v, w \rangle, \quad \forall v, w \in \vectorspace
    $$
    
    In other words, one must require that $J \in O(V, \langle, \rangle)$.
\end{df}    
\end{mdframed}

In a two-dimensional space, the notions of scalar products and almost-complex structures are intimately related, being almost equivalnent. 

\begin{remark}
\end{remark}

\begin{itemize}
    \item Let $\vectorspace$ be a real-valued two-dimensional oriented vector space. If $\langle , \rangle$ is a scalar produt, then there exists a natural almost-complex structure $J$ on $\vectorspace$ associated to it, as follows:
    For any $v \in \vectorspace-{{\bf 0}}$, the vector $J(v) \in \vectorspace$ is uniquely determined by the following three conditions:
        \[
            \langle v , J(v) \rangle = 0, \quad ||J(v)|| = ||v||, \quad \{v, J(v)\} \textnormal{is positively oriented}.
        \]

    In other words, $J$ is the rotation by $\pi/2$. 
    Hence, $J^{2} = -\idop$, with $J \in SO(V)$ being the almost-complex structure. Moreover, two scalar products $\langle ,\rangle$ and $\langle ,\rangle'$ are \textswab{conformally equivalent} if there exists a positive scalar $\lambda$, s.t. 
    $\langle ,\rangle' = \lambda \langle ,\rangle$. 
    Clearly, two conformally equivalent scalar products define the same almost-complex structure. 
    Conversely, for any given almost-complex structure $J$ there always exists a scalar product $\langle ,\rangle$ to which it is associated. \bigbreak
    
    This, then, induces a bijection between the set of conformal equivalence classes of scalar produts on the two-dimensional vector space and the set of almost-complex structures, inducing the given orientation, i.e.

    $$
        \{ \langle \blanky, \blanky \rangle \}/\sim_{\textnormal{conf}} \blanky \longleftrightarrow \blanky \{J \in \generallineargroup(\vectorspace)_{+}\blanky | \blankyJ^2 = -\idop\}.
    $$    
\end{itemize}

Now, returning to the general case, consider a general euclidean vector space $(\vectorspace, \langle \blanky, \blanky \rangle)$ of arbitrary dimension with compatible almost-complex structure $J$. 

\begin{mdframed}[style=MyFrame]

    \begin{df}
        The \textswab{Fundamental Form} associated to $(\vectorspace, \langle, \rangle, J)$ is the form 

        \[
            \omega \equiv -\langle \cdot, J(\cdot) \rangle = \langle J(\cdot), \cdot \rangle
        \]
    \end{df}
    
\end{mdframed}

\clearpage

Using the preceding remarks and results, a Hermitian scalar product $h$ on $\vectorspace$ can be uniquely extended to a symmetric complex $J$-invariant bilinear fo on $\vectorspace^{\C}$ fulliling the following conditions 

\begin{itemize}
    \item $h(\xoverline{v}, \xoverline{v}') = \xoverline{h(v,v')}, \quad v,v' \in \vectorspace^\C$.
    \item $h(v, \xoverline{v}) > 0, \quad v \in \vectorspace^{\C} - \{{\bf 0}\}$.
    \item $h(v, v') = 0$ if $v,v'$ are simultaneously holomorphic or antiholomorphic. 
\end{itemize}

This complex bilinear form, thus, induces a non-degenerate Hermitian form on $\vectorspace^\C$, as follows 

$$
    \langle v | v' \rangle_{h} \equiv h(v, \bar{v}'), 
$$

s.t. the holomorphic and antiholomorphic spaces are mutually orthogonal. Accordingly, the skew-symmetric form $\Omega$, introduced in \cref{Complex_Manifolds_skew_symmetric_form}, can be extendd to $\vectorspace^C$ so that 

\begin{equation}
    \begin{array}{cc}
          \Omega(\xoverline{v}, \xoverline{v}') = \xoverline{\Omega(v,v')}, \quad v,v' \in \vectorspace^\C\\ 
          \Omega(v,v') = 0, \quad v,v' \in \vectorspace^{r,s},\quad r,s=0,1. 
    \end{array}
\end{equation}

\clearpage

\subsection{Almost-complex manifolds}

Let $Z$ be a $2m$-dimensional smooth real-valued manifold, with coordinate basis $(z^i)_{i=1}^{2m}$. 

\begin{df}
    An \textit{almost-complex structure } on $Z$ is defined as a \textbf{vertical bundle automorphism} $J: Z \rightarrow Z$ on the tangent bundle $TZ$ s.t. 

    $$
         J \circ J = - \idop_{TZ}.
    $$
\end{df}

The following statements immediately follow from the previous definition. \medbreak 

Clearly if $J$ is an almost-complex structure, then $J \in \generallineargroup(Z)$. Moreover, if $Z$ is the real vector space underlying a complex vector space then $v \mapsto i\cdot v$ defines an almost complex structure $J$ on $Z$. The converse holds true as well.

\begin{lemma}
    If $J$ is an almost complex structure on a real vector space $Z$, then $Z$ admits in a natural way the structure of a complex vector space. 
\end{lemma}

\begin{proof}
    In effect, the $\C$-module structure on $J$ is defined as $(a+ib)\cdot v = a \cdot v + b \cdot J v$ with $a,b \in \R$. The $\R$-linearity of $J$ and the assumption $J^2 = -\idop$ yields that 
    
    \begin{equation*}
        (a+ib)(c+id)\cdot v = (a+ib) \bigg((c+id) \cdot v\bigg), \quad i(i\cdot v) = -v.  
    \end{equation*}
\end{proof}

Hence, almost complex structures and complex structures are equivalent notions for vector spaces. Moreover, an almost complex structure can only exist on an even dimensional real vector space. 

\begin{lemma}
    Any almost complex structure $J$ on $Z$ induces a natural orientation on $Z$. 
\end{lemma}
      
Hence, the almost-complex structure $J$ on $Z$ can be represented by a tangent-valued form on $Z$, as follows 

\begin{equation}
    J = J^{i}_{\blanky k} dz^k \otimes \partial_i, \quad \textnormal{ s.t. } \quad J^{i}_{\blanky k} J^{k}_{\blanky j} = - \delta^{i}_{\blanky j}. 
\end{equation}

This tangent-valued form defines an automorphism $J$ on the cotangent bundle $T^{*}Z$ of $Z$ s.t. 

\begin{equation*}
    \langle v, J \omega \langle = \langle J v, \omega \langle, \quad \begin{array}{cc}
         v \in T_z Z  \\
         \omega \in T_z^{*}Z 
    \end{array}, \quad z \in Z.
\end{equation*}

Furthermore, an almost-complex structure provides $Z$ with an orientation, associated with the \textbf{adapted fibre bases} for $TZ$. 
The pair $(Z,J)$ is then called an \textit{almost-complex manifold}. A differrmorphism $f: Z \rightarrow Z'$ preserves an almost complex structure $J$ on $Z$ if and only if the tangent morphism $Tf$ commutes with $J$. \medbreak



%\subsection{.}

\bibliography{citations.bib}
\bibliographystyle{apsrev4-1}
