
\section{\textbf{Variational Wave Functions and Parent Hamiltonians}}

\blanky \\

\begin{tcolorbox}[colback =yellow, title = Physical Context]

The ground state for most non-ferromagnetic Heisenberg models are not known. Even those that are known in analytical form, such as the Bethe solution in one dimension, require numerical simulation computations for their spin correlations. Variational wave functions provide an educated guess for the ground state. Physical insight can be gained by a variational calculation, where the energy is minimized with respect to a chosen set of parameters. The variational approach go further beyond the regimes of any particular expansion scheme -semiclassical, large-$N$, or others. \\

A fundamental idea behind the Variational approach is that of the parent Hamiltonian. This is the Hamiltonian for which a particular wave function happens to be the exact ground state. Although the Hamiltonian may differ from the physical model by extra interactions, it serves to enhance the understanding of the system by bringing into light the relation between interactions and ground state correlations. \\

\end{tcolorbox}

\blanky \\

\paragraph{\textbf{Valence Bond States}}

The valence bond states are variational wave functions for antiferromagnetic Heisenberg models, having been studied extensively i nthe context of quantum magnetism and high-temperature superconductivity. Their general form is 

\begin{equation}
    \ket{\{c_{\alpha}\}, S} = \sum_{\alpha} c_{\alpha} \ket{\alpha}, \begin{array}{c}
         \textnormal{ where $c_{\alpha}$ are the variational parameters and }  \\
         \ket{\alpha} = \prod_{i,j \in \Lambda_{\alpha}} \bigg({\bf a}_i^\dagger {\bf b}_i^\dagger - {\bf b}_i^\dagger {\bf a}_i^\dagger \bigg) \ket{0},
    \end{array}
    \label{Valence_bond_states}
\end{equation}

where ${\bf a}_i, {\bf b}_i$ are Schwinger bosons on the $i$-th site, and where $\Lambda_{\alpha}$ is a particular configuration of the (ij)-bonds on the lattice. The condition on $\Lambda_{\alpha}$ is that precisely $2S$ bonds will emanate from each site. In certain cases, the sum on 
\cref{Valence_bond_states} is dominated by a finite number of configurations in the large lattice limit\footnote{

The cases where there are macroscopicaly many configurations were denoted \underline{resonanting valence bonds states} (RVB) by Anderson.

}. \\

From inspection of \cref{Valence_bond_states}, it is clear that all bond operators ${\bf a}_i^\dagger {\bf b}_i^\dagger - {\bf b}_i^\dagger {\bf a}_i^\dagger$ are invariant under global spin rotations, making $\ket{\{c_{\alpha}\}, S}$ a singlet of total spin. A special class of valence bond states is given by 

\begin{equation}
    \ket{\hat{u}, S} = \sum_{\alpha} \bigg(\prod_{(ij) \in \alpha} u_{ij}\bigg) \ket{\alpha},
    \label{special_class_Valence_bond_states }
\end{equation}

wherein there are upto $\frac{N(N-1)}{2}$ independent variational parameters $u_{ij}$, with $N$ being the number of lattice sites. Furthermore, if the bond parameters are both bipartite and positive, this is 

\begin{equation}
    u_{ij} = \left\{\begin{array}{c}
       u_{ij} > 0  \textnormal{ if } \begin{array}{c}
            i \in A  \\
            j \in B
       \end{array}  \\
       0 \textnormal{ if $i,j$ are on the same sublattice}
    \end{array}\right.,
\end{equation}

then the $\ket{\hat{u}, S}$-states obey the Marshall sign criterion \cref{Marshall_theo_1}, as required for the ground state of a bipartite Heisenberg antiferromagnet. 

\clearpage 

A Schwinger boson mean field state is defined as 

\begin{equation}
    \ket{\hat{u}} = \exp \bigg[\frac{1}{2} \sum_{ij} u_{ij} \bigg({\bf a}_i^\dagger {\bf b}_i^\dagger - {\bf b}_i^\dagger {\bf a}_i^\dagger\bigg)\bigg], \textnormal{ with } u_{ij} = -u_{ji}.
    \label{Schwinger_mean_field_state}
\end{equation}

Such states are generated by the Schwinger boson mean field theory of the Heisenberg antiferromagnet, which will be reviewed later on. The $\ket{\hat{u}}$-state contains contributions of different spin sizes at all sites, and therefore is not a bona fide spin state. It can be transformaed, by a Bogoliubov transformation on the Schwinger bosons, into a factorizable Fock state, which is in fact a vacuum of the transformed bosons. Moreover, the correlations of the mean field states can be evaluated analytically. The valence bond state, given by \cref{special_class_Valence_bond_states }, can be constructed from \cref{Schwinger_mean_field_state} via a projection. Said projection is given by the operator defined in \cref{Schwinger_boson_projector},

$$
    \ket{\hat{u}, S} = \mathbf{P}_S \ket{\hat{u}},
$$

where the projected state is non-factorizable, due to correlations introduced by ${\bf P}_S$, said correlation being much easier to evaluate in $\ket{\hat{u}}$ -the Schwinger boson mean field state- than in the special class of valence bond states, $\ket{\hat{u}, S}$. \\

\paragraph{\textit{Spin one-half States}}

Consider the spinor states for $S = \frac{1}{2}$-states, given by 

\begin{align}
    \ket{\uparrow_i} &= {\bf a}_i^\dagger \ket{0}, & \ket{\downarrow_i} &= {\bf b}_i^\dagger  \ket{0}.
\end{align}

Any bipartite valence bond configuration, those given by \cref{Valence_bond_states}, that obeys the Marshall sign criterion can be written in a normalized form as a product of singlet bonds,

\begin{equation}
\begin{split}
    \ket{\alpha}_{S = \frac{1}{2}} = \prod_{(ij) \in \Lambda_{\alpha}}^{{\substack{{i \in A}\\
                                 {i \in B} }}}
    \frac{\ket{\uparrow_i}\ket{\downarrow_j}-\ket{\downarrow_i} \ket{\uparrow_j}}{\sqrt{2}},  \begin{array}{c}
         \textnormal{ where the spin }  \\
         \textnormal{ correlations are } \\
         \textnormal{ are simply } 
    \end{array} \langle \spin_i \spin_j \rangle = \left\{\begin{array}{cc}
         \frac{3}{4} & i = j  \\
         -\frac{3}{4} & (ij) \in \Lambda_{\alpha} \\
         0 & (ij) \notin \Lambda_{\alpha}
    \end{array}\right..
    \label{bipartite_valence_bond_states_corrs}
\end{split}
\end{equation}

Thus, if the bonds in $\Lambda_{\alpha}$ are short range, $\ket{\alpha}_{S=\frac{1}{2}}$ is a disordered spin liquid state. Different valence bond configurations are not orthogonal, for their finite overlap is given by 

\begin{equation}
    \bra{\alpha}\ket{\beta} = \prod_{\ell \in \bra{\alpha}\ket{\beta}} 2^{1- \frac{L_{\ell}}{2}} = 2^{N_{L}^{(\alpha \beta)} - N}, \begin{array}{c}
         \textnormal{ where, in the first equality, the product is made over all loops $\ell$  } \\
         \textnormal{ of $L_{\ell}$-length found in the overlap in the two configurations.  }
    \end{array} 
\end{equation}

For two identical bonds in $\alpha, \beta$, a length $L = 2$ loop is produced. In the second identity of the preceding equation, $N_L$ is the total number of loops and $N$ is the number of sites on the lattice. Since $\ket{\alpha}$ contains one bond per site, and the coordination number of all simple lattices is equal to or larger than two, the $\ket{\alpha}$-states break lattice translational symmetry. This symmetry can be restored in \cref{Valence_bond_states} by summing over $\alpha$. \\

From now, the discusion will be restricted to nearest-neighbour bonds or dimers, for the one and two-dimensional cases separately. \\

\paragraph{\textit{The Majumdar-Ghosh Hamiltonian}}

Majumdar and Ghosh introduced the Hamiltonian,

\begin{equation}
    {\bf H}^{\textnormal{MG}} = \frac{4 |K|}{3} \sum_{i=1}^N \bigg(\spin_i \cdot \spin_{i+1} + \frac{1}{2} \spin_{i} \cdot \spin_{i+2}\bigg) + \frac{1}{N}, \textnormal{ where } \spin_{N+1} = \spin_1,
\end{equation}

and where $i$ labels the sites of one-dimensional chain with even number of sites. It can be proved that this Hamiltonian is the parent Hamiltonian of the two-dimer states, 

\begin{equation}
    \ket{d}_{\pm} = \prod_{n=1}^{\frac{N}{2}} \frac{\ket{\uparrow_{2n}} \ket{\downarrow_{2n\pm1}} - \ket{\downarrow_{2n}} \ket{\uparrow_{2n \pm 1}}}{\sqrt{2}}, \textnormal{ where } {\bf H}^{\textnormal{MG}} \ket{d}_{\pm} = 0, 
    \label{dimer_states}
\end{equation}

and that all other eigenenergies are positive. Thus, should this statement be true, $\ket{d}_{\pm}$ span the ground state manifold of the Majumdar-Ghosh Hamiltonian. \\

This Hamiltonian includes antiferromagnetic interactions between next-nearest neighbours, which partially frustate the nearest neighbour correlations. Thus, the ground state can be expected to be more disordered than the pure nearest-neighbour model. Indeed, the correlations of the Bethe wave function of the nearest-neighbour model decay as an inverse power of distance, while  \cref{bipartite_valence_bond_states_corrs} states that the dimer state correlations vanish beyond a single lattice constant. 

\begin{proof}

Consider a total spin triad of spins at sites $(i-1, i, i+1$) is given by 

\begin{equation}
    {\bf J}_i = \spin_{i-1} + \spin_i + \spin_{i+1},
\end{equation}

whose squares has eigenvalues $J(J+1)$, where $J = \frac{1}{2}, \frac{3}{2}$. The main pinning point of this proof is that the Majumdar-Ghosh Hamiltonian can be written as a sum of projection operators,

\begin{equation}\begin{split}
    {\bf H}^{\textnormal{HG}} = |K| \sum_{i} \mathcal{P}_{\frac{3}{2}}(i-1,i,i+1), \textnormal{ where } 
        \mathcal{P}_{\frac{3}{2}}(i-1,i,i+1) &= \frac{1}{3} \bigg({\bf J}_i^2 - \frac{3}{4}\bigg) \\
        &= \frac{1}{2} + \frac{2}{3} \bigg(\spin_{i-1} \cdot \spin_i + \spin_{i - 1} \cdot \spin_{i+1} + \spin_{i} \cdot \spin_{i+1} \bigg). 
    \end{split}
\end{equation}

Since ${\bf J}_i^2$'s eigenvalues are $J(J+1)$, this $\mathcal{P}_{\frac{3}{2}}(i-1,i,i+1)$-operator clearly annihilates any state with total spin $J = \frac{1}{2}$ of the triad $(i-1, i, i+1)$. Dimer states, given by \cref{dimer_states}, do not contain states with total ${\bf J}^z > \frac{1}{2}$ of any three sites, since two of the three spins have cancelling ${\spin}^z$ quantum numbers.

Consider a triad with ${\bf J}^z = \frac{1}{2}$ but $J = \frac{3}{2}$ and a global $\mathfrak{s}\mathfrak{u}(2)$-rotation on $\ket{d}$. This will admix, by application of ${\bf J}_i^+$, a component of ${\bf J}_i^z = \frac{3}{2}$ into wave function, which contradicts the rotational invariance of $\ket{d}$\footnote{Remember that dimer states are given in terms of Schwinger boson creation operators acting on the system's ground state. These Schwinger boson creation operators, as stated in \cref{Schwinger_bosons_are_su2vectors}, transform as $SU(2)$-vectors. }. Then, there are no triads with total spin $J > \frac{1}{2}$ and therefore each $\mathcal{P}_{\frac{3}{2}}(i-1,i,i+1)$operator annihilates the dimer states, $\ket{d}_{\pm}$. And given that the $|K|\mathcal{P}_{\frac{3}{2}}(i-1,i,i+1)$-operators are non-negative operators, for these are projection operators, $\ket{d}_{\pm}$ span the ground state manifold of the Majumdar-Ghosh Hamiltonian.

\end{proof}

\paragraph{Square Lattice RVB States}

On the square $S=\frac{1}{2}$-lattice, the resonating valence bond states, given by \cref{Valence_bond_states}, fulfill all the requirements of Marshall's theorems,  \cref{Marshall_theo_1} and \cref{Marshall_theo_2}, for the quantum antiferromagnet\footnote{\begin{tcolorbox}[colback=LimeGreen, title = Historical Context]

These states were proposed as candidates for spin liquid ground states by P.W. Anderson. Although the nearest-neighbour antiferromagnets, on the square lattice is ordered at zero temperature, these states are very poplar in the context of doped antiferromagnets and high-temperature superconductivity. 

\end{tcolorbox}}. For RVB states, it is not easy at all to evaluate correlation for the $\ket{\alpha}$-components are not orthogonal. The number of different bond converings on the square lattice increases exponentially with the number of sites. Moreover, M.E. Fisher has calculated that the number of dimer configurations on the square lattice grows exponentially with the lattice size. \\

Numerical Monte Carlo simulations by Liang, Doucot, and Anderson on finite lattices suggest that the RVB state has no long-range order for bonds that decay at least as rapidly as 

$$
    u_{ij} \approx |{\bf x}_i - {\bf x}_j|^{-p}, \textnormal{ with } p \geq 5.
$$

The RVB states can thus be used as variational ground states for both ordered and disordered phases, making these appealing candidates for studying the transition from the Néel antiferromagnet to a paramagnetic phase. \\

\paragraph{Valence Bond Solids and AKLT Models}

The valence bond solids are defined as follows 

\begin{equation}
\ket{\Psi^{\textnormal{VBS}}} = \prod_{\langle ij \rangle} \bigg({\bf a}^\dagger {\bf b}^\dagger - {\bf b}^\dagger {\bf a}^\dagger\bigg)^M \ket{0}, \begin{array}{c}
     \textnormal{where $\langle ij \rangle$ are all the nearest neighbour bonds of the lattice} \\
     \textnormal{ and where $M \in \mathds{Z}$ such that } M  =\frac{2S}{z} \begin{array}{cc}
          \textnormal{ where $z$ is the lattice }  \\
          \textnormal{ coordination number}
     \end{array}.
     \label{valence_bond_solids}
\end{array}
\end{equation}

From the definition it is clear that $S$ is restricted to values depending on the lattice structure. For instance for $(d=1)$-lattices, $S = 1, 2, \cdots$ and for $(d=2)$-lattices, $S = 2,4, \cdots$. \\

Affleck, Kennedy, Lieb and Tasaki (AKLT) constructed the parent Hamiltonians for the VBS states, given by 

\begin{equation}
    {\bf H}^{\textnormal{AKLT}} = \sum_{\langle ij \rangle} \sum_{J = 2S-M+1}^{2S} K_J \mathcal{P}_{J}(ij), \blanky K_J \geq 0.
\end{equation}

The bond-projector $P_J(ij)$ projects the bond spin ${\bf J}_{ij} = \spin_i + \spin_j$ onto the subspace of $J$-magnitude. Any power of $m = 0, 1, 2, \cdots$ of $\spin_i \cdot \spin_j$ can be written in terms of powers of ${\bf J}_{ij}$ and expanded as a linear combination of bond projector operators 

\begin{equation}
    (\spin_i \cdot \spin_j)^m = \sum_{J = 0}^{2S} \bigg[\frac{1}{2} J(J+1) - S(S+1)\bigg]^m \mathcal{P}_J (ij),
\end{equation}

this relation being useful for inverting the bond-projection operators as polynomials of $\spin_i \cdot \spin_j$. \\

\begin{lemma}

The valence bond solids, defined in \cref{valence_bond_solids}, are the eigenstates of the AKLT-Hamiltonian, 

\begin{equation}
     {\bf H}^{\textnormal{AKLT}} \ket{\Psi^{\textnormal{VBS}}} = 0,
\end{equation}

which holds if $\ket{\Psi^{\textnormal{VBS}}}$ has no component with bond spin $J_{ij} > 2S - M $, for any $(ij)$.

\end{lemma}

\begin{proof}

In effect, consider the contribution to $\ket{\Psi^{\textnormal{VBS}}}$ made by the term with maximally possible number of creation Schwinger operators, ${\bf a}^\dagger$ in particular, on a particular bond $(ij)$,

\begin{equation}
    \cdots ({\bf a}_i^\dagger)^{2S-M} \bigg({\bf a}^\dagger {\bf b}^\dagger - {\bf b}^\dagger {\bf a}^\dagger \bigg)^{M} ({\bf a}_j^\dagger)^{2S-M}\cdots.
\end{equation}

By counting the power of the ${\bf a}^\dagger$-operators minus the number of ${\bf b}^\dagger$-operators, the maximum eigenvalues of ${\bf J}_{ij}^z$ acting on \cref{valence_bond_solids}
is 

\begin{equation}
    {\bf J}^z_{\textnormal{max}} = 2S-M
\end{equation}

\end{proof}

TODO: continue it. 

\clearpage

\section{\textbf{Ground States and Excitations}}

\blanky \\

\begin{tcolorbox}[colback =yellow, title = Physical Context]

The focus of previous sections has  been primarily on the ground state of the Heisenberg model. Knowledge of the exact ground state (or in lack of the exact ground state, a good variational state) allows for the calculation of the equal time spin correlations at zero temperature. Experiments, however, can only measure dynamical responses at finite frequencies and finite temperatures. As it turns out, these dynamical correlations depend on the excited states and energies. \\

The ground state correlations can be used to construct certain approximate low-lying excitations. This approach is called the \textbf{single mode approximation} (SMA). Furthermore, the Heisenberg antiferromagnet provides an opportunity to demonstrate the SMA for a system with a highly-correlated ground state. 

\end{tcolorbox}

\begin{tcolorbox}[colback = LimeGreen, title = Historical Context]

The SMA was originally devised by Bijl and Feynmann  to determine the photon-roton dispersion curve in superfluid He4.

\end{tcolorbox}

In this section, the information about spin excitations and correlations is embodied primarily by the dynamical structure factor and some derived quantities, 

\begin{equation}
\begin{split}
    &\mathcal{S}({\bf q}, \omega) = \frac{1}{N} \int_{\mathds{R}} dt e^{i \omega t} \blanky \sum_{ij} e^{i {\bf q} \cdot ({\bf x} - {\bf x}_j)} \langle \spin_i^z(t) \spin_j^z(0) \rangle = \frac{2\pi}{NZ} \sum_{\alpha, \beta} e^{-\beta E_{\alpha}} \bra{\alpha} \spin_{{\bf q}}^z \ket{\beta} \bra{\beta} \spin^z_{-{\bf q}} \ket{\alpha} \delta(\omega + E_{\alpha} - E_{\beta}), \\
    &\mathcal{S}({\bf q}) = \int_{\mathds{R}} \frac{d\omega}{2\pi} \blanky \mathcal{S}({\bf q}, \omega) = \frac{1}{N} \langle \spin_{{\bf q}}^z \spin_{-{\bf q}}^z \rangle,  \begin{array}{c}
         \textnormal{ which is the equal-time}  \\
         \textnormal{ correlation function }
    \end{array} \\
    &\mathcal{F}({\bf q}) = \frac{1}{N} \bigg \langle \bigg[\spin_{{\bf q}}^z, [\spin_{-{\bf q}}^z, {\bf H}]\bigg] \bigg \rangle = \int_{\mathds{R}} \frac{d\omega}{2\pi} \blanky \omega \mathcal{S}({\bf q}, \omega), \begin{array}{cc}
         \textnormal{ which is the double commutator function, previously } \\
         \textnormal{  encountered in the proof of the Mervin-Wagner} \\
         \textnormal{ theorem,  \cref{Mervin-Wagner}.}
    \end{array}
    \label{Goldstone_spin_dynamical_factors}
\end{split}
\end{equation}

Note that, even though $\mathcal{F}({\bf q})$ is a static ground state expectation value, it contains dynamical information through the Hamiltonian. These relations thus show that the $\mathcal{S}({\bf q})$ and $\mathcal{F}({\bf q})$ describe the average and the first frequency moment of $\mathcal{S}({\bf q}, \omega)$, respectively. Hence, a characteristic frequency for spin excitations at ${\bf q}$-momentum is defined as 

\begin{equation}
    \bar{\omega}_{{\bf q}} = \frac{\mathcal{F}({\bf q})}{\mathcal{S}({\bf q})}. 
    \label{Goldstone_char_freq}
\end{equation}

\blanky \\

\paragraph{The Single Mode Approximation}

Consider a quantum system with ground state $\ket{0}$. By translational invariance, all excitations may be labelled by the lattice ${\bf q}$-momentum. A "single mode" is constructed then as 

\begin{equation}
    \ket{{\bf q}} = \spin^z_{{\bf q}} \ket{0}.
\end{equation}

The single mode state approximates a true excitation of the system if $\mathcal{S}({\bf q}, \omega)$ at $T = 0$ is sharply peaked about $\omega = \bar{\omega}_{\bf q}$, i.e.

\begin{equation}
    \mathcal{S}({\bf q}, \omega) \approx 2\pi \mathcal{S}({\bf q}) \delta(\omega- \omega_{\bf q}),
\end{equation}

which will be precise if 

\begin{equation}
    ({\bf H} - E_0)\ket{{\bf q}} = \omega_{{\bf q}} \ket{{\bf q}}.
\end{equation}

In general, even if this holds exactly, the single mode states are only a small subset of all excitations -i.e. those connected to the ground state via $\spin^z_{\bf q}$. If their products are also elementary excitations, they can be considered as weakly interacting elementary excitations. Thus, their energies approximately determine the system's low-frequency response and low-temperature thermodynamics. In \cref{Quantum_disorder_zero_temp_MW}, it was mentioned that a non-zero gap in the excitation spectrum in the Heisenberg model implies the absence of long-range order at zero temperature (ie. in said conditions, the ground state of the Heisenberg model must be disordered). In this case, the dynamical structure factor must vanish below the gap frequency 

\begin{equation}
    \mathcal{S}({\bf q}, \omega) = 0, \textnormal{ for } \omega \in (0, \Delta).
\end{equation}

Then, the single-mode energy $\bar{\omega}_{{\bf q}}$ is an upper bound on the lowest-excitation energy at ${\bf q}$-momentum,

\begin{equation}
    \min_{\alpha} E_{\alpha}({\bf q}) \leq \bar{\omega}_{{\bf q}}.
    \label{Goldstone_Single_mode_approx_ineq}
\end{equation}

However, the existence of an energy gap $\Delta^{\textnormal{SMA}}$ in the single-mode spectrum,

\begin{equation}
    \Delta^{\textnormal{SMA}} \leq  \bar{\omega}_{{\bf q}}, \textnormal{ } \forall {\bf q}.
\end{equation}

does not imply the existence of a true gap for the exact eigenergies $E_{\alpha}$. \\

\paragraph{\textbf{Goldstone's Theorem in Condensed Matter Physics}}

The \cref{Goldstone_Single_mode_approx_ineq}'s inequality serves not to prove the existence of a gap in the excitation spectrum, but rather to disprove its existence ie. it can be used to prove gaplessness. Thus, the lattice-version of the Goldstone theroem can be used on the Heisenberg model. \\

\begin{tcolorbox}[colback = yellow, title = Physical Context]

The essence of Goldstone's theorem is that for a Hamiltonian with short-range interactions and no gauge fields, spontaneously broken continuous symmetry implies the existence of low-energy excitations called \underline{Goldstone modes}. If the ground state has $\bar{\bf q}$-momentum, the energy of the Goldstone mode vanishes as ${\bf q} \rightarrow \bar{\bf q}.$

\begin{tcolorbox}[colback = Bittersweet, title = Example]

For example, 

\begin{itemize}
    \item $\bar{ \bf q} = 0$ for the ferromagnet, 
    \item $\bar{ \bf q} = \pi$ for the Néel antiferromagnet in the cubic lattice.
\end{itemize}

\end{tcolorbox}

In relativistic field theories, Goldstone modes represent massless particles. In condensed matter physics on the other hand, Goldstone modes appear in many systems, eg.

\begin{itemize}
    \item acoustic phonons in solids that break translational symmetry, 
    \item spin waves in O$(n)$ Heisenberg models, or $n > 1$ with $n$ the dimension of the spins. 
\end{itemize}

\end{tcolorbox}

\blanky \\

Consider the short-range Heisenberg Hamiltonian, given in \cref{HH_short_range_MW}, which satisfies the conditions of the Mervin-Wagner theorem, \cref{Mervin-Wagner}. Then, the following theorem holds

\begin{theorem}\label{Goldstone_theo} Goldstone's Theorem \\

If the spin correlation diverges at some ${\bar{\bf q}}$-wave vector, ie.

$$
    \lim_{{\bf q} \rightarrow \bar{\bf q}} \mathcal{S}({\bf q}) \rightarrow \infty,
$$

then there exists a \textnormal{Goldstone mode} labelled by the ${\bf q}$-momentum, whose energy $E({\bf q})$ vanishes at ${\bar {\bf q}}$, ie.

$$
    \lim_{{\bf q} \rightarrow \bar{\bf q}} E({\bf q}) = 0.
$$

\end{theorem}

\begin{proof}

In effect, consider the characteristic frequency for spin excitations at ${\bf q}$-momentum, defined in \cref{Goldstone_char_freq} and \cref{Goldstone_spin_dynamical_factors}. The double commutator, which appears in both relations, has an upper bound given by \cref{Bogoliubov_inequality_double_comm}, which in turn induces a lower bound on the characteristic frequency. Thus, for $h=0$, 

\begin{equation}
    \mathcal{F}({\bf q}) \leq \cancel{h m_{{\bf q}}} + S(S+1) \bar{J} |{\bf q}|^2 \blanky \Longrightarrow \blanky \bar{\omega}_{{\bf q}} \leq \frac{2S(S+1) \bar{J} |{\bf q}|^2}{\mathcal{S}({\bf q})}.
\end{equation}

Then, consider the case in which $\lim_{{\bf q} \rightarrow \bar{\bf q}} \mathcal{S}({\bf q}) \rightarrow \infty$. Furthermore, \cref{Goldstone_Single_mode_approx_ineq} induces an upper bound on the lowest-level excitation energy at a fixed $\bar{\bf q}$-momentum  

$$    
\min_{\alpha} E_{\alpha}({\bf q}) \leq \bar{\omega}_{{\bf q}} \rightarrow \infty \textnormal{ which implies that } \lim_{{\bf q} \rightarrow \bar{\bf q}} E({\bf q}) = 0.
$$
\end{proof}

\blanky \\

\begin{corr}

If the ground state has broken symmetry, there exists Goldstone modes, as in \cref{Goldstone_theo}. 

\end{corr}

\begin{proof}

This statement follows directly from \cref{MW_Broken_Symm_and_Long_range_order}'s red box titled "Spontaneously broken symmetry and true long-range order", in which it is proven that an spontaneously broken symmetry implies true long-range order and the divergence of $\mathcal{S}({\bf q})$ as $N \rightarrow \infty$. 

\end{proof}

\blanky \\

\begin{tcolorbox}[colback = yellow, title = Physical Context]

It is important to note that the converse of the Goldstone theorem is false, ie. the existence of gapless excitations does not imply true long-range order. Notable counterexamples are,

\begin{itemize}
    \item the Fermi gas, which has gapless particle-hole excitations but no long-range order,
    \item and the $S= \frac{1}{2}$-one-dimensional Heisenberg model has gapless excitations which vanish at $q = \{0, \pi\}$ with ${\omega}_{q} \propto |\sin q|$ but is correlation decay algebraically at long distances. 
    \item The Haldane gap in the $S=1$-AKLT models. 
          
          The ground states of the AKLT models are valence bond solids, with single mode energies given by 
          
          $$
              \bar{\omega}_{\bf q} \geq 0.370K.
          $$
          
          As discussed earlier, although the single-mode spectrum has a gap of magnitude $0.370K$, this doesn't imply that a gap actually exists in the exact spectrum. However, numerical simulations for the $S=1$-AKLT model support the existence of a gap of magnitude $\Delta = 0.350K$. The same simulation for the standard Heisenberg model shows the existence of a gap of size $0.325K$. Since the AKLT model has a gap in the SMA-spectrum, this numerical evidence suggests that the additional biquadratic terms $\frac{K}{3} (\spin_i \cdot \spin_j)^2$ do not drastically alter neither the ground state correlations nor the low lying excitation of the $S=1$-Heisenberg model in one dimension. Furthermore, the Haldane gap survives in the thermodynamical limit, this being a general feature of integer spin quantum antierromagnets in low dimensions. 
\end{itemize}

\end{tcolorbox}

\clearpage

\iffalse
%\paragraph{The Haldane Gap and the SMA}

The ground states of the AKLT models are valence-bond solids. The one-dimensional model for $S = 1$ is given by 

\begin{equation}
    {\bf H}^{\textnormal{AKLT}} = K \sum_{\langle ij \rangle} \bigg[\spin_i \cdot \spin_j + \frac{1}{3} (\spin_i \cdot \spin_j)^2 + \frac{2}{3}\bigg]
\end{equation}

The correlation function for $S=1$ VBS states is given by 
\fi
