\documentclass{homework}
\author{Tomás Pérez}
\class{Condensed Matter Theory - Lecture Notes}
\date{\today}
\title{Theory \& Notes}

\graphicspath{{./media/}}

\begin{document} \maketitle

\section{XX-model}

Consider the XX-Heisenberg model, with its Hamiltonian given in terms of the traditional $\frac{1}{2}$-spin operators ie. 

\begin{equation}
    {\bf H} = J \sum_{i=1}^{L} ({\bf S}_{j}^{x} {\bf S}_{j+1}^{x} + {\bf S}_{j}^{y} {\bf S}_{j+1}^{y}) - \lambda \sum_{j=1}^{L} {\bf S}_{j}^{z},
    \label{XX hamiltonian}
\end{equation}

which describes interacting spins in a one-dimensional chain, with periodic boundary conditions. \eqref{XX hamiltonian}'s first terms represents nearest neighbour interactions in the $x$ and $y-$directions  interactions, with $J$ being either positive or negative and quantifying the strength and type of interactions, while the second term represents a magnetic field of strength $\lambda$, applied in the $z$-direction of the spins. \\

In order to solve this problem, it is necessary to rewrite \eqref{XX hamiltonian} and apply a Jordan-Wigner transformation, mapping the spin problem into a fermionic problem. But first, it is convenient to write the spin-operators in terms of the raising and lowering $\mathfrak{su}(2)$-operators, ie.

$$
    \spin_j^{\pm} = \spin_j^{x} \pm i \spin_j^{y} \blanky \Rightarrow \blanky \begin{array}{c}
         \spin_{j}^x = \frac{1}{2} (\spin_j^+ + \spin_j^-), \\
          \\
         \spin_{j}^y = \frac{1}{2i} (\spin_j^+ - \spin_j^-). 
    \end{array}
$$

Then, the XX-Hamiltonian can be re-written as 

\begin{equation}
     {\bf H} = \frac{J}{2} \sum_{i=1}^{L} ({\bf S}_{j}^{+} {\bf S}_{j+1}^{-} + {\bf S}_{j}^{-} {\bf S}_{j+1}^{+}) - \lambda \sum_{j=1}^{L} {\bf S}_{j}^{z}
     \label{raising Hamiltonian}
\end{equation}

where the interacting terms in the first summation, flip neighboring spins if said spins are anti-aligned\footnote{In effect, consider for example, a two-spin problem. Then, the interaction term is given by 

$$
\spin_1^+ \spin_2^- + \spin_1^- \spin_2^+,
$$

and consider an anti-aligned state $\ket{\downarrow\uparrow}$. Then, the action of the previous two-spin operator over this state yields

$$
(\spin_1^+ \spin_2^- + \spin_1^- \spin_2^+) \ket{\downarrow\uparrow} = \ket{\uparrow\downarrow} + 0,
$$

since $\spin_1^- \spin_2^+$ destroys the state. Similarly, $(\spin_1^+ \spin_2^- + \spin_1^- \spin_2^+)\ket{\uparrow\downarrow} = \ket{\downarrow\uparrow}$. However, note that, should both spins be either up or down, the state remain invariant under the action of the two-spin operator. 

\begin{align}
    (\spin_1^+ \spin_2^- + \spin_1^- \spin_2^+) \ket{\downarrow\downarrow} = \ket{\downarrow\downarrow} \textnormal{ and } (\spin_1^+ \spin_2^- + \spin_1^- \spin_2^+) \ket{\uparrow\uparrow} = \ket{\uparrow\uparrow} 
\end{align}}. In addition, the XX-Hamiltonian has a total magnetization symmetry, since the Hamiltonian given by \eqref{raising Hamiltonian} conmutes with the magnetization operator. \\

Now, we use a Jordan-Wigner transformation whereby the spin operators are mapped to fermionic operators, as follows 

\begin{equation}
    \begin{array}{c}
         \spin_j^z = f_j^\dagger f_j - \frac{1}{2}  \\
         \\
         \spin_j^- = \exp\bigg(i\pi \sum_{\ell = 1}^{L-1} f_\ell^\dagger f_\ell \bigg) \\
         \\
         \spin_j^+ = \exp\bigg(-i\pi \sum_{\ell = 1}^{L-1} f_\ell^\dagger f_\ell \bigg) 
    \end{array}
    \label{JW tr}
\end{equation}

Under the Jordan-Wigner map, nearest-neighbours spin flipping is translated into to nearest-neighbours fermionic hopping, ie. $\spin_{j}^{+} \spin_{j+1}^{-} = f_{j}^{\dagger} f_{j+1}$ and $\spin_{j}^{-} \spin_{j+1}^{+} = f_{j+1}^{\dagger} f_{j}$. However, due to the boundary conditions' periodicity, the XX-Hamiltonian cannot be rewritten as a fermionic model yet since (it will contain an additional boundary term), for example, the fermionic counterparts to the $\spin_{L}^{+} \spin_{1}^{-}$ interaction are highly non-local operators and are not desirable. Indeed, under the Jordan-Wigner mapping 

\begin{equation*}
    \spin_{L}^{+} \spin_{1}^{-} = f_L^\dagger \exp\bigg(-i\pi \sum_{\ell = 1}^{L-1} f_\ell^\dagger f_\ell \bigg) f_1,
\end{equation*}

which is not problematic, since it accounts for all $L$-lattice sites. Let

\begin{equation}
    \begin{array}{c}
       \spin_{L}^{+} \spin_{1}^{-} = \mathcal{Q} f_L^\dagger f_1, \\
       \spin_{L}^{-} \spin_{1}^{+} = \mathcal{Q} f_1^\dagger f_L,
    \end{array}
\end{equation}

then \eqref{raising Hamiltonian} can be rewritten as 

\begin{equation}
    {\bf H} = \frac{J}{2} \sum_{i=1}^{L-1} \bigg(f_{j}^{\dagger} f_{j+1} + f_{j+1}^{\dagger} f_j\bigg) - \lambda \sum_{j=1}^{L} \bigg(f_{j}^{\dagger} f_{j} - \frac{1}{2}\bigg) + \frac{J}{2} \mathcal{Q} (f_L^\dagger f_1 + f_1^\dagger f_L),
    \label{fermionic no boundary}
\end{equation}

where the first term accounts for fermionic nearest-neighbour hopping, the second term accounts for the magnetic field, and the third term being the non-local boundary term. Note that this fermionic Hamiltonian hasn't got any type of boundary conditions, since the $L$-lattice site is disconnected in any way whatsoever from the first lattice site. Then, the standard procedure is to add and subtract terms from the Hamiltonian, so that the nearest-neighbour hopping term in \eqref{fermionic no boundary} can also have periodic boundary conditions, thus yielding 

\begin{equation}
    {\bf H} = \frac{J}{2} \sum_{i=1}^{L} \bigg(f_{j}^{\dagger} f_{j+1} + f_{j+1}^{\dagger} f_j\bigg) - \lambda \sum_{j=1}^{L} \bigg(f_{j}^{\dagger} f_{j} - \frac{1}{2}\bigg) + \frac{J}{2} (\mathcal{Q}-1) (f_L^\dagger f_1 + f_1^\dagger f_L),
    \label{true fermionic hamiltonian}
\end{equation}

where now the fermionic hopping term has the standard boundary conditions. The third term, since it does not involve any type of summation over lattice sites, only contributes at $\mathcal{O}\bigg(\frac{1}{L}\bigg)$-order to any microscopic quantity. In the thermodynamic limit, this non-local term can be dropped, thus yielding an $\mathcal{O}(L)$-Hamiltonian given by 

\begin{equation}
    {\bf H} = \frac{J}{2} \sum_{i=1}^{L} \bigg(f_{j}^{\dagger} f_{j+1} + f_{j+1}^{\dagger} f_j - \lambda f_{j}^{\dagger} f_{j}\bigg) + \frac{\lambda L}{2},
    \label{L-order fermionic hamiltonian}
\end{equation}

which is now fully cyclic and where its operators obey fermionic algebras. This Hamiltonian can then be diagonalized via a discrete Fourier transform on the fermionic operators 

\begin{align}
    f_j &= \frac{1}{\sqrt{L}} \sum_{\substack{k = {2\pi m/L}\\
    m \in \mathds{Z}_{[1, L]} }}
                  e^{ijk} d_k, & f_j^{\dagger} &=\frac{1}{\sqrt{L}}  \sum_{\substack{k = {2\pi m/L}\\
    m \in \mathds{Z}_{[1, L]}}} e^{-ijk} d_k^{\dagger},
\end{align}

with the inverse transformation given by 

\begin{align}
    d_k &= \frac{1}{\sqrt{L}} \sum_{j=1}^{L}
                  e^{-ikj} f_j &  d_k^{\dagger} &= \frac{1}{\sqrt{L}} \sum_{j=1}^{L}
                  e^{ikj} f_j^{\dagger}.
\end{align}

Note that the $d_k$-operators follow the standard fermionic anticonmutation algebra. Note as well, that the $f_j$-vacuum state, defined such that $f_j \ket{0}_f = 0, \blanky \forall j$, is the same as the $d_k$-vacuum state, defined such that $d_j \ket{0}_d = 0, \blanky \forall k$, ie. $\ket{0}_f = \ket{0}_d$. Another important relationship is the Fourier transform's consistency condition, ie. 

$$
\sum_{j=1}^{L} e^{i(k-q)j} = L \delta_{kq}.
$$

Under the Fourier transform, \eqref{L-order fermionic hamiltonian}'s terms are mapped as follows 

\begin{align*} 
    \sum_{j=1}^{L} f_j^\dagger f_{j+1} &= \sum_{j=1}^{L} \frac{1}{L} \sum_{k, \blanky q} e^{-ikj} e^{iq(j+1)} d_k^\dagger d_q = \sum_{j=1}^{L} \frac{1}{L} \sum_{k, \blanky q} e^{i(q-k)j} e^{iq} d_k^\dagger d_q \\
    &= \sum_{k, \blanky q} \frac{1}{L} e^{iq} \delta_{qk} d_k^\dagger d_q = \sum_{k} e^{ik} d_k^\dagger d_k \\
    \sum_{j=1}^{L} f_{j+1}^\dagger f_{j} &=  \sum_{j=1}^{L} \frac{1}{L} \sum_{k, \blanky q} e^{-ik(j+1)} e^{iqj} d_k^\dagger d_q = \sum_{j=1}^{L} \frac{1}{L} \sum_{k, \blanky q} e^{i(q-k)j} e^{-ik} d_k^\dagger d_q \\
    &= \sum_{k, \blanky q} \frac{1}{L} e^{-ik} \delta_{qk} d_k^\dagger d_q = \sum_{k} e^{-ik} d_k^\dagger d_k \\
    \sum_{j=1}^{L} f_{j}^\dagger f_{j} &=  \sum_{j=1}^{L} \frac{1}{L} \sum_{k, \blanky q} e^{-ikj} e^{iqj} d_k^\dagger d_q = \sum_{j=1}^{L} \frac{1}{L} \sum_{k, \blanky q} e^{i(q-k)j}  d_k^\dagger d_q \\
    &= \sum_{k, \blanky q} \frac{1}{L} \delta_{qk} d_k^\dagger d_q = \sum_{k} d_k^\dagger d_k \\
\end{align*} 

Therefore, using these identities, the new Hamiltonian is given by 

\begin{align}
    {\bf H} &= \frac{J}{2} \sum_{k} \bigg(e^{ik} d_{k}^{\dagger} d_{k} + e^{-ik} d_{k}^{\dagger} d_k - \lambda d_{k}^{\dagger} d_{k}\bigg) + \frac{\lambda L}{2} \\
    &= \sum_{k} \bigg(J \cos k - \lambda \bigg)d_{k}^{\dagger} d_{k} + \frac{\lambda L}{2},
\end{align}

which can be rewritten as 

\begin{align}
    \alignedbox{{\bf H} }{= \sum_{k} \epsilon_k d_k^\dagger d_{k} + \frac{\lambda L}{2} \begin{array}{c}
         \textnormal{ with the eigenvalues being given by } \epsilon_k = J \cos k - \lambda + \frac{\lambda L}{2} \\
         \textnormal{ and the eigenvectors being given by } \ket{E_n} = \prod_{n} (d_k^\dagger)^n,
         \textnormal{                    with eigenvalue $E_n = \sum_{n} \epsilon_n $}  
    \end{array}},
\end{align}

where, $k = \frac{2\pi m}{L}, \blanky m = -\frac{L}{2} + 1, \cdots \frac{L}{2}, $ ie. $k \in (-\pi, \pi]$, this is called the first Brillouin zone. Thus the problem has been solved \\

As for its thermal properties, since this is a fermionic model, the fermions will obey the Fermi-Dirac distribution, ie. 

\begin{equation}
    \mathcal{N}_{jk} = \langle d_j^\dagger d_k \rangle_{\textnormal{th}} =  \frac{1}{1+e^{\beta \epsilon_k + \mu}} \delta_{jk} 
\end{equation}

\clearpage

\subsection{Quantum Phase Transition}

The system's groundstate is defined such that 

\begin{equation}
    \bra{E_G} d_k^\dagger d_k \ket{E_G} = 1 \textnormal{ if } \epsilon_k \leq 0 \textnormal{ and } \bra{E_G} d_k^\dagger d_k \ket{E_G} = 0 \textnormal{ if } \epsilon_k > 0.
\end{equation}

Then the groundstate energy is given by 

\begin{equation}
    E_G = \sum_{k} \epsilon_k \bra{E_G} d_k^\dagger d_k \ket{E_G}.
\end{equation}

Now, there exists a momenta $k_F$ such that $\epsilon_{k_F} = 0$, with $k_F$ being the Fermi momentum, ie. $\epsilon_{k_f} = J \cos k_f - \lambda$. The previous equation depends both on the coupling strength $J$ and the magnetic field strength $\lambda$. From the previous equation some special cases arise, namely 

\begin{itemize}
    \item when $\lambda < -J$, which yields a Fermi momentum $k_F = \pi$,
    \item when $|\lambda| \leq J$, which yields a Fermi momentum $k_F = \arccos \bigg(\frac{\lambda}{J}\bigg)$,
    \item when $\lambda > J$ which yields $k_F = 0$.
\end{itemize}

Therefore, the system groundstate will be defined by the following relations, 

\begin{equation}
    \begin{array}{c}
         \bra{E_G} d_k^\dagger d_k \ket{E_G} = 1  \\
         \\
         \textnormal{ if } k \in [-\pi, -k_F] \cup [k_F, \pi],
    \end{array} \textnormal{ and }  \begin{array}{c}
         \bra{E_G} d_k^\dagger d_k \ket{E_G} = 0  \\
         \\
         \textnormal{ if } k \in [-k_F, k_F].
    \end{array}
\end{equation}

\blanky \\

The total magnetization operator is defined as the sum of the $\spin_z$ operators, along the entire chain ie. ${\bf M}^z = \sum_{j=1} \spin^z_j$. Under the Jordan-Wigner transformation, $\spin^z_j = f_j^\dagger f_j - \frac{1}{2}$. In turn, these fermionic number operators can be mapped to fermionic number operators acting on momentum space, ie. $\sum_{j=1}^{L} f_j^\dagger f_j = \sum_k d_k^\dagger d_k$. This entails

\begin{equation}
    {\bf M}^z = \sum_{k} \bigg(d_k^\dagger d_k - \frac{1}{2}\bigg).
\end{equation}

Then the total magnetization operator's expectation value for the groundstate can be calculated. Given the system's fermionic nature, momentum eigenmodes are either occupied or unoccupied, contributing only one or zeros respectively. Then, this expectation value yields

\begin{equation}
    \bra{E_G} {{\bf M}^z} \ket{E_G} = \sum_{k=-\pi}^{-k_F} 1 + \sum_{k=k_F}^{\pi} 1 - \frac{1}{2} \sum_{k=-\pi}^{\pi} 1 \\
    = \frac{1}{2} \bigg( \sum_{k=-\pi}^{-k_F} 1 + \sum_{k=k_F}^{\pi} 1 - \sum_{k = -k_F}^{k_F} 1  \bigg), \label{total magnetization ev}
\end{equation}

where the total magnetization expectation value is just half the difference of two competing summations, related to the negative and positive eigenmodes. Defining the total magnetization per site operator, ${{\bf m}}^z = \frac{1}{L} {\bf M}^z$. Now, as the chain gets progressively bigger, with more and more lattice sites, the summations in \eqref{total magnetization ev} can be approximated by Riemannian integrals, ie.

\begin{equation*}
    \sum_{k} f(k) = \frac{1}{\Delta k} \sum_{k} f(k) \Delta k \overset{L \rightarrow \infty}{\rightarrow} \frac{L}{2\pi}\int_{k \in \mathds{B}\mathds{Z}} dk f(k) \textnormal{ where } \Delta k = \frac{2\pi}{L}.
\end{equation*}

Using this trick, \eqref{total magnetization ev} can be rewritten as 

\begin{align}
    \langle {{\bf m}}^{z} \rangle_{G} &= \frac{1}{4\pi} \bigg(\int_{-\pi}^{-k_F} dk + \int_{k_F}^{\pi} dk - \int_{-k_F}^{k_F} dk \bigg) = \frac{1}{4\pi} \bigg(-k_F + \pi + \pi - k_F - k_F + k_F\bigg) \\
    &= \frac{1}{2} - \frac{k_F}{\pi},
\end{align}

which yields the total magnetization per lattice site. Note that at $\lambda < J$, $k_F = \pi$ whereby there is total polarization ie. $\langle {{\bf m}}^{z} \rangle_{G} = -\frac{1}{2}$. Similarly, for $\lambda > J$, $k_F = 0$ and then $\langle {{\bf m}}^{z} \rangle_{G} = \frac{1}{2}$. However, if $|\lambda| \leq J$, then $k_F = \arccos\bigg(\frac{\lambda}{J}\bigg)$, which then yields a total magnetization per lattice site given by 
\begin{equation} \langle {{\bf m}}^{z} \rangle_{G}  = \left\{
    \begin{array}{cc}
          \frac{1}{2} - \frac{1}{\pi}\arccos\bigg(\frac{\lambda}{J}\bigg)&  \textnormal{ if } |\lambda| \leq J  \\
          \\
          -\frac{1}{2} & \textnormal{ if } \lambda < J  \\
          \\
          \frac{1}{2} & \textnormal{ if } \lambda > J  \\
    \end{array} \right.
\end{equation}

Another interesting quantity is the total magnetic susceptibility, which can be written in terms of the total magnetization per lattice site as follows 

\begin{align}
    \chi = \bigg(\frac{\partial \langle {{\bf m}}^{z} \rangle_{G}}{\partial \lambda}\bigg) \Rightarrow \chi = \left\{ \begin{array}{cc} 
         \frac{1}{J \pi \sqrt{1- \frac{\lambda^2}{J^2}}} \textnormal{ if } |\lambda| \leq J  \\
         \\
          0 \textnormal{ if } \lambda < J \\ 
         \\
          0 \textnormal{ if } \lambda > J  \\
              \end{array} \right., \label{}
\end{align}

which is divergent if $\lambda = J$. This shows that the XX-Heisenberg model has a second order (since there is no latent heat involved it can't be a first order phase transition) quantum phase transition, with an associated power law with a critical exponent.  \\

\section{Numerical solution to Fermionic models}

Consider a Hamiltonian describing a fermionic system, given by 

\begin{equation}
    {\bf H} = J \summation (f_{j}^{\dagger}f_{j+1} + f_{j+1}^{\dagger}f_{j}) + \sum_{j=1} \lambda_j f_{j}^{\dagger} f_j, \begin{array}{c}
         \textnormal{with the usual } \\
         \textnormal{ conmutation rules} 
    \end{array}
    \begin{array}{c}
         \{f_j, f_k\} = \{f_j^\dagger, f_k^\dagger\} = 0  \\
         \{f_j, f_k^\dagger\} = \delta_{jk}
    \end{array}
    \label{fermionic hamiltonian}
\end{equation}

where $L$ indicates the number of lattice sites, $J$ is the hopping strength, which could be either positive or negative, and where $\lambda_j$ is the on-site potential strength\footnote{The $\lambda_j$-term frequently appears in many condensed matter models, with different numerical values and interpretations, eg.

\begin{itemize}
    \item In the XX model, $\lambda_j = \lambda \blanky \forall j$. 
    \item While for the Anderson model $\lambda_j \in \mathcal{U}_{\R_{[-W, W]}}$, a uniform random variable, with $W$ being the disorder strength. 
    \item In the Aubry-André model, $\lambda_j = \lambda \cos(2\pi\sigma j)$, with $\sigma \in \mathds{I}$ and $\lambda$ quantifying the disorder strength. 
\end{itemize}}. Said Hamiltonian has open boundaries conditions since there is no hopping term across the boundary. Note that we can rewrite \eqref{fermionic hamiltonian} as 

\begin{equation}
    {\bf H} = \sum_{i, j = 1}^{L} \M_{ij} f_i^{\dagger} f_j \textnormal{ with } \begin{array}{c}
         \M \in \textnormal{GL}(L, \R), \\
         \M_{ij} = \left\{\begin{array}{cc}
             \lambda_i & \textnormal{ if } i=j  \\
              J & \textnormal{ if } j=i+1 \textnormal{ or } i = j + 1 \\
              0 & \textnormal{ otherwise}
         \end{array} \right.
    \end{array},
\end{equation}

which is a positive-defined tri-diagonal matrix.
Let ${\bf f} = (f_1 \blanky f_2 \blanky \cdots f_L)\transpose$ be a vector of the $L$ fermionic operators. Then, \eqref{fermionic hamiltonian} can be rewritten as 

\begin{align}
    {\bf H} = {\bf f}^\dagger \M {\bf f}.
\end{align}

Since $\M$ is symmetric, then it can be diagonalized  $\M = A \mathcal{D} A\transpose$, where $A \in \mathds{R}^{L \times L}$ is a real orthogonal matrix and with $\mathcal{D}_{ij} \in \mathds{R}^{L \times L} \blanky | \blanky \mathcal{D}_{ij}  = \epsilon_i \delta_{ij}$. In this context, the $A$-matrix acts on the fermionic operator as a Bogoliubov transformation, allowing for \eqref{fermionic hamiltonian} to be rewritten as 

\begin{equation}
     {\bf H} = {\bf f}^\dagger A \mathcal{D} A\transpose {\bf f} = {\bf d}^\dagger \mathcal{D} {\bf d}
     \label{fermionic matrix hamiltonian}
\end{equation}

where ${\bf d} = A\transpose {\bf f}$. Since the $A$-matrix is orthogonal, the new $d_k$-operators are fermionic operators as well, satisfying \eqref{fermionic hamiltonian}'s anti-commutation rules. Then, the new fermionic operators are 

\begin{equation*}
    \begin{array}{c}
         d_k = \sum_{j=1}^{L} A_{jk} f_j  \\ 
         \\
         f_j = \sum_{k=1}^{L} A_{jk} d_k 
    \end{array} \textnormal{ since } A\transpose A = \sum_{j,k = 1}^{L} A_{jk} A_{kj} = \mathds{1}_{L}.
\end{equation*}

Then, we can expand \eqref{fermionic matrix hamiltonian} in terms of the lattices, as follows 

\begin{equation}
    {\bf H} = \sum_{k = 1}^{L} \epsilon_k d_k^\dagger d_k,
\end{equation}

which is a sum of number operators with potentials. The eigenstates can then be constructed from the the theory's vacuum state, by applying the $d_k^{\dagger}$-fermionic operators. In the Heisenberg-picture, the $d_k$-operators can be evolved via the Heisenberg equation of motion

\begin{equation}
\frac{d}{dt} d_k = i [{\bf H}, d_k],
\label{H eom}
\end{equation}

and using that $d_k^2 = 0$, it turns out that \eqref{H eom}'s solution is simply $d_k(t) = e^{-i\epsilon_k t} d_k$. The system's correlation can be easily found by analyzing the following matrix. Let $\mathcal{N}_{jk} = \langle d_j^\dagger d_k \rangle$, where the expectation value is taken via calculating the operator's trace along the Fock space, which takes the following values 

\begin{equation}
    \mathcal{N}_{jk} = \langle d_j^\dagger d_k \rangle = \left\{\begin{array}{c}
        0 \textnormal{ or } 1 \textnormal{ if } j=k  \\
        0 \textnormal{ if } j \neq k 
    \end{array} \right.,
\end{equation}

ie. different lattice-sites are not correlated and there can only be a single fermion at most per lattice site, in accordance with Pauli's principle. A ground state, for example, would choose to turn on all fermions in the eigenmode $d$-space such that $\epsilon_k < 0$. If instead, the expectation value is taken with thermal states, the Fermi-Dirac distribution is returned, 

\begin{equation}
    \mathcal{N}_{jk} = \langle d_j^\dagger d_k \rangle_{\textnormal{th}} =  \frac{1}{1+e^{\beta \epsilon_k + \mu}} \delta_{jk}.
\end{equation}

Another interesting quality is a system with an initial configuration where the system's initial state, in real space, is known. In this setting, $\mathcal{N}_{jk}$ is known for all lattices. Consider for example the Anderson model, where the system's initial state is given by a single tensor product of $n$-fermionic states in real space, with $n<L$. Then, for all lattice sites, we have that $\N_{jj}$ is either zero or one. The $\N_{jk}$-matrix entries can then be evaluated as 

\begin{align*}
    \langle d_j^\dagger d_k \rangle = \sum_{i,j = 1}^{n < L} A_{ik} A_{jl} \langle f_i^\dagger f_j \rangle = \sum_{j=1} A_{jk} A_{jl} \langle f_j^\dagger f_j \rangle,
    \label{real system correlations}
\end{align*}

which can then be numerically computed to obtain the LHS expectation value. In general, this $\N_{jk}$-matrix will not be diagonal, which is reasonable since the system's real configuration is not an eigenstate. In principle and in practice, by inverting \eqref{real system correlations}, we can evolve any number operator or two-body correlation operator, ie.

\begin{align}
    \langle f_j^\dagger f_k \rangle = \sum_{k,l =1}^{n} A_{jk} A_{jl} \langle d_k^\dagger d_l \rangle.
\end{align}

This quantities' time evolution can then be found out to be 

\begin{equation}
    \langle f_j^\dagger(t) f_k(t) \rangle = \sum_{k,l =1}^{L} e^{i(\epsilon_k - \epsilon_l)t}A_{jk} A_{jl} \langle d_k^\dagger d_l \rangle,
\end{equation}

which can then be numerically solved. 

\clearpage 

\section{Independent Electrons and Static Crystals}

\subsection{Crystal lattices} 

The mathematical concept which best describes an actual crystal lattice is that of a Bravais lattice, defined as a set of mathematical points corresponding to the discrete positions in space given by 

\begin{equation}
    \{\Rb \in \R^{3} \blanky | \blanky \R = \sum_{i=1}^{3} n_i {\bf a}_i \textnormal{ with } n_i \in \mathds{Z}\},
\end{equation}
    

where the $\{a_i\}$-vector are the so-called primitive vector. The Bravais lattice is invariant under the operation 

$$
\Rb \rightarrow \Rb + {\bf T} \textnormal{ where } {\bf T} = \sum_{i=1}^{3} L_i {\bf a}_i \blanky | \blanky L_i \in \mathds{Z}
$$

\clearpage

\section{Bethe ansatz}

The Hamiltonian matrix which defines the Heisenberg XYZ-chain can be experessed in terms of an $R$-matrix which is a solution of the Yang-Baxter equation, which leads to the model's integrability. The most convenient approach is to construct a transfer matrix -a one parameter commutative family of operators acting on the full state space of the Heisenberg spin chain. \\

\paragraph{\textbf{The Lax operator}}

The algebraic Bethe ansatz' basic tool is the lax operator $\bm{\mathcal{L}}$. Consider a chain with $N$ sites and a corresponding Hilbert space given by 

\begin{equation}
    \mathds{H} \sim \bigotimes_{n=1}^{N} {\mathds{V}_n} \textnormal{ where } {\mathds{V}_n} \sim \ctwo
\end{equation}

To these spaces, an additional auxiliary space $\mathds{V}_a \sim \ctwo$ can be added. Then, the lax operator $\bm{\mathcal{L}}$ can be defined as an operator which involves the local quantum space $\mathds{V}_n$ and the auxiliary space $\mathds{V}_a$, as follows 

\begin{equation}
    \begin{array}{c}
         \bm{\mathcal{L}} : \mathds{V}_n \otimes \mathds{V}_a \rightarrow \mathds{V}_n \otimes \mathds{V}_a \\
         \\
         \bm{\mathcal{L}}_{n, a} = u(\mathds{1}_n \otimes \mathds{1}_a) + i \sum_{\alpha} \spin_n^\alpha \otimes \sigma_{a}^\alpha.
    \end{array}
    \label{lax operator}
\end{equation}


where $\mathds{1}_n$ and $\spin_n^\alpha$ act on $\mathds{V}_n $ while $\mathds{1}_a$ and $\spin_a^\alpha$ act upon $\mathds{V}_a$. The $u$-parameter is the spectral parameter, a complex number. \eqref{lax operator} can be rewritten as 


\begin{align}
    \bm{\mathcal{L}}_{n, a} &= u(\mathds{1}_n \otimes \mathds{1}_a) + i \sum_{\alpha} \spin_n^\alpha \otimes \sigma_{a}^\alpha \\
    &= u \left(\begin{array}{cc}
        \mathds{1}_n & 0  \\
        0 & \mathds{1}_a 
    \end{array}\right) + i \spin^{x}_{n} \otimes \left(\begin{array}{cc}
        0 & 1 \\
        1 & 0
    \end{array}\right)_a + i \spin^{y}_{n} \otimes \left(\begin{array}{cc}
        0 & -i \\
        i & 0 
    \end{array}\right)_a + i \spin^{z}_{n} \otimes \left(\begin{array}{cc}
        1 & 0  \\
        0 & -1
    \end{array}\right)_a \\
    &= \left(\begin{array}{cc}
       u \mathds{1}_n + i \spin^{z}_{n} & i \spin^{x}_{n} + \spin^{y}_{n} \\
       i \spin^{x}_{n} - \spin^{y}_{n}  & u \mathds{1}_n - i \spin^{z}_{n}
    \end{array}\right)_a = \left(\begin{array}{cc}
       u \mathds{1}_n + i \spin^{z}_{n} & i \spin^{-}_{n} \\
       i \spin^{+}_{n} & u \mathds{1}_n - i \spin^{z}_{n}
    \end{array}\right)_a,
\end{align}
    
which is a $2 \times 2$-matrix in the auxiliary space $\mathds{V}_a$ and the matrix entries are operators acting on the physical Hilbert space $\mathds{V}_n$. \\

Consider the permutation operator

\begin{equation}
    \bm{\mathcal P}_{n,a} = \frac{1}{2} \bigg(\mathds{1}_n \otimes \mathds{1}_a + \sum_{\alpha}\sigma_n^{\alpha}  \otimes \sigma_n^{\alpha} \bigg),
    \label{permutation op}
\end{equation}

which can then be written as a $2 \times 2$-matrix in the auxiliary space $\mathds{V}_a$, 
\begin{align}
    \bm{\mathcal P}_{n,a} &= \frac{1}{2} \bigg(\mathds{1}_n \otimes \mathds{1}_a + \sum_{\alpha}\sigma_n^{\alpha}  \otimes \sigma_n^{\alpha} \bigg)  \\
    &= \frac{1}{2} \left[ \left(\begin{array}{cc}
        \mathds{1}_n & 0  \\
        0 & \mathds{1}_a 
    \end{array}\right) + \sigma^{x}_{n} \otimes \left(\begin{array}{cc}
        0 & 1 \\
        1 & 0
    \end{array}\right)_a + \sigma^{y}_{n} \otimes \left(\begin{array}{cc}
        0 & -i \\
        i & 0 
    \end{array}\right)_a + \sigma^{z}_{n} \otimes \left(\begin{array}{cc}
        1 & 0  \\
        0 & -1
    \end{array}\right)_a \right] \\
    &= \left(\begin{array}{cc}
       \mathds{1}_n + \sigma^{z}_{n} &  \sigma^{x}_{n} - i\sigma^{y}_{n} \\
       \sigma^{x}_{n} + i\sigma^{y}_{n}  & \mathds{1}_n - \sigma^{z}_{n}
    \end{array}\right)_a. 
\end{align}

Using this definition for the permutation operator, the lax operator can then be rewritten as

\begin{align*}
       \bm{\mathcal{L}}_{n, a} &= u(\mathds{1}_n \otimes \mathds{1}_a) + i \sum_{\alpha} \spin_n^\alpha \otimes \sigma_{a}^\alpha. \\
       &= u(\mathds{1}_n \otimes \mathds{1}_a) + \frac{i}{2} \sum_{\alpha} \sigma_n^\alpha \otimes \sigma_{a}^\alpha \\
       &= u(\mathds{1}_n \otimes \mathds{1}_a) - \frac{i}{2} (\mathds{1}_n \otimes \mathds{1}_a) + \frac{i}{2} (\mathds{1}_n \otimes \mathds{1}_a) + \frac{i}{2} \sum_{\alpha} \sigma_n^\alpha \otimes \sigma_{a}^\alpha \\
       &= \bigg(u-\frac{i}{2}\bigg) (\mathds{1}_n \otimes \mathds{1}_a) + \frac{i}{2} \bigg( \mathds{1}_n \otimes \mathds{1}_a + \sum_{\alpha}\sigma_n^{\alpha}  \otimes \sigma_n^{\alpha} \bigg) \\
       &= \bigg(u-\frac{i}{2}\bigg) \mathds{1}_{n, a} + i \bm{\mathcal{P}}_{n, a}.
\end{align*}

The permutation operator indeed permutes the factors in the tensor product the factors in the tensor product by calculating the sum of tensor product of the Pauli matrices with respect to the standard four-dimensional basis, ie. if $\x = \left(\begin{array}{c}
     a \\
     b 
\end{array}\right)$ and ${\bf y} = \left(\begin{array}{c}
     c \\
     d 
\end{array}\right)$n then $\x \otimes {\bf y} = \left( \begin{array}{cc}
    ac & ad  \\
    bc & bd 
\end{array} \right)$ and  ${\bf y} \otimes \x = \left( \begin{array}{cc}
    ac & bc  \\
    ad & bd 
\end{array} \right)$. Therefore,

\begin{equation}
    \bm{\mathcal{P}}_{n, a} (\x \otimes {\bf y} ) = {\bf y} \otimes \x \textnormal{ and } \bm{\mathcal{P}}_{n, a} ( {\bf y} \otimes \x) = \x \otimes {\bf y}.
\end{equation}

\blanky \\

\paragraph{Fundamental commutation Relations and Monodromy Matrix}

Consider two distinct lax operators,

\begin{equation*}
    \begin{array}{c}
        \bl_{n, a_1}(u_1): \blanky \vds_n \otimes \vds_{a_1} \rightarrow \vds_n \otimes \vds_{a_1}  \\
        \\
        \bl_{n, a_2}(u_2): \blanky \vds_n \otimes \vds_{a_2} \rightarrow \vds_n \otimes \vds_{a_2} 
    \end{array},
\end{equation*}

both of which act on the physical local quantum state and on two different auxiliary spaces as well. The product of these two operator is then a triple tensor product in $\mathds{V}_{n} \otimes \mathds{V}_{a_1} \otimes \mathds{V}_{a_2}$. Consider an $\bm{\mathcal{R}}$-matrix which relates the permutation and spectral parameters of two auxiliary spaces, which is given by  

\begin{equation}
\begin{array}{c}
     \bm{\mathcal{R}}: \blanky \mathds{V}_{a_1} \otimes \mathds{V}_{a_2} \otimes \mathds{V}_{a_1} \otimes \mathds{V}_{a_2} \\ 
     \\
     \bm{\mathcal{R}}_{a_1, a_2}(u_1-u_2) = (u_1-u_2) \mathds{1}_{a_1, a_2} + i \bm{\mathcal{P}}_{a_1, a_2}, \\
\end{array}
\end{equation}

note that $\bm{\mathcal{R}}$ can be rewritten as a $\twobytwo$-matrix in either of the auxiliary spaces, as follows

\begin{equation}
    \bm{\mathcal{R}}_{a1, a2} = \left(\begin{array}{cc}
        \bigg(u + \frac{i}{2}\bigg) \mathds{1}_{a_2} + i \spin_{a_2}^z & i \spin_{a_2}^{-} \\
        i \spin_{a_2}^{+}  & \bigg(u + \frac{i}{2}\bigg) \mathds{1}_{a_2} - i \spin_{a_2}^z 
    \end{array} \right)_{a_1} = \left(\begin{array}{cc}
        \bigg(u + \frac{i}{2}\bigg) \mathds{1}_{a_1} + i \spin_{a_1}^z & i \spin_{a_1}^{-} \\
        i \spin_{a_1}^{+}  & \bigg(u + \frac{i}{2}\bigg) \mathds{1}_{a_1} - i \spin_{a_2}^z 
    \end{array} \right)_{a_2}.
\end{equation}

\blanky \\

An interesting relationship between the products of the two lax operators and the $\bm{\mathcal{R}}$-matrix can then be found\footnote{Note that the following permutation properties hold 

\begin{equation}
\bm{\mathcal{P}}_{n, a_1}\bm{\mathcal{P}}_{n, a_2} = \bm{\mathcal{P}}_{a_1, a_2}\bm{\mathcal{P}}_{n, a_1} = \bm{\mathcal{P}}_{n, a_2}\bm{\mathcal{P}}_{a_2, a_1} \textnormal{ and } \bm{\mathcal{P}}_{a, b} = \bm{\mathcal{P}}_{b, a}
\end{equation}

}, consider 

\begin{align*}
    \bm{\mathcal{R}}_{a_1, a_2}(u_1-u_2) & \bm{\mathcal{L}}_{n, a_1} (u_1) \bm{\mathcal{L}}_{n, a_2} (u_2) \\ 
    &= {\bigg((u_1-u_2) \mathds{1}_{a_1, a_2} 
    + i \bm{\mathcal{P}}_{a_1, a_2}\bigg)}   \left(\bigg(u-\frac{i}{2}\bigg) \mathds{1}_{n, a_1} + i \bm{\mathcal{P}}_{n, a_1}\right) \left(\bigg(u-\frac{i}{2}\bigg) \mathds{1}_{n, a_2} + i \bm{\mathcal{P}}_{n, a_2}\right) \\
    &= \left({(u_1-u_2) \mathds{1}_{a_1, a_2} \bigg(u-\frac{i}{2}\bigg) \mathds{1}_{n, a_1}} + {(u_1-u_2) \mathds{1}_{a_1, a_2} i \bm{\mathcal{P}}_{n, a_1}} + i \bm{\mathcal{P}}_{a_1, a_2} \bigg(u-\frac{i}{2}\bigg) \mathds{1}_{n, a_1} + i^2 \bm{\mathcal{P}}_{a_1, a_2}  \bm{\mathcal{P}}_{n, a_1} \right) \\
    & \times \left(\bigg(u-\frac{i}{2}\bigg) \mathds{1}_{n, a_2} + i \bm{\mathcal{P}}_{n, a_2}\right) \\
    &= {(u_1-u_2) \mathds{1}_{a_1, a_2} \bigg(u-\frac{i}{2}\bigg) \mathds{1}_{n, a_1}} \bigg(u-\frac{i}{2}\bigg) \mathds{1}_{n, a_2} + {(u_1-u_2) \mathds{1}_{a_1, a_2} \bigg(u-\frac{i}{2}\bigg) \mathds{1}_{n, a_1}} i \bm{\mathcal{P}}_{n, a_2} \\
    &+ {(u_1-u_2) \mathds{1}_{a_1, a_2} i \bm{\mathcal{P}}_{n, a_1}} \bigg(u-\frac{i}{2}\bigg) \mathds{1}_{n, a_2} + {(u_1-u_2) \mathds{1}_{a_1, a_2} i \bm{\mathcal{P}}_{n, a_1}} i \bm{\mathcal{P}}_{n, a_2} \\ 
    &+ i \bm{\mathcal{P}}_{a_1, a_2} \bigg(u-\frac{i}{2}\bigg)\mathds{1}_{n, a_1}  \bigg(u-\frac{i}{2}\bigg) \mathds{1}_{n, a_2} + i \bm{\mathcal{P}}_{a_1, a_2} \bigg(u-\frac{i}{2}\bigg) \mathds{1}_{n, a_1} i \bm{\mathcal{P}}_{n, a_2} \\
    & - \bm{\mathcal{P}}_{a_1, a_2}  \bm{\mathcal{P}}_{n, a_1}  \bigg(u-\frac{i}{2}\bigg) \mathds{1}_{n, a_2} - \bm{\mathcal{P}}_{a_1, a_2}  \bm{\mathcal{P}}_{n, a_1} i \bm{\mathcal{P}}_{n, a_2} \\
    & = {(u_1-u_2) \mathds{1}_{a_1, a_2}  \bigg(u-\frac{i}{2}\bigg) \mathds{1}_{n, a_2}} \times \bigg(u-\frac{i}{2}\bigg) \mathds{1}_{n, a_1} + {(u_1-u_2) \mathds{1}_{a_1, a_2} i \bm{\mathcal{P}}_{n, a_2}} \times \bigg(u-\frac{i}{2}\bigg) \mathds{1}_{n, a_1} \\
    &+ {(u_1-u_2) \mathds{1}_{a_1, a_2} \bigg(u-\frac{i}{2}\bigg) \mathds{1}_{n, a_2}} \times  i \bm{\mathcal{P}}_{n, a_1} + {(u_1-u_2) \mathds{1}_{a_1, a_2} i \bm{\mathcal{P}}_{n, a_2}} \times i \bm{\mathcal{P}}_{n, a_1} \\
    &+ i \bm{\mathcal{P}}_{a_1, a_2} \bigg(u-\frac{i}{2}\bigg) \mathds{1}_{n, a_1}  \bigg(u-\frac{i}{2}\bigg) \mathds{1}_{n, a_2} + i \bm{\mathcal{P}}_{a_1, a_2} \bigg(u-\frac{i}{2}\bigg) \mathds{1}_{n, a_1} i \bm{\mathcal{P}}_{n, a_2} \\
    & - \bm{\mathcal{P}}_{a_1, a_2}  \bm{\mathcal{P}}_{n, a_1}  \bigg(u-\frac{i}{2}\bigg) \mathds{1}_{n, a_2} - \bm{\mathcal{P}}_{a_1, a_2}  \bm{\mathcal{P}}_{n, a_1} i \bm{\mathcal{P}}_{n, a_2} \\
    &= \left({(u_1-u_2) \mathds{1}_{a_1, a_2}  \bigg(u-\frac{i}{2}\bigg) \mathds{1}_{n, a_2}} + {(u_1-u_2) \mathds{1}_{a_1, a_2} i \bm{\mathcal{P}}_{n, a_2}}\right) \bigg(u-\frac{i}{2}\bigg) \mathds{1}_{n, a_1} \\ 
    &+ \left({(u_1-u_2) \mathds{1}_{a_1, a_2} \bigg(u-\frac{i}{2}\bigg) \mathds{1}_{n, a_2}} + {(u_1-u_2) \mathds{1}_{a_1, a_2} i \bm{\mathcal{P}}_{n, a_2}} \right) \times  i \bm{\mathcal{P}}_{n, a_1} \\
    &+ i \bm{\mathcal{P}}_{a_1, a_2} \bigg(u-\frac{i}{2}\bigg) \mathds{1}_{n, a_1} \bigg(u-\frac{i}{2}\bigg) \mathds{1}_{n, a_2} + i \bm{\mathcal{P}}_{a_1, a_2} \bigg(u-\frac{i}{2}\bigg) \mathds{1}_{n, a_1} i \bm{\mathcal{P}}_{n, a_2} \\
    & - \bm{\mathcal{P}}_{a_1, a_2}  \bm{\mathcal{P}}_{n, a_1}  \bigg(u-\frac{i}{2}\bigg) \mathds{1}_{n, a_2} - \bm{\mathcal{P}}_{a_1, a_2}  \bm{\mathcal{P}}_{n, a_1} i \bm{\mathcal{P}}_{n, a_2} \\
    &= \bigg((u_1-u_2) \mathds{1}_{a_1, a_2} \bigg(u-\frac{i}{2}\bigg) \mathds{1}_{n, a_2} + {(u_1-u_2) \mathds{1}_{a_1, a_2} i \bm{\mathcal{P}}_{n, a_2}} \bigg) \left( \bigg(u-\frac{i}{2}\bigg) \mathds{1}_{n, a_2} + i \bm{\mathcal{P}}_{n, a_1} \right) \\
    &+ i \bm{\mathcal{P}}_{a_1, a_2} \bigg(u-\frac{i}{2}\bigg) \mathds{1}_{n, a_1} \bigg(u-\frac{i}{2}\bigg) \mathds{1}_{n, a_2} + i \bm{\mathcal{P}}_{a_1, a_2} \bigg(u-\frac{i}{2}\bigg) \mathds{1}_{n, a_1} i \bm{\mathcal{P}}_{n, a_2} \\
    & - \bm{\mathcal{P}}_{a_1, a_2}  \bm{\mathcal{P}}_{n, a_1}  \bigg(u-\frac{i}{2}\bigg) \mathds{1}_{n, a_2} - \bm{\mathcal{P}}_{a_1, a_2}  \bm{\mathcal{P}}_{n, a_1} i \bm{\mathcal{P}}_{n, a_2} \\
    &= \bigg( \bigg(u-\frac{i}{2}\bigg) \mathds{1}_{n, a_2}  + { i \bm{\mathcal{P}}_{n, a_2}} \bigg) \left( \bigg(u-\frac{i}{2}\bigg) \mathds{1}_{n, a_2} + i \bm{\mathcal{P}}_{n, a_1} \right) (u_1-u_2) \mathds{1}_{a_1, a_2}  \\
    &+ \bigg( \bigg(u-\frac{i}{2}\bigg) \mathds{1}_{n, a_2}  + { i \bm{\mathcal{P}}_{n, a_2}} \bigg) \left( \bigg(u-\frac{i}{2}\bigg) \mathds{1}_{n, a_2} + i \bm{\mathcal{P}}_{n, a_1} \right)  i \bm{\mathcal{P}}_{n, a_2}  \\
    &= \bigg( \bigg(u-\frac{i}{2}\bigg) \mathds{1}_{n, a_2}  + { i \bm{\mathcal{P}}_{n, a_2}} \bigg) \left( \bigg(u-\frac{i}{2}\bigg) \mathds{1}_{n, a_2} + i \bm{\mathcal{P}}_{n, a_1} \right) \bigg(\bigg(u-\frac{i}{2}\bigg) \mathds{1}_{n, a_2} + i \bm{\mathcal{P}}_{n, a_2} \bigg), 
\end{align*}

which proves that 

\begin{align}
 & & \alignedbox{\br_{a_1, a_2}(u_1-u_2) \bm{\mathcal{L}}_{n, a_1} (u_1) \bm{\mathcal{L}}_{n, a_2} (u_2)}{= \bm{\mathcal{L}}_{n, a_1} (u_2) \bm{\mathcal{L}}_{n, a_1} (u_1) \bm{\mathcal{R}}_{a_1, a_2}(u_1-u_2)}.   
\label{fundamental commutation relation}
\end{align}

From the fundamental commutation relation it can then be shown that the $\bm{\mathcal{R}}$-matrix follows the quantum Yang-Baxter equation. Consider then the product of the following $\bl$-operators,

\begin{equation} 
    \begin{split}
    \bl_{n, 1} \bl_{n, 2} \bl_{n, 3} &= \br_{12}^{-1} \bl_{n, 2} \bl_{n, 1} \bl_{n, 3} \br_{12} \\
    &= \br_{12}^{-1} \br_{13}^{-1} \bl_{n, 2} \bl_{n, 3} \bl_{n, 1} \br_{13}\br_{12} \\
    &= \br_{12}^{-1} \br_{13}^{-1} \br_{23}^{-1} \bl_{n, 3} \bl_{n, 2} \bl_{n, 1} \br_{23}\br_{13} \br_{12} \\
    &= (\br_{23}\br_{13}\br_{12})^{-1} \bl_{n, 3} \bl_{n, 2} \bl_{n, 1}\br_{23}\br_{13}\br_{12},
\end{split}
\end{equation}

and similarly 

\begin{equation}  
    \begin{split}
    \bl_{n, 1} \bl_{n, 2} \bl_{n, 3} &= \br_{23}^{-1} \bl_{n, 1} \bl_{n, 3} \bl_{n, 2} \br_{23} \\
    &= \br_{23}^{-1} \br_{13}^{-1} \bl_{n, 3} \bl_{n, 1} \bl_{n, 2} \br_{13}\br_{23} \\
    &= \br_{23}^{-1} \br_{13}^{-1} \br_{12}^{-1} \bl_{n, 3} \bl_{n, 2} \bl_{n, 1} \br_{12} \br_{13}\br_{23} \\
    &= (\br_{12}\br_{13}\br_{23})^{-1} \bl_{n, 3} \bl_{n, 2} \bl_{n, 1}\br_{12}\br_{13}\br_{23},
\end{split}
\end{equation}

both of which then yield an important constraint on the $\br$-matrix, namely 

\begin{align}
    \alignedbox{\br_{12}\br_{13}\br_{23}}{= \br_{23}\br_{13}\br_{12}},
\end{align}

which is the quantum Yang-Baxter equation. \\

Let the monodromy matrix $\bm{\mathcal{T}}_{N,a}$ be defined as a product of subsequent of $\bl$-operators, ie.

\begin{equation}
    \begin{array}{c}
         \bmt_{N,a} = \prod^{j=N}_{1} \bl_{j,a} \\
         \\
         \bmt_{N,a} = \prod^{j=N}_{1} \left(\begin{array}{cc}
            u \mathds{1}_j + i \spin_j^z & i \spin_j^-  \\
            i \spin_j^+ & u \mathds{1}_j - i \spin_j^z
         \end{array}\right)) \equiv \left(\begin{array}{cc}
            A(u) & B(u) \\
            C(u) & D(u)
         \end{array}\right) 
    \end{array},
\end{equation}

which obeys the fundamental relation as well\footnote{ For example, consider $N=2$, then

\begin{equation}
    \begin{split}
        {\br_{a_1, a_2}(u_1-u_2) \bm{\mathcal{T}}_{N, a_1} (u_1) \bm{\mathcal{T}}_{N, a_2} (u_2)} &= {\br_{a_1, a_2}(u_1-u_2) \bm{\mathcal{T}}_{2, a_1} (u_1) \bm{\mathcal{T}}_{2, a_2} (u_2)} \\
        &= {\br_{a_1, a_2}(u_1-u_2) \bl_{2, a_1} (u_1) \bl_{2, a_1} (u_1) \bl_{2, a_2} (u_2) \bl_{1, a_2} (u_2)} \\
        &= \bigg(\br_{a_1, a_2}(u_1-u_2) \bl_{2, a_1} (u_1) \bl_{2, a_2} (u_2)\bigg) \bl_{1, a_1} (u_1) \bl_{1, a_2} (u_2) \\
        &=  \bl_{2, a_2} (u_2) \bl_{2, a_1} (u_1) \bigg(\br_{a_1, a_2}(u_1-u_2)\bl_{1, a_1} (u_1) \bl_{1, a_2} (u_2) \bigg) \\
        &= \bl_{2, a_2} (u_2) \bl_{2, a_1} (u_1) \bl_{1, a_2} (u_2) \bl_{1, a_1} (u_1) \br_{a_1, a_2}(u_1-u_2) \\
        &=  \bl_{2, a_2} (u_2) \bl_{1, a_2} (u_2) \bl_{2, a_1} (u_1) \bl_{1, a_1} (u_1) \br_{a_1, a_2}(u_1-u_2) \\
        &= \bm{\mathcal{T}}_{n, a_1} (u_2) \bm{\mathcal{T}}_{n, a_1} (u_1) \bm{\mathcal{R}}_{a_1, a_2}(u_1-u_2),
    \end{split}
\end{equation}

where the fundamental commutation relation given by \eqref{fundamental commutation relation} was used. Note as well that $\bl$-operators acting on different spaces commute as well, eg. $[\bl_{n_1, a_1}, \bl_{n_2, a_2}] \propto \delta_{n_1, n_2}$. The general fundamental relation can then be proved via mathematical induction for the general case.}, ie.

\begin{align}
 & & {\br_{a_1, a_2}(u_1-u_2) \bm{\mathcal{T}}_{N, a_1} (u_1) \bm{\mathcal{T}}_{N, a_2} (u_2)}{= \bm{\mathcal{T}}_{N, a_2} (u_2) \bm{\mathcal{T}}_{N, a_1} (u_1) \bm{\mathcal{R}}_{a_1, a_2}(u_1-u_2)}.   
\label{fundamental commutation relation on R}
\end{align}

 \blanky \\

Now, let the transfer $\mathfrak{t}$-matrix be defined as the $\bm{\mathcal{T}}$-matrix's partial trace over the auxiliary space $\vds_{a}$, ie.

\begin{equation}
    \mathfrak{t}(u) = \Tr_{\vds_{a}} \bm{\mathcal{T}}_{N, a} (u) = \Tr_{\vds_{a}} \left(\begin{array}{cc}
            A(u) & B(u) \\
            C(u) & D(u)
         \end{array}\right)  = A(u) + D(u).
\end{equation}
 
\eqref{fundamental commutation relation on R}'s double trace can then be written i nterms of the transfer matrix, as follows 

\begin{align*}
    \Tr_{\vds_{a_1}} \Tr_{\vds_{a_2}} \bigg[\br_{a_1, a_2}(u_1-u_2) \bm{\mathcal{T}}_{N, a_1} (u_1) \bm{\mathcal{T}}_{N, a_2} (u_2)\bigg] = \Tr_{\vds_{a_1}} \Tr_{\vds_{a_2}} \bigg[{\bm{\mathcal{T}}_{N, a_1} (u_2) \bm{\mathcal{T}}_{N, a_1} (u_1) \bm{\mathcal{R}}_{a_1, a_2}(u_1-u_2)}\bigg] \\
    \Rightarrow \Tr_{\vds_{a_1}} \Tr_{\vds_{a_2}} \bigg[\br_{a_1, a_2}(u_1-u_2) \bm{\mathcal{T}}_{N, a_1} (u_1) \bm{\mathcal{T}}_{N, a_2} (u_2) - {\bm{\mathcal{T}}_{N, a_1} (u_2) \bm{\mathcal{T}}_{N, a_1} (u_1) \bm{\mathcal{R}}_{a_1, a_2}(u_1-u_2)} \bigg] = 0 \\ 
    \Rightarrow \Tr_{\vds_{a_1}} \Tr_{\vds_{a_2}} \bigg[\br_{a_1, a_2}(u_1-u_2) \bigg(\bm{\mathcal{T}}_{N, a_1} (u_1) \bm{\mathcal{T}}_{N, a_2} (u_2) - {\bm{\mathcal{T}}_{N, a_2} (u_2) \bm{\mathcal{T}}_{N, a_1} (u_1)\bigg)} \bigg] = 0 \\ 
    \textnormal{ which implies }
    \Rightarrow \mathfrak{t}(u_1)\mathfrak{t}(u_2) = \mathfrak{t}(u_2)\mathfrak{t}(u_1).
\end{align*}

Therefore the transfer matrix commutes with itself for different values of the spectral parameter. It turns out that the monodromy matrix can be expanded in a power series around any point $z_0 \in \mathds{C}$, generating an infinite set of linearly independent commuting operators acting on the full quantum space. This formally leads to the model's integrability. \\

\paragraph{\textbf{Monodromy matrix and the Hamiltonian}}

The Hamiltonian operator belongs to the family of transfer matrices. Let $z_0 = \frac{i}{2}$, then the monodromy matrix can be expanded around said complex point yielding (up to first order)

\begin{equation} 
    \begin{split}
         \bmt_{N,a}(z_0) &= \prod_1^{j=N} \bigg[ \cancel{\bigg(z_0-\frac{i}{2}\bigg) \mathds{1}_{j, a}} + i \bm{\mathcal{P}}_{j, a}\bigg] = i^N \bm{\mathcal{P}}_{N, a} \bm{\mathcal{P}}_{N-1, a} \cdots \bm{\mathcal{P}}_{1, a} = \\
         &= i^N \bm{\mathcal{P}}_{1, 2} \bm{\mathcal{P}}_{2, 3} \cdots \bm{\mathcal{P}}_{N-1, N} \bm{\mathcal{P}}_{N, a}\\
         &= i^N \prod_{j=1}^{N-1} \bm{\mathcal{P}}_{j, \blanky j+1} \bm{\mathcal{P}}_{N, a}
    \end{split}
\end{equation}

where the second line holds given the properties of the permutation operators. Taking the partial trace over the auxiliary space yields the transfer matrix, as follows 

\begin{equation}
    \begin{split}
        \mathfrak{t}(z_0) &= \Tr_{\vds_a}  \bmt_{N,a}(z_0) = \Tr_{\vds_a} i^N \bm{\mathcal{P}}_{1, 2} i^N \prod_{j=1}^{N-1} \bm{\mathcal{P}}_{j, \blanky j+1} \bm{\mathcal{P}}_{N, a} \\
        &= i^N \prod_{j=1}^{N-1} \bm{\mathcal{P}}_{j, \blanky j+1}
        \Tr_{\vds_a} \bm{\mathcal{P}}_{N, a}    = i^N \prod_{j=1}^{N-1} \bm{\mathcal{P}}_{j, \blanky j+1} \Tr_{\vds_a} \frac{1}{2}\left(\begin{array}{cc}
       \mathds{1}_n + \sigma^{z}_{n} &  \sigma^{x}_{n} - i\sigma^{y}_{n} \\
       \sigma^{x}_{n} + i\sigma^{y}_{n}  & \mathds{1}_n - \sigma^{z}_{n} \\
        \end{array}\right)_a. \\
        &= i^N \prod_{j=1}^{N-1} \bm{\mathcal{P}}_{j, \blanky j+1} \mathds{1}_N.
    \end{split}
\end{equation}

Then $\mathcal{U} = i^{-N} \mathfrak{t}(z_0) = \prod_{j=1}^{N-1} \bm{\mathcal{P}}_{j, \blanky j+1}$ is a shift operator in the full quantum Hilbert space $\mathds{H}$. In effect, note that 
$$
 \bm{\mathcal{P}}_{n_1, \blanky n_2} {\bf X}_{n_2} \bm{\mathcal{P}}_{n_1, \blanky n_2} = {\bf X}_{n_1},
$$

ie. this permutation moves ${\bf X}$ one step back. Furthermore, this operator is unitary (given that the permutations are, by definition, unitary operators). Then, according to Stone's theorem on one-parameter unitary groups, the shift operator is related to the momentum operator 

\begin{equation}
    \mathcal{U} = e^{iP}
\end{equation}

\blanky \\

The next order in the expansion of the transfer matrix can then be found via the derivative of the monodromy matrix at $z_0 = \frac{i}{2}$, 

\begin{equation}
    \begin{split}
    \frac{d \bmt_{N,a}(u)}{du}\bigg|_{u = z_0} &= \frac{d}{du} \prod_{1}^{j=N} \left(\begin{array}{cc}
            u \mathds{1}_j + i \spin_j^z & i \spin_j^-  \\
            i \spin_j^+ & u \mathds{1}_j - i \spin_j^z
         \end{array}\right)\bigg|_{u = z_0} \\
        &=  \frac{d}{du} \left[\left(\begin{array}{cc}
            u \mathds{1}_N + i \spin_N^z & i \spin_N^-  \\
            i \spin_N^+ & u \mathds{1}_N - i \spin_N^z
         \end{array}\right)\right]\bigg|_{u = z_0} \prod_{1}^{j=N-1} \left(\begin{array}{cc}
            u \mathds{1}_j + i \spin_j^z & i \spin_j^-  \\
            i \spin_j^+ & u \mathds{1}_j - i \spin_j^z
         \end{array}\right) \\
         &+ \left(\begin{array}{cc}
            u \mathds{1}_N + i \spin_N^z & i \spin_N^-  \\
            i \spin_N^+ & u \mathds{1}_N - i \spin_N^z
         \end{array}\right) \frac{d}{du} \left[\left(\begin{array}{cc}
            u \mathds{1}_{N-1} + i \spin_{N-1}^z & i \spin_{N-1}^-  \\
            i \spin_{N-1}^+ & u \mathds{1}_{N-1} - i \spin_{N-1}^z
         \end{array}\right)\right]\bigg|_{u = z_0} \\
         &\times \prod_{1}^{j=N-2} \left(\begin{array}{cc}
            u \mathds{1}_j + i \spin_j^z & i \spin_j^-  \\
            i \spin_j^+ & u \mathds{1}_j - i \spin_j^z
         \end{array}\right) \\
         &+ \cdots +  \prod^{j=N}_{3}\left(\begin{array}{cc}
            u \mathds{1}_j + i \spin_j^z & i \spin_j^-  \\
            i \spin_j^+ & u \mathds{1}_j - i \spin_j^z
         \end{array}\right) \frac{d}{du} \left[\left(\begin{array}{cc}
            u \mathds{1}_{2} + i \spin_{2}^z & i \spin_{2}^-  \\
            i \spin_{2}^+ & u \mathds{1}_{2} - i \spin_{2}^z
         \end{array}\right)\right]\bigg|_{u = z_0} \\
         & \times \left(\begin{array}{cc}
            u \mathds{1}_{1} + i \spin_{1}^z & i \spin_{1}^-  \\
            i \spin_{1}^+ & u \mathds{1}_{1} - i \spin_{1}^z
         \end{array}\right)
         \\
         &+ \prod^{j=N}_{2}\left(\begin{array}{cc}
            u \mathds{1}_j + i \spin_j^z & i \spin_j^-  \\
            i \spin_j^+ & u \mathds{1}_j - i \spin_j^z
         \end{array}\right) \frac{d}{du} \left[\left(\begin{array}{cc}
            u \mathds{1}_{1} + i \spin_{1}^z & i \spin_{1}^-  \\
            i \spin_{1}^+ & u \mathds{1}_{1} - i \spin_{1}^z
         \end{array}\right)\right]\bigg|_{u = z_0} \\
         &= i^{N-1} \bigg[ \prod_{1}^{j=N-1} \bm{\mathcal{P}}_{j} + \prod^{j=N}_{N} \bm{\mathcal{P}}_{j}\times  \prod_{1}^{j=N-2} \bm{\mathcal{P}}_{j} + \cdots + \prod^{j=N}_{k} \bm{\mathcal{P}}_{j}\times  \prod^{k+2}_{1} \bm{\mathcal{P}}_{j} \\
         &+ \cdots + \prod^{j=N}_{3} \bm{\mathcal{P}}_{j} \times \prod^{j=1}_{1} \bm{\mathcal{P}}_{j} + \prod^{j=N}_{2} \bm{\mathcal{P}}_{j} \bigg] \\
         &= i^{N-1} \sum_{n \in \mathds{N}} \bm{\mathcal{P}}_{N, a} \cdots \bm{\mathcal{P}}_{n-1, a} \bm{\mathcal{P}}_{n+1, a} \cdots \bm{\mathcal{P}}_{1, a} \\
         &= i^{N-1} \sum_{n \in \mathds{N}} \bm{\mathcal{P}}_{1, 2} \bm{\mathcal{P}}_{2, 3}\cdots \bm{\mathcal{P}}_{n-1, n+1}
         \cdots \bm{\mathcal{P}}_{N-1, N} \bm{\mathcal{P}}_{N, a} \\
         &= i^{N-1} \sum_{n \in \mathds{N}} \prod_{j=1}^{n-2} \bm{\mathcal{P}}_{j, \blanky j+1}  \times \bm{\mathcal{P}}_{n-1, \blanky n+1} \times \prod_{j=n+2}^{N-1}  \bm{\mathcal{P}}_{j, \blanky j+1} \bm{\mathcal{P}}_{N, a}.
    \end{split} 
\end{equation}

Therefore,

\begin{equation}
    \begin{split}
        \frac{d \mathfrak{t}(u)}{du}\bigg|_{u = z_0} &= \frac{d \Tr_{\vds_{a}} \bmt_{N,a}(u)}{du}\bigg|_{u = z_0} \\
        &= i^{N-1} \sum_{n \in \mathds{N}} \prod_{j=1}^{n-2} \bm{\mathcal{P}}_{j, \blanky j+1} 
        \times \bm{\mathcal{P}}_{n-1, \blanky n+1} \times \prod_{j=n+2}^{N-1}  \bm{\mathcal{P}}_{j, \blanky j+1} \times \Tr_{\vds_{a}} \bm{\mathcal{P}}_{N, a} \\
        &= i^{N-1} \sum_{n \in \mathds{N}} \prod_{j=1}^{n-2} \bm{\mathcal{P}}_{j, \blanky j+1} 
        \times \bm{\mathcal{P}}_{n-1, \blanky n+1} \times \prod_{j=n+2}^{N-1}  \bm{\mathcal{P}}_{j, \blanky j+1}.
    \end{split}
\end{equation}

Now, note that 

\begin{equation}
    \begin{split}
        \frac{d}{du} \log(\mathfrak{t}(u))\bigg|_{u = z_0} &= \frac{d}{du} \log \Tr_{\vds_{a}} \bm{\mathcal{T}}(u) \bigg|_{u=z_0} \\
        &= \frac{d}{du}\bigg(\mathfrak{t}(u)\bigg) \mathfrak{t}(u)^{-1}\bigg|_{u = z_0} \\
        &= i^{N-1} \sum_{n \in \mathds{N}} \bigg( \prod_{j=1}^{n-2} \bm{\mathcal{P}}_{j, \blanky j+1} 
        \times \bm{\mathcal{P}}_{n-1, \blanky n+1} \times \prod_{j=n+2}^{N-1}  \bm{\mathcal{P}}_{j, \blanky j+1} \bigg) \times \bigg(i^N \prod_{j=1}^{N-1} \bm{\mathcal{P}}_{j, \blanky j+1} \mathds{1}_N\bigg)^{-1} \\
        &= \frac{1}{i} \sum_{n \in \mathds{N}} \bigg(\prod_{j=1}^{n-2} \bm{\mathcal{P}}_{j, \blanky j+1} 
        \times \bm{\mathcal{P}}_{n-1, \blanky n+1} \times \prod_{j=n+2}^{N-1}  \bm{\mathcal{P}}_{j, \blanky j+1} \bigg) \times \prod_{j=1}^{N-1} \bm{\mathcal{P}}_{N-1-j, N-j}^{-1} \\
        &= \frac{1}{i} \sum_{n \in \mathds{N}} \bigg(\prod_{j=1}^{n-2} \bm{\mathcal{P}}_{j, \blanky j+1} 
        \times \bm{\mathcal{P}}_{n-1, \blanky n+1} \times \prod_{j=n+2}^{N-1}  \bm{\mathcal{P}}_{j, \blanky j+1} \bigg) \times \prod_{1}^{j = N-1} \bm{\mathcal{P}}_{j, j+1}  \\
        &= \frac{1}{i} \sum_{n \in \mathds{N}} \bigg(\cancel{\bm{\mathcal{P}}_{1,2}} \cancel{\bm{\mathcal{P}}_{2,3}} \cdots \bm{\mathcal{P}}_{n-1, \blanky n+1} \cdots \cancel{\bm{\mathcal{P}}_{N-1,N} \times \bm{\mathcal{P}}_{N-1, \blanky N}} \cdots \bm{\mathcal{P}}_{n,\blanky n+1} \cdots \cancel{\bm{\mathcal{P}}_{2,3}} \cancel{\bm{\mathcal{P}}_{1,2}}\bigg) \\
        &= \frac{1}{i}  \sum_{n \in \mathds{N}} \bm{\mathcal{P}}_{N-1, \blanky N}
    \end{split}
\end{equation}

Note that \eqref{XXX model} can be rewritten in terms of the permutation operator as 

\begin{equation}
    {\bf H} = \frac{J}{2} \sum_{n}\bigg( \bm{\mathcal{P}}_{n, \blanky n+1} - \mathds{1}^{\otimes N}\bigg),
\end{equation}

which in turn can be rewritten as 

\begin{equation}
    {\bf H} = \frac{J}{2} \bigg(i \frac{d}{du} \log(\mathfrak{t}(u))\bigg|_{u=z_0} - N \mathds{1}^{\otimes N}\bigg),
\end{equation}

which shows that the XXX Hamiltonian belongs to the family of $N-1$ commuting operators generated by the trace of the monodromy matrix $\bmt$. As a result, the Hamiltonian commutes with the transfer matrix 

\begin{equation}
    [{\bf H}, \mathfrak{t}(u)] = 0.
\end{equation}

\blanky \\

\paragraph{\textbf{Diagonalizing the Hamiltonian}}

The only task left is to diagonalize the Hamiltonian, by diagonalizing the transfer matrix. The monodromy matrix $\bm{\mathcal{t}}_{N, \blanky a}$ is a $\twobytwo$-matrix in the auxiliary space $\vds_{a}$ whose entries are operators in the full quantum space. 

\clearpage 

\section{.}
 
\end{document}
