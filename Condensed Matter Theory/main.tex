\documentclass{homework}
\author{Tomás Pérez}
\class{Condensed Matter Theory - Lecture Notes}
\date{\today}
\title{Theory \& Notes}

\graphicspath{{./media/}}

\begin{document} \maketitle

\section{Spin Chain models}

\section{Spin Chain models}

\subsection{Classical Spin Systems}

Classical spin systems are idealized versions of magnets. Although many magnetic phenomena in materials are inherently quantum mechanical, many properties are well described at least qualitatively by classical spin systems. \\

Each degrees of freedom of a spin system models a magnetic moment at a fixed location. A classical spin is simply the angular momentum vector of a spinning particle, and so can be represented by an arrow, which is assumed to have fixed length. For a spin system to be well-defined, the following requirements are needed 

\begin{itemize}
    \item constraints on where the arrow is allowed to point. For example, in the XY-model, the arrow is constrained to point in a two-diensional plane, and in the Ising model, the arrow is allowed to point in only directions. More complicated examples are the sigma-models, where instead of an arrow, the degrees of freedom take values on some manifold. In this language, the spins in the XY model take values on a circle. \\
    \item Where the spins physically are. For example, they may be at each point of some lattice in some dimension or they could be continuously distributed (ie. an arrow for every point in space eg. a field theory). \\
    \item How the spins themselves interact with one another ie. the energies associated with the possible spin configurations. Two major types of interaction are the ferromagnetic, where the energy is lowered when two spin are aligned, and antiferromagnetic, where the energy is lowered when they point in opposite directions. A given model can include interactions of both types. \\
\end{itemize}

Many many-body physical systems can effectively be treated as being on a lattice. For example, many of these systems are often well treated by utilizing a tight-binding model, where each electron is treated as being located at a fixed nucleus and so live at particular points in space. In some situations, the physics itself arises from the interplay between the degrees of freedom and the particular lattice they live on eg. geometrical frustration in the two-dimensional antiferromagnetic Ising model on a triangular lattice. Calling the two allowed directions in the Ising model as "up" and "down", antiferromagnetic interactions make adjacent spins different. On the square lattice, it is possible to make a low-energy state by alternating up and down spins, but on a triangular lattice this is not the case. Around a triangle, there must be at least two mutually up or down spins adjacent to each other. Such a bond in said to be unsatisfied and so the spins are frustrated, yielding completely new properties different fro the unfrustrated model. \\

%\subsubsection{The partition function and correlators}

%\paragraph{\textbf{The partition function and correlators}}

\subsubsection{Formalization of Statistical Mechanics}

A formal definition for a classical general spin chain, as a dynamical system, requires the use of measure theory terminology. \\

\paragraph{\textbf{Some technical aspects}} 
\underline{\textbf{An introduction to Measure Theory}}

A \textbf{measure} is a mathematical device which reflects the notion of quantity for a given set. Let ${\bf X}$ be a set, then each subset ${\bf U} \in {\bf X}$ is assigned a positive real number $\mu[{\bf U}]$. Thus, the measure is a function

$$
\mu: \textnormal{Dom}(\mu) \subset 2^{\bf X} \rightarrow \mathds{R},
$$

where $2^{\bf X} = \{{\bf S} \in {\bf X}\}$ is the power set of ${\bf X}$. However, it's usually impossible to define a satisfactory notion of quantity for all subsets of ${\bf X}$. Therefore, it is useful to consider 
$\textnormal{Dom}(\mu) \in {\bf X}$ ie. only some subsets of ${\bf X}$ will be measurable. Before defining a proper measure, it's domain must be specified first. This domain will be a collection of subsets of the space ${\bf X}$, called a $\sigma$-\textbf{algebra}. \\

Let ${\bf X}$ be a nonempty set, which in the following sections will be a sample space. A $\sigma$-\textbf{algebra} over ${\bf X}$ is a collection $\mathcal{F} = \{{\bf U}_i\}_{i \in \mathds{N}}$ of subsets of ${\bf X}$ with the following properties:

\begin{itemize}
    \item $\mathcal{F}$ contains the set ${\bf X}$: ${\bf X} \subset \mathcal{F}$. \\
    \item { $\mathcal{F}$ is closed under \underline{complementation}: if ${\bf U} \subset \mathcal{F}$, then } $ 
    {\bf U}^c = ({\bf X} / {\bf U}) \subset \mathcal{F}.
    $\\
    \item $\mathcal{F}$ is closed under \underline{countable unions} ie. 
    $\{{\bf U}_i\}_{i \in \mathds{N}} \textnormal{ } | \textnormal{ } {\bf U}_i \subset \mathcal{F} \Rightarrow \underset{i \in \mathds{N}}{\cup} {\bf U}_i \subset \mathcal{F}.
    $\\
    \item $\mathcal{F}$ is closed under \underline{countable intersections} ie. $
    \textnormal{}
    \{{\bf U}_i\}_{i\in \mathds{N}} \textnormal{ } | \textnormal{ } {\bf U}_i \subset \mathcal{F} \Rightarrow \underset{i\in \mathds{N}}{\cap} {\bf U}_i \subset \mathcal{F}. 
    $\\
\end{itemize}

Technically, the fourth property, is a corollary and follows directly from the second and third properties and due to De Morgan's laws. \\

A \textbf{measurable space} is an ordered pair $({\bf X}, \mathcal{F})$, where ${\bf X}$ is a set and where $\mathcal{F}$ is a $\sigma$-algebra on ${\bf X}$. Some of the most common examples of $\sigma$-algebras are

\begin{itemize}
    \item Two examples of trivial $\sigma$-algebras are:
    \begin{itemize}
        \item for any set ${\bf X}$ the collection $\{\emptyset, {\bf X}\}$ is a $\sigma$-algebra. This first example is far too small to be of any use.
        \item The power set $2^{\bf X}$ is also a $\sigma$-algebra. Note that for large sets, the power set becomes too large to be manageable.\\
    \end{itemize}
    
    \item A more manageable $\sigma$-algebra is the one induced by the (co-)countable sets. Let $\mathcal{M}$ be the most conservative collection of "manageable" sets, this is 
    
    $$
    \mathcal{M} = \bigg\{\{x\}: x \in {\bf X} \bigg\},
    $$
    
    ie. the set of all the singleton subsets of ${\bf X}$. Then $\mathcal{C} = \sigma(\mathcal{M}))$ is the $\sigma$-algebra of \textbf{countable and co-countable sets} 
    
    $$
    \mathcal{C} = \{{\bf C} \subset {\bf X} | \textnormal{ either } {\bf C} \textnormal{ is countable, or } {\bf X}/{\bf C} \textnormal{ is countable}\}.
    $$
    
    If ${\bf X}$ is itself finite or countable, then $\mathcal{C} = \mathcal{P}({\bf X})$. \\
    
    \item Another example are the partition algebras. Let ${\bf X}$ be a set. Then a \textbf{partition} of ${\bf X}$ is a collection $\mathcal{P} = \{{\bf P}_i\}_{i=1}^{N}$ of disjoint subsets, such that ${\bf X}= \sqcup_{n=1}^{N} {\bf P}_N$. These subsets ${\bf P}_i$ are called the \textbf{atoms} of the partition. Then, the $\sigma$-algebra generated by $\mathcal{P}$ is the collection of all possible unions of $\mathcal{P}$-atoms:
    
    $$
    \sigma(\mathcal{P}) = \{\sqcup_{j=1}^{k}{\bf P}_{n_j}\textnormal{ }|\textnormal{ } \{n_j\}_{j=1}^{k} \in \mathds{N}_{[1, \cdots, N]}\}.
    $$
    
    Therefore if $\textnormal{card}[{\bf P}] = N$, then $\textnormal{card}[\sigma({\bf P})] = 2^N$. \\
    
    If $\mathcal{Q}$ is another partition, we say $\mathcal{Q}$ redefines $\mathcal{P}$ ($\mathcal{Q}\prec \mathcal{P}$) if, for every ${\bf P}\in \mathcal{Q}$, there are $\{{\bf Q}_i\}_{i=1}^{N} \in \mathcal{Q}$ so that ${\bf P} = \sqcup_{j=1}^{N} {\bf Q}_{j}$. In said case we have 
    
    $$
    \mathcal{P}\prec \mathcal{Q} \Leftrightarrow \sigma(\mathcal{P}) \subset \sigma(\mathcal{Q}).
    $$ 
    
    \item \textbf{Borel } $\sigma$\textbf{-algebra of } $\mathds{R}$: \\
    
    Let ${\bf X} =\mathds{R}$ be the real numbers and let $\mathcal{M}$ be the set of all open intervals in $\mathds{R}$:
    
    $$
    \mathcal{M} = \{(a,b): -\infty \leq a < b \leq \infty\}, 
    $$
    
    then the $\sigma$-algebra $\mathcal{B} = \sigma(\mathcal{M})$ contains all open subsets of $\mathds{R}$, all closed subset, all countable intersections of open subsets, countable unions of closed subsets, etc. For example, $\mathcal{B}$ contains, as elements, the set $\mathds{Z}$ of integers, the set $\mathds{Q}$ of rationals and the set $\mathds{I}$ of irrationals. Then, $\mathcal{B}$ is called the \textbf{Borel} $\sigma$\textbf{-algebra} of $\mathds{R}$. \\
\end{itemize}

In general, let ${\bf X}$ be a topological space and let $\mathcal{M}$ be the set of all open subsets of ${\bf X}$. The $\sigma$-algebra $\sigma(\mathcal{M)}$ is the \textbf{Borel} $\sigma$\textbf{-algebra} of {\bf X} and is denoted by $\mathcal{B}({\bf X})$. It contains all open sets and closed subsets of ${{\bf X}}$, all countable intersections of open sets (called $G\delta$ sets), all countable unions of closed sets (called $F\sigma$ sets). For example, if ${\bf X}$ is Hausdorff, then $\mathcal{B}({\bf X})$ contains all countable and co-countable sets.\\

Let $({\bf X}, \mathcal{F})$ be a measurable space. A \textbf{measure} on $\mathcal{F}$ is a map $\mu: \mathcal{F} \rightarrow \mathds{R}_{+}$, which is \textbf{countably additive} ie. 

$$
\textnormal{If } \{{\bf Y}_i\}_{i=1} \textnormal{ }| \textnormal{ } {\bf Y}_i \in \mathcal{F} \rightarrow \mu\bigg[\sqcup_{n=1}^{\infty}{\bf Y}_n\bigg] = \sum_{n=1}^{\infty} \mu[{\bf Y}_n].
$$

Then a \textbf{measure space} is an ordered triple $({\bf X}, \mathcal{F}, \mu)$, where ${\bf X}$ is a set, $\mathcal{F}$ is a $\sigma$-algebra and $\mu$ is a measure on $\mathcal{F}$. Thus, $\mu$ assigns a size to the $\mathcal{F}$-measurable subsets of ${\bf X}$. Some important measures and measureable spaces are

\begin{itemize}
    \item \textbf{The counting measure} assigns, to any set, the cardinality of that set. 
    
    $$
    \mu[{\bf S}] =\textnormal{card}[{\bf S}].
    $$
    
    This measure provides no means of distinguishing between sets of the same cardinality, only being useful in finite measure spaces. \\
    
    \item A \textbf{finite measure space} is made up by a finite set ${\bf X}$, and a $\sigma$-algebra $\mathcal{F}= {\bf X}$. Then a measure $\mu$ on ${\bf X}$ is entirely defined by some function $f: {\bf X} \rightarrow \mathds{R}_{+}$. For any subset $\{x_i\}_{i=1}^{N}$ we then define 
    
    $$
    \mu\bigg(\{x_i\}_{i=1}^{N}\bigg)= \sum_{i=1}^{N}f(x_i).
    $$
    
    Discrete probability theory is based on at most countable sample spaces $\Omega$, where all subsets of $\Omega$ may be treated as events, and thus elements of the $\sigma$-algebra $\mathcal{F}$. \\
    
    \item \textbf{Discrete measures}: If, instead, the sample space is uncountable, (eg. it has a discrete part a non-discrete part), it can be decomposed into \textbf{atoms}. The atoms are an at most countable (maybe even empty) set, whose probability is the sum of probabilities of all atoms. If this sum is equal to 1, then all other points can safely be excluded from the sample space, returning to the discrete case. Otherwise, if the sum of probabilities of all atoms is between 0 and 1, then the measure space decomposes into a discrete atomic part and a non-atomic part.    If $({\bf X}, \mathcal{F},\mu)$ is a measure space, then an \textbf{atom} of $\mu$ is a subset $\mathbf{A} \in \mathcal{F}$ such that
    
    \begin{itemize}
        \item it has positive measure $\mu[{\bf A}] = A > 0$.
        \item and it contains no set of smaller positive measure, ie. $\forall {\bf B} \subset {\bf A}$, either $\mu[{\bf B}] = A$ or $\mu[{\bf B}] = 0$.\\
    \end{itemize}
    
    For example, in the finite measure space above, the singleton set $\{x_n\}$ is an atom if $f(x_n) > 0$. The measure space $({\bf X}, \mathcal{F},\mu)$ is called discrete if it can be written as 
    
    $$
    {\bf X} = {\bf Z} \sqcup (\sqcup_{n=1}^{\infty} A_n),
    $$
    
    where $\mu[{\bf Z}] = 0$ and where $\{A_n\}_{n=1}^{\infty}$ is a collection of atoms. Note that any finite measure space is discrete. \\
    
    For example, consider the set ${\bf X}$ = \{1, 2, ..., 9, 10\} and let the $\sigma$-algebra ${\displaystyle \Sigma }$ be the power set of ${\bf X}$. Define the measure ${\displaystyle \mu }$  of a set to be its cardinality, that is, the number of elements in the set. Then, each of the singletons ${i}$, for i = 1, 2, ..., 9, 10 is an atom. On the other hand, consider the Lebesgue measure on the real line. This measure has no atoms. \\
    
    %\item \textbf{The Lebesgue measure}, the \textbf{Haar measures} and the \textbf{Hausdorff measure}. \\ 
    
    %If $\mathds{R}$ is seen as a group, then the Lebesgue measure arises as a Haar measure, instead, if $\mathds{R}$ is interpreted as a metric space, the Lebesgue measure arises as a Hausdorff measure. Instead, if it is treated as an ordered set, then the Lebesgue measure arises from a Stieltjes measure.\\
    
    %Given an ordered set $({\bf X}, <)$, a $\sigma$-algebra $\mathcal{F}$ can be defined to be the $\sigma$-algebra generated by all left-open intervals of the form $(a,b ]$\footnote{If ${\bf X}=\mathds{R}$ with the usual linear ordering, then this $\mathcal{F}$ is just the usual Borel $\sigma$-algebra.}. Now suppose that $f: {\bf X}\rightarrow \mathds{R}    $ is a  right-continuous, non-decreasing function, we can define the measure of any interval $(a,b]$ to be simply the difference between the value of $f$ at the two end-points, $a$ and $b$ :
    
    %$$
    %\mu_f (a,b] = f(b) - f(a).
    %$$
    
    %This measure can then be extended to the rest of the elements of $\mathcal{F}$ by approximating them with disjoint union of left-open intervals. \\
    
    %A measure $\mu_f$ is a \textbf{Stieltjes measure} and $f$ the \textbf{accumulation function} or \textbf{cumulative distribution} of $\mu_f$. Under suitable conditions, every measure on $({\bf X}, \mathcal{F})$ can be generated in this way: \\
    %Starting with an arbitrary measure, $\mu$, we can find a zero-point $x_0 \in {\bf X}$ so that 
    
    %\begin{itemize}
    %    \item $\mu(x_0, x]$ is finite for all $x > x_0$, \\
    %    \item $\mu(x_0, x]$ is finite for all $x < x_0$ \\
    %    \item Then define the function $f: {\bf X} \rightarrow \mathds{R}$ as 
    %$$
    %f(x) = \bigg\{ \begin{array}{ccc}
    %   \mu(x_0, x]  & \textnormal{if } x > x_0  \\
    %   -\mu(x_0, x]  & \textnormal{if } x < x_0 
    %\end{array}
    %$$.
    %\end{itemize}
    
    \item \textbf{Density Functions}: Let $\rho : \mathds{R}^n \rightarrow \mathds{R}$ be a positive, integrable function on $\mathds{R}^n$. For any ${\bf B} \in \mathcal{B}(\mathds{R}^n)$, then a measure can be defined as
    
    $$
    \mu_{\rho}({\bf B}) = \int_{{\bf B}} \rho,
    $$
    
    where $\rho$ is the \textbf{density function} for $\mu$. \\

\end{itemize}

Then, the ordered triple $({\bf X}, \mathcal{F}, \mu)$ is a \textbf{probability space} if $\mu$ is a \textbf{probability measure}. This is, a measure $\mu$ on ${\bf X}$ is a \textbf{probability measure} if $\mu[{\bf X}] = 1$ \footnote{\textbf{Stochastic processes} are a particular kind of probability measures, which represent a system randomly evolving in time. Let $\mathcal{S}$ be any  randomly-evolving complex system, let ${\bf X}$ be the set of all possible states of the system $\mathcal{S}$ and let $\mathds{T}$ be a set representing time. For example, 
    
    \begin{itemize}
        \item If $\mathcal{S}$ is a rolling die, then ${\bf X}= \{1,2,3,4,5,6\}$ and $\mathds{T}=\mathds{N} $ indexes the successive dice rolls. \\
        \item If $\mathcal{S}$ is a publically traded stock, then its state is its price. Thus ${\bf X} = \mathds{R}$. If we assume trading occurs continuously when the market is open, and let each trading day have length $c < 1$, then one representation of market time is $\mathds{T} = \sqcup_{n=1}^{\infty} [n, n+c]$. \\
        \item If $\mathcal{S}$ is a weather system, then its state can be representated by a large array of data ${\bf x} = [x_1,\cdots,x_n]$. Thus ${\bf X} = \mathds{R}^n$ and since the weather evolves continuously $\mathds{T}= \mathds{R}$.\\
    \end{itemize}
    
    The random evolution of $\mathcal{S}$ is represented by assigning a probability to every possible \textit{history}. A history is an assignment of a state in ${\bf X}$ to every moment in time (ie. in $\mathds{T}$). In other words, it's a function $f: \mathds{T} \rightarrow {\bf X}$. The set of all possibly histories is ${\bf H} = {\bf X}^{\mathds{T}}$ ie. the set of all functions $f: \mathds{T} \rightarrow {\bf X}$. The $\sigma$-algebra on ${\bf H}$ is usually a \textbf{cylinder algebra}. 
    
    \begin{itemize}
        \item Let $({\bf X}_\lambda, \mathcal{X}_\lambda)$ be measurable spaces for all $\lambda \in \Lambda$, where $\Lambda$ is some (possibly uncountably infinite) indexing set. Consider the cartesian product  $\bigtimes\limits_{\lambda \in \Lambda}{\bf X}_\lambda$. Let 
        
        $$
        \mathcal{M} = \bigg\{\bigtimes\limits_{\lambda \in \Lambda}{\bf U}_\lambda \textnormal{ } | \textnormal{ } \forall \lambda \in \Lambda \textnormal{ and } {\bf U}_{\lambda} \in \mathcal{F}_{\lambda} \textnormal{ and } {\bf U}_{\lambda} = {\bf X}_{\lambda} \textnormal{ for all but finitely many } \lambda \bigg\},
        $$
        
    such subsets are called \textbf{cylinder sets} in ${\bf X}$ and $\sigma(\mathcal{M})$ is the \textbf{cylinder} $\sigma$\textbf{-algebra} denoted by $\bigtimes\limits_{\lambda \in \Lambda}{\bf X}_\lambda \mathcal{F}_{\lambda}$. If the ${\bf X}_{\lambda}$ are topological spaces with Borel $\sigma$-algebras $\mathcal{F}_{\lambda}$, and we endow ${\bf X}$ with the Tychonoff product topology, then $\bigtimes\limits_{\lambda \in \Lambda} \mathcal{F}_{\lambda}$ is the Borel $\sigma$-algebra of ${\bf X}$. \\
    \end{itemize}
    
    Suppose that ${\bf X}$ has a $\sigma$-algebra $\mathcal{F}$, then it follows ${\bf H}$'s $\sigma$-algebra is $\mathcal{H}= \bigtimes\limits_{t \in \mathds{T}}\mathcal{F}_t$. An \textbf{event} is an element of $\mathcal{F}$ and thus corresponds to a cylinder set, a countable union of cylinder sets etc. Suppose, for all ${t\in \mathds{T}}$, that ${\bf U}_t \in \mathcal{F}$ with ${\bf U}_t = X$ for all but finitely many $t$. The cylinder set ${\bf U} = \prod_{t \in \mathds{T}} {\bf U}_t$ thus corresponds to the assertion: "for every ${t\in \mathds{T}}$, at time $t$, the state of $\mathcal{S}$ was inside ${\bf U}_t$. A probability measure on $(\bf H, \mathcal{H})$ is then a way of assigning probabilities to such assertions. }. \\
    
\paragraph{\textbf{Gibbs Random Fields}}

A \textbf{Gibbs random field} is defined on a lattice, as follows

\begin{itemize}
    \item A lattice, which is a countable set $\mathds{L}$ and a set $\mathcal{L}$, the set of all finite subsets of $\mathds{L}$.
    \item A single-spin space, which is a probability space $(S, \bm{\mathcal{S}}, \lambda)$, where $S$ is the set of all possible spin-values and is assumed to be discrete, $\bm{\mathcal{S}}$ is the $\sigma$-algebra associated with this set. By definition, $\bm{\mathcal{S}} \subseteq 2^{S}$, where $2^{S}$ is the powerset of $S$, and where $\lambda$ is the probability measure on this single-spin space.
    \item A configuration space $(\Omega, \mathcal{F})$, where $\Omega = S^{\mathds{L}}$ is the event space and where $\mathcal{F} = \mathcal{S}^{\mathds{L}}$.
    \item Given a configuration $\omega \in \Omega$ and a subset $\Lambda \subset \mathds{L}$, the restriction of $\omega$ to $\Lambda$ is 
    
    $$
        \omega_{\Lambda} = (\omega(t))_{t \in \Lambda}.
    $$
    
    In particular, if $A_1 \cap A_2 = \emptyset$ and $A_1 \cup A_2 = \mathds{L}$, then the configuration $\omega_{\Lambda_1} \omega_{\Lambda_2}$ is the configuration whose restrictions to $\Lambda_1$ and $\Lambda_2$ are $\omega_{\Lambda_1}$ and $\omega_{\Lambda_2}$, respectively. \\
    \item For each subset $\Lambda \subset \mathds{L}$, $\mathcal{F}_{\Lambda}$ is the $\sigma$-algebra generated by the family of functions 
    
    $$
        (\sigma(t))_{t\in \Lambda}, \textnormal{   where   } \sigma(t) (\omega) = \omega(t). 
    $$
    
    The union of these $\sigma$-algebras as $\Lambda$ varies over $\mathcal{L}$ is the algebra of cylinder sets on the lattice. 
    \item The potential is a family of functions 
    
    \begin{equation*}
        \Phi = (\Phi_A)_{A \in \mathcal{L}} \textnormal{   where   } \Phi_A : \Omega \rightarrow \mathds{R},
    \end{equation*}
    
    such that 
    
    \begin{enumerate}
        \item $\forall A \in \mathcal{L}$, $\Phi_A$ is $\mathcal{F}_\Lambda$-measurable, meaning it depends only on the restriction $\omega_A$, and does so measurably. 
        \item $\forall A \in \mathcal{L}, \blanky \forall \omega \in \Omega$, then 
        
        $$
        \exists H_{\Lambda}^{\Phi}(\omega) = \sum_{\substack{A \in \mathcal{L} \\
              A \cap \Lambda \neq \emptyset}} \Phi_A (\omega),
        $$
        
        where $\Phi_A$ is interpreted as the contribution to the total energy, ie. the Hamiltonian, associated to the interaction among all the points of finite set $A$. Then, $H_{\Lambda}^{\Phi}(\omega)$ is the contribution to the total energy of all the finite sets $A$ that meet $\Lambda$. Typically, the total energy may be infinite. \\
    
    \item The Hamiltonian in $\Lambda \in \mathcal{L}$ with boundary conditions $\bar \omega$, for the potential $\Phi$, is defined by 
    
    $$
     H_{\Lambda}^{\Phi}(\omega | \bar{\omega}) =  H_{\Lambda}^{\Phi}(\omega_{\Lambda} \bar{\omega}_{\Lambda^c}) \textnormal{    where   } \Lambda^c = \mathds{L} / \Lambda.
    $$
    \end{enumerate}
    
    \item The partition function in $\Lambda \in \mathds{L}$ with boundary conditions $\bar \omega$ and inverse temperature $\beta > 0$, for the $\Phi$-potential and $\lambda$ previously defined, is given by 
    
    \begin{equation*}
        \mathcal{Z}_{\Lambda}^{\Phi}(\bar{\omega}) = \int \lambda^{\Lambda}(d\omega) exp(-\beta   H_{\Lambda}^{\Phi}(\omega | \bar{\omega} ), \textnormal{ where } \lambda^\Lambda(d\omega) = \prod_{t\in\Lambda} \lambda(d\omega(t)) \textnormal{ is the product measure}.
    \end{equation*}
\end{itemize}

\blanky \\

As dynamical systems, the Ising model or lattice gas can be obtained by first ignoring the momenta and assuming that the particles' positions are restricted to a discrete subset of $\mathds{R}^d$. In general, this subset is taken to be the $d$-\textbf{dimensional cubic lattice}. Then,

\begin{equation}
    \textnormal{ the lattice is simply } \mathds{L} = \mathds{Z}^d, \textnormal{ where }
    \mathds{Z}^d = \{{\bf i} = (i_1, \cdots, i_d) \in \mathds{R}^d \blanky  |  \blanky i_k \in \mathds{Z} \blanky \forall k \in \mathds{N}_{[1, d]}\}.
\end{equation}

In other words, the system is imagined as living on $\mathds{R}^d$ and said space is composed of small cells, so that each cell can accommodate at most one particle. On each one of these cells, a single-spin space is then defined as $S_i = \{-1, 1\}^{i}, \blanky i \in \mathds{Z}$. To describe the model's microstates, consider a finite region $\Lambda \subset \mathds{Z}^d$, representing the vessel and an \textbf{occupation number} $n_i$ defined as 

\begin{equation}
    n_i : \Lambda \rightarrow \{0,1\} \Rightarrow \textnormal{ then the set of microstates is } \Omega_{\Lambda} = \{0, 1\}^{\times \Lambda}.
\end{equation}

The model automatically includes a short-range repulsion between the particles since no two particles are allowed to share a same cell. The attractive part of the interaction can then be included into the Hamiltonian:

$$
    \forall {\bf n} = (n_i)_{i \in \Lambda}, n_i \in \Omega_{\Lambda}, \blanky {\bf H}({\bf n}) = \sum_{\{i,j\} \subset \Lambda} \mathcal{J}(i-j) n_i n_j, 
$$

which is completely similar to a classical, continuous, Hamiltonian ${\bf H}({\bf q})$ and where the function $\mathcal{J}: \mathds{Z}^d \rightarrow \mathds{R}$ depends, \textit{a priori}, only on the distance between the $i$-th and $j$-th cells. Note that the contribution of a pair of cells $\{i,j\}$ is zero if they do not both contain a particle. The number of particles in $\Lambda$ is given by 

$$
    N_{\Lambda}({\bf n}) = \sum_{i\in\Lambda} n_i.
$$

Thus the basic object classical statistical mechanics is the Boltzmann weight $\prob(n)$. In thermal equilibrium, this is a probability measure that gives the probability that a system will be in a certain configuration ${\bf n} \in \Omega_{\Lambda}$, in terms of said configuration's energy and system's temperature. In other words,

\begin{equation}
    \begin{split}
        \prob_{\Lambda, \beta, N} :{\bf n} \in \Omega_{\Lambda}\rightarrow \R_{[0,1]} \textnormal{ where }
         \prob_{\Lambda, \beta, N} {(\bf n)} = \frac{e^{- {\bf H}({\bf n})}}{\mathcal{Z}_{\Lambda, \beta, N}},
    \end{split}
\end{equation}

where $\mathcal{Z}_{\Lambda, \beta, N}$ is the partition function defined on the finite region $\Lambda \subset \mathds{Z}^d$, at inverse temperature $\beta$ with $N$ particles. The Boltzmann weight is defined by the requirement that the probabilities sum to one. Then, the two relevant ensemble partition functions can be defined, as follows

\begin{align*}
        \mathcal{Q}_{\Lambda, \beta, N} = \sum_{\eta \in \Omega_{\Lambda, N}} e^{-\beta {\bf H}({\bf n})} & \blanky & 
        \Theta_{\Lambda, \beta, N} = \sum_{N} e^{\beta \mu N} \sum_{\eta \in \Omega_{\Lambda, N}} e^{-\beta {\bf H}({\bf n})} \\
        \blanky & \textnormal{ where } \Omega_{\Lambda, N} = \{n \in \Omega_{\Lambda} | N_{\Lambda} (n) = N\} & \blanky
\end{align*}


Note that if the degrees of freedom take on continuous values, or the model is defined in the continuum, then the sums are replaced by integrals. Note that, indeed, the probability of a given configuration increases as the energy gets lower, and conversely, that as the temperature gets higher and higher, the energies involved must get larger and larger to make a difference. Note as well that if all the energies are shifted by some constant $E_0$, the probabilities are left unchanged, since both the numerator and denominator are multiplied by the same constant, namely $e^{-\beta E_0}$. \\


Consider then the Ising model, with nearest-neighbour interactions (coupling constant $J$) and a magnetic field $(h)$. Its regular crystalline structure, corresponding to the positions of the magnet's atoms by a finite, non-oriented graph $G = (\Lambda, \mathcal{E})$, whose set of vertices is $\Lambda \subset \mathds{Z}^d$. Then, the \textbf{box of radius $n$} is 

$$
    B({n}) = \{-n, \cdots, n\}^d
$$

The edges of the graph will most often be between nearest neighbours, that is, pairs of vertices $i, j$ such that $||j-i||_{1} = 1$ \footnote{In this discrete setting, the norm is defined such that $||i||_1 = \sum_{k=1}^d |i_k|$}. Two nearest-neighbours spins are denoted by $i \sim j$. Then, the set of edges in the box $B(n)$ is $\{\{i,j\} \subset B(n) : i \sim j\}$. \\

The Ising model is defined by first assuming that a spin is located at each vertex of a graph $G = (\Lambda, \mathcal{E})$, such that each single-spin space has cardinality two, ie. each spin can only take two values. It follows, then, that to describe a microstate, a variable $\sigma_i$ taking two possible values ($\pm1$) is associated to each vertex $i \in \Lambda$, which is called the \textbf{spin} at $i$. A microstate of the system is called a \textbf{configuration}, and is thus an element

$$
    \sigma \in \Omega_{\Lambda} = \{0, 1\}^{\times \Lambda}.
$$

An $N$-site chain, with finite $N$, has a configuration space, which is a discrete space, which is a 0-manifold. This 0-manifold has zero topological dimension and $2^N$ cardinality. The microscopic interactions among the spins are defined such that 

\begin{enumerate}
    \item There are only interactions between pairs of sins located at neighbouring vertices. That is, it is assumed that the spins at two distinct vertices $i, j \in \Lambda$ interact if and only if the pair $\{i, j\}$ is an edge of the graph. \\
    
    \item The interaction favours agreement of spin values. In the simplest form of the Ising model, this is done in the simplest possible way: a pair of spins at the endpoints $i$ and $j$ of an edge, decreases the overall configuration's energy if they agree $\sigma_i = \sigma_j$ and increases if they differ. More precisely, the spins at the endpoints of the $\{i,j\}-$edge contribute to the total energy by an amount 
    
    $$
        - \sigma_i \sigma_j.
    $$
    
    Therefore, configurations in which most pairs of neighbours are aligned to have smaller energy. 
    
    \item Spins align with the external magnetic field. Assume that a constant external magnetic field of $h \in \mathds{R}$ intensity, oriented along the same direction as the spins, acts on the system. Its interaction with the $i$-th spin contributed an
    
    $$
        -h \sigma_i
    $$
    
    to the total energy. That is, when the magnetic field is positive, the configurations with most of their spins equal to +1 have smaller energy. \\
\end{enumerate}

Then, the energy of a given configuration $\sigma$ is obtained by summing the interactions over all pairs and by adding the interaction of each spin with the external magnetic field, thus yielding the \textbf{Ising Hamiltonian}

\begin{equation}
    {\bf H}_{\Lambda; h}(\sigma) = - J \sum_{\substack{i,j \in \Lambda} \\
    i \sim j} \sigma_i \sigma_j - h \sum_{i \in \Lambda} \sigma_i, \textnormal{ with } \sigma \in \Omega_{\Lambda}. 
\end{equation}

Since it favours local alignment of the spins, the Hamiltonian of the model is said to be ferromagnetic, which is not to say it behaves like a ferromagnet. The Gibbs measure/distribution is then denoted by 

\begin{equation}
    \begin{split}
        \prob_{\Lambda, \beta, h} :{\bf n} \in \Omega_{\Lambda}\rightarrow \R_{[0,1]} \textnormal{ where }
         \prob_{\Lambda, \beta, h}(\sigma) = \frac{e^{-\beta {\bf H}_{\Lambda; h}(\sigma) }}{\mathcal{Z}_{\Lambda, \beta, h}}. 
    \end{split}
\end{equation}

Note that, in the absence of a magnetic field, even though local spin alignment is favoured by the Hamiltonian, neither of the orientations (+1 or -1) is globally favoured. Namely, if $-\sigma$ denotes the spin-flipped configuration in which $(-\sigma)_{i} = -\sigma_i$, then ${\bf H}_{\Lambda; 0}(\sigma) = {\bf H}_{\Lambda; 0}(-\sigma)$, this implies 

$$
    \prob_{\Lambda, \beta, 0} (-\sigma) = \prob_{\Lambda, \beta, 0}. 
$$

The model is then said to be invariant under global spin flip. When $h\neq0$, this symmetry no longer holds. Then the expectation of an observable $f : \Omega_\Lambda \rightarrow \mathds{R}$, under $\mu_{\Lambda, \beta, h}$, is denoted by $\langle f \rangle_{\Lambda, \beta, h}$. 
The parameter $J$ is the coupling strength, if $J>0$ this models describes a ferromagnetic coupling, while $J<0$ describes an antiferromagnetic coupling. 
From the partition function, the expectation values of physical quantities can be calculated. For example

\begin{itemize}
    \item Internal energy $\langle E\rangle_{\Lambda, \beta, h} = \frac{1}{\mathcal{Z}} \sum_{n} E_n e^{-\beta E_n} = - \frac{\partial}{\partial \beta} \log \mathcal{Z}$. \\
    \item Specific heat $C = \frac{\partial}{\partial T} \langle E\rangle_{\Lambda, \beta, h} = k \frac{\partial}{\partial T} T^2 \frac{\partial \log \mathcal{Z}}{\partial T}$ \\
    \item Two-point correlator in the Ising model, 
    
    $$
    \langle \sigma_a \sigma_b \rangle_{\Lambda, \beta, h} = \sum_{\sigma_i} \sigma_a \sigma_b \prob_{\Lambda, \beta, h}({\sigma_i}) = \frac{1}{\mathcal{Z}} \sum_{\sigma_i = \pm 1} \sigma_a \sigma_b e^{-\beta {\bf H}_{\Lambda; h}(\sigma)},
    $$
    
    which describes how the degrees of freedom at one point are affected by the degrees of freedom at another point. If $ \langle \sigma_a \sigma_b \rangle = 1$, the two spins are aligned, while if $ \langle \sigma_a \sigma_b \rangle = -1$ the two spins are anti-aligned, and if $ \langle \sigma_a \sigma_b \rangle = 0$, they are uncorrelated. 
\end{itemize}

%The partition function is then 
%\begin{equation}
%    \mathcal{Z} = \sum_{\sigma_i = \pm 1} e^{-\beta E(\{\sigma_i\})},
%\end{equation}
%where the sum is performed over all $2^N$ different configurations. 

%Consider the Ising model, an interacting many-body system, where the energies $E_n$ depends on mutual properties of the degrees of freedom. This is a spin system where the spin is contrained to point in one of two directions, these directions are describied by a variable $\sigma_i = 1$ for the up direction and $\sigma_i = -1$ for the down direction. For an $N$-site lattice, there are therefore $2^N$ different configurations in the model. The simplest interaction is to assign one energy unit if the neighbouring spins are the same, and another if neighbouring spins are different, that is 

%\begin{equation}
%    E(\{\sigma_i\}_{i=1}^{N}) = -J \sum_{\langle ij \rangle} \sigma_i \sigma_j, \textnormal{ where } \begin{array}{c}
%        \{\sigma_i\}_{i=1}^{N} \in \bigtimes_{i=1}^{N}  T^* \mathcal{M}_i \\
%    \end{array}
%\end{equation}

%and where the sum is performed on all nearest-neighbour sites $i$ and $j$ ie. this sum is not over all sites, but rather over all bonds. Note that on the square lattice, there are two bonds for every site. 

\blanky \\

\paragraph{\textbf{Three kinds of phases}}

Even in simple interacting systems, the two point function will not have such simple values except in some extreme limits. However, knowing its dependence on the separation is valuable information, and it implies that the behaviour of the correlators is one of the best ways of understanding and characterising different phases of matter. Consider a correlator defined so that in the non-interacting limit, ie. the $T \rightarrow \infty$, it vanishes. Letting $L$ be the system size, there are three main types of behaviour of such a two-point function as $L >> |a-b|>> 1$, as follows 

\begin{itemize}
    \item \textbf{ordered:} 
    
    $$
      \langle \sigma_a \sigma_b \rangle \sim C \neq 0,
    $$
    
    
In an ordered spin system, the spins are correlated arbitrarily far from each other: if a spin, at a given position, points in some direction, then another spin is going to most likely point in the same direction. For example, in an antiferromagnet, order typically means that spins at an odd number of sites apart, are antialigned, while those an ever number of sites apart are aligned. 

Another way to characterize order is in terms of a one-point function, of a given variable whose thermodynamic average vanishes in the non-interacting limit. In the Ising model, this is simply $\langle \sigma_a \rangle$, which is the probability that a given spin is either up or down. In principle, this is easier to compute than the two-point function but, in many cases including the Ising model, it vanishes for all values of $a, b, J$. Indeed, this follows from the Ising model's $\mathds{Z}_2$-symmetry $    \sigma_i \rightarrow -\sigma_i$. Then, $E(\{\sigma_i\}_{i=1}^{N}) =  E(-\{\sigma_i\}_{i=1}^{N})$. Therefore, as long as the boundary conditions remain invariant under spin flip, it holds that 

\begin{equation}
    -\langle \sigma_a \rangle = - \sum_{\sigma_i = \pm 1} \sigma_i e^{-\beta E(\{\sigma_i\})} = \sum_{\sigma_i = \mp 1} s_a e^{-\beta E(\{s_i\})} \textnormal{ where } s_i = -\sigma_i.
\end{equation}

However, since the sum is performed over a finite number of lattice sites, the $s_i$-variables on the last equality may be renamed as $\sigma_i$. Thus, the last expression is simply $\langle \sigma_a \rangle=0$. \\
    
    \item \textbf{disordered:} 
    
    $$
      \langle \sigma_a \sigma_b \rangle \sim e^{-\frac{|a-b|}{\epsilon}},
    $$
    
    A more general way of characterizing order is to use the previous definition, ie. a non-vanishing value of a two-point function at arbitrarily large separation. Then, the order parameter can be defined as the square root of this absolute value. \\
    
    \item \textbf{critical:} 
    
    $$
      \langle \sigma_a \sigma_b \rangle \sim \frac{1}{|a-b|^{2x}}.
    $$
    
    In a disordered phase, the two-point function class off to zero exponentially fast. Thus a conveniet measure is the correlation length $\epsilon$. A useful heuristic argument is to think of the degrees of freedom closer than $\epsilon$ as being correlated, and those further apart as being uncorrelated. One thing to bear in mind is that not all disordered systems behave in the same way. Some systems are disordered by definition, but possess a topological order, where an expectation value of some non-local quantities has a non-vanishing expectation value. \\

The system's criticality is the most difficult and physically meaningful case. The correlator is said to be algebraically decaying. In particular, in the algebraic decay, the spatial dependence does not involve in any way a dimensional parameter, but rather only a dimensionless parameter $x$. Thus, this behaviour is characteristic of a critical phase, where the system is scale-invariant, given that there is no parameter like a correlation length. The reason for the factor of 2 in the $x$-quantity's definition stems from this picture, in the continuum limit, $x$ can be thought of as a "scaling dimension" of the spin field. \\
\end{itemize}

Note that, this picture is oversimplified since different types of correlators can occur in the same system. \\

%A simple phase diagram in statistical mechanics. 

A common case in statistical mechanics occurs when, at temperatures much larger than any parameter with dimensions of energy - like $J$ in the Ising model-, one can effectively neglect the energy altogether. The partition function is then a sum over all allowed configurations with the same weight each. Then, the two-point function of any local operators vanishes quickly as they are brought apart (as long as non-local constraints are not present). The system is then in a disordered phase at sufficiently large temperatures. At low enough temperatures, the opposite occurs: since the Boltzmann weight is proportional to $e^{-E/T}$, the very low-energy configurations are much more likely. This can lead to order, the lowest-energy configurations have some regular pattern and so there is a non-vanishing local order parameter. When there is a disordered phase at high-temperature and an ordered phase at high temperature as well, there must be a phase transition in between. There are two categories of phase transitions, often referred to as first-order and continuous. In the latter case, correlators become algebraically decaying, and then the system is critical. In the former case, the system remains disordered at the phase transition and abruptly changes its behaviour. \\

\subsubsection{Exact solution to the Ising Models}

\paragraph{\textbf{Zero-dimensional Ising model}}

A system with a finite number of spins, no matter in which directions, is commonly referred to a zero-dimensional model. In the $d=0$-Ising model, there are just two degrees of freedom, called $s_1$ and $s_2$, both of which can only take two values $s_i \in S = \{\pm 1\}$ and where the configuration space is given by $(\Omega_\Lambda, \mathcal{F})$, where $\Omega_\Lambda = \{-1, 1\}^{\times 2}$ and where the $\sigma$-algebra $\mathcal{F} = 2^{\Omega_\Lambda}$. To compute the partition function, $\mathcal{Z}_{\Lambda, \beta, B}$, the system's energy must be known for every configuration $(s_1, s_2) \in \Omega_{\Lambda}$, which is assumed to be 

\begin{equation}
    E({\bf s}) = - J s_1 s_2 - B(s_1+s_2).
\end{equation}

This energy models a system made of magnetic moments. Although, in general, the moments can point in any direction, in the Ising model they point up or down one axis - say the $z$-axis - along which an external magnetic field $B$ is applied. The constant $J$ has units of energy and $B$ has units of magnetic moment. If $J>0$, the model is ferromagnetic and the energy favors aligned spins while if $J<0$ it models an antiferromagnetic system, which favours anti-parallel spins. In both cases, $B>0$ is chosen so that the spins tend to align with the external field. \\

The partition function may be evaluated as 

\begin{equation}
    \mathcal{Z}_{\Lambda, K, h} = \sum_{{\bf s} \in \Omega_{\Lambda}} e^{K s_1 s_2 + h(s_1+s_2)} \textnormal{ where } \begin{array}{c}
         K = \beta J  \\
         h = \beta B.
    \end{array}
\end{equation}

It is readily seen that 

\begin{equation}
    \mathcal{Z}_{\Lambda, K, h}  = \sum_{m=-\frac{1}{2}}^{\frac{1}{2}} e^{\beta h m} 2 \cosh\bigg(\frac{h + Km}{2}\bigg) = 2 \cosh(2h) e^{K} + 2e^{-K}.
\end{equation}

At very low temperatures (very low $T$ or very high $\beta$), the state of lowest energy will dominate the sum and thus, the spins are expected to be aligned with $B$ and also with each other. At very high temperatures and vanishing $\beta$, all four states get equal weight and the spins will fluctuate independently of each other and the applied field. In effect, let the system's (spin-) average magnetization be $M = \frac{1}{2} (s_1+s_2)$, in any given configuration\footnote{Note that this is the average over all spins in the system, and not the thermal average}. The thermal average can then be found to be 

\begin{equation}
    \begin{split}
        \langle M \rangle_{\Lambda, K, h} &= \frac{1}{ \mathcal{Z}_{\Lambda, \beta, h}} \sum_{{\bf s} \in \Omega_\Lambda} \frac{1}{2} (s_1+s_2) e^{K s_1 s_2 + h(s_1+s_2)} = \frac{1}{2 \mathcal{Z}_{\Lambda, K, h} }\frac{\partial  \mathcal{Z}_{\Lambda, K, h} }{\partial h} \\
        &= \frac{1}{2} \frac{\partial \log  \mathcal{Z}_{\Lambda, K, h}}{\partial h}.
    \end{split}
\end{equation}

In terms of the free energy $F(K,h)$, defined as $\mathcal{Z}_{\Lambda, K, h} = e^{-\beta F}$, where $-\beta F = \log(2 \cosh(2h) e^{K} + 2e^{-K})$ it yields

\begin{equation}
    \langle M \rangle_{\Lambda, K, h}  = \frac{\partial [-\beta F(K,h)]}{\partial h} = \frac{\sinh 2h}{\cosh 2h + e^{-2K}} \textnormal{ which, as expected} \begin{array}{c}
         \langle M \rangle \underset{h, K \rightarrow \infty}{\rightarrow} 1   \\
         \langle M \rangle \underset{h, K \rightarrow 0}{\rightarrow} h. 
    \end{array}
\end{equation}

\blanky \\

The thermal average of a particular spin can also be calculated, provided a soruce term of the form $h_1 s_1 + h_2 s_2$ is added in the Boltzmann weight, which couples the $i$
th spin to its own independent field $h_i$. Then,

\begin{align}
    \mathcal{Z}_{\Lambda, K, h} &= \sum_{{\bf s} \in \Omega_\Lambda} e^{K s_1 s_2 + {\bf h} \cdot {\bf s}} = e^{-\beta F(K, {\bf h})} \\ \Rightarrow \langle s_i \rangle = & \frac{\partial [-\beta F(K,{\bf h})]}{\partial h_i} = \frac{\partial \log \mathcal{Z}_{\Lambda, K, h}}{\partial h_i} = \frac{1}{\mathcal{Z}_{\Lambda, K, h}} \frac{\partial \mathcal{Z}_{\Lambda, K, h}}{\partial h_i} .
\end{align}

Taking the mixed derivative with respect to both $h_1$ and $h_2$ yields 

\begin{equation}
    \begin{split}
        \frac{\partial^2 [-\beta F(K,{\bf h})]}{\partial h_1 \partial h_2} &= \frac{\partial}{\partial h_1} \bigg(\frac{1}{\mathcal{Z}_{\Lambda, K, h}} \frac{\partial \mathcal{Z}_{\Lambda, K, h}}{\partial h_2} \bigg)  \\
        &= \frac{1}{\mathcal{Z}_{\Lambda, K, h}} \frac{\partial^2 \mathcal{Z}_{\Lambda, K, h}}{\partial h_1 \partial h_2} - \frac{1}{\mathcal{Z}^2} \frac{\partial \mathcal{Z}_{\Lambda, K, h}}{\partial h_1} \frac{\partial \mathcal{Z}_{\Lambda, K, h}}{\partial h_2} \\
        &= \langle s_1 s_2 \rangle - \langle s_1 \rangle \langle s_2 \rangle \\
        &= \langle s_1 s_2 \rangle_c \textnormal{ which is the connected correlation function.}
    \end{split}
\end{equation}

Similarly, if there are four spins coupled to four independent magnetic field $h_1, h_2, h_3, h_4$ then

\begin{equation}
    \begin{split}
        \frac{\partial^4 [-\beta F(K,{\bf h})]}{\partial h_1 \partial h_2 \partial h_3 \partial h_4}\bigg|_{h=0} &= \langle s_1 s_2 s_3 s_4 \rangle_c \\
        &= \langle s_1 s_2 s_3 s_4 \rangle - \langle s_1 s_2 \rangle \langle s_3 s_4 \rangle - \langle s_1 s_3 \rangle \langle s_2 s_4 \rangle - \langle s_1 s_4 \rangle \langle s_2 s_3 \rangle.
    \end{split}
\end{equation}

Since $h=0$ in the end, there will be no correlators with an odd number of spins. Note that the previous derivations are valid even if $s$ is replaced by some other variable, with eg. continuous values, coupled to the corresponding magnetic field in the Boltzmann weight, since the fact that $s = \pm 1$ was not used. The thermal averages may be computed in the case of non-zero zeros. In particular, if the field is uniform or zero. In that case, the $h_i$-derivatives are calculated and then set to $h_i = h$ or $0$, for all $i$. Then, for example, if there were a uniform external field $h$, the connected correlation function is 

\begin{equation}
    \begin{split}
        \langle s_1 s_2 \rangle - \langle s_1 \rangle \langle s_2 \rangle = \langle s_1 s_2 \rangle_c = \frac{\partial^2 [-\beta F(K,{\bf h})]}{\partial h_1 \partial h_2}\bigg|_{h_i = h, \blanky \forall i.}
    \end{split}
\end{equation}

The correlation function $\langle s_1 s_2 \rangle$ as a $K$-derivative, 

\begin{equation}
    \begin{split}
        \frac{\partial[-\beta F(K,{\bf h})]}{\partial K} &= \frac{\partial \log \mathcal{Z}_{\Lambda, K, h}}{\partial K} = \frac{1}{\mathcal{Z}_{\Lambda, K, h}}\frac{\partial \mathcal{Z}_{\Lambda, K, h}}{\partial K} = \frac{1}{\mathcal{Z}_{\Lambda, K, h}} \frac{\partial}{\partial K}  \sum_{{\bf s} \in \Omega_{\Lambda}} e^{K s_1 s_2 + h(s_1+s_2)} \\
        &= \frac{1}{\mathcal{Z}_{\Lambda, K, h}} \sum_{\mathcal{M}^{\times 2}} s_1 s_2 e^{K s_1 s_2 + h(s_1+s_2)} = \langle s_1 s_2 \rangle \\
        &= \frac{e^K \cosh 2h - e^{-K}}{e^K \cosh 2h + e^{-K}},
    \end{split}
\end{equation}

which may interpreted as quantifying the average interaction energy, which happens to coincide with the correlation of neighbouring spins. In a model with more spins, correlations of not-neighbouring spins (eg. the correlation between $s_5$ and $s_{92}$) cannot be obtained via a $K$-derivative. The distinction between the average interaction energy and generic spin correlations is blurred in this toy-model. \\

Higher-order derivatives with respect to $K$ or $h$ (for uniform $h$) give additional information about fluctuactions about the mean. Thus, 

\begin{equation}
    \chi = \frac{1}{2} \frac{ \partial \langle M \rangle}{\partial h} = \frac{1}{4} \frac{\partial^2 [-\beta F]}{\partial h^2} = \langle M^2 \rangle_{\Lambda, K, h}- \langle M \rangle^2_{\Lambda, K, h}
\end{equation}

measures the magnetization's fluctuation about its mean and is the magnetic susceptibility, and quantifies the rate of change of the average magnetization with the applied field. \\

\paragraph{\textbf{One-dimensional Ising model}}

The standard way for solving the one-dimensional Ising model is to use the transfer matrix. The physical motivation behind it is very elegant: it consists on singling out a given spatial direction, say the $x$-direction. Then the partition function is built up by "evolving" the system from one value of $x=x_0$ to the next value $x=x_1$. Repeating this process yields the full partition function after all values of $x$ are summed over. This method is not only useful for classical statistical mechanics, but by replacing the spatial evolution with a real time-evolution, yields a connection between quantum and classical statistical mechanics. \\

Now, any real system is finite, one may ask why we should focus on infinite systems. The answer lies that systems with a very large number of degrees of freedom look more like infinite systems than finite ones in the following sense.  Phenomenologically, system at a temperature lower than some Curie temperature $T_C$ will magnetize. But it can be shown rigorously that a finite system of Ising spins can never magnetize: if it is polarized one way (say the $z$-axis) for some time, it can jump to the opposite polarization after some time, so that on average it is not magnetized. This rigorous result can be reconciled with reality by noting that the time to flip can be as large as the age of the universe, so that in human time, magnetization is possible. The nice thing about infinite system is that remote possibilities are rendered impossible, making them a better model of real life. \\








Eventually, the thermodynamic limit $N \rightarrow \infty$ may be taken. The reason for lining up the lattice points on a vertical axis is that later it will become the time axis for a related problem. The energy of a given configuration may be written as 

\begin{equation}
    E = -J \sum_{i=0}^{N-1} s_i s_{i+1}.
\end{equation}

In the one-dimensional Ising model, there is only a single site at a fixed $x$-vale, so that the transfer matrix therefore "evolves" the system from one spin to the next. Therefore, consider two nearest-neighbour sites in the Ising model. There are, then, four configurations on these two sites, $++, +-, -+, --$. The Boltzmann weights $\prob(\sigma_1 \sigma_2)$ for these four configurations are 

\begin{equation}
    \prob(++) = \prob(--) = e^{\beta J} \textnormal{  and  } \prob(+-) = \prob(-+) = e^{-\beta J}.
\end{equation}

Now consider for example an Ising model on three sites, with the simplest boundary conditions, where both spins $\sigma_1$ and $\sigma_3$ are both fixed at specific values. Summing over all values of the middle spin $\sigma_2 = \pm 1$, the partition function with both end spins fixed is 

\begin{equation}
    \sum_{\sigma_2 = \pm 1} \prob(+\sigma_2) \prob(\sigma_2+) = e^{2J} + e^{-2J}.
\end{equation}

The general expression for the partition function of an $\ell$-site Ising model with fixed boundary conditions can be found to be 

\begin{equation*}
    \begin{array}{c}
        \textnormal{If $N = 3$ then } \mathcal{Z}_3(\textnormal{fixed}) = \sum_{\sigma_2 = \pm 1} \prob(\sigma_1\sigma_2) \prob(\sigma_2\sigma_3).  \\
        \\
        \textnormal{If $N = 4$ then } \mathcal{Z}_4(\textnormal{fixed}) = \sum_{\substack{\sigma_2 = \pm 1\\
    \sigma_3 = \pm 1}} \prob(\sigma_1\sigma_2) \prob(\sigma_2\sigma_3) \prob(\sigma_3s\sigma_4). \\
    \end{array}
\end{equation*}

These expression look exactly like matrix multiplications, written out in terms of the matrix elements. The matrix being multiplied is called the transfer matrix and includes all the nearest-neighbour interactions. For the one-dimensional Ising model it is given by 

\begin{equation}
    \bm{\mathcal{T}} = \left( \begin{array}{cc}
        \prob(++) & \prob(+-) \\
         \prob(-+) &  \prob(--)
    \end{array} \right) = \left(\begin{array}{cc}
        e^J & e^{-J}  \\
        e^{-J} & e^J
    \end{array}\right).
\end{equation}

To compute the partition function with fixed boundaries conditions, from the transfer matrix, the boundary conditions are treated as vectors, on which the transfer $\bm{\mathcal{T}}$-matrix acts on. Here, the basis elements are the two spin values $+$ and $-$. These two vectors span a $\ctwo$-vector space. The partition function of an $N$-site system with fixed boundary conditions can be found out to be 

\begin{itemize}
    \item if $N=2$,
        $
        \mathcal{Z}_2(\textnormal{fixed}) = \bra{\sigma_1} \bm{\mathcal{T}} \ket{\sigma_2} = \prob(\sigma_1 \sigma_2),
        $
    \item if $N=3$,
        $
        \mathcal{Z}_3(\textnormal{fixed}) = \bra{\sigma_1} \bm{\mathcal{T}}^2 \ket{\sigma_3},
        $
    \item in general,
        $
        \mathcal{Z}_N(\textnormal{fixed}) = \bra{\sigma_1} \bm{\mathcal{T}}^{N-1} \ket{\sigma_N}.
        $
\end{itemize}

Note that each time the transfer matrix is applied, it corresponds to adding one bond to the system. The transfer matrix now can easily be used to find analogous expressions for other boundary conditions. For example, \\

\begin{itemize}
    \item for free boundary conditions, allowing the extreme-most spins to take on all allowed value, the partition function is given by 
    
    \begin{equation}
    \begin{split}
     \mathcal{Z}_N(\textnormal{free}) &= \sum_{\substack{\sigma_1 = \pm 1\\
    \sigma_N = \pm 1}} \bra{\sigma_1} \bm{\mathcal{T}}^{N-1} \ket{\sigma_N} =  \left(\begin{array}{cc}
        1 & 0 \\
    \end{array}\right) \bm{\mathcal{T}}^{N-1} \left(\begin{array}{c}
        1 \\
        0
    \end{array}\right) +\left(\begin{array}{cc}
        1 & 0 \\
    \end{array}\right) \bm{\mathcal{T}}^{N-1} \left(\begin{array}{c}
        0 \\
        1
    \end{array}\right) \\
    & + \left(\begin{array}{cc}
        0 & 1 \\
    \end{array}\right) \bm{\mathcal{T}}^{N-1} \left(\begin{array}{c}
        1 \\
        0 
    \end{array}\right).
    \left(\begin{array}{cc}
        0 & 1 \\
    \end{array}\right) \bm{\mathcal{T}}^{N-1} \left(\begin{array}{c}
        0 \\
        1 
    \end{array}\right)
    \\
    &= \left(\begin{array}{cc}
        1 & 1 \\
    \end{array}\right) \bm{\mathcal{T}}^{N-1} \left(\begin{array}{c}
        1 \\
        1 
    \end{array}\right).
    \end{split}
    \label{Ising model - Free PF}
    \end{equation}
    
    \item for periodic boundary conditions, where an extra interaction $-J \sigma_N \sigma_1$ is added to the energy, including a bond between the two most-extreme spins. In this case, the model is translation invariant, shifting all the spins by one site mod $N$ doesn't change the energy. The transfer $\bm{\mathcal{T}}$ must therefore evolve the first spin back to the $N$-th one. This is equivalent to having fixed boundary conditions on $N+1$-sites where $\sigma_{N+1} = \sigma_1$, and then summing over both possible values, thus yielding 
    
    \begin{equation}
    \begin{split}
    \mathcal{Z}_N(\textnormal{periodic}) &= \sum_{\sigma_1 = \pm 1} \bra{\sigma_1} \bm{\mathcal{T}} \ket{\sigma_1} =  \left(\begin{array}{cc}
        1 & 0 \\
    \end{array}\right) \bm{\mathcal{T}}^{N} \left(\begin{array}{c}
        1 \\
        0 
    \end{array}\right) +  \left(\begin{array}{cc}
        0 & 1 \\
    \end{array}\right) \bm{\mathcal{T}}^{N} \left(\begin{array}{c}
        0 \\
        1 
    \end{array}\right) \\
    &= \Tr_{\mathds{C}^2} \bm{\mathcal{T}}^{N}.
    \end{split}
    \end{equation}
\end{itemize}

The partition functions with different boundary conditions are simply related, eg. $\mathcal{Z}_N(\textnormal{periodic}) = \mathcal{Z}_N(++) + \mathcal{Z}_N(--)$. The usefulness of the transfer matrix in a one-dimensional system lies in that the partition function can be computed simply by diagonalizing the transfer matrix. For the one-dimensional Ising model, the diagonalization yields  

\begin{equation}
    \begin{split}
    \bm{\mathcal{T}} = \left(\begin{array}{cc}
        1 & 1 \\
        -1 & 1 
    \end{array}\right)\left(\begin{array}{cc}
        2\cosh(\beta J) & 0 \\
        0 & 2\sinh(\beta J)
    \end{array}\right) \left(\begin{array}{cc}
        1 & 1 \\
        -1 & 1 
    \end{array}\right)^{-1}
    \end{split},
    \label{Ising model - T matrix diag}
\end{equation}

and given that the transfer matrix is hermitian, the eigenvalues are real. Then, the partition functions for the different boundary conditions can then be found

\begin{itemize}
    \item For the free case, the $\ket{\sigma_j}$-vectors can be decomposed in terms of the $\bm{\mathcal{T}}$- matrix's eigenvectors, which can be written as 
    
    $$ 
     v_{\pm}= \frac{1}{2} \bigg(\left(\begin{array}{cc}
     1 & 1 \end{array}\right)^{\textnormal{T}} + \left(\begin{array}{cc}
     1 & -1 \end{array}\right)^{\textnormal{T}} \bigg) 
     $$
     
     Therefore, plugging in these results in \cref{Ising model - Free PF} yields,
     
     \begin{equation}
         \begin{split}
             \mathcal{Z}_N(\textnormal{free}) &= \left(\begin{array}{cc}
        1 & 1 \\
    \end{array}\right) \bm{\mathcal{T}}^{N-1} \left(\begin{array}{c}
        1 \\
        1 
    \end{array}\right) = 2(2\cosh \beta J)^{N-1}.
         \end{split}
     \end{equation}
     
    \item For the periodic-boundary conditions, the partition function may be calculated as the sum of $\mathcal{Z}_N(++)$ and $\mathcal{Z}_N(--)$, which can be easily calculated from \cref{Ising model - T matrix diag}, as follows 
    
    \begin{equation}
    \begin{array}{c}
             \mathcal{Z}_N(++) = \left(\begin{array}{cc}
        1 & 0 \\
    \end{array}\right) \bm{\mathcal{T}}^{N} \left(\begin{array}{c}
        1 \\
        0 
    \end{array}\right)  = \frac{1}{2}\bigg[\bigg(2 \cosh \beta J\bigg)^N + \bigg(2 \sinh \beta J\bigg)^N \bigg] \\
    \mathcal{Z}_N(+-) = \left(\begin{array}{cc}
        1 & 0 \\
    \end{array}\right) \bm{\mathcal{T}}^{N} \left(\begin{array}{c}
        0 \\
        1 
    \end{array}\right)  = \frac{1}{2}\bigg[\bigg(2 \cosh \beta J\bigg)^N - \bigg(2 \sinh \beta J\bigg)^N \bigg]
    \\
    %\end{array} \begin{array}{c}
    \mathcal{Z}_N(-+) = \left(\begin{array}{cc}
        0 & 1 \\
    \end{array}\right) \bm{\mathcal{T}}^{N} \left(\begin{array}{c}
        1 \\
        0 
    \end{array}\right)  = \frac{1}{2}\bigg[\bigg(2 \cosh \beta J\bigg)^N - \bigg(2 \sinh \beta J\bigg)^N \bigg] \\
    \mathcal{Z}_N(--) = \left(\begin{array}{cc}
        0 & 1  \\
    \end{array}\right) \bm{\mathcal{T}}^{N} \left(\begin{array}{c}
        0 \\
        1 
    \end{array}\right)  = \frac{1}{2}\bigg[\bigg(2 \cosh \beta J\bigg)^N + \bigg(2 \sinh \beta J\bigg)^N \bigg] \\
    \end{array}
    \end{equation}
    
    Then, given that $\mathcal{Z}_N(\textnormal{periodic}) = \mathcal{Z}_N(++) + \mathcal{Z}_N(--)$, the periodic boundary condition partition functions is given by 
    
    \begin{equation*}
        \begin{split}
          \mathcal{Z}_N(\textnormal{periodic}) &= (2\cosh \beta J)^N + (2\sinh \beta J)^N \\
          &= (2\cosh \beta J)^N \bigg(1+ (2\tanh \beta J)^N\bigg) \\
          &\approx (2\cosh \beta J)^N \textnormal{ for } N >> 1. 
        \end{split}
    \end{equation*}
\end{itemize}

Note as well that the free energy per site is proportional to $\frac{\log \mathcal{Z}}{N}$, and is independent of boundary conditions when $N >> 1$. In effect, the free energy per site in the thermodynamic limit may be written as

\begin{equation}
    F(\beta, K) = - \lim_{N \rightarrow \infty} \frac{\log \mathcal{Z}}{N}.
\end{equation}

\blanky \\

It is straight forward to compute the two-point function in terms of these partition functions. Consider the following special cases.

\begin{itemize}
    \item First, consider an Ising spin chain with free boundary conditions. Then, the two-point function between the two end-spins can be computed by summing over the four possible configurations. But these spins are opposite, then 

\begin{equation}
    \langle \sigma_1 \sigma_N \rangle = \frac{\mathcal{Z}_N(++) + \mathcal{Z}_N(--) - \mathcal{Z}_N(+-) - \mathcal{Z}_N(-+)}{\mathcal{Z}_{\textnormal{free}}} =
    (\tanh \beta J)^{N-1},
    \end{equation}
    
   which falls off exponentially with distance, so even for small distances, the end spins are effectively uncorrelated.
  \\
   \item Now, consider an Ising chain with periodic boundary condtions. Then, the the two-point function between the $j$-th and the $i$-th spins, with $j>i$, may be written as

    \begin{equation}
       \langle \sigma_j \sigma_i \rangle_{\textnormal{periodic}} =  \frac{2 \mathcal{Z}_{|j-i|}(++)\mathcal{Z}_{N-|j-i|}(--) + 2\mathcal{Z}_{|j-i|}(+-) \mathcal{Z}_{N-|j-i|}(-+)}{\mathcal{Z}_N{\textnormal{periodic}}},
    \end{equation}
\end{itemize}

\blanky \\

The general two-point correlator is similar and is given by 

\begin{equation}
    \langle \sigma_j \sigma_i \rangle_{j > i} = \frac{1}{\mathcal{Z}} \sum_{k} \sigma_j \sigma_i e^{\sum_{k} \beta J \sigma_k \sigma_{k+1}},
\end{equation}

which measures how likely $s_i$ and $s_j$ are, on average, to point in the same direction. In any ferromagnetic system, the average will be positive for a pair of neighbouring spins since the Boltzmann weight is biased toward parallel values. Surely, if a spin can influence its neighbours to be parallel to it, they in turn will act similarly on their own neighbours, therefore long-range correlations may be present. In addition, if there is an external magnetic field present, it will enhance this correlation further. Naturally, if $j>i$ it holds that 

$$
\sigma_j \sigma_i = \sigma_i \sigma_j = \sigma_i \sigma_{i+1} \sigma_{i+1} \cdots \sigma_{j-1} \sigma_j = \mathfrak{t}_{i}\mathfrak{t}_{i+1}\cdots\mathfrak{t}_{j-1} \textnormal{ where } \mathfrak{t}_i = \sigma_i _{i+1}, 
$$

then 

\begin{equation}
    \langle \sigma_j \sigma_i \rangle = \langle \mathfrak{t}_i \rangle \langle \mathfrak{t}_{i+1} \rangle \cdots \langle \mathfrak{t}_{j-1} \rangle.
\end{equation}

Note that the answer factorizes over $i$ since the Boltzmann weight factorizes over $i$ as well, when written in terms of the $\mathfrak{t}_i$\footnote{In effect, note that the partition function may be written as 

\begin{equation}
\begin{split}
    \mathcal{Z} &= \sum_{\sigma_i = \pm 1} \exp \bigg(\sum_{i=0}^{N-1} \beta J \sigma_i \sigma_{i+1} \bigg) = \sum_{\mathfrak{t}_i = \pm 1} \exp \bigg(\sum_{i=0}^{N-1} \beta J \mathfrak{t}_i \bigg) \\
    &= \sum_{\mathfrak{t}_i = \pm 1} \prod_{i=0}^{N-1} e^{\beta J \mathfrak{t}_i},
\end{split}
\end{equation}

where the exponential has naturally factorized into a product over $i$.}. Then, the thermal average for any one $\mathfrak{t}$ can be easily calculated as 

\begin{equation}
\begin{array}{ccc}
    \langle \mathfrak{t} \rangle = \frac{1}{\mathcal{Z}}\sum_{\mathfrak{t} =  \pm 1} \mathfrak{t} e^{E\{\mathfrak{t}\}}\\
     \langle \mathfrak{t} \rangle  = \frac{1 e^{\beta J} - 1 e^{-\cdot \beta J}}{e^{\beta J} +  e^{-\cdot \beta J}} = \tanh \beta J & \Rightarrow & \langle \sigma_j \sigma_i \rangle = (\tanh \beta J)^{|j-i|} = \exp \bigg(|j-i| \log \tanh \beta J\bigg).
\end{array}
\end{equation}

At any finite $\beta J$, since $\tanh \beta J < 1$, $\langle \sigma_j \sigma_i \rangle  \underset{|j-i| \rightarrow \infty}{\longrightarrow} 0$. Only at $T=0$ or $\beta J \rightarrow \infty$, when $\tanh \beta J = 1$, does the correlation not exponentially decay but remain flat. In other words, note that 

\begin{equation}
    \langle \sigma_j \sigma_i \rangle \sim e^{\frac{-|j-i|}{\zeta}} \textnormal{ where } \zeta = -\frac{1}{\log \tanh \beta J} \textnormal{ is the correlation length.}
\end{equation}

Thus, the one-dimensional Ising model is in a disordered phase for any non-zero temperature. The zero-temperature case is pathological, since there is no meaning of equilibrium in a zero-temperature classical system. 
\\

The $\langle \sigma_j \sigma_i \rangle$-correlator depends on just the difference in coordinates, ie. it presents translational invariance. This is not a generic result for a finite chain, but a peculiarity of this model in particular. In general, for an $N+1$-site chain, correlations between two spins will generally depend on where the two points are in relation to the ends. On the other hand, for $N\rightarrow\infty$-models, translational invariance is expected to hold for correlations of spins far from the ends. In order for translational invariance to hold for a finite system, periodic boundary conditions must be imposed on the system. In said case, correlation functions will now only depend on the difference between the two coordinates, but will not decay monotonically with separation since as one point start moving away from one extreme, it starts approaching the extreme spin from the other side. \\

\clearpage

\paragraph{\textbf{The Monte-Carlo method}}

The partition function and two-point functions may be numerically calculated with the Monte-Carlo method. Consider the two point function, 

\begin{equation}
    \langle \sigma_i \sigma_j \rangle_{\Lambda, \beta, h} = \frac{\sum_{{\bm \sigma} \in \Omega_{\Lambda}} \sigma_i \sigma_j e^{-\beta {\bf H}_{\Lambda, \beta, h}({\bm \sigma})}}{\sum_{{\bm \sigma} \in \Omega_{\Lambda}} e^{-\beta {\bf H}_{\Lambda, \beta, h}({\bm \sigma})}} = \sum_{{\bm \sigma} \in \Omega_{\Lambda}} \sigma_i \sigma_j({\bm \sigma}) \prob_{\Lambda, \beta, h}{({\bm \sigma})},
\end{equation}

where $\sigma_i \sigma_j({\bm \sigma})$ is the value of $\sigma_i \sigma_j$ at the ${\bm \sigma} \in \Omega_{\Lambda}$ configuration. The Monte-Carlo method is useful for numerically computing multidimensional integrals or sums. Each configuration will occur at a rate proportional to its Boltzmann weight. Then, for each configuration, the value of $\sigma_i \sigma_j$ can be computed. Since the configurations are already weighted, the arithmetic average of these numbers will yield the weighted thermal average. From this correlation function, the correlation length $\varepsilon$ can be calculated. By the same method, any other quantity of interest may be calculated eg. the magnetization. This can be done for any problem with a real, positive, Boltzmann weight, not just for Ising models. One specific algorithm for calculating these weighted configurations is the \textbf{Metropolis Method}, as follows \\

\begin{algorithm}[H]
 \KwData{System's parameters: $J, h, \beta$}
 \KwResult{}
 Start with some configuration ${\bm \sigma}_i \in \Omega_{\Lambda}$ with energy $E_i$;\
 \While{not at end of this document}{
  Consider another configuration ${\bm \sigma}_j \in \Omega_{\Lambda}$ with energy $E_j$, obtained by changing some degrees of freedom\;
  \eIf{ $E_j < E_i$}{
   jump to $j$ with unit probability\;
   }
   {{
    jump to $j$ with probability $e^{-\beta(E_j - E_i)}$
  }}
 }
 \caption{Metropolis Algorithm}
\end{algorithm}

After a great number of iterations, the configurations appear with probabilities obeying 

$$
    \frac{p(i)}{p(j)} = e^{-\beta (E_i - E_j)}.
$$

In effect, the rate equation reads 

\begin{equation*}
    \frac{dp(i)}{dt} = -p(i) \sum_{j} R(i\rightarrow j) + \sum_{j} p(j) R(j\rightarrow i),
\end{equation*}

where $R(i\rightarrow j)$ is the rate of jumping from the $i$-th configuration to the $j$-th configuration and vice-versa. In equilibrium or steady state, $\dot{p(i)} = 0$. One way to impose such constrain on the right-hand side is by considering a detailed balance approach, to

\clearpage

\subsection{Quantum Spin Systems}

A general spin chain is a dynamical system. In particular, an $N$-site chain, with finite $N$, has a configuration space, which is a discrete space or also known as a 0-manifold.

Consider a general spin chain as a dynamical system. Its spins can take any value in a given configuration space, which is a finite-dimensional, discrete space, also called as a 0-%or infinite-dimensional 
manifold $\mathcal{M}$. Therefore, assuming that $\mathcal{M}$ is a smooth manifold, the tangent space at $\ket{s}$ is denoted by $T_{\ket{s}} \mathcal{M}$. Then, the cotangent space at $\ket{s}$ is defined as the dual space of $T_{\ket{s}} \mathcal{M}$, 

\begin{equation}
    T_{\ket{s}}^* \mathcal{M} = (T_{\ket{s}} \mathcal{M})^*. \begin{array}{c}
         \textnormal{ Concretely, elements of the cotangent  }  \\
         \textnormal{ space are linear functionals on $T_{\ket{s}} \mathcal{M}$ } \\
         \textnormal{  that is, every element $\bra{s} \in T_{\ket{s}}^* \mathcal{M}$ is a linear map } 
    \end{array} 
\bra{s} : T_{\ket{s}} \mathcal{M} \rightarrow \mathds{C}.
\end{equation}

Then, for for each value $\ket{\mathfrak{s}} \in \mathcal{M}$, the "momentum" $\bra{\mathfrak{s}}$ of the system would take values in the cotangent space $T_s^* \mathcal{M}$ of that space. Thus, the phase space is naturally represented here by the cotangent bundle 

$$
T^* \mathcal{M} = \{(\ket{\mathfrak{s}}, \bra{\mathfrak{s}}) : \ket{\mathfrak{s}} \in \mathcal{M} \textnormal{ and } \bra{\mathfrak{s}} \in T_{\ket{s}}^* \mathcal{M}\}
$$

Thus the basic object classical statistical mechanics is the Boltzmann weight $\prob(n)$. In thermal equilibrium, this is a probability measure that gives the probability that a system will be in a certain configuration $n \in \mathcal{M}$, in terms of said configuration's energy and system's temperature. In other words,

\begin{equation}
    \begin{split}
        \prob : T^* \mathcal{M} \rightarrow \R_{[0,1]} \textnormal{ where }
        \prob{(n)} = \frac{e^{-\beta E_n}}{\mathcal{Z}},
    \end{split}
\end{equation}

where $\mathcal{Z}$ is the partition function, defined by the requirement that the probabilities sum to one, that is

$$
\mathcal{Z} = \sum_{n \in T^* \mathcal{M}} e^{-\beta E_n}.
$$

Note that if the degrees of freedom take on continuous values, or the model is defined in the continuum, then this sum is replaced by an integral. Note that, indeed, the probability of a given configuration increases as the energy gets lower, and conversely, that as the temperature gets higher and higher, the energies involved must get larger and larger to make a difference. Note as well that if all the energies are shifted by some constant $E_0$, the probabilities are left unchanged, since both the numerator and denominator are multiplied by the same constant, namely $e^{-\beta E_0}$. \\


The Hilbert space of a quantum spin is defined by choosing a representation for the spin operators. A representation of a Lie algebra is a set of three matrices satisfying the commutation relations of the $\mathfrak{su}(2)$-algebra. An \textbf{irreducible representation} is a set of matrices such that no unitary transformation $\mathcal{U} {\bf S}^{a} \mathcal{U}^\dagger$ block-diagonalizes all three matrices. It is known that for the $\mathfrak{su}(2)$-Lie algebra there is exactly one set (up to unitary transformations) of irreducible complex $n\times n$-matrices, for each integer $n$. It is customary to write $n=2s+1$ for all integers and half-integers $s$. A single spin-$s$ quantum particle at a fixed point in space therefore has a Hilbert space $\mathds{C}^{2s+1}$, so the matrices ${\bf S}^a$ are all $(2s+1)\times(2s+1)$. An orthonormal basis is given by the eigenstates of any one of the matrices. 

\begin{itemize}
    \item For $s=0$, the matrices all consist of the number zero, thus this is the trivial representation. 
    \item For $s = \frac{1}{2}$, the chosen basis is ${\bf S}^a = \frac{\hbar}{2} \sigma^a$, this is the fundamental representation.
    \item For $s=1$, the matrices can be written to have entries $({\bf S}^a)_{bc} = i \epsilon_{abc}$, yielding the adjoint representation.
\end{itemize}

In a given representation, an interesting invariant is given by the quadratic Casimir operator, 

$$
{\bf K} = {\bf S} \cdot {\bf S},
$$

which commutes with each of the representation's generators. As a result, it must be proportional to the identity in a given irreducible representation. This is a fundamental consequence of Schur's lemma on theory of representations 

\begin{lemma}Schur's lemma. 
    Let $\mathds{V}$ be a $\mathds{C}$-vector space, associated with a finite-dimensional irreducible representation of an algebra $\mathfrak{A}$ over $\mathds{C}$. Then, let $\phi : \mathds{V} \rightarrow \mathds{V}$ be a homorphism ie. $\phi(av) = a\phi(v), \forall a \in \mathfrak{A}, v \in \mathds{V}$. Then, $\phi = \lambda \mathds{1}_{\mathds{V}}.$ \\
\end{lemma}

\paragraph{\textbf{Spin chain types}}

Some of the most common examples of spin chains are treated as follows, 

\begin{itemize}
    \item The simplest example of a $SU(2)$-symmetric spin Hamiltonian is therefore the nearest-neighbor Heisenberg model, where 

\begin{equation}
    \Hamiltonian_{\mathds{X}\mathds{X}\mathds{X}} = -J \sum_{\langle ij \rangle} {\spin}_i \cdot {\spin}_j = -J \sum_{i=1}^{N} \bigg(\frac{1}{2} (\sigma_i^+ \sigma_{i+1}^{-} + \sigma_i^- \sigma_{i+1}^+) + \frac{1}{4} \sigma_i^z \sigma_{i+1}^z\bigg),
\end{equation}

where in the ferromagnetic case the diagonal terms in the Hamiltonian favour aligned spins, while in the antiferromagnetic case, the diagonal terms favour antialignement. The XXX-Hamiltonian commutes with the magnetization operator by construction. Note that Schur's lemma does not immediately apply since the magnetization operator is reducible. \\

\item The XXZ model is a deformation of the Heisenberg model, breaking the $SU(2)$-model symmetry down to a $U(1)$-subgroup. The XXZ-hamiltonian reads

\begin{equation}
    \Hamiltonian_{\mathds{X}\mathds{X}\mathds{Z}} = - \sum_{\langle jk \rangle} (J_\perp (\spin_j^x \spin_k^x + \spin_j^y \spin_k^y) + J_z \spin_j^z \spin_k^z) = - \sum_{j=1}^{N} \bigg( J_\perp (\sigma_{j}^+ \sigma_{j+1}^- + \sigma_{j}^- \sigma_{j+1}^+) + \frac{\Delta}{2} \sigma_j^z \sigma_{j+1}^z \bigg),
    \label{XXZ hamiltonian}
\end{equation}

which is the most general $U(1)$-symmetric nearest-neighbor interaction for spin-$\frac{1}{2}$ particles. \\

Although this model is no longer $SU(2)$-symmetric, the magnetization operator commutes with the hamiltonian, thus generating a $U(1)$-symmetry. Even though the full $SU(2)$-symmetry is broken, the degrees of freedom are still typically referred to as spins. The sign of $J_z$ alone determines whether the model is ferromagnetic, $J_z > 0$, or antiferromagnetic, $J_z < 0$, since the sign of $J_\perp$ is unimportant in any bipartite lattice, redefining the states by changing the overall sign of the $\spin^x$ and $\spin^y$-operators on half the lattice sites, leaves the algebra unchanged but flips the sign of $J_\perp$. Therefore, the physically meaningful coupling is $\Delta = \frac{J_z}{|J_\perp|}$, so that $\Delta = \pm 1$ for the ferromagnetic and antiferromagnetic spin models, respectively. In the $\Delta \rightarrow \pm \infty$-limit, only the $J_z$ remains, and the model is effectively classical. For the ferromagnetic case $J_z > 0$, all the spins simply line up with the maximum value of ${\bf M}^z$. In the antiferromagnetic case, $\Delta \rightarrow -\infty$ on a bipartite lattice, the spins take their maximum opposite values on every other site. \\

It is easy to check that this Hamiltonian commutes with the magnetization operator, and so preserves a $U(1)\times \mathds{Z}_2$-symmetry. In a classical notion, the $U(1)$-symmetry corresponds to rotations around the $z$-axis, while the $\mathds{Z}_2$ corresponds to flipping all the spins ${\spin_j^a \rightarrow -\spin_j^a}$. \\

\item A still more general Hamiltonian is given by the XYZ-model, which only preserves the spin-flip symmetry. 
\end{itemize}

None of these Hamiltonians correspond to a quantum version of the Ising model. \\

\paragraph{\textbf{Ferromagnets and antiferromagnets}}

Unless geometric frustration is present, in classical systems, there is no significant difference between ferromagnets and antiferromagnets. Eg, with NN-interaction, geometric frustration occurs for lattices that are not bipartite. In a bipartite lattice, the sites can be divided into two sub-lattices such that nearest neighbours always belong to different sub-lattices. A nearest-neighbour antiferromagnetic interaction in a classical model on a bipartite lattice can typically be changed into a ferromagnetic one, by redefining the spin via a flip on all sites on one of the sub-lattices but not on the other (eg. $\uparrow \rightarrow \downarrow$ in the Ising model). The physics of such classical antiferromagnets is therefore essentially equivalent to that of the ferromagnets. \\

Antiferromagnetic quantum systems on non-bipartite lattices also exhibit interesting behaviour. But the interesting thing is that on bipartite lattices, there are a number of important differences between quantum ferromagnets and antiferromagnets. \\

\paragraph{\textbf{Quantum Ferromagnets}}

Consider the Heisenberg interaction $-J \vec{\spin}_1 \cdot \vec{\spin}_2$ across a single bond. For spin-$\frac{1}{2}$ particles, this is a simple $4 \times 4$-matrix acting on the computational basis $\mathcal{B} = \{\ket{\uparrow\uparrow}, \ket{\uparrow \downarrow}, \ket{\downarrow \uparrow}, \ket{\downarrow\downarrow}\}$, yielding 

\begin{equation*} \vec{\spin}_1 \cdot \vec{\spin}_2 = \frac{1}{4} \left(\begin{array}{*{11}c}1 & 0 & 0 & 0 \\0 & -1 & 4 & 0 \\0 & 4 & -1 & 0\\0 & 0 & 0 & 1\\\end{array}\right)\end{equation*}

Diagonalizing this matrix will yield its eigenvalues and its eigenvectors. Given the Clebsh-Gordan decomposition rule $\frac{1}{2} \otimes \frac{1}{2} = 0 \oplus 1$, these eigenvectors can be grouped into the $s=1$-triplet representation and the $s=0$-singlet representation. In effect,

\begin{equation}
    \begin{split}
        \textnormal{triplet}: \ket{\uparrow \uparrow}, \frac{1}{\sqrt{2}} \bigg(\ket{\downarrow \uparrow} + \ket{\uparrow\downarrow}\bigg), \ket{\downarrow\downarrow}, \textnormal{ with } \lambda_i = -\frac{J}{4} \\
        \textnormal{singlet}: \frac{1}{\sqrt{2}} \bigg(\ket{\downarrow \uparrow} - \ket{\uparrow\downarrow}\bigg), \textnormal{ with } \mu_i = +\frac{J}{4}
    \end{split}
\end{equation}

It is simple to check that ${\bf K} = s(s+1)$ in both cases, and that while the singlet is annihilated by all three generators $\vec{\bf S}$, acting with ${\bf S}^+$ and ${\bf S}^-$ takes members of the triplet to each other. One important difference between the ferromagnetic ($J > 0$) and the antiferromagnetic ($J < 0$) models now arises,

\begin{itemize}
    \item if $J>0$, there are multiple ground states for the ferromagnet, each member of the triplet having minimum energy $-J$,
    \item but if $J<0$, the antiferromagnetic ground state, the singlet, is unique. Moreover, the antiferromagnetic ground state is invariant under the $SU(2)$-symmetry, whereas the ferromagnetic ground states are not. \\
\end{itemize}

Additional insight comes from solving the Heisenberg model on a four-site chain, with its Hilbert space being 16-dimensional, by exploiting the model's inner symmetries. The magnetization operator ${\bf M}^z$ commutes with the Hamiltonian, which makes them simultaneously diagonalizable. For perioidic boundary conditions, the translational invariance plays a powerful role. The translation operator is defined by shifting the spins by one site modulo $N$, this is 
\begin{equation}
\begin{split}
    \mathcal{T} : \mathfrak{A}_i \rightarrow \mathfrak{A}_j \\
    \mathcal{T}^{-1} \vec{\bf S}_i \mathcal{T} = \vec{\spin}_{i+1}
\end{split}
\begin{array}{c}
     \textnormal{ where } \mathfrak{A}_i, \mathfrak{A}_j \simeq \mathfrak{su}(2) \blanky \forall i,j. 
\end{array}
\end{equation}

The Hamiltonian commutes with the translation operator when the boundary conditions are periodic, and commutes with the magnetization as well. Thus the Hamiltonian can be broken into blocks acting on states with a fixed eigenvalue of the traslation operator and of the magnetization. \\

Since $\mathcal{T}^N = \mathds{1}$ for an $N$-site chain, the eigenvalues of $\mathcal{T}$ are $\lambda_i = e^{\frac{2\pi i n}{N}}, \forall n \in \mathds{Z}$, from which the corresponding momentum can be defined as $k = \frac{2\pi n}{N}$. Then, the following holds 
$$
    \sum_{n \in \mathds{Z}_{[0, N-1]}} e^{-\frac{2\pi i n}{N}} \mathcal{T}^n \ket{s} = \ket{s} + e^{-\frac{2\pi i}{N}} \ket{s} + \cdots + e^{-\frac{2\pi i (N-1)}{N}}\mathcal{T}^{N-1} \ket{s}.
$$

For example, for a four-site chain, the translational invariance diagonalizes the enire Hamiltonian save for $m=0$ and $k = 0$ or $k = \pi$, since the corresponding eigenstates are 

\begin{equation}
    \begin{split}
        \ket{A} = \frac{1}{\sqrt{2}} \bigg({\uparrow\downarrow\uparrow\downarrow} + e^{-i\pi k} {\downarrow\uparrow\downarrow\uparrow}\bigg), \\
        \ket{B} = \frac{1}{2}  \bigg(\ket{{\uparrow\uparrow\downarrow\downarrow}} + \ket{\downarrow \uparrow \uparrow \downarrow} + \cdots \bigg)
    \end{split} \begin{array}{c}
         \textnormal{with the hamiltonian} \\ \textnormal{ on these} \\ \textnormal{ $k=0$ and $k=\pi$} \\ \textnormal{states being }  
    \end{array}-J \left(\begin{array}{cc}
        -1 & \sqrt{2} \cos\left(\frac{\pi k}{2}\right)  \\
        \sqrt{2} \cos\left(\frac{\pi k}{2}\right) & 0
    \end{array}\right).
\end{equation}

These sector's eigenvalues are therefore $-J$ and $2J$ for $k=0$ and $J = 0$ for $k=\pi$. Again organising the eigenstates into $\mathfrak{su}(2)$-multiplets yield the energy levels divided by $-J$, as follows 

\begin{equation}
    \begin{split}
        & \textnormal{quintuplet}: 1 \\
        & \textnormal{triplets}: \cos(\pi k) (k \neq 0) \\
        & \textnormal{singlets}: -2,0 \textnormal{ for } k=0, \pi \textnormal{ respectively}. 
    \end{split}
\end{equation}

This $\mathfrak{su}(2)$-invariance allows for the construction of the entire multiplet once one of the states is known. \\

As with the two-site chain, the ferromagnetic ground state is a multiplet, whereas the antiferromagnetic is a singlet. For a general $N$-site ferromagnet, the completely ferromagnetic states (all spins up or all spins down) are exact ground states of the Hamiltonian. This suggest using an order parameter for ferromagnetism. 

 . \\

\clearpage

\subsection{XX-model}

Consider the XX-Heisenberg model, with its Hamiltonian given in terms of the traditional $\frac{1}{2}$-spin operators ie. 

\begin{equation}
    {\bf H} = J \sum_{i=1}^{L} ({\bf S}_{j}^{x} {\bf S}_{j+1}^{x} + {\bf S}_{j}^{y} {\bf S}_{j+1}^{y}) - \lambda \sum_{j=1}^{L} {\bf S}_{j}^{z},
    \label{XX hamiltonian}
\end{equation}

which describes interacting spins in a one-dimensional chain, with periodic boundary conditions. \eqref{XX hamiltonian}'s first terms represents nearest neighbour interactions in the $x$ and $y-$directions  interactions, with $J$ being either positive or negative and quantifying the strength and type of interactions, while the second term represents a magnetic field of strength $\lambda$, applied in the $z$-direction of the spins. \\

\paragraph{\textbf{Exact solution}}

In order to solve this problem, it is necessary to rewrite \eqref{XX hamiltonian} and apply a Jordan-Wigner transformation, mapping the spin problem into a fermionic problem. But first, it is convenient to write the spin-operators in terms of the raising and lowering $\mathfrak{su}(2)$-operators, ie.

$$
    \spin_j^{\pm} = \spin_j^{x} \pm i \spin_j^{y} \blanky \Rightarrow \blanky \begin{array}{c}
         \spin_{j}^x = \frac{1}{2} (\spin_j^+ + \spin_j^-), \\
          \\
         \spin_{j}^y = \frac{1}{2i} (\spin_j^+ - \spin_j^-). 
    \end{array}
$$

Then, the XX-Hamiltonian can be re-written as 

\begin{equation}
     {\bf H} = \frac{J}{2} \sum_{i=1}^{L} ({\bf S}_{j}^{+} {\bf S}_{j+1}^{-} + {\bf S}_{j}^{-} {\bf S}_{j+1}^{+}) - \lambda \sum_{j=1}^{L} {\bf S}_{j}^{z}
     \label{raising Hamiltonian}
\end{equation}

where the interacting terms in the first summation, flip neighboring spins if said spins are anti-aligned\footnote{In effect, consider for example, a two-spin problem. Then, the interaction term is given by 

$$
\spin_1^+ \spin_2^- + \spin_1^- \spin_2^+,
$$

and consider an anti-aligned state $\ket{\downarrow\uparrow}$. Then, the action of the previous two-spin operator over this state yields

$$
(\spin_1^+ \spin_2^- + \spin_1^- \spin_2^+) \ket{\downarrow\uparrow} = \ket{\uparrow\downarrow} + 0,
$$

since $\spin_1^- \spin_2^+$ destroys the state. Similarly, $(\spin_1^+ \spin_2^- + \spin_1^- \spin_2^+)\ket{\uparrow\downarrow} = \ket{\downarrow\uparrow}$. However, note that, should both spins be either up or down, the state remain invariant under the action of the two-spin operator. 

\begin{align}
    (\spin_1^+ \spin_2^- + \spin_1^- \spin_2^+) \ket{\downarrow\downarrow} = \ket{\downarrow\downarrow} \textnormal{ and } (\spin_1^+ \spin_2^- + \spin_1^- \spin_2^+) \ket{\uparrow\uparrow} = \ket{\uparrow\uparrow} 
\end{align}}. In addition, the XX-Hamiltonian has a total magnetization symmetry, since the Hamiltonian given by \eqref{raising Hamiltonian} conmutes with the magnetization operator. \\

Now, we use a Jordan-Wigner transformation whereby the spin operators are mapped to fermionic operators, as follows 

\begin{equation}
    \begin{array}{c}
         \spin_j^z = f_j^\dagger f_j - \frac{1}{2}  \\
         \\
         \spin_j^- = \exp\bigg(i\pi \sum_{\ell = 1}^{L-1} f_\ell^\dagger f_\ell \bigg) \\
         \\
         \spin_j^+ = \exp\bigg(-i\pi \sum_{\ell = 1}^{L-1} f_\ell^\dagger f_\ell \bigg) 
    \end{array}
    \label{JW tr}
\end{equation}

Under the Jordan-Wigner map, nearest-neighbours spin flipping is translated into to nearest-neighbours fermionic hopping, ie. $\spin_{j}^{+} \spin_{j+1}^{-} = f_{j}^{\dagger} f_{j+1}$ and $\spin_{j}^{-} \spin_{j+1}^{+} = f_{j+1}^{\dagger} f_{j}$. However, due to the boundary conditions' periodicity, the XX-Hamiltonian cannot be rewritten as a fermionic model yet since (it will contain an additional boundary term), for example, the fermionic counterparts to the $\spin_{L}^{+} \spin_{1}^{-}$ interaction are highly non-local operators and are not desirable. Indeed, under the Jordan-Wigner mapping 

\begin{equation*}
    \spin_{L}^{+} \spin_{1}^{-} = f_L^\dagger \exp\bigg(-i\pi \sum_{\ell = 1}^{L-1} f_\ell^\dagger f_\ell \bigg) f_1,
\end{equation*}

which is not problematic, since it accounts for all $L$-lattice sites. Let

\begin{equation}
    \begin{array}{c}
       \spin_{L}^{+} \spin_{1}^{-} = \mathcal{Q} f_L^\dagger f_1 \textnormal{  and  }
       \spin_{L}^{-} \spin_{1}^{+} = \mathcal{Q} f_1^\dagger f_L,
    \end{array}
\end{equation}

then \eqref{raising Hamiltonian} can be rewritten as 

\begin{equation}
    {\bf H} = \frac{J}{2} \sum_{i=1}^{L-1} \bigg(f_{j}^{\dagger} f_{j+1} + f_{j+1}^{\dagger} f_j\bigg) - \lambda \sum_{j=1}^{L} \bigg(f_{j}^{\dagger} f_{j} - \frac{1}{2}\bigg) + \frac{J}{2} \mathcal{Q} (f_L^\dagger f_1 + f_1^\dagger f_L),
    \label{fermionic no boundary}
\end{equation}

where the first term accounts for fermionic nearest-neighbour hopping, the second term accounts for the magnetic field, and the third term being the non-local boundary term. Note that this fermionic Hamiltonian hasn't got any type of boundary conditions, since the $L$-lattice site is disconnected in any way whatsoever from the first lattice site. Then, the standard procedure is to add and subtract terms from the Hamiltonian, so that the nearest-neighbour hopping term in \eqref{fermionic no boundary} can also have periodic boundary conditions, thus yielding 

\begin{equation}
    {\bf H} = \frac{J}{2} \sum_{i=1}^{L} \bigg(f_{j}^{\dagger} f_{j+1} + f_{j+1}^{\dagger} f_j\bigg) - \lambda \sum_{j=1}^{L} \bigg(f_{j}^{\dagger} f_{j} - \frac{1}{2}\bigg) + \frac{J}{2} (\mathcal{Q}-1) (f_L^\dagger f_1 + f_1^\dagger f_L),
    \label{true fermionic hamiltonian}
\end{equation}

where now the fermionic hopping term has the standard boundary conditions. The third term, since it does not involve any type of summation over lattice sites, only contributes at $\mathcal{O}\bigg(\frac{1}{L}\bigg)$-order to any microscopic quantity. In the thermodynamic limit, this non-local term can be dropped, thus yielding an $\mathcal{O}(L)$-Hamiltonian given by 
\begin{equation}
    {\bf H} = \frac{J}{2} \sum_{i=1}^{L} \bigg(f_{j}^{\dagger} f_{j+1} + f_{j+1}^{\dagger} f_j - \lambda f_{j}^{\dagger} f_{j}\bigg) + \frac{\lambda L}{2},
    \label{L-order fermionic hamiltonian}
\end{equation}

which is now fully cyclic and where its operators obey fermionic algebras. This Hamiltonian can then be diagonalized via a discrete Fourier transform on the fermionic operators 

\begin{align}
    f_j &= \frac{1}{\sqrt{L}} \sum_{\substack{k = {2\pi m/L}\\
    m \in \mathds{Z}_{[1, L]} }}
                  e^{ijk} d_k, & f_j^{\dagger} &=\frac{1}{\sqrt{L}}  \sum_{\substack{k = {2\pi m/L}\\
    m \in \mathds{Z}_{[1, L]}}} e^{-ijk} d_k^{\dagger},
\end{align}

with the inverse transformation given by 

\begin{align}
    d_k &= \frac{1}{\sqrt{L}} \sum_{j=1}^{L}
                  e^{-ikj} f_j &  d_k^{\dagger} &= \frac{1}{\sqrt{L}} \sum_{j=1}^{L}
                  e^{ikj} f_j^{\dagger}.
\end{align}

Note that the $d_k$-operators follow the standard fermionic anticonmutation algebra. Note as well, that the $f_j$-vacuum state, defined such that $f_j \ket{0}_f = 0, \blanky \forall j$, is the same as the $d_k$-vacuum state, defined such that $d_j \ket{0}_d = 0, \blanky \forall k$, ie. $\ket{0}_f = \ket{0}_d$. Another important relationship is the Fourier transform's consistency condition, ie. 

$$
\sum_{j=1}^{L} e^{i(k-q)j} = L \delta_{kq}.
$$

Under the Fourier transform, \eqref{L-order fermionic hamiltonian}'s terms are mapped as follows 

\begin{align*} 
    \sum_{j=1}^{L} f_j^\dagger f_{j+1} &= \sum_{j=1}^{L} \frac{1}{L} \sum_{k, \blanky q} e^{-ikj} e^{iq(j+1)} d_k^\dagger d_q = \sum_{j=1}^{L} \frac{1}{L} \sum_{k, \blanky q} e^{i(q-k)j} e^{iq} d_k^\dagger d_q \\
    &= \sum_{k, \blanky q} \frac{1}{L} e^{iq} \delta_{qk} d_k^\dagger d_q = \sum_{k} e^{ik} d_k^\dagger d_k \\
    \sum_{j=1}^{L} f_{j+1}^\dagger f_{j} &=  \sum_{j=1}^{L} \frac{1}{L} \sum_{k, \blanky q} e^{-ik(j+1)} e^{iqj} d_k^\dagger d_q = \sum_{j=1}^{L} \frac{1}{L} \sum_{k, \blanky q} e^{i(q-k)j} e^{-ik} d_k^\dagger d_q \\
    &= \sum_{k, \blanky q} \frac{1}{L} e^{-ik} \delta_{qk} d_k^\dagger d_q = \sum_{k} e^{-ik} d_k^\dagger d_k \\
    \sum_{j=1}^{L} f_{j}^\dagger f_{j} &=  \sum_{j=1}^{L} \frac{1}{L} \sum_{k, \blanky q} e^{-ikj} e^{iqj} d_k^\dagger d_q = \sum_{j=1}^{L} \frac{1}{L} \sum_{k, \blanky q} e^{i(q-k)j}  d_k^\dagger d_q \\
    &= \sum_{k, \blanky q} \frac{1}{L} \delta_{qk} d_k^\dagger d_q = \sum_{k} d_k^\dagger d_k \\
\end{align*} 

Therefore, using these identities, the new Hamiltonian is given by 

\begin{equation}
\begin{split}
    {\bf H} &= \frac{J}{2} \sum_{k} \bigg(e^{ik} d_{k}^{\dagger} d_{k} + e^{-ik} d_{k}^{\dagger} d_k - \lambda d_{k}^{\dagger} d_{k}\bigg) + \frac{\lambda L}{2} \\
    &= \sum_{k} \bigg(J \cos k - \lambda \bigg)d_{k}^{\dagger} d_{k} + \frac{\lambda L}{2},
\end{split}
\end{equation}

which can be rewritten as 

\begin{align}
    \alignedbox{{\bf H} }{= \sum_{k} \epsilon_k d_k^\dagger d_{k} + \frac{\lambda L}{2} \begin{array}{c}
         \textnormal{ with the eigenvalues being given by } \epsilon_k = J \cos k - \lambda + \frac{\lambda L}{2} \\
         \textnormal{ and the eigenvectors being given by } \ket{E_n} = \prod_{n}  (d_k^\dagger)^n, \\
         \textnormal{                    with eigenvalue $E_n = \sum_{n} \epsilon_n $}  
    \end{array}}
\end{align}

where, $k = \frac{2\pi m}{L}, \blanky m = -\frac{L}{2} + 1, \cdots \frac{L}{2}, $ ie. $k \in (-\pi, \pi]$, this is called the first Brillouin zone. Thus the problem has been solved \\

As for its thermal properties, since this is a fermionic model, the fermions will obey the Fermi-Dirac distribution, ie. 

\begin{equation}
    \mathcal{N}_{jk} = \langle d_j^\dagger d_k \rangle_{\textnormal{th}} =  \frac{1}{1+e^{\beta \epsilon_k + \mu}} \delta_{jk} 
\end{equation}

\clearpage

\subsubsection{Quantum Phase Transition}

The system's groundstate is defined such that 

\begin{equation}
    \bra{E_G} d_k^\dagger d_k \ket{E_G} = 1 \textnormal{ if } \epsilon_k \leq 0 \textnormal{ and } \bra{E_G} d_k^\dagger d_k \ket{E_G} = 0 \textnormal{ if } \epsilon_k > 0.
\end{equation}

Then the groundstate energy is given by 

\begin{equation}
    E_G = \sum_{k} \epsilon_k \bra{E_G} d_k^\dagger d_k \ket{E_G}.
\end{equation}

Now, there exists a momenta $k_F$ such that $\epsilon_{k_F} = 0$, with $k_F$ being the Fermi momentum, ie. $\epsilon_{k_f} = J \cos k_f - \lambda$. The previous equation depends both on the coupling strength $J$ and the magnetic field strength $\lambda$. From the previous equation some special cases arise, namely 

\begin{itemize}
    \item when $\lambda < -J$, which yields a Fermi momentum $k_F = \pi$,
    \item when $|\lambda| \leq J$, which yields a Fermi momentum $k_F = \arccos \bigg(\frac{\lambda}{J}\bigg)$,
    \item when $\lambda > J$ which yields $k_F = 0$.
\end{itemize}

Therefore, the system groundstate will be defined by the following relations, 

\begin{equation}
    \begin{array}{c}
         \bra{E_G} d_k^\dagger d_k \ket{E_G} = 1  \\
         \\
         \textnormal{ if } k \in [-\pi, -k_F] \cup [k_F, \pi],
    \end{array} \textnormal{ and }  \begin{array}{c}
         \bra{E_G} d_k^\dagger d_k \ket{E_G} = 0  \\
         \\
         \textnormal{ if } k \in [-k_F, k_F].
    \end{array}
\end{equation}

\blanky \\

The total magnetization operator is defined as the sum of the $\spin_z$ operators, along the entire chain ie. ${\bf M}^z = \sum_{j=1} \spin^z_j$. Under the Jordan-Wigner transformation, $\spin^z_j = f_j^\dagger f_j - \frac{1}{2}$. In turn, these fermionic number operators can be mapped to fermionic number operators acting on momentum space, ie. $\sum_{j=1}^{L} f_j^\dagger f_j = \sum_k d_k^\dagger d_k$. This entails

\begin{equation}
    {\bf M}^z = \sum_{k} \bigg(d_k^\dagger d_k - \frac{1}{2}\bigg).
\end{equation}

Then the total magnetization operator's expectation value for the groundstate can be calculated. Given the system's fermionic nature, momentum eigenmodes are either occupied or unoccupied, contributing only one or zeros respectively. Then, this expectation value yields

\begin{equation}
    \bra{E_G} {{\bf M}^z} \ket{E_G} = \sum_{k=-\pi}^{-k_F} 1 + \sum_{k=k_F}^{\pi} 1 - \frac{1}{2} \sum_{k=-\pi}^{\pi} 1 \\
    = \frac{1}{2} \bigg( \sum_{k=-\pi}^{-k_F} 1 + \sum_{k=k_F}^{\pi} 1 - \sum_{k = -k_F}^{k_F} 1  \bigg), \label{total magnetization ev}
\end{equation}

where the total magnetization expectation value is just half the difference of two competing summations, related to the negative and positive eigenmodes. Defining the total magnetization per site operator, ${{\bf m}}^z = \frac{1}{L} {\bf M}^z$. Now, as the chain gets progressively bigger, with more and more lattice sites, the summations in \eqref{total magnetization ev} can be approximated by Riemannian integrals, ie.

\begin{equation*}
    \sum_{k} f(k) = \frac{1}{\Delta k} \sum_{k} f(k) \Delta k \overset{L \rightarrow \infty}{\rightarrow} \frac{L}{2\pi}\int_{k \in \mathds{B}\mathds{Z}} dk f(k) \textnormal{ where } \Delta k = \frac{2\pi}{L}.
\end{equation*}

Using this trick, \eqref{total magnetization ev} can be rewritten as 

\begin{align}
    \langle {{\bf m}}^{z} \rangle_{G} &= \frac{1}{4\pi} \bigg(\int_{-\pi}^{-k_F} dk + \int_{k_F}^{\pi} dk - \int_{-k_F}^{k_F} dk \bigg) = \frac{1}{4\pi} \bigg(-k_F + \pi + \pi - k_F - k_F + k_F\bigg) \\
    &= \frac{1}{2} - \frac{k_F}{\pi},
\end{align}

which yields the total magnetization per lattice site. Note that at $\lambda < J$, $k_F = \pi$ whereby there is total polarization ie. $\langle {{\bf m}}^{z} \rangle_{G} = -\frac{1}{2}$. Similarly, for $\lambda > J$, $k_F = 0$ and then $\langle {{\bf m}}^{z} \rangle_{G} = \frac{1}{2}$. However, if $|\lambda| \leq J$, then $k_F = \arccos\bigg(\frac{\lambda}{J}\bigg)$, which then yields a total magnetization per lattice site given by 
\begin{equation} \langle {{\bf m}}^{z} \rangle_{G}  = \left\{
    \begin{array}{cc}
          \frac{1}{2} - \frac{1}{\pi}\arccos\bigg(\frac{\lambda}{J}\bigg)&  \textnormal{ if } |\lambda| \leq J  \\
          \\
          -\frac{1}{2} & \textnormal{ if } \lambda < J  \\
          \\
          \frac{1}{2} & \textnormal{ if } \lambda > J  \\
    \end{array} \right.
\end{equation}

Another interesting quantity is the total magnetic susceptibility, which can be written in terms of the total magnetization per lattice site as follows 

\begin{align}
    \chi = \bigg(\frac{\partial \langle {{\bf m}}^{z} \rangle_{G}}{\partial \lambda}\bigg) \Rightarrow \chi = \left\{ \begin{array}{cc} 
         \frac{1}{J \pi \sqrt{1- \frac{\lambda^2}{J^2}}} \textnormal{ if } |\lambda| \leq J  \\
         \\
          0 \textnormal{ if } \lambda < J \\ 
         \\
          0 \textnormal{ if } \lambda > J  \\
              \end{array} \right., \label{}
\end{align}

which is divergent if $\lambda = J$. This shows that the XX-Heisenberg model has a second order (since there is no latent heat involved it can't be a first order phase transition) quantum phase transition, with an associated power law with a critical exponent.  \\

%\subsection{XXZ-model}

\clearpage

\subsection{Bethe ansatz}

\paragraph{\textbf{Heisenberg XXX-chain}}

Consider a one-dimensional lattice with $N$ lattice sites and a $\frac{1}{2}$-spin particle positioned at every lattice site, which have a nearest neighbour spin-spin interaction. Each particles either has spin up or spin down, generating a two-dimensional local Hilbert space $\vds_n$. Strictly speaking there are two possible topologies for a one-dimensional chain, open or closed. Consider the XXX-Hamiltonian with closed topology, given by 

\begin{equation}
\begin{split}
    {\bf H} &= J \sum_{n=1}^{N} \bigg(\vec{\spin}_n \cdot \vec{\spin}_{n+1} - \frac{1}{4} \mathds{1}^{\otimes N} \bigg) \\
    &= \frac{J}{4}\sum_{n=1}^{N} \bigg(\vec{\sigma}_n \cdot \vec{\sigma}_{n+1} - \frac{1}{4} \mathds{1}^{\otimes N} \bigg).
    \label{XXX model}
\end{split}
\end{equation}

Note that $J < 0$ models a ferromagnetic state while $J > 0$ models an anti-ferromagnetic state. A ferromagnetic state refers to a state where all the spins are alligned while an anti-ferromagnetic state refers to a state where adjacent spins are anti-aligned within the domain. 


The main point of the Bethe solution is that XXX Heisenberg Hamiltonian, which defines the model, can be expressed in terms of a complex-valued $\bm{\mathcal R}$-matrix which is a solution of the Yang-Baxter equation, thus leading to the model's integrability. The most convenient approach is to construct a transfer $\bm{\mathcal{T}}$-matrix -a one parameter commutative family of operators acting on the full state space of the Heisenberg spin chain. \\

\paragraph{\textbf{Some fundamentals of Matrix Theory}}

Let $\mathds{V}$ be an $n$-dimensional $\mathds{C}$-vector space and let $k \in \mathds{1}_n \textnormal{ such that } \in \mathds{N}_{[0, k]}$. Let $\mathfrak{J} = \mathds{1}_n / \mathds{1}_k = \{k+1,\cdots, n\}$. Let $A \in \textnormal{GL}(n, \mathds{C})$, which can be written in block notation as 
\begin{equation}
    A = \left(\begin{array}{cc}
        A_{\mathcal{I}} & A_{\mathcal{I} \mathfrak{J}}  \\
        A_{\mathfrak{J} \mathcal{I}} & A_{\mathfrak{J}}
    \end{array}\right) \begin{array}{c}
         \mathds{C}^k  \\
         \mathds{C}^{n-k}
    \end{array} \textnormal{ where } \begin{array}{cc}
        A_{\mathcal{I}} \in \textnormal{GL}(k, \mathds{C}),  &  A_{\mathcal{I} \mathfrak{J}} \in \textnormal{GL}(k \rightarrow n-k, \mathds{C}), \\
        A_{\mathfrak{J} \mathcal{I}} \in  \textnormal{GL}(n-k \rightarrow k, \mathds{C}), & A_{\mathfrak{J}}  \in \textnormal{GL}(n-k, \mathds{C}),
    \end{array} 
\end{equation}

This notation is consistent with the common matrix operations. The same idea holds for writing down a  block matrix representation for a linear operator with respect to a given subspace. Let $\mathcal{S} \subset \mathds{C}^n$ be a $k$-dimensional vector subspace, such that $\mathds{C}^n = \mathcal{S} \oplus \mathcal{S}^{\perp}$. Then, every linear operator in $\textnormal{End}(\mathds{C}^n)$ can be rewritten as a $\twobytwo$-block-matrix, with its entries being matrices of a given dimension ie. \footnote{
For example, consider the orthogonal projector over $\mathcal{S}$, labelled $P_{\mathcal{S}}$. In this notation, its matrix representation is given by 

\begin{equation*}
    P_{\mathcal{S}} = \left( \begin{array}{cc}
       \mathcal{I} & 0 \\
       0 & 0
    \end{array} \right) \begin{array}{c}
         \mathcal{S} \\
         \mathcal{S}^\perp
    \end{array} 
\end{equation*}}

\begin{equation*}
\begin{split}
    A \in \textnormal{End}(\mathds{C}^n) &\rightarrow \exists! \left\{\begin{array}{cc}
        A_{\mathcal{S}} \in \textnormal{GL}(k, \mathds{C}),  &  A_{\mathcal{S}, \mathcal{S}^\perp} \in \textnormal{GL}(k \rightarrow n-k, \mathds{C}), \\
        A_{\mathcal{S}^\perp, \mathcal{S}}  \in  \textnormal{GL}(n-k \rightarrow k, \mathds{C}), & A_{\mathcal{S}^\perp} \in \textnormal{GL}(n-k, \mathds{C}),
    \end{array} \right\} / \\
   &A = \left( \begin{array}{cc}
       A_{\mathcal{S}} & A_{\mathcal{S}, \mathcal{S}^\perp}  \\
       A_{\mathcal{S}^\perp, \mathcal{S}} & A_{\mathcal{S}^\perp} 
    \end{array} \right) \begin{array}{c}
         \mathcal{S} \\
         \mathcal{S}^\perp
    \end{array}
    \end{split}
\end{equation*}

\textnormal{ } \\

\paragraph{\textbf{The Lax operator}} Consider an XXX spin chain with $N$ sites and a corresponding Hilbert space given by 

\begin{align}
    \mathds{H} \simeq \bigotimes_{n=1}^{N} {\mathds{V}_n} \textnormal{ where } {\mathds{V}_n} \simeq \ctwo \begin{array}{c}
         \textnormal{To these individual Hilbert spaces, additional}  \\
         \textnormal{auxiliary, non-physical, spaces $\mathds{V}_a \simeq \ctwo$, can be added. } \\
    \end{array}
\end{align}

The algebraic Bethe ansatz' basic tool is the lax operator $\bm{\mathcal{L}}$, whose definition is as follows. 

\begin{df}
   The lax operator $\bm{\mathcal{L}}$ can be defined as an operator which involves the local quantum space $\mathds{V}_n$ and the auxiliary space $\mathds{V}_a$, as follows 

\begin{equation}
    \begin{array}{c}
         \bm{\mathcal{L}} : \mathds{V}_n \otimes \mathds{V}_a \rightarrow \mathds{V}_n \otimes \mathds{V}_a \\
         \\
         \bm{\mathcal{L}}_{n, a} = u(\mathds{1}_n \otimes \mathds{1}_a) + i \sum_{\alpha} \spin_n^\alpha \otimes \sigma_{a}^\alpha, \blanky \forall n \leq N \\
         %= u^{\mu} {S}_{\nu} \sigma^{\nu} \textnormal{ where $u^{\mu} = (u, i,i,i)$}
    \end{array} \begin{array}{c}
         \textnormal{ where $\mathds{1}_n$ and $\spin_n^\alpha$ act on $\mathds{V}_n $} \\
         \\
         \textnormal{ while $\mathds{1}_a$ and $\spin_a^\alpha$ act upon $\mathds{V}_a$ and where the}\\
         \\
         \textnormal{ complex-valued $u$-parameter is the spectral parameter.}
    \end{array}
    \label{lax operator}
\end{equation}
\end{df}

 Note that \cref{lax operator} can be rewritten as 

\begin{equation}
    \begin{split}
    \bm{\mathcal{L}}_{n, a} &= u(\mathds{1}_n \otimes \mathds{1}_a) + i \sum_{\alpha} \spin_n^\alpha \otimes \sigma_{a}^\alpha \\
    %&= u \left(\begin{array}{cc}
    %    \mathds{1}_n & 0  \\
    %    0 & \mathds{1}_a 
    %\end{array}\right) + i \spin^{x}_{n} \otimes \left(\begin{array}{cc}
    %    0 & 1 \\
    %    1 & 0
    %\end{array}\right)_a + i \spin^{y}_{n} \otimes \left(\begin{array}{cc}
    %    0 & -i \\
    %    i & 0 
    %\end{array}\right)_a + i \spin^{z}_{n} \otimes \left(\begin{array}{cc}
    %    1 & 0  \\
    %    0 & -1
    %\end{array}\right)_a \\
    &= \left(\begin{array}{cc}
       u \mathds{1}_n + i \spin^{z}_{n} & i \spin^{x}_{n} + \spin^{y}_{n} \\
       i \spin^{x}_{n} - \spin^{y}_{n}  & u \mathds{1}_n - i \spin^{z}_{n}
    \end{array}\right)_a = \left(\begin{array}{cc}
       u \mathds{1}_n + i \spin^{z}_{n} & i \spin^{-}_{n} \\
       i \spin^{+}_{n} & u \mathds{1}_n - i \spin^{z}_{n}
    \end{array}\right)_a,
    \end{split}
\end{equation}
    
which is a $2 \times 2$-matrix in the auxiliary space $\mathds{V}_a$ and the matrix entries are operators acting on the physical Hilbert space $\mathds{V}_n$. \\

Consider now the following mathematical object. 

\begin{df}
    Let the permutation operator be
    
    \begin{equation}
        \bm{\mathcal P}_{n,a} = \frac{1}{2} \bigg(\mathds{1}_n \otimes \mathds{1}_a + \sum_{\alpha}\sigma_n^{\alpha}  \otimes \sigma_n^{\alpha} \bigg) = \frac{1}{2} \sigma^{n \mu}\sigma^{\alpha}_{\mu}
        \label{permutation op}
    \end{equation}
    
    which \footnote{Note that the permutation operator can be written as a $2 \times 2$-matrix in the auxiliary space $\mathds{V}_a$,
\begin{equation} 
\begin{split}
    \bm{\mathcal P}_{n,a} &= \frac{1}{2} \left[ \left(\begin{array}{cc}
        \mathds{1}_n & 0  \\
        0 & \mathds{1}_a 
    \end{array}\right) + \sigma^{x}_{n} \otimes \left(\begin{array}{cc}
        0 & 1 \\
        1 & 0
    \end{array}\right)_a + \sigma^{y}_{n} \otimes \left(\begin{array}{cc}
        0 & -i \\
        i & 0 
    \end{array}\right)_a + \sigma^{z}_{n} \otimes \left(\begin{array}{cc}
        1 & 0  \\
        0 & -1
    \end{array}\right)_a \right] \\
    &= \left(\begin{array}{cc}
       \mathds{1}_n + \sigma^{z}_{n} &  \sigma^{x}_{n} - i\sigma^{y}_{n} \\
       \sigma^{x}_{n} + i\sigma^{y}_{n}  & \mathds{1}_n - \sigma^{z}_{n}
    \end{array}\right)_a = \left(\begin{array}{cccc}
        1 & 0 & 0 & 0   \\
        0 & 0 & 1 & 0   \\
        0 & 1 & 0 & 0   \\
        0 & 0 & 0 & 1
    \end{array}\right)
    \end{split}
\end{equation}} acts on both the physical and auxiliary systems.
\end{df}

Using this definition for the permutation operator, the lax operator can then be rewritten as

\begin{equation}
    \begin{split}
       \bm{\mathcal{L}}_{n, a} &= u(\mathds{1}_n \otimes \mathds{1}_a) + i \sum_{\alpha} \spin_n^\alpha \otimes \sigma_{a}^\alpha = u(\mathds{1}_n \otimes \mathds{1}_a) + \frac{i}{2} \sum_{\alpha} \sigma_n^\alpha \otimes \sigma_{a}^\alpha \\
       &= u(\mathds{1}_n \otimes \mathds{1}_a) - \frac{i}{2} (\mathds{1}_n \otimes \mathds{1}_a) + \frac{i}{2} (\mathds{1}_n \otimes \mathds{1}_a) + \frac{i}{2} \sum_{\alpha} \sigma_n^\alpha \otimes \sigma_{a}^\alpha \\
       &= \bigg(u-\frac{i}{2}\bigg) (\mathds{1}_n \otimes \mathds{1}_a) + \frac{i}{2} \bigg( \mathds{1}_n \otimes \mathds{1}_a + \sum_{\alpha}\sigma_n^{\alpha}  \otimes \sigma_n^{\alpha} \bigg) \\
       &= \bigg(u-\frac{i}{2}\bigg) \mathds{1}_{n, a} + i \bm{\mathcal{P}}_{n, a}.
       \end{split}
       \label{l-matrix def}
\end{equation}

\blanky \\

\paragraph{\textbf{Fundamental commutation Relations and Monodromy Matrix}}

Consider the following two lax operators

\begin{equation*}
    \begin{array}{c}
        \bl_{n, a_1}(u_1): \blanky \vds_n \otimes \vds_{a_1} \rightarrow \vds_n \otimes \vds_{a_1}  \\
        \\
        \bl_{n, a_2}(u_2): \blanky \vds_n \otimes \vds_{a_2} \rightarrow \vds_n \otimes \vds_{a_2} 
    \end{array} \begin{array}{c}
         \textnormal{both of which act on the physical quantum}\\
         \textnormal{state and on two different auxiliary spaces as well}\footnote{
The permutation operator indeed permutes the factors in the tensor product the factors in the tensor product by calculating the sum of tensor product of the Pauli matrices with respect to the standard four-dimensional basis, ie. if $\x = \left(\begin{array}{c}
     a \\
     b 
\end{array}\right)$ and ${\bf y} = \left(\begin{array}{c}
     c \\
     d 
\end{array}\right)$ then $\x \otimes {\bf y} = \left( \begin{array}{cc}
    ac & ad  \\
    bc & bd 
\end{array} \right)$ and  ${\bf y} \otimes \x = \left( \begin{array}{cc}
    ac & bc  \\
    ad & bd 
\end{array} \right)$. Therefore,

\begin{equation}
    \bm{\mathcal{P}}_{n, a} (\x \otimes {\bf y} ) = {\bf y} \otimes \x \textnormal{ and } \bm{\mathcal{P}}_{n, a} ( {\bf y} \otimes \x) = \x \otimes {\bf y}.
\end{equation}}
. 
    \end{array}
\end{equation*}

The product of these two operator is then a triple tensor product in $\mathds{V}_{n} \otimes \mathds{V}_{a_1} \otimes \mathds{V}_{a_2}$. Now, the following matrix arises 

\begin{df}
Consider an $\bm{\mathcal{R}}$-matrix which relates the permutation and spectral parameters of two auxiliary spaces, which is given by  

\begin{equation}
\begin{array}{c}
     \bm{\mathcal{R}}: \blanky \mathds{V}_{a_1} \otimes \mathds{V}_{a_2} \otimes \mathds{V}_{a_1} \otimes \mathds{V}_{a_2} \\ 
     \\
     \bm{\mathcal{R}}_{a_1, a_2}(u_1-u_2) = (u_1-u_2) \mathds{1}_{a_1, a_2} + i \bm{\mathcal{P}}_{a_1, a_2}, \\
     \label{R-matrix def}
\end{array}
\end{equation}
 
which\footnote{Note that $\bm{\mathcal{R}}$ can be rewritten as a $\twobytwo$-matrix in either of the auxiliary spaces, as follows

\begin{equation}
    \bm{\mathcal{R}}_{a1, a2} = \left(\begin{array}{cc}
        \bigg(u + \frac{i}{2}\bigg) \mathds{1}_{a_2} + i \spin_{a_2}^z & i \spin_{a_2}^{-} \\
        i \spin_{a_2}^{+}  & \bigg(u + \frac{i}{2}\bigg) \mathds{1}_{a_2} - i \spin_{a_2}^z 
    \end{array} \right)_{a_1} = \left(\begin{array}{cc}
        \bigg(u + \frac{i}{2}\bigg) \mathds{1}_{a_1} + i \spin_{a_1}^z & i \spin_{a_1}^{-} \\
        i \spin_{a_1}^{+}  & \bigg(u + \frac{i}{2}\bigg) \mathds{1}_{a_1} - i \spin_{a_2}^z 
    \end{array} \right)_{a_2}.
\end{equation}} doesn't act on the physical system at all, but on the auxiliary, phantom, systems.
\end{df}


An interesting relationship between the products of the two lax operators and the $\bm{\mathcal{R}}$-matrix can then be found\footnote{Note that the following permutation properties hold 

\begin{equation}
\bm{\mathcal{P}}_{n, a_1}\bm{\mathcal{P}}_{n, a_2} = \bm{\mathcal{P}}_{a_1, a_2}\bm{\mathcal{P}}_{n, a_1} = \bm{\mathcal{P}}_{n, a_2}\bm{\mathcal{P}}_{a_2, a_1} \textnormal{ and } \bm{\mathcal{P}}_{a, b} = \bm{\mathcal{P}}_{b, a}
\end{equation}

}, consider 

\begin{align*}
    \bm{\mathcal{R}}_{a_1, a_2}(u_1-u_2) & \bm{\mathcal{L}}_{n, a_1} (u_1) \bm{\mathcal{L}}_{n, a_2} (u_2) \\ 
    &= {\bigg((u_1-u_2) \mathds{1}_{a_1, a_2} 
    + i \bm{\mathcal{P}}_{a_1, a_2}\bigg)}   \left(\bigg(u-\frac{i}{2}\bigg) \mathds{1}_{n, a_1} + i \bm{\mathcal{P}}_{n, a_1}\right) \left(\bigg(u-\frac{i}{2}\bigg) \mathds{1}_{n, a_2} + i \bm{\mathcal{P}}_{n, a_2}\right) \\
    &= \left({(u_1-u_2) \mathds{1}_{a_1, a_2} \bigg(u-\frac{i}{2}\bigg) \mathds{1}_{n, a_1}} + {(u_1-u_2) \mathds{1}_{a_1, a_2} i \bm{\mathcal{P}}_{n, a_1}} + i \bm{\mathcal{P}}_{a_1, a_2} \bigg(u-\frac{i}{2}\bigg) \mathds{1}_{n, a_1} + i^2 \bm{\mathcal{P}}_{a_1, a_2}  \bm{\mathcal{P}}_{n, a_1} \right) \\
    & \times \left(\bigg(u-\frac{i}{2}\bigg) \mathds{1}_{n, a_2} + i \bm{\mathcal{P}}_{n, a_2}\right) \\
    &= {(u_1-u_2) \mathds{1}_{a_1, a_2} \bigg(u-\frac{i}{2}\bigg) \mathds{1}_{n, a_1}} \bigg(u-\frac{i}{2}\bigg) \mathds{1}_{n, a_2} + {(u_1-u_2) \mathds{1}_{a_1, a_2} \bigg(u-\frac{i}{2}\bigg) \mathds{1}_{n, a_1}} i \bm{\mathcal{P}}_{n, a_2} \\
    &+ {(u_1-u_2) \mathds{1}_{a_1, a_2} i \bm{\mathcal{P}}_{n, a_1}} \bigg(u-\frac{i}{2}\bigg) \mathds{1}_{n, a_2} + {(u_1-u_2) \mathds{1}_{a_1, a_2} i \bm{\mathcal{P}}_{n, a_1}} i \bm{\mathcal{P}}_{n, a_2} \\ 
    &+ i \bm{\mathcal{P}}_{a_1, a_2} \bigg(u-\frac{i}{2}\bigg)\mathds{1}_{n, a_1}  \bigg(u-\frac{i}{2}\bigg) \mathds{1}_{n, a_2} + i \bm{\mathcal{P}}_{a_1, a_2} \bigg(u-\frac{i}{2}\bigg) \mathds{1}_{n, a_1} i \bm{\mathcal{P}}_{n, a_2} \\
    & - \bm{\mathcal{P}}_{a_1, a_2}  \bm{\mathcal{P}}_{n, a_1}  \bigg(u-\frac{i}{2}\bigg) \mathds{1}_{n, a_2} - \bm{\mathcal{P}}_{a_1, a_2}  \bm{\mathcal{P}}_{n, a_1} i \bm{\mathcal{P}}_{n, a_2} \\
    & = {(u_1-u_2) \mathds{1}_{a_1, a_2}  \bigg(u-\frac{i}{2}\bigg) \mathds{1}_{n, a_2}} \times \bigg(u-\frac{i}{2}\bigg) \mathds{1}_{n, a_1} + {(u_1-u_2) \mathds{1}_{a_1, a_2} i \bm{\mathcal{P}}_{n, a_2}} \times \bigg(u-\frac{i}{2}\bigg) \mathds{1}_{n, a_1} \\
    &+ {(u_1-u_2) \mathds{1}_{a_1, a_2} \bigg(u-\frac{i}{2}\bigg) \mathds{1}_{n, a_2}} \times  i \bm{\mathcal{P}}_{n, a_1} + {(u_1-u_2) \mathds{1}_{a_1, a_2} i \bm{\mathcal{P}}_{n, a_2}} \times i \bm{\mathcal{P}}_{n, a_1} \\
    &+ i \bm{\mathcal{P}}_{a_1, a_2} \bigg(u-\frac{i}{2}\bigg) \mathds{1}_{n, a_1}  \bigg(u-\frac{i}{2}\bigg) \mathds{1}_{n, a_2} + i \bm{\mathcal{P}}_{a_1, a_2} \bigg(u-\frac{i}{2}\bigg) \mathds{1}_{n, a_1} i \bm{\mathcal{P}}_{n, a_2} \\
    & - \bm{\mathcal{P}}_{a_1, a_2}  \bm{\mathcal{P}}_{n, a_1}  \bigg(u-\frac{i}{2}\bigg) \mathds{1}_{n, a_2} - \bm{\mathcal{P}}_{a_1, a_2}  \bm{\mathcal{P}}_{n, a_1} i \bm{\mathcal{P}}_{n, a_2} \\
    &= \left({(u_1-u_2) \mathds{1}_{a_1, a_2}  \bigg(u-\frac{i}{2}\bigg) \mathds{1}_{n, a_2}} + {(u_1-u_2) \mathds{1}_{a_1, a_2} i \bm{\mathcal{P}}_{n, a_2}}\right) \bigg(u-\frac{i}{2}\bigg) \mathds{1}_{n, a_1} \\ 
    &+ \left({(u_1-u_2) \mathds{1}_{a_1, a_2} \bigg(u-\frac{i}{2}\bigg) \mathds{1}_{n, a_2}} + {(u_1-u_2) \mathds{1}_{a_1, a_2} i \bm{\mathcal{P}}_{n, a_2}} \right) \times  i \bm{\mathcal{P}}_{n, a_1} \\
    &+ i \bm{\mathcal{P}}_{a_1, a_2} \bigg(u-\frac{i}{2}\bigg) \mathds{1}_{n, a_1} \bigg(u-\frac{i}{2}\bigg) \mathds{1}_{n, a_2} + i \bm{\mathcal{P}}_{a_1, a_2} \bigg(u-\frac{i}{2}\bigg) \mathds{1}_{n, a_1} i \bm{\mathcal{P}}_{n, a_2} \\
    & - \bm{\mathcal{P}}_{a_1, a_2}  \bm{\mathcal{P}}_{n, a_1}  \bigg(u-\frac{i}{2}\bigg) \mathds{1}_{n, a_2} - \bm{\mathcal{P}}_{a_1, a_2}  \bm{\mathcal{P}}_{n, a_1} i \bm{\mathcal{P}}_{n, a_2} \\
    &= \bigg((u_1-u_2) \mathds{1}_{a_1, a_2} \bigg(u-\frac{i}{2}\bigg) \mathds{1}_{n, a_2} + {(u_1-u_2) \mathds{1}_{a_1, a_2} i \bm{\mathcal{P}}_{n, a_2}} \bigg) \left( \bigg(u-\frac{i}{2}\bigg) \mathds{1}_{n, a_2} + i \bm{\mathcal{P}}_{n, a_1} \right) \\
    &+ i \bm{\mathcal{P}}_{a_1, a_2} \bigg(u-\frac{i}{2}\bigg) \mathds{1}_{n, a_1} \bigg(u-\frac{i}{2}\bigg) \mathds{1}_{n, a_2} + i \bm{\mathcal{P}}_{a_1, a_2} \bigg(u-\frac{i}{2}\bigg) \mathds{1}_{n, a_1} i \bm{\mathcal{P}}_{n, a_2} \\
    & - \bm{\mathcal{P}}_{a_1, a_2}  \bm{\mathcal{P}}_{n, a_1}  \bigg(u-\frac{i}{2}\bigg) \mathds{1}_{n, a_2} - \bm{\mathcal{P}}_{a_1, a_2}  \bm{\mathcal{P}}_{n, a_1} i \bm{\mathcal{P}}_{n, a_2} \\\end{align*}
\begin{align*}
    &= \bigg( \bigg(u-\frac{i}{2}\bigg) \mathds{1}_{n, a_2}  + { i \bm{\mathcal{P}}_{n, a_2}} \bigg) \left( \bigg(u-\frac{i}{2}\bigg) \mathds{1}_{n, a_2} + i \bm{\mathcal{P}}_{n, a_1} \right) (u_1-u_2) \mathds{1}_{a_1, a_2}  \\
    &+ \bigg( \bigg(u-\frac{i}{2}\bigg) \mathds{1}_{n, a_2}  + { i \bm{\mathcal{P}}_{n, a_2}} \bigg) \left( \bigg(u-\frac{i}{2}\bigg) \mathds{1}_{n, a_2} + i \bm{\mathcal{P}}_{n, a_1} \right)  i \bm{\mathcal{P}}_{n, a_2}  \\
    &= \bigg( \bigg(u-\frac{i}{2}\bigg) \mathds{1}_{n, a_2}  + { i \bm{\mathcal{P}}_{n, a_2}} \bigg) \left( \bigg(u-\frac{i}{2}\bigg) \mathds{1}_{n, a_2} + i \bm{\mathcal{P}}_{n, a_1} \right) \bigg(\bigg(u-\frac{i}{2}\bigg) \mathds{1}_{n, a_2} + i \bm{\mathcal{P}}_{n, a_2} \bigg), 
\end{align*}

which proves that 

\begin{align}
 & & \alignedbox{\br_{a_1, a_2}(u_1-u_2) \bm{\mathcal{L}}_{n, a_1} (u_1) \bm{\mathcal{L}}_{n, a_2} (u_2)}{= \bm{\mathcal{L}}_{n, a_1} (u_2) \bm{\mathcal{L}}_{n, a_1} (u_1) \bm{\mathcal{R}}_{a_1, a_2}(u_1-u_2)}.   
\label{fundamental commutation relation}
\end{align}

From the fundamental commutation relation it can then be shown that the $\bm{\mathcal{R}}$-matrix follows the quantum Yang-Baxter equation. Consider then the product of the following $\bl$-operators,

\begin{equation} 
    \begin{split}
    \bl_{n, 1} \bl_{n, 2} \bl_{n, 3} &= \br_{12}^{-1} \bl_{n, 2} \bl_{n, 1} \bl_{n, 3} \br_{12} \\
    &= \br_{12}^{-1} \br_{13}^{-1} \bl_{n, 2} \bl_{n, 3} \bl_{n, 1} \br_{13}\br_{12} \\
    &= \br_{12}^{-1} \br_{13}^{-1} \br_{23}^{-1} \bl_{n, 3} \bl_{n, 2} \bl_{n, 1} \br_{23}\br_{13} \br_{12} \\
    &= (\br_{23}\br_{13}\br_{12})^{-1} \bl_{n, 3} \bl_{n, 2} \bl_{n, 1}\br_{23}\br_{13}\br_{12},
\end{split}
\end{equation}

and similarly 

\begin{equation}  
    \begin{split}
    \bl_{n, 1} \bl_{n, 2} \bl_{n, 3} &= \br_{23}^{-1} \bl_{n, 1} \bl_{n, 3} \bl_{n, 2} \br_{23} \\
    &= \br_{23}^{-1} \br_{13}^{-1} \bl_{n, 3} \bl_{n, 1} \bl_{n, 2} \br_{13}\br_{23} \\
    &= \br_{23}^{-1} \br_{13}^{-1} \br_{12}^{-1} \bl_{n, 3} \bl_{n, 2} \bl_{n, 1} \br_{12} \br_{13}\br_{23} \\
    &= (\br_{12}\br_{13}\br_{23})^{-1} \bl_{n, 3} \bl_{n, 2} \bl_{n, 1}\br_{12}\br_{13}\br_{23},
\end{split}
\end{equation}

both of which then yield an important constraint on the $\br$-matrix, namely 

\begin{align}
    \alignedbox{\br_{12}\br_{13}\br_{23}}{= \br_{23}\br_{13}\br_{12}},
\end{align}

which is the quantum Yang-Baxter equation. \\

\begin{df}
    Let the monodromy matrix $\bm{\mathcal{T}}_{N,a}$ be defined as a product of subsequent of $\bl$-operators, ie.

\begin{equation}
    \begin{array}{c}
         \bmt_{N,a} = \prod^{j=N}_{1} \bl_{j,a} \\
         \\
         \bmt_{N,a} = \prod^{j=N}_{1} \left(\begin{array}{cc}
            u \mathds{1}_j + i \spin_j^z & i \spin_j^-  \\
            i \spin_j^+ & u \mathds{1}_j - i \spin_j^z
         \end{array}\right) \equiv \left(\begin{array}{cc}
            A(u) & B(u) \\
            C(u) & D(u)
         \end{array}\right) 
    \end{array},
\end{equation}
\end{df}

This matrix obeys the fundamental relation as well\footnote{ For example, consider $N=2$, then

\begin{equation}
    \begin{split}
        {\br_{a_1, a_2}(u_1-u_2) \bm{\mathcal{T}}_{N, a_1} (u_1) \bm{\mathcal{T}}_{N, a_2} (u_2)} &= {\br_{a_1, a_2}(u_1-u_2) \bm{\mathcal{T}}_{2, a_1} (u_1) \bm{\mathcal{T}}_{2, a_2} (u_2)} \\
        &= {\br_{a_1, a_2}(u_1-u_2) \bl_{2, a_1} (u_1) \bl_{2, a_1} (u_1) \bl_{2, a_2} (u_2) \bl_{1, a_2} (u_2)} \\
        &= \bigg(\br_{a_1, a_2}(u_1-u_2) \bl_{2, a_1} (u_1) \bl_{2, a_2} (u_2)\bigg) \bl_{1, a_1} (u_1) \bl_{1, a_2} (u_2) \\
        &=  \bl_{2, a_2} (u_2) \bl_{2, a_1} (u_1) \bigg(\br_{a_1, a_2}(u_1-u_2)\bl_{1, a_1} (u_1) \bl_{1, a_2} (u_2) \bigg) \\
        &= \bl_{2, a_2} (u_2) \bl_{2, a_1} (u_1) \bl_{1, a_2} (u_2) \bl_{1, a_1} (u_1) \br_{a_1, a_2}(u_1-u_2) \\
        &=  \bl_{2, a_2} (u_2) \bl_{1, a_2} (u_2) \bl_{2, a_1} (u_1) \bl_{1, a_1} (u_1) \br_{a_1, a_2}(u_1-u_2) \\
        &= \bm{\mathcal{T}}_{n, a_1} (u_2) \bm{\mathcal{T}}_{n, a_1} (u_1) \bm{\mathcal{R}}_{a_1, a_2}(u_1-u_2),
    \end{split}
\end{equation}

where the fundamental commutation relation given by \eqref{fundamental commutation relation} was used. Note as well that $\bl$-operators acting on different spaces commute as well, eg. $[\bl_{n_1, a_1}, \bl_{n_2, a_2}] \propto \delta_{n_1, n_2}$. The general fundamental relation can then be proved via mathematical induction for the general case.}, ie.

\begin{align}
 & & \alignedbox{\br_{a_1, a_2}(u_1-u_2) \bm{\mathcal{T}}_{N, a_1} (u_1) \bm{\mathcal{T}}_{N, a_2} (u_2)}{= \bm{\mathcal{T}}_{N, a_2} (u_2) \bm{\mathcal{T}}_{N, a_1} (u_1) \bm{\mathcal{R}}_{a_1, a_2}(u_1-u_2)}.   
\label{fundamental commutation relation on R}
\end{align}

 \blanky \\

\begin{df}
    The transfer $\mathfrak{t}$-matrix is defined as the $\bm{\mathcal{T}}$-matrix's partial trace over the auxiliary space $\vds_{a}$, ie.

\begin{equation}
    \mathfrak{t}(u) = \Tr_{\vds_{a}} \bm{\mathcal{T}}_{N, a} (u) = \Tr_{\vds_{a}} \left(\begin{array}{cc}
            A(u) & B(u) \\
            C(u) & D(u)
         \end{array}\right)  = A(u) + D(u),
\end{equation}
    where $\mathfrak{t}, A, B, C, D$ all are $\twobytwo$-matrices in the physical system, the auxiliary systems being present no more. 
\end{df}
 
With the previous definitions in mind, \cref{fundamental commutation relation on R}'s double trace can then be written in terms of the transfer matrix, as follows 

\begin{align*}
    \Tr_{\vds_{a_1}} \Tr_{\vds_{a_2}} & \bigg[\br_{a_1, a_2}(u_1-u_2) \bm{\mathcal{T}}_{N, a_1} (u_1) \bm{\mathcal{T}}_{N, a_2} (u_2)\bigg] &= \Tr_{\vds_{a_1}} \Tr_{\vds_{a_2}} \bigg[{\bm{\mathcal{T}}_{N, a_1} (u_2) \bm{\mathcal{T}}_{N, a_1} (u_1) \bm{\mathcal{R}}_{a_1, a_2}(u_1-u_2)}\bigg] \\
    &\Rightarrow \Tr_{\vds_{a_1}} \Tr_{\vds_{a_2}} \bigg[\br_{a_1, a_2}(u_1-u_2) \bm{\mathcal{T}}_{N, a_1} (u_1) \bm{\mathcal{T}}_{N, a_2} (u_2) - {\bm{\mathcal{T}}_{N, a_1} (u_2) \bm{\mathcal{T}}_{N, a_1} (u_1) \bm{\mathcal{R}}_{a_1, a_2}(u_1-u_2)} \bigg] = 0 \\ 
    &\Rightarrow \Tr_{\vds_{a_1}} \Tr_{\vds_{a_2}} \bigg[\br_{a_1, a_2}(u_1-u_2) \bigg(\bm{\mathcal{T}}_{N, a_1} (u_1) \bm{\mathcal{T}}_{N, a_2} (u_2) - {\bm{\mathcal{T}}_{N, a_2} (u_2) \bm{\mathcal{T}}_{N, a_1} (u_1)\bigg)} \bigg] = 0 \\ 
    \textnormal{ which implies }
    & \mathfrak{t}(u_1)\mathfrak{t}(u_2) = \mathfrak{t}(u_2)\mathfrak{t}(u_1).
\end{align*}

Therefore the transfer matrix commutes with itself for different values of the spectral parameter. It turns out that the monodromy matrix can be expanded in a power series around any point $z_0 \in \mathds{C}$, generating an infinite set of linearly independent commuting operators acting on the full quantum space. This formally leads to the model's integrability. \\

\paragraph{\textbf{Monodromy matrix and the Hamiltonian}}

The Hamiltonian operator belongs to the family of transfer matrices. Let $z_0 = \frac{i}{2}$, then the monodromy matrix can be expanded around said complex point yielding (up to first order)

\begin{equation} 
    \begin{split}
         \bmt_{N,a}(z_0) &= \prod_1^{j=N} \bigg[ \cancel{\bigg(z_0-\frac{i}{2}\bigg) \mathds{1}_{j, a}} + i \bm{\mathcal{P}}_{j, a}\bigg] = i^N \bm{\mathcal{P}}_{N, a} \bm{\mathcal{P}}_{N-1, a} \cdots \bm{\mathcal{P}}_{1, a} = \\
         &= i^N \bm{\mathcal{P}}_{1, 2} \bm{\mathcal{P}}_{2, 3} \cdots \bm{\mathcal{P}}_{N-1, N} \bm{\mathcal{P}}_{N, a}\\
         &= i^N \prod_{j=1}^{N-1} \bm{\mathcal{P}}_{j, \blanky j+1} \bm{\mathcal{P}}_{N, a}
    \end{split}
\end{equation}

where the second line holds given the properties of the permutation operators. Taking the partial trace over the auxiliary space yields the transfer matrix, as follows 

\begin{equation}
    \begin{split}
        \mathfrak{t}(z_0) &= \Tr_{\vds_a}  \bmt_{N,a}(z_0) = \Tr_{\vds_a} i^N \bm{\mathcal{P}}_{1, 2} i^N \prod_{j=1}^{N-1} \bm{\mathcal{P}}_{j, \blanky j+1} \bm{\mathcal{P}}_{N, a} \\
        &= i^N \prod_{j=1}^{N-1} \bm{\mathcal{P}}_{j, \blanky j+1}
        \Tr_{\vds_a} \bm{\mathcal{P}}_{N, a}    = i^N \prod_{j=1}^{N-1} \bm{\mathcal{P}}_{j, \blanky j+1} \Tr_{\vds_a} \frac{1}{2}\left(\begin{array}{cc}
       \mathds{1}_n + \sigma^{z}_{n} &  \sigma^{x}_{n} - i\sigma^{y}_{n} \\
       \sigma^{x}_{n} + i\sigma^{y}_{n}  & \mathds{1}_n - \sigma^{z}_{n} \\
        \end{array}\right)_a. \\
        &= i^N \prod_{j=1}^{N-1} \bm{\mathcal{P}}_{j, \blanky j+1} \mathds{1}_N.
    \end{split}
\end{equation}

Then $\mathcal{U} = i^{-N} \mathfrak{t}(z_0) = \prod_{j=1}^{N-1} \bm{\mathcal{P}}_{j, \blanky j+1}$ is a shift operator in the full quantum Hilbert space $\mathds{H}$. In effect, note that 
$$
 \bm{\mathcal{P}}_{n_1, \blanky n_2} {\bf X}_{n_2} \bm{\mathcal{P}}_{n_1, \blanky n_2} = {\bf X}_{n_1},
$$

ie. this permutation moves ${\bf X}$ one step back. Furthermore, this operator is unitary (given that the permutations are, by definition, unitary operators). Then, according to Stone's theorem on one-parameter unitary groups, the shift operator is related to the momentum operator 

\begin{equation}
    \mathcal{U} = e^{iP}
\end{equation}

\blanky \\

The next order in the expansion of the transfer matrix can then be found via the derivative of the monodromy matrix at $z_0 = \frac{i}{2}$, 

\begin{equation*}
    \begin{split}
    \frac{d \bmt_{N,a}(u)}{du}\bigg|_{u = z_0} &= \frac{d}{du} \prod_{1}^{j=N} \left(\begin{array}{cc}
            u \mathds{1}_j + i \spin_j^z & i \spin_j^-  \\
            i \spin_j^+ & u \mathds{1}_j - i \spin_j^z
         \end{array}\right)\bigg|_{u = z_0} \\
        &=  \frac{d}{du} \left[\left(\begin{array}{cc}
            u \mathds{1}_N + i \spin_N^z & i \spin_N^-  \\
            i \spin_N^+ & u \mathds{1}_N - i \spin_N^z
         \end{array}\right)\right]\bigg|_{u = z_0} \prod_{1}^{j=N-1} \left(\begin{array}{cc}
            u \mathds{1}_j + i \spin_j^z & i \spin_j^-  \\
            i \spin_j^+ & u \mathds{1}_j - i \spin_j^z
         \end{array}\right) \\
         &+ \left(\begin{array}{cc}
            u \mathds{1}_N + i \spin_N^z & i \spin_N^-  \\
            i \spin_N^+ & u \mathds{1}_N - i \spin_N^z
         \end{array}\right) \frac{d}{du} \left[\left(\begin{array}{cc}
            u \mathds{1}_{N-1} + i \spin_{N-1}^z & i \spin_{N-1}^-  \\
            i \spin_{N-1}^+ & u \mathds{1}_{N-1} - i \spin_{N-1}^z
         \end{array}\right)\right]\bigg|_{u = z_0} \\
         &\times \prod_{1}^{j=N-2} \left(\begin{array}{cc}
            u \mathds{1}_j + i \spin_j^z & i \spin_j^-  \\
            i \spin_j^+ & u \mathds{1}_j - i \spin_j^z
         \end{array}\right) \\
         &+ \cdots +  \prod^{j=N}_{3}\left(\begin{array}{cc}
            u \mathds{1}_j + i \spin_j^z & i \spin_j^-  \\
            i \spin_j^+ & u \mathds{1}_j - i \spin_j^z
         \end{array}\right) \frac{d}{du} \left[\left(\begin{array}{cc}
            u \mathds{1}_{2} + i \spin_{2}^z & i \spin_{2}^-  \\
            i \spin_{2}^+ & u \mathds{1}_{2} - i \spin_{2}^z
         \end{array}\right)\right]\bigg|_{u = z_0} \\
         & \times \left(\begin{array}{cc}
            u \mathds{1}_{1} + i \spin_{1}^z & i \spin_{1}^-  \\
            i \spin_{1}^+ & u \mathds{1}_{1} - i \spin_{1}^z
         \end{array}\right)
         \\
         &+ \prod^{j=N}_{2}\left(\begin{array}{cc}
            u \mathds{1}_j + i \spin_j^z & i \spin_j^-  \\
            i \spin_j^+ & u \mathds{1}_j - i \spin_j^z
         \end{array}\right) \frac{d}{du} \left[\left(\begin{array}{cc}
            u \mathds{1}_{1} + i \spin_{1}^z & i \spin_{1}^-  \\
            i \spin_{1}^+ & u \mathds{1}_{1} - i \spin_{1}^z
         \end{array}\right)\right]\bigg|_{u = z_0} s    \end{split} 
\end{equation*}
\begin{equation*}
    \begin{split}
         &= i^{N-1} \bigg[ \prod_{1}^{j=N-1} \bm{\mathcal{P}}_{j} + \prod^{j=N}_{N} \bm{\mathcal{P}}_{j}\times  \prod_{1}^{j=N-2} \bm{\mathcal{P}}_{j} + \cdots + \prod^{j=N}_{k} \bm{\mathcal{P}}_{j}\times  \prod^{k+2}_{1} \bm{\mathcal{P}}_{j} \\
         &+ \cdots + \prod^{j=N}_{3} \bm{\mathcal{P}}_{j} \times \prod^{j=1}_{1} \bm{\mathcal{P}}_{j} + \prod^{j=N}_{2} \bm{\mathcal{P}}_{j} \bigg] \\
         &= i^{N-1} \sum_{n \in \mathds{N}} \bm{\mathcal{P}}_{N, a} \cdots \bm{\mathcal{P}}_{n-1, a} \bm{\mathcal{P}}_{n+1, a} \cdots \bm{\mathcal{P}}_{1, a} \\ 
        &= i^{N-1} \sum_{n \in \mathds{N}} \bm{\mathcal{P}}_{1, 2} \bm{\mathcal{P}}_{2, 3}\cdots \bm{\mathcal{P}}_{n-1, n+1}
         \cdots \bm{\mathcal{P}}_{N-1, N} \bm{\mathcal{P}}_{N, a} \\
         &= i^{N-1} \sum_{n \in \mathds{N}} \prod_{j=1}^{n-2} \bm{\mathcal{P}}_{j, \blanky j+1}  \times \bm{\mathcal{P}}_{n-1, \blanky n+1} \times \prod_{j=n+2}^{N-1}  \bm{\mathcal{P}}_{j, \blanky j+1} \bm{\mathcal{P}}_{N, a}.
    \end{split} 
\end{equation*}

Therefore,

\begin{equation}
    \begin{split}
        \frac{d \mathfrak{t}(u)}{du}\bigg|_{u = z_0} &= \frac{d \Tr_{\vds_{a}} \bmt_{N,a}(u)}{du}\bigg|_{u = z_0} \\
        &= i^{N-1} \sum_{n \in \mathds{N}} \prod_{j=1}^{n-2} \bm{\mathcal{P}}_{j, \blanky j+1} 
        \times \bm{\mathcal{P}}_{n-1, \blanky n+1} \times \prod_{j=n+2}^{N-1}  \bm{\mathcal{P}}_{j, \blanky j+1} \times \Tr_{\vds_{a}} \bm{\mathcal{P}}_{N, a} \\
        &= i^{N-1} \sum_{n \in \mathds{N}} \prod_{j=1}^{n-2} \bm{\mathcal{P}}_{j, \blanky j+1} 
        \times \bm{\mathcal{P}}_{n-1, \blanky n+1} \times \prod_{j=n+2}^{N-1}  \bm{\mathcal{P}}_{j, \blanky j+1}.
    \end{split}
\end{equation}

Now, note that 

\begin{equation}
    \begin{split}
        \frac{d}{du} \log(\mathfrak{t}(u))\bigg|_{u = z_0} &= \frac{d}{du} \log \Tr_{\vds_{a}} \bm{\mathcal{T}}(u) \bigg|_{u=z_0} \\
        &= \frac{d}{du}\bigg(\mathfrak{t}(u)\bigg) \mathfrak{t}(u)^{-1}\bigg|_{u = z_0} \\
        &= i^{N-1} \sum_{n \in \mathds{N}} \bigg( \prod_{j=1}^{n-2} \bm{\mathcal{P}}_{j, \blanky j+1} 
        \times \bm{\mathcal{P}}_{n-1, \blanky n+1} \times \prod_{j=n+2}^{N-1}  \bm{\mathcal{P}}_{j, \blanky j+1} \bigg) \times \bigg(i^N \prod_{j=1}^{N-1} \bm{\mathcal{P}}_{j, \blanky j+1} \mathds{1}_N\bigg)^{-1} \\
        &= \frac{1}{i} \sum_{n \in \mathds{N}} \bigg(\prod_{j=1}^{n-2} \bm{\mathcal{P}}_{j, \blanky j+1} 
        \times \bm{\mathcal{P}}_{n-1, \blanky n+1} \times \prod_{j=n+2}^{N-1}  \bm{\mathcal{P}}_{j, \blanky j+1} \bigg) \times \prod_{j=1}^{N-1} \bm{\mathcal{P}}_{N-1-j, N-j}^{-1} \\
        &= \frac{1}{i} \sum_{n \in \mathds{N}} \bigg(\prod_{j=1}^{n-2} \bm{\mathcal{P}}_{j, \blanky j+1} 
        \times \bm{\mathcal{P}}_{n-1, \blanky n+1} \times \prod_{j=n+2}^{N-1}  \bm{\mathcal{P}}_{j, \blanky j+1} \bigg) \times \prod_{1}^{j = N-1} \bm{\mathcal{P}}_{j, j+1}  \\
        &= \frac{1}{i} \sum_{n \in \mathds{N}} \bigg(\cancel{\bm{\mathcal{P}}_{1,2}} \cancel{\bm{\mathcal{P}}_{2,3}} \cdots \bm{\mathcal{P}}_{n-1, \blanky n+1} \cdots \cancel{\bm{\mathcal{P}}_{N-1,N} \times \bm{\mathcal{P}}_{N-1, \blanky N}} \cdots \bm{\mathcal{P}}_{n,\blanky n+1} \cdots \cancel{\bm{\mathcal{P}}_{2,3}} \cancel{\bm{\mathcal{P}}_{1,2}}\bigg) \\
        &= \frac{1}{i}  \sum_{n \in \mathds{N}} \bm{\mathcal{P}}_{N-1, \blanky N}
    \end{split}
\end{equation}

Note that \eqref{XXX model} can be rewritten in terms of the permutation operator as 

\begin{equation}
    {\bf H} = \frac{J}{2} \sum_{n}\bigg( \bm{\mathcal{P}}_{n, \blanky n+1} - \mathds{1}^{\otimes N}\bigg),
\end{equation}

which in turn can be rewritten as 

\begin{align}
    \alignedbox{{\bf H}}{ = \frac{J}{2} \bigg(i \frac{d}{du} \log(\mathfrak{t}(u))\bigg|_{u=z_0} - N \mathds{1}^{\otimes N}\bigg)},
    \label{Hamiltonian monodromy}
\end{align}

which shows that the XXX Hamiltonian belongs to the family of $N-1$ commuting operators generated by the trace of the monodromy matrix $\bmt$. As a result, the Hamiltonian commutes with the transfer matrix $ [{\bf H}, \mathfrak{t}(u)] = 0.$ \\

\paragraph{\textbf{Diagonalizing the Hamiltonian}}

The only task left is to diagonalize the Hamiltonian, by diagonalizing the transfer matrix. The monodromy matrix ${\mathfrak{t}}_{N,a}$ is a $\twobytwo$-matrix in the physical space $\vds_{a}$. Recalling the fundamental commutation relation between the $\bm{\mathcal{R}}$ and $\bm{\mathcal{T}}$-matrices, then the following theorem holds,

\begin{theo} \label{theo monodromy}
Let $\bm{\mathcal{T}}$ be the monodromy matrix for the Heisenberg model. Then, $\forall u_1, u_2 \in \mathds{C}$ then \\

\begin{itemize}
    \item $[\bm{\mathcal{T}}_{ij}(u_1), \bm{\mathcal{T}}_{ij}(u_2)] = 0, \blanky i,j = 1, 2$, which entails that $[B(u_1), B(u_2)] = \cdots = [D(u_1), D(u_2)] = 0$ \\
    \item $A(u_1) B(u_2) = f(u_1-u_2) B(u_2) A(u_1) + g(u_1-u_2) B(u_1)A(u_2)$\\
    \item $D(u_1) B(u_2) = h(u_1-u_2) B(u_2) D(u_1) + k(u_1-u_2) B(u_1)D(u_2)$,
\end{itemize}

where $f,g,h,$ and $k$ are given by 

\begin{align*}
    f(u) &= \frac{u-i}{u}, & g(u) &= \frac{i}{u}, & h(u) &= \frac{u + i}{u}, & k(u) &= -\frac{i}{u}.
\end{align*}
\end{theo}

\begin{proof}
Consider two different monodromy matrices, labelled $\bm{\mathcal{T}}_{N, a_1} (u_1)$ and  $\bm{\mathcal{T}}_{N, a_2} (u_2)$, acting on two different auxiliary spaces, labelled $\vds_1$ and $\vds_2$ respectively. Then, these $\bmt$-matrices can be written as 

\begin{equation} \begin{array}{cc}
   \bm{\mathcal{T}}_{N, a_1} (u_1) = \left( \begin{array}{cc}
        A(u_1) & B(u_1)  \\
        C(u_1) & D(u_1) 
    \end{array} \right) \otimes \mathds{1}_{\vds_2}, & \bm{\mathcal{T}}_{N, a_2} (u_2) = \mathds{1}_{\vds_1} \otimes \left( \begin{array}{cc}
        A(u_2) & B(u_2)  \\
        C(u_2) & D(u_2) 
    \end{array} \right) \\
    = \left( \begin{array}{cccc}
        A(u_1) & 0 & B(u_1) & 0  \\
        0 & A(u_1) & 0 & B(u_1) \\
        C(u_1) & 0 & D(u_1) & 0  \\
        0 & C(u_1) & 0 & D(u_1) \\
    \end{array} \right), & \left( \begin{array}{cccc}
        A(u_2) & B(u_2) & 0 & 0  \\
        C(u_2) & D(u_2) & 0 & 0 \\
        0 & 0 & A(u_2) & B(u_2) \\
        0 & 0 & C(u_2) & D(u_2) \\
    \end{array} \right)
\end{array}
\end{equation}

 Recall that the $\bm{\mathcal{R}}$-matrix is given by \eqref{R-matrix def}, which can be rewritten as 

$$
\bm{\mathcal{R}}_{a_1, a_2}(u_1-u_2) = 
\left(\begin{array}{cccc}
        a (u_1-u_2) & 0 & 0 & 0  \\
        0 & b(u_1-u_2) & c(u_1-u_2) & 0  \\
        0 & c(u_1-u_2) & b(u_1-u_2) & 0 ) \\
        0 & 0 & 0 & a (u_1-u_2) \\
    \end{array} \right) \textnormal{ where } \begin{array}{c}
         a(\lambda) = \lambda + i  \\
         b(\lambda) = \lambda \\
         c(\lambda) = i 
    \end{array}, 
$$

Then the fundamental commutation relation \eqref{fundamental commutation relation on R} establishes a relationship between the monodromy $\bmt$-matrices and the $\bm{\mathcal{R}}$-matrix. Then, plugging in the previous results yields

\begin{equation*}
    \Rightarrow \begin{split}
    \left(\begin{array}{cccc}
        a_{12} & 0 & 0 & 0  \\
        0 & b_{12} & c_{12} & 0  \\
        0 & c_{12} & b_{12} & 0  \\
        0 & 0 & 0 & a_{12} \\
    \end{array} \right) \left( \begin{array}{cccc}
        A(u_1) A(u_2) & A(u_1) B(u_2) & B(u_1) A(u_2) & B(u_1) B(u_2)  \\
        A(u_1) C(u_2) & A(u_1) D(u_2) & B(u_1) C(u_2) & B(u_1) D(u_2) \\
        C(u_1) A(u_2) & C(u_1) B(u_2) & D(u_1) A(u_2) & D(u_1) B(u_2) \\
        C(u_1) C(u_2) & C(u_1) D(u_2) & D(u_1) C(u_2) & D(u_1) D(u_2) \\
    \end{array} \right) \\ = \left( \begin{array}{cccc}
        A(u_2) A(u_1) & B(u_2)A(u_1) & A(u_2) B(u_1) & B(u_2) B(u_1)  \\
        C(u_2) A(u_1) & D(u_2) A(u_1) & C(u_2) B(u_1) & D(u_2) B(u_1) \\ 
        A(u_2) C(u_1) & B(u_2) C(u_1) & A(u_2) D(u_1) & B(u_2) D(u_1) \\ 
        C(u_2) C(u_1) & D(u_2) C(u_1) & C(u_2) D(u_1) & D(u_2) D(u_1) 
    \end{array} \right) \left(\begin{array}{cccc}
        a_{12} & 0 & 0 & 0  \\
        0 & b_{12} & c_{12} & 0  \\
        0 & c_{12} & b_{12} & 0  \\
        0 & 0 & 0 & a_{12} \\
    \end{array} \right) \\
   \Rightarrow 
   \left(\begin{array}{cccc}
        a_{12} A_1 A_2 & a_{12} A_1 B_2 & a_{12} B_1 A_2 & a_{12} B_1 B_2 \\
        b_{12} A_1 C_2 + c_{12} C_1 A_2 & b_{12} A_1 D_2 + c_{12} C_1 B_2 & b_{12} B_1 C_2 + c_{12} D_1 A_2 & b_{12} B_1 D_2 + c_{12} D_1 B_2 \\
        c_{12} A_1 C_2 + b_{12} C_1 A_2 & b_{12} A_1 D_2 + c_{12} C_1 B_2 & c_{12} B_1 C_2 + b_{12} D_1 A_2 & c_{12} B_1 D_2 + b_{12} D_1 B_2 \\
        a_{12} C_1 C_2 & a_{12} C_1 D_2 & a_{12} D_1 C_2 & a_{12} D_1 D_2
    \end{array} \right) \\
    = \left(\begin{array}{cccc}
        a_{12} A_2 A_1 & b_{12} B_2 A_1 + c_{12} A_2 B_1 & c_{12} B_2 A_1 + b_{12} A_2 B_1 & a_{12} B_2 B_1\\ 
        a_{12} C_2 A_1 & b_{12} D_2 A_1 + c_{12} C_2 B_1 & c_{12} D_2 A_1 + b_{12} C_2 B_1 & a_{12} D_2 B_1\\
        a_{12} A_2 C_1 & b_{12} B_2 C_1 + c_{12} A_2 D_1 & c_{12} B_2 C_1 + b_{12} A_2 D_1 & a_{12} B_2 D_1\\
        a_{12} C_2 C_1 & b_{12} D_2 C_1 + c_{12} C_2 D_1 & c_{12} D_2 C_1 + b_{12} C_2 D_1 & a_{12} D_2 D_1
    \end{array} \right)
    \end{split}
\end{equation*}

\[
    \Rightarrow \text{\small $ \left(  \scalemath{0.7}{\begin{array}{cccc}
    a_{12} [A_1, A_2] & a_{12} A_1 B_2 - (b_{12} B_2 A_1 + c_{12} A_2 B_1) & a_{12} B_1 A_2 - (c_{12} B_2 A_1 + c_{12} A_2 B_1) & a_{12} [B_1, B_2]  \\
    (b_{12} A_1 C_2 + c_{12} C_1 A_2) - a_{12} C_2 A_1 & b_{12} [A_1, D_2] + c_{12} [C_1 B_2] & (b_{12} [B_1, C_2] + c_{12} [D_1, A_2] & b_{12} B_1 D_2 + c_{12} D_1 B_2 - a_{12} D_2 B_1 \\
    c_{12} A_1 C_2 + b_{12} C_1 A_2 - a_{12} A_2 C_1 & (b_{12} A_1 D_2 + c_{12} C_1 B_2) - b_{12} B_2 C_1 + c_{12} A_2 D_1) & c_{12} [B_1, C_2] + b_{12} [D_1, A_2] & c_{12} B_1 D_2 + b_{12} D_1 B_2 - a_{12} B_2 D_1 \\
    a_{12} [C_1, C_2] & a_{12} C_1 D_2 - (b_{12} D_2 C_1 + c_{12} C_2 D_1) & a_{12} D_1 C_2 - (c_{12} D_2 C_1 + b_{12} C_2 D_1) & a_{12} [D_1, D_2]
    \end{array}} \right) = {\bf 0} $},
\]

from which, it's been proven that $[\bm{\mathcal{T}}_{ij}(u_1), \bm{\mathcal{T}}_{ij}(u_2)] = 0, \blanky i,j = 1, 2$ as long as $a (u_1 - u_2) \neq 0$. From the other matrix-entries, it's immediately seen that 

$$
a(u_1-u_2) B(u_1) A(u_2) = c(u_1-u_2) B(u_2) A(u_1) + b(u_1-u_2) A(u_2) B(u_1).  
$$

Swapping $u_1 \rightarrow u_2$ implies $u_1 - u_2 \rightarrow -(u_1 - u_2)$. Now, note that $c(u_1)/b(u_1) = g(u_1) = g(-u_1)$ and similarly $a(-u_1)/b(-u_1) = h(-u_1) \equiv f(u_1)$, which yields the second line of theorem \ref{theo monodromy}. The third claim can then be similarly derived. 

\end{proof} 

\blanky \\

\paragraph{\textbf{Eigenvalues of the Bethe Ansatz}}

Theorem \ref{theo monodromy}'s relations shows the model's inner structure, identifying an underlying algebraic structure somewhat similar to that of the Harmonic oscillator, where $A+D$ are associated with the eigenenergies, and where $B$ and $C$ can be interpreted as ladder operators. \\

Then, a Fock-like state space can be constructed for the $N$-body system, with its ground state satisfying $C \Omega^{\otimes N} = 0$. Then, $\omega_+^{i} = \left( \begin{array}{cc}
    1 & 0 
\end{array}\right)_{i}^{\dagger}$ describes $i$-th spin's state, with up spin. Then, 

$$
\spin^{+} \omega_+^{i} = 0,
$$

Therefore, consider the $\bl$-matrix, defined by \eqref{l-matrix def}, its action on a tensor product of states can be written out as 

\begin{equation}
    \bl_{N, a} v \otimes \omega_+^{i} \left(\begin{array}{cc}
        \lambda + \frac{i}{2} & \cdots \\
        0 &  \lambda - \frac{i}{2}
    \end{array}\right) v \otimes \omega_+^{i}, \blanky \forall v \in \mathds{V}_a,
\end{equation}

since $\spin^{+}$ annihilates $\omega_+^i$. Thus, the ground sate can be written as the physical ferromagnetic vacuum state with all its spins up. Said state is an eigenstate of $A(u)$ and $D(u)$ and is annihilated by $C(u)$, ie. 

\begin{align}
    \Omega^{\otimes N} = \bigotimes_{i=1}^{N} \left( \begin{array}{c} 
         1 \\
         0 
    \end{array} \right), \textnormal{ where } \begin{array}{c}
    A(u) \Omega^{\otimes N} = \bigg(u + \frac{i}{2}\bigg)^{N} \Omega^{\otimes N}, \\
    \\
    D(u) \Omega^{\otimes N} = \bigg(u - \frac{i}{2}\bigg)^{N} \Omega^{\otimes N}, \\
    \end{array}
    C(u) \Omega^{\otimes N} = 0. 
\end{align}

The $B(u)$ operators can then be used to construct the Bethe state

\begin{equation}
    \ket{u_1, \cdots u_M} = \prod_{j=1}^{M < N} B(u_j) \Omega^{\otimes N}, \textnormal{ for some set } \{u_j\}_{j=1}^{M < N} \subset \mathds{C}.
    \label{bethe state}
\end{equation}

A priori, not all complex-valued sequences are allowed since the Bethe states must be eigenvectors of $A(u) + D(u)$, which yields some algebraic conditions. \\

\paragraph{\textbf{Conditions on } $\{u_j\}_{j=1}^{M < N}$}

Consider the Bethe state given by \cref{bethe state}. Then, 

\begin{equation*}
    A(v) \ket{u_1, \cdots u_M} = A(v) \prod_{j=1}^{M < N} B(u_j) \Omega^{\otimes N}.
\end{equation*}

Using the fundamental commutation relations given \cref{theo monodromy},

$$
    A(v) B(u) = f(v-u) B(u) A(v) + g(v-u) B(v) A(u).
$$

Therefore,

\begin{equation}
    \begin{split}
        A(v) \ket{u_1, \cdots u_M} &= \bigg[f(v-u_1) B(u_1) A(v) + g(v-u_1) B(v) A(u_1)\bigg] \prod_{j=2}^{M < N} B(u_j) \Omega^{\otimes N} \\
        &= \left[\prod_{k=1}^{\ell} f(v-u_k)\right] \alpha^N(v) \ket{u_1, \cdots u_M} + \mathcal{O}({A \times D}),
    \end{split}
\end{equation}

wherein, the first term is an scalar times the Bethe state, said scalar being the eigenvalue if and only if the second term, the term with products of $A$ and $D$, cancels out. In that case, the Bethe state is an eigenvector of $A + D$. Then, continuing the previous calculation 

\begin{equation}
    \begin{split}
         A(v) \ket{u_1, \cdots u_M} &=  \bigg[f(v-u_1) B(u_1) A(v) + g(v-u_1) B(v) A(u_1)\bigg] B(u_2) \prod_{j=3}^{M < N} B(u_j) \Omega^{\otimes N} \\
         &= \bigg[f(v-u_1)f(v-u_2) B(u_1) B(u_2) A(v) + f(v-u_1)g(v-u_2) B(u_1) B(v) A(u_2) \\
         & + g(v-u_1)f(u_1-u_2) B(v) B(u_2) A(u_1) + g(v-u_1)g(v-u_2) B(v) B(u_2)
         \bigg] \prod_{j=3}^{M < N} B(u_j) \Omega^{\otimes N}. 
    \end{split}
\end{equation}

Now, note that 

\begin{align*}
    %\begin{split}
        \bigg(A(v) + D(v) \bigg)\ket{u_1, \cdots u_M} &= \bigg[\bigg(\prod_{k=1}^{\ell} f(v-u_k)\bigg)\alpha^N(v) + \bigg(\prod_{k=1}^{\ell} h(v-u_k)\bigg) \delta^N(v) \bigg] \ket{u_1, \cdots u_M} \\
        &+ \sum_{k=1}^{\ell} \bigg[\bigg(g(v-u_k) \bigg[ \prod_{j \neq k}^{\ell} f(u_k - u_j)\bigg] \alpha^N(u_k)\bigg) + h(v-u_k) \bigg(\prod_{j \neq k}^{\ell} h(u_k - u_j)\bigg) \delta^N(u_j) \\
        &\times \prod_{\substack{j = 1 \\
                                 j \neq k}}^{\ell} \Omega^{\otimes N} B(u_i)\bigg]\\
    &\equiv \bigg[\bigg(\prod_{k=1}^{\ell} f(v-u_k)\bigg)\alpha^N(v) + \bigg(\prod_{k=1}^{\ell} h(v-u_k)\bigg) \delta^N(v) \bigg] \ket{u_1, \cdots u_M} + 0
    %\end{split} 
\end{align*}

\begin{align*}
    \Rightarrow \sum_{k=1}^{\ell} \bigg[\bigg(g(v-u_k) \bigg[ \prod_{j \neq k}^{\ell} f(u_k - u_j)\bigg] \alpha^N(u_k)\bigg) + h(v-u_k) \bigg(\prod_{j \neq k}^{\ell} h(u_k - u_j)\bigg)  \delta^N(u_j) \bigg] &= 0 \\
    \sum_{k=1}^{\ell} \bigg[\bigg(g(v-u_k) \bigg[ \prod_{j \neq k}^{\ell} f(u_k - u_j)\bigg] \alpha^N(u_k)\bigg) - g(v-u_k) \bigg(\prod_{j \neq k}^{\ell} h(u_k - u_j)\bigg)  \delta^N(u_j) \bigg] &= 0 \\
    \sum_{k=1}^{\ell} g(v-u_k) \bigg[ \prod_{j \neq k}^{\ell} f(u_k - u_j)\bigg] \alpha^N(u_k) + \prod_{j \neq k}^{\ell} h(u_k - u_j) \delta^N(u_j) \bigg] &= 0,
\end{align*}

which must be true for all $k$,

\begin{align}
    & & \bigg[\prod_{j \neq k}^{\ell} f(v- u_k)\bigg] \alpha^N(u_j) = \bigg[\prod_{j \neq k}^{\ell} h(v - u_k) \bigg]  \delta^N(u_j).
\end{align}

Now using \cref{theo monodromy} functions and the that $\alpha(u) = u + \frac{i}{2}$ and $\delta(u) = u - \frac{i}{2}$, the previous equation can be rewritten as 

\begin{equation}
    \begin{split}
        \bigg[\prod_{j \neq k}^{\ell} &f(u_k - u_j)\bigg] \alpha^N(u_j) = \bigg[\prod_{j \neq k}^{\ell} h(u_k - u_j) \bigg]  \delta^N(u_j) \\
        \prod_{j \neq k}^{\ell} &\frac{u_k - u_j - i}{u_k - u_j} \bigg(u_j + \frac{i}{2}\bigg) = \prod_{j \neq k}^{\ell} \frac{u_k - u_j + i}{u_k - u_j} \bigg(u_j - \frac{i}{2}\bigg)\\
        0 &= \prod_{j \neq k}^{\ell} \frac{u_k - u_j - i}{u_k - u_j} \bigg(u_j + \frac{i}{2}\bigg) - \prod_{j \neq k}^{\ell} \frac{u_k - u_j + i}{u_k - u_j} \bigg(u_j - \frac{i}{2}\bigg),
    \end{split}
\end{equation}

which can be rearranged to yield the Bethe ansatz equations, 

\begin{align}
    \alignedbox{\left(\frac{u_k+\frac{i}{2}}{u_k - \frac{i}{2}}\right)^N}{= \prod_{j\neq k}^{\ell} \frac{u_k - u_j + i}{ u_k - u_j -i}},
    \label{Bethe coeff equations}
\end{align}

which is a set of $\ell$ non linear equations on the coefficients $u_k \in \mathds{C}$. \\

\paragraph{\textbf{Momentum and Energy}}

Let $\{u_k\}_{k=1}^{\ell < N} \subset \mathds{C}$ satisfy the Bethe equations. Then,  

\begin{equation}
\begin{split}
    (A(v) + D(v)) \ket{u_1, \cdots u_M} &= \bigg[\bigg(\prod_{k=1}^{\ell} f(v-u_k)\bigg)\alpha^N(v) + \bigg(\prod_{k=1}^{\ell} h(v-u_k)\bigg) \delta^N(v) \bigg] \ket{u_1, \cdots u_M} \\
    (A(v) + D(v)) \ket{u_1, \cdots u_M} &= \bigg[ \left(v + \frac{i}{2}\right)^N \prod_{k=1}^{\ell} \frac{v - u_k - i}{v-u_k} + \left(v - \frac{i}{2}\right)^N \prod_{k=1}^{\ell} \frac{v - u_k + i}{v-u_k} \bigg]\ket{u_1, \cdots u_M} \\
    \left(A\left(\frac{i}{2}\right) + D\left(\frac{i}{2}\right)\right) \ket{u_1, \cdots u_M} &= i^N \prod_{k=1}^{\ell} \frac{u_k + \frac{i}{2}}{u_k - \frac{i}{2}} \ket{u_1, \cdots u_M} \equiv \lambda(\{u_k\}_{k=1}^{\ell < N})\ket{u_1, \cdots u_M}
\end{split}
\end{equation}

Therefore, via the transfer matrix, it holds that $U = e^{iP} = i^{-N} \mathfrak{t}\left(\frac{i}{2}\right) = i^{-N} \left(A\left(\frac{i}{2}\right) + D\left(\frac{i}{2}\right) \right)$, from which 

\begin{equation}
    P \ket{u_1, \cdots u_\ell} = \sum_{k=1}^{\ell} p(u_k) \ket{u_1, \cdots u_\ell} \textnormal{ where } p(u_k) = \frac{1}{i} \log \frac{u_k + \frac{i}{2}}{u_k - \frac{i}{2}}.
\end{equation}

Now, \cref{Hamiltonian monodromy} establishes that the XXX Hamiltonian belongs to the monodromy matrices' family. Furthermore, it provides a link between the $A+D$'s eigenvalues and the eigenenergies. In effect, said eigenenergies may be found by differentiating the $A+D$'s eigenvalues over $u$ and setting $u = z_0 = \frac{i}{2}$. Let the former 

$$
\Lambda(v) = \left(v + \frac{i}{2}\right)^N \prod_{k=1}^{\ell} \frac{v - u_k - i}{v-u_k} + \left(v - \frac{i}{2}\right)^N \prod_{k=1}^{\ell} \frac{v - u_k + i}{v-u_k},
$$

then, 

\begin{align*}
    \frac{d \log \Lambda(v)}{dv} \bigg|_{v=z_0} &= \frac{1}{\Lambda(v)} \frac{d \Lambda(v)}{dv} \bigg|_{v=z_0} \\
    &= \frac{1}{\Lambda(v)\bigg|_{v=z_0}} \bigg[N i^{N-1} \prod_{k=1}^{\ell} \frac{-u_k - \frac{i}{2}}{\frac{i}{2} - u_k} + i^N \sum_k \frac{i}{(\frac{i}{2} - u_k)^2} \prod_{\substack{j=1 \\
                      j \neq k}}^{\ell} \frac{-u_j - \frac{i}{2}}{\frac{i}{2} - u} \bigg] \\
    &= i \bigg(\sum_{k=1}^{\ell} \frac{1}{u_k^2 + \frac{1}{4}} - N\bigg).
\end{align*}

Then, using \cref{Hamiltonian monodromy}, yields the eigenenergies

\begin{equation}
    H \ket{u_1, \cdots u_M} = \left[ \frac{i}{2} i \bigg(\sum_{k=1}^{\ell} \frac{1}{u_k^2 + \frac{1}{4}} - N\bigg) - \frac{N}{2} \right] \ket{u_1, \cdots u_M} \\
\end{equation}

or, equivalently, 

\begin{align}
    \alignedbox{{\bf H} \ket{u_1, \cdots u_M}}{= \sum_{k =1}^{\ell} \epsilon(u_k) \ket{u_1, \cdots u_M} \textnormal{ where } \epsilon(u_k) = - \frac{1}{2} \frac{1}{u_k^2 + \frac{1}{4}}},
\end{align}

which concludes the Bethe solution for the XXX Heisenberg chain.

\clearpage 


\subsection{Numerical solution to Fermionic models}

Consider a Hamiltonian describing a fermionic system, given by 

\begin{equation}
    {\bf H} = J \summation (f_{j}^{\dagger}f_{j+1} + f_{j+1}^{\dagger}f_{j}) + \sum_{j=1} \lambda_j f_{j}^{\dagger} f_j, \begin{array}{c}
         \textnormal{with the usual } \\
         \textnormal{ conmutation rules} 
    \end{array}
    \begin{array}{c}
         \{f_j, f_k\} = \{f_j^\dagger, f_k^\dagger\} = 0  \\
         \{f_j, f_k^\dagger\} = \delta_{jk}
    \end{array}
    \label{fermionic hamiltonian}
\end{equation}

where $L$ indicates the number of lattice sites, $J$ is the hopping strength, which could be either positive or negative, and where $\lambda_j$ is the on-site potential strength\footnote{The $\lambda_j$-term frequently appears in many condensed matter models, with different numerical values and interpretations, eg.

\begin{itemize}
    \item In the XX model, $\lambda_j = \lambda \blanky \forall j$. 
    \item While for the Anderson model $\lambda_j \in \mathcal{U}_{\R_{[-W, W]}}$, a uniform random variable, with $W$ being the disorder strength. 
    \item In the Aubry-André model, $\lambda_j = \lambda \cos(2\pi\sigma j)$, with $\sigma \in \mathds{I}$ and $\lambda$ quantifying the disorder strength. 
\end{itemize}}. Said Hamiltonian has open boundaries conditions since there is no hopping term across the boundary. Note that we can rewrite \eqref{fermionic hamiltonian} as 

\begin{equation}
    {\bf H} = \sum_{i, j = 1}^{L} \M_{ij} f_i^{\dagger} f_j \textnormal{ with } \begin{array}{c}
         \M \in \textnormal{GL}(L, \R), \\
         \M_{ij} = \left\{\begin{array}{cc}
             \lambda_i & \textnormal{ if } i=j  \\
              J & \textnormal{ if } j=i+1 \textnormal{ or } i = j + 1 \\
              0 & \textnormal{ otherwise}
         \end{array} \right.
    \end{array},
\end{equation}

which is a positive-defined tri-diagonal matrix.
Let ${\bf f} = (f_1 \blanky f_2 \blanky \cdots f_L)\transpose$ be a vector of the $L$ fermionic operators. Then, \eqref{fermionic hamiltonian} can be rewritten as 

\begin{align}
    {\bf H} = {\bf f}^\dagger \M {\bf f}.
\end{align}

Since $\M$ is symmetric, then it can be diagonalized  $\M = A \mathcal{D} A\transpose$, where $A \in \mathds{R}^{L \times L}$ is a real orthogonal matrix and with $\mathcal{D}_{ij} \in \mathds{R}^{L \times L} \blanky | \blanky \mathcal{D}_{ij}  = \epsilon_i \delta_{ij}$. In this context, the $A$-matrix acts on the fermionic operator as a Bogoliubov transformation, allowing for \eqref{fermionic hamiltonian} to be rewritten as 

\begin{equation}
     {\bf H} = {\bf f}^\dagger A \mathcal{D} A\transpose {\bf f} = {\bf d}^\dagger \mathcal{D} {\bf d}
     \label{fermionic matrix hamiltonian}
\end{equation}

where ${\bf d} = A\transpose {\bf f}$. Since the $A$-matrix is orthogonal, the new $d_k$-operators are fermionic operators as well, satisfying \eqref{fermionic hamiltonian}'s anti-commutation rules. Then, the new fermionic operators are 

\begin{equation*}
    \begin{array}{c}
         d_k = \sum_{j=1}^{L} A_{jk} f_j  \\ 
         \\
         f_j = \sum_{k=1}^{L} A_{jk} d_k 
    \end{array} \textnormal{ since } A\transpose A = \sum_{j,k = 1}^{L} A_{jk} A_{kj} = \mathds{1}_{L}.
\end{equation*}

Then, we can expand \eqref{fermionic matrix hamiltonian} in terms of the lattices, as follows 

\begin{equation}
    {\bf H} = \sum_{k = 1}^{L} \epsilon_k d_k^\dagger d_k,
\end{equation}

which is a sum of number operators with potentials. The eigenstates can then be constructed from the the theory's vacuum state, by applying the $d_k^{\dagger}$-fermionic operators. In the Heisenberg-picture, the $d_k$-operators can be evolved via the Heisenberg equation of motion

\begin{equation}
\frac{d}{dt} d_k = i [{\bf H}, d_k],
\label{H eom}
\end{equation}

and using that $d_k^2 = 0$, it turns out that \eqref{H eom}'s solution is simply $d_k(t) = e^{-i\epsilon_k t} d_k$. The system's correlation can be easily found by analyzing the following matrix. Let $\mathcal{N}_{jk} = \langle d_j^\dagger d_k \rangle$, where the expectation value is taken via calculating the operator's trace along the Fock space, which takes the following values 

\begin{equation}
    \mathcal{N}_{jk} = \langle d_j^\dagger d_k \rangle = \left\{\begin{array}{c}
        0 \textnormal{ or } 1 \textnormal{ if } j=k  \\
        0 \textnormal{ if } j \neq k 
    \end{array} \right.,
\end{equation}

ie. different lattice-sites are not correlated and there can only be a single fermion at most per lattice site, in accordance with Pauli's principle. A ground state, for example, would choose to turn on all fermions in the eigenmode $d$-space such that $\epsilon_k < 0$. If instead, the expectation value is taken with thermal states, the Fermi-Dirac distribution is returned, 

\begin{equation}
    \mathcal{N}_{jk} = \langle d_j^\dagger d_k \rangle_{\textnormal{th}} =  \frac{1}{1+e^{\beta \epsilon_k + \mu}} \delta_{jk}.
\end{equation}

Another interesting quality is a system with an initial configuration where the system's initial state, in real space, is known. In this setting, $\mathcal{N}_{jk}$ is known for all lattices. Consider for example the Anderson model, where the system's initial state is given by a single tensor product of $n$-fermionic states in real space, with $n<L$. Then, for all lattice sites, we have that $\N_{jj}$ is either zero or one. The $\N_{jk}$-matrix entries can then be evaluated as 

\begin{align*}
    \langle d_j^\dagger d_k \rangle = \sum_{i,j = 1}^{n < L} A_{ik} A_{jl} \langle f_i^\dagger f_j \rangle = \sum_{j=1} A_{jk} A_{jl} \langle f_j^\dagger f_j \rangle,
    \label{real system correlations}
\end{align*}

which can then be numerically computed to obtain the LHS expectation value. In general, this $\N_{jk}$-matrix will not be diagonal, which is reasonable since the system's real configuration is not an eigenstate. In principle and in practice, by inverting \eqref{real system correlations}, we can evolve any number operator or two-body correlation operator, ie.

\begin{align}
    \langle f_j^\dagger f_k \rangle = \sum_{k,l =1}^{n} A_{jk} A_{jl} \langle d_k^\dagger d_l \rangle.
\end{align}

This quantities' time evolution can then be found out to be 

\begin{equation}
    \langle f_j^\dagger(t) f_k(t) \rangle = \sum_{k,l =1}^{L} e^{i(\epsilon_k - \epsilon_l)t}A_{jk} A_{jl} \langle d_k^\dagger d_l \rangle,
\end{equation}

which can then be numerically solved. 

\clearpage 

%\section{Independent Electrons and Static Crystals}

%\subsection{Crystal lattices} 

%The mathematical concept which best describes an actual crystal lattice is that of a Bravais lattice, defined as a set of mathematical points corresponding to the discrete positions in space given by 

%\begin{equation}
 %   \{\Rb \in \R^{3} \blanky | \blanky \R = \sum_{i=1}^{3} n_i {\bf a}_i \textnormal{ with } n_i \in \mathds{Z}\},
%\end{equation}
    
%where the $\{a_i\}$-vector are the so-called primitive vector. The Bravais lattice is invariant under the operation 

%$$
%\Rb \rightarrow \Rb + {\bf T} \textnormal{ where } {\bf %T} = \sum_{i=1}^{3} L_i {\bf a}_i \blanky | \blanky L_i %\in \mathds{Z}
%

\clearpage

\section{Photons: coupling to electrons}

Phonons are quantized vibrations and play a fundamental role in sound physics, specific heat, elasticity and electrical resistivity of solids. More surprisingly, the electron-phonon coupling is the cause of conventional superconductivity. There are two distinct models for electron-phonon coupling, the jellium model, where the ions are represented by a smeared-out continuous positive background, and the lattice model where the ions oscillate around their equilibrium positions forming a crystal lattice. \\

Phonons are basically harmonic oscillators, they are bosons. Moreover, these naturally occur at finite temperature, so the thermal distribution function for bosons will be used. \\

\paragraph{\textbf{Jellium oscillations and Einstein phonons}}

Let $\rho^0_{\textnormal{ion}}$ be the particle density of the ion jellium and let $\rho^0_{\textnormal{el}} = Z \rho^0_{\textnormal{ion}}$ that of the homogeneous electron gas. Consider slow ionic density oscillations in a static electron gas, where the restoring force is the long range Coulomb interaction, where the electron dynamics are neglected as well. This entails that $\rho_{\textnormal{ext}}$ is a fixed constant. In the limit of small harmonic deviations from the equilibrium, $\delta \rho_{\textnormal{ion}}(x^\mu) = \delta \rho_{\textnormal{ion}} ({\bf x}) e^{-i\Omega t}$. The equations of motion are linear up to first order in the ionic density variations, thus its solutions can be written as 

\begin{equation}
    \rho_{\textnormal{ion}}(x^\mu) = \rho^0_{\textnormal{ion}} + \delta \rho_{\textnormal{ion}} ({\bf x}) e^{-i\Omega t}.
\end{equation}

A non-zero $\delta \rho_{\textnormal{ion}}$ corresponds to a charge density $Z e \delta \rho_{\textnormal{ion}}$ and hence is associated with an electric field ${\bf E}$ which must obey Maxwell's equations, ie.

\begin{align}
    \nabla \cdot {\bf E} &= \frac{Ze}{\epsilon_0} \delta \rho_{\textnormal{ion}} & \Rightarrow  \nabla \cdot \bm{\mathfrak{f}} &= \frac{Z^2 e^2 \rho^0_{\textnormal{ion}} }{\epsilon_0} \delta \rho_{\textnormal{ion}},
\end{align}

where $\bm{\mathfrak{f}} = Z e \rho_{\textnormal{ion}} {\bf E} \approx Z e \rho^0_{\textnormal{ion}} {\bf E}$ is a force density written upto first order in $\delta \rho_{\textnormal{ion}}$. This force equation is supplemented by the continuity equation, $\partial_{\mu} J^\mu = 0$, where $J^\mu = (\rho_{\textnormal{ion}}, \rho_{\textnormal{ion}} {\bf v})$. This equation can be rewritten as 

$$
\partial_t \delta \rho_{\textnormal{ion}} +  \rho^0_{\textnormal{ion}} \nabla \cdot {\bf v} + \mathcal{O}(\delta^2 \rho_{\textnormal{ion}}) = 0,
$$

since the velocity is already a small quantity. Differentiating this with respect to time once and again and using Newton's second law $\bm{\mathfrak{t}} = M \rho_{\textnormal{ion}} \partial_t {\bf v}$ yields 


\clearpage
\section{Linear Response Theory}

Linear response theory is an extremely widely used concept in physics, stating that the response to a weak external perturbation is proportional to the perturbation, and therefore the quantity of interest is the proportionality constant. The physical question to ask is thus: supposing some perturbation $H'$, what is the measured consequence for an observable quantity ${\bf A}$. In other words, what is $\langle {\bf A} \rangle$ to linear order in $H'$? 

Among the numerous physical application of the linear response formalism, one can mention charge and spin susceptibilities of eg. electron systems due to external electric or magnetic fields. Responses to external mechanical forces or vibrations can also be calculated using the same formalism. \\

\paragraph{\textbf{The general Kubo formula}}

Consider a quantum system described by a time independent Hamiltonian ${\bf H}_0$ in thermodynamic equilibrium. This means that an expectation value of a physical quantity, described by the operator ${\bf A}$, which can be evaluated as 

\begin{equation}
    \begin{split}
        \langle {\bf A} \rangle = \frac{1}{\mathcal{Z}_0} \textnormal{Tr }_{\mathds{H}} \rho_0 {\bf A}  = \frac{1}{\mathcal{Z}_0} \sum_{n \in \Lambda} \bra{n} {{\bf A}} \ket{n} e^{-\beta E_n} \\
        \rho_0 = e^{-\beta {\bf H}_0} = \sum_{n \in \Lambda} \ket{n} \bra{n} e^{-\beta E_n},
    \end{split} \begin{array}{c}
         \textnormal{ where $\{\ket{n}\}_{n \in \Lambda}$ is a complete} \\
         \\
         \textnormal{set of eigenstates}.
    \end{array}
\end{equation}

Suppose now that at some time, $t = t_0$, an external perturbation is applied to the system, driving it out of equilibrium. The perturbation is described by an additional time dependent term in the Hamiltonian 

\begin{equation}
    \Hamiltonian(t) = \Hamiltonian_0 + \Hamiltonian'(t) \theta(t-t_0)
\end{equation}

Now, the interest lies in finding the expectation value of the ${\bf A}$ operator at times $t$ greater that $t_0$-. In order to do so, the time evolution of the density matrix must be found, or equivalently the time evolution of the eigenstates of the unperturbed Hamiltonian. Once $\ket{n(t)}$ is found, the time-dependent expectation value can be found as 

\begin{align}
        \langle {\bf A}(t) \rangle = \frac{1}{\mathcal{Z}_0} \textnormal{Tr }_{\mathds{H}} \rho(t) {\bf A}  = \frac{1}{\mathcal{Z}_0} \sum_{n \in \Lambda} \bra{n(t)} {{\bf A}} \ket{n(t)} e^{-\beta E_n} \\
        \rho_0 = e^{-\beta {\bf H}_0} = \sum_{n \in \Lambda} \ket{n(t)} \bra{n(t)} e^{-\beta E_n},
\end{align}

The physical idea behind this expression is as follows. The initial states of the system are distributed according to the usual Boltzmann distribution $\frac{e^{-\beta E_{0n}}}{\mathcal{Z_0}}$. At later times, the system is described by the same distribution of states but the states are now time-dependent and they have evolved according to the new Hamiltonian. The time dependence of the states $\ket{n(t)}$ is governed by the Schr\"odinger equation. Since $\Hamiltonian'$ is regarded to be a small perturbation, the interaction picture representation is suitable for this setting. In this representation, the time dependence is given by 

\begin{equation}
    \ket{n(t)} = e^{-i\Hamiltonian_0 t} \ket{n(t)}_I = e^{-i\Hamiltonian_0 t} \mathcal{U}(t, t_0) \ket{\hat{n}(t_0)},
\end{equation}

where by definition $\ket{\hat{n}(t_0)} = e^{i \Hamiltonian_0 t_0} \ket{n(t_0)} = \ket{n}$.

Up to linear order in $\Hamiltonian'$, the time evolution operator $\mathcal{U}(t, t_0)$ can be written 

$$
    \mathcal{U}(t, t_0) = \mathds{1} - i \int_{t_0}^t dt' \Hamiltonian'(t') + \mathcal{O}(\Hamiltonian^{'2})
$$

Then, using this in the time-dependent expectation value yields 

\begin{equation} \begin{split}
    \langle {\bf A}(t) \rangle &= \frac{1}{\mathcal{Z}_0} \sum_{n \in \Lambda} \bra{n(t)} {{\bf A}} \ket{n(t)} e^{-\beta E_n} \\
    &= \frac{1}{\mathcal{Z}_0} \sum_{n \in \Lambda} \bra{n(t_0)} \bigg(\mathds{1} - i \int_{t_0}^t dt' \Hamiltonian'(t')\bigg) e^{-i \Hamiltonian_0 t_0} {{\bf A}} e^{i \Hamiltonian_0 t_0} \bigg(\mathds{1} - i \int_{t_0}^t dt' \Hamiltonian'(t')\bigg) \ket{n(t_0)}  e^{-\beta E_n}+ \mathcal{O}(\Hamiltonian^{'2}) \\
    &= \frac{1}{\mathcal{Z}_0} \sum_{n \in \Lambda} \bra{n(t_0)} {{\bf A}} \ket{n(t_0)} - \frac{i}{\mathcal{Z}_0} \int_{t_0}^{t} dt' \sum_{n \in \Lambda} e^{-\beta E_n} \bigg( e^{-i \Hamiltonian_0 t_0} {{\bf A}} e^{i \Hamiltonian_0 t_0} \Hamiltonian'(t') - \Hamiltonian'(t')e^{-i \Hamiltonian_0 t_0} {{\bf A}} e^{i \Hamiltonian_0 t_0} \ket{n(t_0)} \bigg) \\
    &= \langle {{\bf A}} \rangle_{0} + \int_{t_0}^{t} \frac{dt'}{i} \langle[{{\bf A}}(t), \Hamiltonian'(t')]\rangle_{0},
\end{split}
\end{equation}

where the brackets $\langle \rangle_{0}$ means an equilibrium average with respect to the Hamiltonian $\Hamiltonian$. This is in fact a remarkable and very useful result since the inherently non-equilibrium quantity $\langle {{\bf A}}(t) \rangle$ has been expressed as a correlation function of the system in equilibrium. The physical reason for this is that the interaction between excitations created in the non-equilibrium state is an effect to second order in the weak perturbation, hence not included in the linear response. \\

The correlation function is the retarded correlation function, which can be rewritten as 
the difference between $\langle {{\bf A}}(t) \rangle \textnormal{ and } \langle {{\bf A}}\rangle_0 $ ie. 

\begin{align} 
        & \alignedbox{\delta \langle {{\bf A}}(t) \rangle }{
        = \int_{t_0}^{\infty} dt' \mathcal{C}_{AH'}^{R}(t,t') e^{-\eta(t-t')} \textnormal{ where } \mathcal{C}_{AH'}^{R}(t,t') = -i \theta(t-t') \langle[{{\bf A}}(t), \Hamiltonian'(t')]\rangle_{0}},
\end{align}

which is the Kubo formula. This expresses the linear response to a perturbation $\Hamiltonian'$. Note that the factor $e^{-\eta(t-t')}$, with an infinitesimal positive parameter $\eta$, has been included to force the response at time $t$ due to the influence of $\Hamiltonian'$ at time $t'$ to decay when $t >> t'$. At the end of the calculation, the limit $\eta \rightarrow 0^+$,. This is so since the retarded effect of a perturbation must decrease in time\footnote{The other, the advanced correlation function is non-physical and must be ditched. The other one, the retarded correlation function, decreases exponentially with time, the exponential factor thus picks out the physically relevant solution by introducing an artificial relaxation mechanism.}. \\

\paragraph{\textbf{Kubo fromula in the frequency domain}}

It is often convenient to express the response to an external disturbance in the frequency domain via Fourier transformations\footnote{Consider the $L^1$-space, ie. the space of all integrable functions on the real line. Then the Fourier transform and the Fourier-anti trasnform can be defined as 
\begin{align}
    \mathcal{F}[f(t)](\omega) = f(\omega) = \int_{\R} dt e^{i\omega t} f(t) \blanky \textnormal{ and } \blanky \mathcal{F}^{-1}[f(\omega)](t) = f(t) = \int_{\R} \frac{d\omega}{2\pi} e^{-i\omega t} f(\omega)
\end{align}

}. Therefore, consider the perturbation Hamiltonian $\Hamiltonian'$, which can be rewritten in terms of its Fourier components 

\begin{equation}
    \Hamiltonian'(t) = \int_{\Omega \subset \R} \frac{d\omega}{2\pi} e^{-i\omega t} \Hamiltonian'_\omega,
\end{equation}

such that the retarded correlation function becomes 

\begin{equation}
    \mathcal{C}_{AH'}^{R}(t,t') = \int_{\mathds{R}} \frac{d\omega}{2\pi} e^{-i\omega t'} \mathcal{C}_{AH'_\omega}^{R}(t-t') \begin{array}{c}
         \textnormal{since $\langle[{{\bf A}}(t), \Hamiltonian'(t')_{\omega'}]\rangle_{0}$ only depends } \\
         \textnormal{ on the difference between $t$ and $t'$. }
    \end{array}
\end{equation}

Therefore, inserting this result into the Kubo formula yields

\begin{equation}
    \begin{split}
        \delta \langle {{\bf A}}(t) \rangle
        &= \int_{t_0}^{\infty} dt' \mathcal{C}_{AH'_\omega}^{R}(t,t') e^{-\eta(t-t')} \\
        &= \int_{\R} dt' \int_{\R} \frac{d\omega}{2\pi} e^{-i\omega t} e^{-i(\omega + i\eta) (t'-t)} \mathcal{C}_{AH'_\omega}^{R}(t-t') \\
        &= \int_{\R} \frac{d\omega}{2\pi} e^{-¿i\omega t} \bigg(\int_{\R} d(t'-t) e^{-i(\omega + i\eta)(t'-t)} \mathcal{C}_{AH'_\omega}^{R}(t-t')\bigg) \\
        &= \int_{\R} \frac{d\omega}{2\pi} e^{-i\omega t} \mathcal{C}_{AH'_\omega}^{R}(\omega; \eta) 
    \end{split}
\end{equation}

which can be inverted to yield the final result in the frequency domain

\begin{equation}
    \begin{split}
        \int_{\R} {dt} e^{i\nu t}
        \delta \langle {{\bf A}}(t) \rangle &= \int_{\R} {dt} e^{i\nu t} \int_{\R} \frac{d\omega}{2\pi} e^{-i\omega t} \mathcal{C}_{AH'_\omega}^{R}(\omega; \eta) \\
        \langle {{\bf A}}_\nu \rangle &= \int_{\R} \frac{d\omega}{2\pi} \int_{\R} dt e^{i(\nu - \omega)t} \mathcal{C}_{AH'_\omega}^{R}(\omega; \eta) \\
        &= \int_{\R} {d\omega} \delta(\nu - \omega) \mathcal{C}_{AH'_\omega}^{R}(\omega; \eta) \\
        &= \mathcal{C}_{AH'_\nu}^{R}(\nu; \eta) 
    \end{split}
\end{equation}

\begin{align}
       & \alignedbox{\Rightarrow \delta \langle {{\bf A}}_\omega \rangle = \mathcal{C}_{AH'_\omega}^{R}(\omega) \textnormal{ with } \mathcal{C}_{AH'_\omega}^{R}(\omega)}{ = \int_{\R} dt e^{i\omega t} e^{-\eta t} \mathcal{C}_{AH'_\omega}^{R}(t)},
\end{align}

{where the infinitesimal $\eta$ parameter is incorporated } { in order to ensure the correct physical result, ie.} {the retarded response function decays at $t>>1$.} \\

\paragraph{\textbf{Kubo formula for conductivity}}

Consider a system of charged particles, eg. electrons, which is subjected t an external electromagnetic field. The electromagnetic field induces a current and the conductivity is the linear response coefficient. In the general case, the conductivity is a non-local quantity in both time and space, such that the electric current ${{\bf J}_e}$ at some point ${\bf x}$ at time $t$ depends on the electric field at points ${\bf y}$ at times $t'$, eg.\footnote{Note that this section's mathematical treatment is not Lorentz-covariant. This is, the spacetime is treated as $\R^4$ with the metric being $g_{\mu \nu} = \delta_{\mu \nu}$} 

\begin{equation}
    J^{\alpha}_{e}(x^\mu) = \int_{\Omega \subset \R^4 } dx^{\mu} \sum_{\alpha \beta} \sigma_{\beta}(x^\mu, y^{\mu}) E^{\beta} (y^{\mu}), \begin{array}{c} 
         \textnormal{ where $\sigma_{\alpha \beta}(x^\mu, y^{\mu})$ is the conductivity tensor, which  } \\ 
         \textnormal{ describes the current response in the } \\
         \textnormal{ ${\bf e}_\alpha$-direction to an applied }
         \textnormal{ electric field in the ${\bf e}_\beta$-direction.}
    \end{array}
\end{equation}

The electric field ${\bf E}$ is given by the electric potential $\phi_{ext}$ and the vector potential ${\bf A}_{\textnormal{ext}}$ as $
{\bf E}(x^\mu) = -\partial_{\mu} A^{\mu}$. For electrons, the current density can be written as ${\bf J}_e = -e \langle {\bf J} \rangle$. The perturbing term in the Hamiltonian due to the external electromagnetic field is given by the coupling of the electrons to both the scalar potential and the vector potential. Then, upto linear order in the external potential, 

$$
\Hamiltonian_{\textnormal{ext}} = -e \int_{\R^3} d{\bf x} J_{\mu}({\bf x}) A^{\mu}_{\textnormal{ext}}(x^{\mu}).
$$

Let, ${\bf A}_0$ denote the vector potential in the equilibrium ie. prior to the onset of the perturbation ${\bf A}_0(x^{\mu}$ and let ${\bf A}_0(x^{\mu})$ denote the total vector potential. Then, 

$$
   {\bf A}(x^{\mu}) = {\bf A}_{0}(x^{\mu}) +  {\bf A}_{\textnormal{ext}}(x^{\mu}).
$$

The current operator can be decomposed in two components, the diamagnetic and the paramagnetic terms, as follows 

\begin{equation}
    {\bf J} = {\bf J}^{\nabla}({\bf x}) + \frac{e}{m} {\bf A}({\bf x}) \rho({\bf x}).  
\end{equation}

For simplicity and using gauge invariance, the external electric potential can be set to zero. The conductivity is most easily expressed in the frequency domain via a Fourier transformation of the perturbation. Since 

\begin{equation} \begin{array}{cc}
     \partial_t \overset{\mathcal{F}}{\rightarrow} -i\omega \blanky & \textnormal{ then } \blanky {\bf A}_{\textnormal{ext}}({\bf x}, \omega) \overset{\mathcal{F}}{\rightarrow}  \frac{1}{i\omega} {\bf E}_{\textnormal{ext}}({\bf x}, \omega) \\
\end{array}
     \Rightarrow \Hamiltonian_{\textnormal{ext}, \omega} = \frac{e}{i\omega} \int_{\Omega \subset \R^3} d{\bf x} \blanky {\bf J}({\bf x}) \cdot {\bf E}_{\textnormal{ext}}({\bf x}, \omega).
\end{equation}

In order to exploit the frequency domain formulation of linear response theory, it is desirable to find the corresponding formula for the conductivity tensor in frequency-space. The conductivity tensor is a property of the equilibrium system and can thus onyly depend on time differences $ \sigma_{\alpha \beta}(x^\mu, y^{\mu}) =  \sigma_{\alpha \beta}({\bf x},{\bf y}, t-t')$. The frequency transform of the conductivity yields

\begin{equation}
    J^{\alpha}_{e}({\bf x}, \omega) = \int_{\Omega \subset \R^3 } d{\bf y} \blanky \sum_{ \beta} \sigma_{\alpha \beta}({\bf x},{\bf y}, \omega) E^{\beta}({\bf y}, \omega).
\end{equation}

Now, given that he external perturbation, written in frequency space, is already linear in the external potential ${\bf E}_{\textnormal{ext}}$, and given that the interest lies only on terms proportional to said perturbation, the conductivity can be rewritten as 

\clearpage

\section{.}

\end{document}
