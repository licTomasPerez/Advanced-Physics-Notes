\documentclass{homework}
\author{Tomás Pérez}
\class{Condensed Matter Theory - Lecture Notes}
\date{\today}
\title{Theory \& Notes}

\graphicspath{{./media/}}

\begin{document} \maketitle

\section{Cuentitas}

Given a physical system, a density operator for it is a positive semi-definite, self-adjoint operator of trace one acting on the system's Hilbert space, denoted by $\mathds{H}$. The set of all density operators has the structure of a vector space $\mathcal{C}(\mathds{H})$,

$$
\mathcal{C}(\mathds{H}) = \{ \rho \in \textnormal{GL}(N, \mathds{C}) \blanky | \blanky \rho^\dagger = \rho, \blanky \rho \geq 0, \blanky \textnormal{Tr } \rho = 1 \},
$$

where $\textnormal{GL}(N, \mathds{C})$ is the general linear group over the complex number field, whose elements are squared matrices of $N \times N$-dimension. The following statements can then be proved:

\begin{enumerate}
    \item $\mathcal{C}(\mathds{H})$ is a topological space. This is, this space can be imbued with a topology $\mathcal{T}$ which satisfies a set of axioms. 
    \begin{itemize}
        \item In effect, the desired topology may be chosen to be the trivial topology $\mathcal{T} = \{\emptyset, \mathcal{C}(\mathds{H})\}$,
        \item or it may be chosen out to be the discrete topology, ie. any collection of $\tau$-sets, subsets of $\mathcal{C}(\mathds{H})$, so that that $\mathcal{T} = \bigcup \tau$ adheres to the topological space's axioms. 
        \item Another interesting election is to define a metric on this space, allowing for the construction of the metric topology. More on this later.   
    \end{itemize}
    \item $\topospace$ is a Hausdorff space, allowing for the distinction of elements via disjoint neighbourhoods, 
    \item $\topospace$ is a topological manifold. \item $\topospace$ is a differentiable manifold,
    \item and is a Riemannian non-convex manifold
\end{enumerate}

Density operators can either describe pure or mixed states, which are deffined as follows 

\begin{itemize}
    \item Pure states can be written as an outer product of a vector state with itself, this is 
    
    $$
    \rho \textnormal{ is a pure state if } \exists \ket{\psi} \in \hilbert \blanky | \blanky \rho \propto \ket{\psi} \bra{\psi}. 
    $$
    
    In other words, $\rho$ is a rank-one orthogonal projection. Equivalently, a density matrix is a pure state if there exists a unit vector in the Hilbert space such that $\rho$ is the orthogonal projection onto the span of $\psi$. \\
    
    Note as well that 
    
    $$
       \ket{\psi} \bra{\psi} \in \hilbert \otimes \hilbert^{\star}, \textnormal{ but } \hilbert \otimes \hilbert^{\star} \sim \textnormal{End}(\hilbert)
    $$
    
    ie. the tensor space $\hilbert \otimes \hilbert^{\star}$ is canonically isomorphic to the vector space of endormorphisms in $\hilbert$, ie. to the space of linear operators from $\mathds{H}$ to $\mathds{H}$.
    It's important to note that this isomorphism is only strictly valid in finite-dimensional Hilbert spaces, wherein for infinite-dimensional Hilbert spaces, the isomorphism holds as well provided the density operators are redefined as being trace-class.
    \item Mixed states do not adhere to the previous properties. 
\end{itemize}

Let $\mathcal{B}$ be the set of all operators which are endomorphisms on $\statespace$, ie. 

$$
\mathcal{B} = \{{\bf O} | \blanky {\bf O}: \statespace \rightarrow \statespace\}.
$$

Note that, by definition, $\statespace \subset \mathcal{B}$. Consider an $N$-partite physical system, then its associated Hilbert space will have $\mathcal{O}(2^N)$ dimension and its associated density operator space will have $\mathcal{O}(2^{2N})$ dimension. Then, all linear operators acting on $\statespace$ can be classified as $k$-body operators, with $k \leq N$. This is, in essence, operators whose action is non-trivial only for a total of $k$ particles. Therefore, the $N$-partite Hilbert space can be written as 

$$
    \hilbert = \bigotimes_{j=1}^{N} \mathfrak{H}_j,
$$ 

where $\mathfrak{H}_j$ is the $j$-th subsytem's Hilbert space. This definition thus allows for systems with different particles species (eg. fermions, bosons, spins etc.). Then, 

$$
\mathcal{B}_1(\mathds{H})=\{\hat{\bf O}| \hat{\bf O}:\mathfrak{H}_j \rightarrow \mathfrak{H}_j, \blanky \forall j \leq N \}
$$ 

is the space of all one-body operators. Then the
space of $k$-body operators can be recursively defined in terms of this set, 

\begin{align*}
\mathcal{B}_k(\mathds{H}) = \{\otimes_{i=1}^{k} {\bf O}_i | {\bf O}_i \in \mathcal{B}_1(\mathds{H}) \}, \textnormal{ where } \mathcal{B}(\mathds{H}) = \bigsqcup_{i=1}^{N} \mathcal{B}_i(\mathds{H}).
\end{align*}

%Let $(\mathcal{B}(\mathds{H}), ||\cdot||)$ denote the space of all linear operators acting on the Hilbert space, noting that the sub-space of all linear bounded operators is a Banach space. 

If $\hilbert$ is a Hilbert space and $A \in \mathcal{B}$ is a non-negative self-adjoint operator on $\hilbert$, then it can be shown that $A$ has a well-defined, but possible infinite, trace. Now, if ${\bf A}$ is a bounded operator, then ${\bf A}^\dagger {\bf A}$ is self-adjoint and non-negative. An operator ${\bf A}$ is said to be Hilbert-Schmidt if $\textnormal{Tr } \bf A^\dagger {\bf A} < \infty$. Naturally, the space of all Hilbert-Schmidt operators form a vector space, labelled by $\mathfrak{H}\mathfrak{s}(\hilbert)$. Then, the Hilbert Schmidt inner product can be defined as 

\begin{align*}
\langle \cdot , \cdot \rangle_{\textnormal{HS}}: \blanky \mathfrak{H}\mathfrak{s}(\hilbert) \times \mathfrak{H}\mathfrak{s}(\hilbert) \rightarrow \mathds{C}, & \textnormal{ where } &
    \begin{array}{c} 
         \langle {\bf A}, {\bf B} \rangle_{\textnormal{HS}} = \textnormal{Tr } \bf A^\dagger {\bf B} \\
         ||{\bf A}||_{\textnormal{HS}} = \sqrt{\textnormal{Tr } \bf A^\dagger {\bf A}}.
    \end{array}
\end{align*}

If the Hilbert space is finite-dimensional, the trace is well defined and if the Hilbert space is infinite-dimensional, then the trace can be proven to be absolutely convergent and independent of the orthonormal basis choice\footnote{In effect, given a non-negative, self-adjoint operator, its trace is always invariant under orthogonal change of basis. Should the trace be a finite number, then it is called a trace class. Any given operator ${\bf A} \in \mathcal{B}$ is trace-class if the non-negative self-adjoint operator $\sqrt{{\bf A^\dagger} {\bf A}}$ is trace class as well. Now, given two Hilbert-Schmidt operators ${\bf A}, {\bf B} \in \mathfrak{H}\mathfrak{s}(\hilbert)$, then the new operator $\bf A^\dagger{\bf B}$ is a trace-class operator, meaning that the sum 

$$
\textnormal{Tr } {\bf A^\dagger}{\bf B} = \sum_{\lambda \in \Lambda} \langle {\bf e}_{\lambda}, {\bf A}^\dagger {\bf B} {\bf e}_{\lambda} \rangle 
$$

is absolutely convergent and the value of the sum is independent of the choice of orthonormal basis $\{{\bf e}_{\lambda}\}_{\lambda \in \Lambda}$. 
}. \\

This inner product implies that $(\mathfrak{H}\mathfrak{s}(\hilbert), \langle \cdot , \cdot \rangle_{\textnormal{HS}})$ is a

\begin{itemize}
    \item inner product space since the norm is the square root of the inner product of a vector and itself ie.
    
    $$
    ||{\bf A}||_{\textnormal{HS}} = \langle {\bf A} , {\bf A} \rangle_{\textnormal{HS}} = \sqrt{\textnormal{Tr } \bf A^\dagger {\bf A}}
    $$
    
    \item and is a normed vector space since the norm is always well defined over $\mathfrak{h}\mathfrak{s}(\hilbert)$. 
\end{itemize}

\colorbox{red}{Acá va un comentario "importante": no tiene sentido que dos vectores estén infinitamente lejos, no?} 

\colorbox{red}{entónces tengo que definir esto producto interno y métrico solo en HS(H) y no sobre B(H)} \\

Now, every inner product space is a metric space. In effect, since the function 

\begin{align*}
    \begin{array}{c}
         {\bf A} \rightarrow \sqrt{\textnormal{Tr } {\bf A^\dagger}{\bf A}}  \\
         \textnormal{is a well-defined norm}  
    \end{array} & \textnormal{ then }  \begin{array}{c}
         {\bf A}, {\bf B} \overset{d}{\rightarrow} \sqrt{\textnormal{Tr } {\bf A}^{\dagger} {\bf B}}  \\
         \textnormal{is a well-defined distance}
    \end{array}
\end{align*}

\begin{equation*}
\begin{split}
   d_{\textnormal{HS}}(\cdot, \cdot): \blanky \mathfrak{H}\mathfrak{s}(\hilbert) \times \mathfrak{H}\mathfrak{s}(\hilbert) \rightarrow \mathds{R} \\
   d_{\textnormal{HS}}({\bf A}, {\bf B}) = \sqrt{\textnormal{Tr } {\bf A}^\dagger {\bf B}}
\end{split}
\end{equation*}

With this metric thus defined, then $(\mathfrak{H}\mathfrak{s}(\hilbert), d_{\textnormal{HS}})$ is a metric space. Every metric space can be modified, via the completions of its metric, in such a way that 
$(\mathfrak{H}\mathfrak{s}(\hilbert)^{\star}, d_{\textnormal{HS}}^{\star})$ is a complete metric space, in the sense of the convergence of Cauchy series, where $\mathfrak{H}\mathfrak{s}(\hilbert) \subset \mathfrak{H}\mathfrak{s}(\hilbert)^{\star}$. In this particular case, given that the metric over $\mathfrak{H}\mathfrak{s}(\hilbert)$ is always a finite number -having removed those elements with infinite trace-, then it is already complete $\mathfrak{H}\mathfrak{s}(\hilbert) \sim \mathfrak{H}\mathfrak{s}(\hilbert)^{\star}$. Therefore, $\mathfrak{H}\mathfrak{s}(\hilbert)$ is a Hilbert space with respect to the Hilbert-Schmidt inner product $\langle \cdot , \cdot \rangle_{\textnormal{HS}}$ (or with respect to the Hilbert-Schmidt distance $d_{\textnormal{HS}}$). \\

Thus defined, the Hilbert-Schmidt inner product is complex-valued, thus not immediately suited for our calculations.

\begin{theo}
Consider the modified Hilbert-Schmidt product given by 
 
 \begin{align*}
 \begin{array}{c}
    \langle \cdot , \cdot \rangle_{\textnormal{HS}}^{\rho_0}: \blanky \mathfrak{H}\mathfrak{s}(\hilbert) \times \mathfrak{H}\mathfrak{s}(\hilbert) \rightarrow \mathds{R} \\
    \\
    \langle {\bf A}, {\bf B} \rangle_{\textnormal{HS}}^{\rho_0} = \frac{1}{2}\textnormal{Tr } \rho_0 \bf \{A^\dagger, {\bf B}\}
 \end{array} & \textnormal{ where } \rho_0 \in \statespace.
\end{align*}

We claim this is a valid inner product over the space of all linear trace-class endomorphisms on $\hilbert$, $\mathfrak{H}\mathfrak{s}(\hilbert)$.
\end{theo}

\begin{proof}

In order to prove this is a well-defined inner product over the space of trace-class operators, we must prove that it linear in it second argument and sesquilinear in its first argument, hermitian and positive-defined. 

%First and foremost, it is clear that this inner product is real-valued. In effect, consider

%$$
%    \textnormal{Tr } \bigg(\sqrt{\rho}{\bf A} {\bf A}^{\dagger} + {\bf A}^\dagger {\bf A} \sqrt{\rho} \bigg) = \textnormal{Tr } \bigg({\bf A} {\bf A}^{\dagger} \sqrt{\rho} + \sqrt{\rho} {\bf A}^\dagger {\bf A} \bigg)^{\dagger} \textnormal{Tr } \bigg({\bf A} {\bf A}^{\dagger} \sqrt{\rho} + \sqrt{\rho} {\bf A}^\dagger {\bf A} \bigg)4
%   $$

In effect, 

\begin{enumerate}
    \item the linearity and sesquilinearity is self evident. \\
    \item Is it hermitian? Yes
    
    \begin{equation}
    \begin{split}
        \langle {\bf A}, {\bf B} \rangle_{\textnormal{HS}}^{\rho_0} &= \frac{1}{2}\textnormal{Tr } \rho_0 \{{\bf A}^\dagger, {\bf B}\} = \frac{1}{2} \textnormal{Tr } \rho_0 \bigg({\bf A}^\dagger {\bf B} + {\bf B} {\bf A}^\dagger \bigg)  \\
        &= \frac{1}{2} \textnormal{Tr } \rho_0 \bigg( {\bf B}^\dagger {\bf A}  + {\bf A} {\bf B}^\dagger \bigg)^{\dagger} =  \frac{1}{2}\textnormal{Tr } \rho_0 \{{\bf B}^\dagger, {\bf A}\}^\dagger \\
        &= (\langle {\bf B}, {\bf A} \rangle_{\textnormal{HS}}^{\rho_0})^{*}
        \end{split}
    \end{equation} \\
    
    \item Is it positive-defined? Yes, in effect,
    
    \begin{equation}
        \begin{split}
            \langle {\bf A}, {\bf A} \rangle_{\textnormal{HS}}^{\rho_0}  &=  \frac{1}{2}\textnormal{Tr } \rho_0 \{{\bf A}^\dagger, {\bf B}\} = \textnormal{Tr } \sqrt{\rho} \frac{{\bf A} {\bf A}^{\dagger} + {\bf A}^{\dagger} {\bf A}}{2}\sqrt{\rho} \\
        \end{split}
    \end{equation}
\end{enumerate}
\end{proof}

In our context, since the calculations are to be computed via the (modified) Hilbert-Schmidt inner product and metric, only trace-class operators are allowed. Therefore, some redefinitions are needed. 
\begin{equation}
    \begin{array}{c}
        \mathfrak{H}\mathfrak{s}_1(\hilbert)=\{\hat{\bf O} | \hat{\bf O} \in \mathcal{B}_1 \land ||{\bf O}||_{\textnormal{HS}}^{\rho_0} < \infty , \blanky \forall j \leq N \} \\
        \\
        \mathfrak{H}\mathfrak{s}_k(\hilbert)=\{\otimes_{i=1}^k \hat{\bf O}_i | \hat{\bf O}_i \in \mathfrak{H}\mathfrak{s}_1(\hilbert) \blanky \forall i,k \leq N \} 
    \end{array} \begin{array}{c}
         \textnormal{is the space of all} \\
         \textnormal{one-body operators, } \\
         \\
         \textnormal{is the space of all} \\
         \textnormal{$k$-body operators } 
    \end{array} \begin{array}{c}
         \textnormal{and where} \\ 
         \\
         \mathfrak{H}\mathfrak{s}(\hilbert) = \bigsqcup_{k=1}^N \mathfrak{H}\mathfrak{s}_k(\hilbert)
    \end{array}
\end{equation} 

\blanky \\ 

Consider now an $N$-body quantum system, where correlations, entanglements, and interactions are present. Different particle species are allowed. The One-body and Two-body Max-Ent frameworks are defined as follows


\begin{tcolorbox}[title = N\"aive one-body Max-Ent]
    In the N\"aive one-body Max-Ent framework, there are $N$ sets of one-body operators, each one corresponding to one the $N$ subsystems, which are assumed to be the local basis. These operators must be local operators, acting non-trivially in only one Hilbert subspace, and must be trace-class. The framework thus allows for interactions between different particle species since the basis may have different dimension, this is
    
    \begin{equation*}
        \begin{array}{c}
            \{\mathbf{O}_i^{(1)}\}_{i=1}^{n_1}\in \mathfrak{H}\mathfrak{s}_1(\hilbert^{(1)})  \\
            \vdots \\
            \{\mathbf{O}_i^{(\ell)}\}_{i=1}^{n_{\ell}}\in \mathfrak{H}\mathfrak{s}_1(\hilbert^{(\ell)}) \\
            \vdots \\
            \{\mathbf{O}_i^{(N)}\}_{i=1}^{n_N}\in \mathfrak{H}\mathfrak{s}_1(\hilbert^{(N)})
        \end{array} \begin{array}{c}
             \textnormal{ The operators are then }\\
             \textnormal{redefined so that } \\
             \textnormal{they act on the global} \\
             \textnormal{Hilbert space $\hilbert^{\otimes N}$}
        \end{array} \begin{array}{c}
            \mathfrak{b}_1 = \{\mathbf{O}_i^{(1)} \otimes \bigotimes_{k=2}^{N} \mathds{1}^{(k)}\}_{i=1}^{n_1}\subset \mathfrak{H}\mathfrak{s}(\hilbert^{\otimes N}) \\
            \vdots \\
            \mathfrak{b}_\ell =  \{\otimes_{k=1}^{\ell} \mathds{1}^{(k)} \otimes \mathbf{O}_i^{(\ell)} \otimes_{k' = \ell + 1}^{N} \mathds{1}^{(k')}\}_{i=1}^{n_\ell}\subset \mathfrak{H}\mathfrak{s}(\hilbert^{\otimes N})\\
            \vdots \\
            \mathfrak{b}_N = \{ \bigotimes_{k=1}^{N-1} \mathds{1}^{(k)} \otimes  \mathbf{O}_i^{(N)}\}_{i=1}^{n_N}\subset \mathfrak{H}\mathfrak{s}(\hilbert^{\otimes N})
        \end{array}
    \end{equation*}
    
    Then, the one-body Max-Ent basis $\mathfrak{B}_{ME_1}$ is defined as the union of these sets 
    
    $$
        \mathfrak{B}_{ME_1} = \bigsqcup_{k=1}^{N} \mathfrak{b}_k,
    $$
    
    whose dimension is given by the sum of the $\mathfrak{b}$-basis dimensions ie. $\textnormal{dim}(\mathfrak{B}_{ME_1}) = \sum_{k=1}^N \textnormal{dim}(\mathfrak{b}_k) \sim \mathcal{N}$. Then, the one-body Max-Ent states are given by 
    
    \begin{equation}
         \bm{\mathcal{S}}_{ME,1}( \mathfrak{B}_{ME_1}) = \{ \rho \in \mathcal{C}(\mathds{H}) \blanky | \blanky \exists \{\lambda_k\}_{k=1}^{\dim \mathfrak{B}_{ME_1}} \subset \mathds{R}  \textnormal{ such that }  \rho \propto \exp(-\sum_{i}\lambda_i {\bf O}_i) \}.
    \end{equation}
\end{tcolorbox}

\begin{tcolorbox}[title = N\"aive two-body Max-Ent]

Here, similarly to the n\"aive one-body Max-Ent framework,
we have $N$ sets of local one-body operators at our disposal, which must be trace-class and must only non-trivially act in only one Hilbert subspace. If the one-body local operators are  

\begin{equation}
    \begin{array}{c}
            \{\mathbf{O}_i^{(1)}\}_{i=1}^{n_1}\in \mathfrak{H}\mathfrak{s}_1(\hilbert^{(1)})  \\
            \vdots \\
            \{\mathbf{O}_i^{(\ell)}\}_{i=1}^{n_{\ell}}\in \mathfrak{H}\mathfrak{s}_1(\hilbert^{(\ell)}) \\
            \vdots \\
            \{\mathbf{O}_i^{(N)}\}_{i=1}^{n_N}\in \mathfrak{H}\mathfrak{s}_1(\hilbert^{(N)})
        \end{array}, \begin{array}{c}
             \textnormal{ then the global}\\
             \textnormal{one-body operators are}
        \end{array}   \begin{array}{c}
            \mathfrak{b}_1 = \{\mathbf{O}_i^{(1)} \otimes \bigotimes_{k=2}^{N} \mathds{1}^{(k)}\}_{i=1}^{n_1}\subset \mathfrak{H}\mathfrak{s}(\hilbert^{\otimes N})\\
            \vdots \\
            \mathfrak{b}_\ell =  \{\otimes_{k=1}^{\ell} \mathds{1}^{(k)} \otimes \mathbf{O}_i^{(\ell)} \otimes_{k' = \ell + 1}^{N} \mathds{1}^{(k')}\}_{i=1}^{n_\ell}\subset \mathfrak{H}\mathfrak{s}(\hilbert^{\otimes N})\\
            \vdots \\
            \mathfrak{b}_N = \{ \bigotimes_{k=1}^{N-1} \mathds{1}^{(k)} \otimes  \mathbf{O}_i^{(N)}\}_{i=1}^{n_N}\subset \mathfrak{H}\mathfrak{s}(\hilbert^{\otimes N})
        \end{array} 
\end{equation}

\blanky\\

However, unlike the previous case, we now allow for two-body operators to be included. These new sets of two-body operators may be defined as follows

\begin{equation*}
\begin{array}{c}
             \mathfrak{c}_{11} = \{\mathbf{O}_i^{(1)} \mathbf{O}_j^{(1)} \bigotimes_{k=2}^{N} \mathds{1}^{(k)}\}_{\substack{i, j= 1}}^{\Gamma(n_1, n_1)} \blanky | \blanky {\bf O}_i^{(1)} \in \mathfrak{b}_{1}  \\
             \\
             \mathfrak{c}_{12} = \{\mathbf{O}_i^{(1)} \otimes \mathbf{Q}_j^{(2)} \bigotimes_{k=3}^{N} \mathds{1}^{(k)}\}_{\substack{i, j= 1}}^{\Gamma(n_1, n_2)} \blanky | \blanky {\bf O}_i^{(1)} \in \mathfrak{b}_{1}, {\bf Q}_j^{(2)} \in \mathfrak{b}_{2}\\
             \vdots \\
             \mathfrak{c}_{\ell \ell'} =  \{\mathbf{O}_i^{(\ell)} \otimes \mathbf{Q}_j^{(\ell)} \bigotimes_{\substack{k=1 \\
                                   k \neq \ell             \ell'}}^{N} 
            \mathds{1}^{(k)}\}_{\substack{i, j= 1}}^{\Gamma(n_\ell, n_{\ell'})} \blanky | \blanky {\bf O}_i^{(1)} \in \mathfrak{b}_{\ell}, {\bf Q}_j^{(2)} \in \mathfrak{b}_{\ell'},\\
            \vdots \\
            \mathfrak{c}_{NN} = \{\bigotimes_{k=1}^{N-1} \mathds{1}^{(k)} \otimes \mathbf{O}_i^{(N)} \mathbf{O}_j^{(N)}\}_{\substack{i, j= 1}}^{\Gamma(n_N, n_N)} \blanky | \blanky {\bf O}_i^{(N)} \in \mathfrak{b}_{N}  \\
        \end{array}
\end{equation*}

\blanky\\

where $\Gamma(n_a, n_b)$ counts all the possible, non-repeating, order notwithstanding, pair combinations of elements from a $n_a$-cardinality set with elements from a $n_b$-cardinality set. In other words, 

\begin{equation}
    \Gamma(n_a, n_b) = 
        n_a + n_b + \frac{n_a(n_b-1)}{2}
\end{equation} 

Then, the two-body Max-Ent basis is 

$$
    \mathfrak{B}_{ME_2} = \bigsqcup_{k=1}^{N} \mathfrak{b}_k \cup \bigsqcup_{k, k' = 1}^{N} \mathfrak{c}_{kk'},
$$

from which the two-body Max-Ent states are given by 

\begin{equation}
    \bm{\mathcal{S}}_{ME,2} (\mathfrak{B}_{ME_2}) = \{ \rho \in \mathcal{C}(\mathds{H}) \blanky | \blanky \exists \{\lambda_k\}_{k=1}^{\ell \leq N}, \{\gamma_{mn}\}_{\substack{m,n=1\\
                      {m \mbox{\textless} n}}}^{\ell \leq N} \subset \mathds{R} \textnormal{ such that }  \rho \propto \exp(-\sum_{i,j}\lambda_i {\bf O}_i - \gamma_{ij} {\bf O}_i {\bf O}_j ) \}. 
\end{equation}

Note that $\dim \mathfrak{B}_{ME_2} = \sum_{k=1}^N \dim(\mathfrak{b}_k) + \sum_{i, j=1}^N \dim(\mathfrak{c}_{ij}) \sim \mathcal{O}(N^2)$
\end{tcolorbox}

Both of these techniques require substantially fewer parameters than the exact dynamics, which requires $\mathcal{O}(2^{2N})$ complex-valued entries (or alternatively $\mathcal{O}(2^{2N + 1})$ real-valued parameters. 

%which requires $2^{2n} - 1 $

\clearpage

Consider a closed quantum many-body system described by a Hamiltonian ${\bf H}$ and with its initial state, $\rho_0$, given by

$$
\rho_0 = e^{-{\bf K}} \begin{array}{c}
     \textnormal{ where $\rho_0 \in \mathcal{C}(\hilbert^{\otimes N}))$ } \\
     \\
     \textnormal{ and with ${\bf K} \in \mathfrak{H}\mathfrak{s}(\hilbert^{\otimes N})$} \Longleftrightarrow {\bf K} = - \log \rho
\end{array}
$$

The system's time evolution of course governed by the Schr\"odinger equation. 

$$
i \frac{d\rho(t)}{dt} =  [{\bf H}, \rho]
$$

For the time being, consider $\rho_0$ as a one-body Max-Ent state, with respect to some general Max-Ent 1 basis composed of a collection of one-body local operators

\begin{equation}
    \rho \in \bm{\mathcal{S}}_{ME,1}( \mathfrak{B}_{ME_1}) \begin{array}{c}
         \textnormal{which in turn implies that the ${\bf K}$-operator } \\
         \textnormal{ can be uniquely decomposed,} \\
         \textnormal{upto phase factors, as } 
    \end{array}
    {\bf K} = \sum_{\mu}^{\ell} \phi^{\mu}(t) \bm{\mathcal{O}}_{\mu},
\end{equation}

where we have chosen the Schrodinger picture for the operators. Now we claim the following 

\begin{theo}
    Since $\rho = e^{-{\bf K}}$ is a well-defined density operator $\rho_0 \in \mathcal{C}(\hilbert^{\otimes N}))$, the ${\bf K}$-operator's time evolution is governed by a Schr\"odinger equation as well, this is 
    
    $$
        i \frac{d{\bf K}}{dt} =  [{\bf K}, \rho].
    $$
\end{theo}

\begin{proof}
    If $\rho = e^{-{\bf K}}$ then, by definition 
    
    \begin{equation}
    \begin{split}
            &\rho = \mathds{1} - {\bf K} + \frac{1}{2}{\bf K}^2 - \frac{1}{3!} {\bf K}^3 + \cdots \\
            &d\rho = 0 - d{\bf K} + \frac{1}{2} \bigg(d{\bf K} {\bf K} + {\bf K} d{\bf K} \bigg) - \frac{1}{3!} \bigg((d{\bf K}) {\bf K}^2 + {\bf K} (d{\bf K}) {\bf K} + {\bf K}^2 (d{\bf K}) \bigg) + \cdots 
    \end{split}
    \end{equation}
    
    which, if we are willing to assume that $[{\bf K},d{\bf K}] = 0$, yields 
    
    \begin{equation}
    \begin{split}
        d\rho &= 0 - d{\bf K} + \frac{1}{2} \bigg(d{\bf K} {\bf K} + {\bf K} d{\bf K} \bigg) - \frac{1}{3!} \bigg((d{\bf K}) {\bf K}^2 + {\bf K} (d{\bf K}) {\bf K} + {\bf K}^2 (d{\bf K}) \bigg) + \cdots \\
        &= - d{\bf K} + {\bf K}d{\bf K} - \frac{1}{2} {\bf K}^2 d{\bf K} + \cdots \\
        &= -\bigg( \mathds{1} - {\bf K} + \frac{1}{2} {\bf K}^2 + \cdots\bigg) d{\bf K} = -e^{-{\bf K}} d{\bf K} \\
        &\Rightarrow \frac{d\rho}{dt} = -e^{-{\bf K}} \frac{d{\bf K}}{dt},
    \end{split}
    \end{equation}
    
    and given that $\rho$'s time evolution is governed by the Schr\"odinger equation, this yields
    
    \begin{equation}
        \begin{split}
            &i\frac{d\rho}{dt} = [{\bf H}, \rho] \\
            & i e^{-{\bf K}} \frac{d{\bf K}}{dt}  = [{\bf H}, e^{-{\bf K}}]
        \end{split}
    \end{equation}
    
    \notaFTBP{Acá algunas cosas me hacen ruido. Primero, la identidad $\frac{d\rho}{dt} = -e^{-{\bf K}} \frac{d{\bf K}}{dt}$ se mantiene sí y solo sí asumo que $dK$ y $K$ conmutan, lo cual en general no es el caso. Por otro lado, si intento hacer la cuenta por el lado de la derivada del logaritmo tengo que $d \log \rho/dt = \rho^{-1} d\rho/dt $}
    
    \begin{align*}
        \frac{d}{dt} \log \rho(t) &= \lim_{\Delta t \rightarrow 0 } \frac{\log [ \rho + \rho' \Delta t ] - \log \rho}{\Delta t} \\
        &= \lim_{\Delta t \rightarrow 0 } \frac{\log [ \rho \rho^{-1} + \rho' \rho^{-1} \Delta t ]}{\Delta t} \\
        &=  \lim_{\Delta t \rightarrow 0 } \frac{\log [ \mathds{1} + \rho' \rho^{-1} \Delta t ]}{\Delta t} \\
        &= \lim_{\Delta t \rightarrow 0 } {\log \bigg([ \mathds{1} + \rho' \rho^{-1} \Delta t ]}\bigg)^{\frac{1}{\Delta t}} \\
        &= \lim_{\sigma \rightarrow 0 } {\log \bigg([ \mathds{1} + \sigma ]}\bigg)^{\rho' \rho^{-1}\sigma^{-1}} \textnormal{ where $\sigma = \rho' \rho^{-1} \Delta t$ } \\
        &= \rho' \rho^{-1} \lim_{\sigma \rightarrow 0 } {\log \bigg([ \mathds{1} + \sigma ]}\bigg)^{\sigma^{-1}} \\
        &= \rho' \rho^{-1} \lim_{\sigma \rightarrow 0 } \log e \\
        &=  \rho' \rho^{-1}
    \end{align*}
    \notaFTBP{ pero lo que no me gusta de acá es que es que usé que $\log AB^{-1} = \log A - \log B$, que vale solo si conmutan y si $\rho' \rho^{-1}$ conmutan dentro del límite. Me hace ruido}
\end{proof}

\end{document}
