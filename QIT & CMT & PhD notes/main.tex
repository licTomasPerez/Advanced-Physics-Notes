\documentclass{homework}
\author{Tomás Pérez}
\class{Condensed Matter Theory - Lecture Notes}
\date{\today}
\title{Theory \& Notes}

\graphicspath{{./media/}}

\begin{document} \maketitle

\section{Some topology definitions and results}

First, consider $X$ to be a topological space.\\

\paragraph{\textbf{Path-connected}}

A \textbf{\underline{path}} from a point $p$ to a point $q$ in a topological space $X$ is a continuous function $f: \mathds{R}_{[0,1]} \to X$, with $f(0) = x$ and $f(1) = y$. A \textbf{\underline{path-component}} of $X$ is an equivalence class of $X$ under the equivalence relation which makes $x$ equivalent to $y$ is there there is a path from $x$ to $y$. Hence, $X$ is said to be \textbf{\underline{path-connected}} if there is exactly one-path component, ie. if there is a path joining any two points $X$. Some important results regarding path-connectedness are enunciated, as follows.

\begin{itemize}
    \item Let $X$ and $Y$ be topological spaces and let $f: X \to Y$ be a continuous function. If $X$ is path-connected then its image $f(X)$ is path-connected as well. 
    \item Every path-connected space is connected
    \item The closure of a connected subset is connected. Furthermore, any subset between a connected subset and its closure is connected. 
    \item Every product of a family of path-connected  spaces is path-connected.
    \item Every manifold is locally path-connected. \\
\end{itemize}

\paragraph{\textbf{Simply connected}}

$X$ is \textbf{\underline{simply connected}} (or 1-connected) if it is path-connected and every path between two points can be continuously transformed (intuitively for embedded spaces, staying within the space) into any other such path, while preserving the two endpoints in question. The fundamental group $\pi_1(X)$ is an indicator of the failure of a topological space to be simply connected: a path-connected topological space is simply connected if and only if $\pi_1(X) \simeq \mathds{Z}$. \\

An equivalent definition is: $X$ is called simply-connected if

\begin{itemize}
    \item it is path-connected, 
    \item and any loop in $X$ defined by $f: S^{1} \to X$ can be contracted to a point; ie. $\exists F: D^{2} \to X \textnormal{ such that } F\bigg|_{S^{1}} = f$, where ${S^{1}}$ denotes the unit circle and $D^2$ denotes the closed unit disk in the Euclidean plane respectively. In other words, there exists a continuous map from the closed unit disk to $X$ such that $F$ restricted to $S^1$ is precisely $f$.
\end{itemize}

An equivalent formulation is this: $X$ is simply connected if and only if 

\begin{itemize}
    \item it is path-connected,
    \item and whenever two arbitrary paths (ie. continuous maps) $p: \mathds{R}_{[0,1]} \to X$ and $q: \mathds{R}_{[0,1]} \to X$ with the same endpoints, $p(0) = q(0) \textnormal{ and } p(1) = q(1)$, can be continuously deformed into one another. Explicitly, there exists a homotopy $F: \mathds{R}_{[0,1]} \to X$ such that $F(x,0) = p(x)$ and $F(x,1) = q(x)$. \\
\end{itemize}

Equivalently, $X$ is simply connected if and only if $X$ is path-connected and $\pi_1 (X) \simeq \mathds{Z}$ at each point. \\

\paragraph{\textbf{Contractibility}}

 Then, said topological space $X$ is \underline{\textbf{contractible}} if the identity map on $X$ is null-homotopic, ie. if it is homotopic to some constant map. Intuitively, a contractible space is one that can be continously shrunk to a point within that space.

A contractible space is precisely one with the homotopy type of a point. It follows that all the the associated homotopy groups of a contractible space are trivial. Therefore, any space with a non-trivial homotopy group cannot be contractible. For a topological space $X$, the following statements are all equivalent:

\begin{itemize}
    \item $X$ is contractible (ie. the identity map is null-homotopic). 
    \item $X$ is homotopy-equivalent to a one-point space. 
    \item For any space $Y$, any two maps $f, g: Y \rightarrow X$ are homotopic.
    \item For any space $Y$, any map $f: Y \rightarrow X$ is null-homotopic. 
    \item The cone on a space $X$ is always contractible. Therefore any space can be embedded in a contractible one (which also illustrates that subspaces of contractible spaces need not be contractible themselves).
    \item $X$ is a contractibe if and only if there exists a retraction from the cone of $X$ to $X$.
    \item Every contractible space is path-connected and simply connected. Moreover, since all the higher homotopy groups, vanish. Every contractible space is $n$-connected, for all $n \geq 0$.\\ 
\end{itemize}

\paragraph{\textbf{Homotopy invariants}}

Formally, a homotopy between two continuous functions $f, g: X \to Y$, wherein both $X$ and $Y$ are topological spaces, is defined to be a continuous function $H: X \times \mathds{R}_{[0,1]} \to Y$, from the product of the space $X$ with the unit interval to $Y$, such that 

$$
    H(x,0) = f(x) \textnormal{ and } H(x, 1) = g(x), \forall x \in X 
$$

In practical terms, if $H$'s second argument is thought of as time, when $H$ describes a continuous deformation of $f$ into $g$, at time 0 we have the function $f$ and at time 1 we have the function $g$, in such a way that there is a smooth transition from $f$ to $g$. \\

Given two topological spaces $X$ and $Y$, a homotopy equivalence between $X$ and $Y$ is a pair of continuous maps $f: X \to Y$ and $g: Y \to X$, such that $ g \circ f $ is homotopic to the identity map on $X$, $\textnormal{id}_X$ and $ f \circ g $ is homotopic to the identity map on $Y$, $\textnormal{id}_Y$. If such a pair exists, then $X$ and $Y$ are homotopy-equivalent, o r of the same homotopy type. Intuitively, two space X and Y are homotopy equivalent if they can be transformed into one another by bending, shrinking and expanding operations. Spaces that are homotopy-equivalent to a point are called contractible.

Note that a homeomorphism is a special case of a homotopy equivalence, in which $g \circ f$ is exactly equal to the identity map $\textnormal{id}_X$ (not only homotopic to it), and $f \circ g$ is equal to $\textnormal{id}_Y$. Therefore, if $X$ and $Y$ are homeomorphic then they are homotopy-equivalent, but the opposite is not true. Some examples:

\begin{itemize}
    \item A solid disk is homotopy-equivalent to a single point, since you can deform the disk along radial lines continuously to a single point. However, they are not homeomorphic, since there is no bijection between them (since one is an infinite set, while the other is finite).
    \item The Möbius strip and an untwisted (closed) strip are homotopy equivalent, since you can deform both strips continuously to a circle. But they are not homeomorphic. \\
\end{itemize}

A function $f$ is said to be null-homotopic if it is homotopic to a constant function. (The homotopy from $f$ to a constant function is then sometimes called a null-homotopy.) For example, a map $f$ from the unit circle $S^1$ to any space $X$ is null-homotopic precisely when it can be continuously extended to a map from the unit disk $D^2$ to $X$ that agrees with $f$ on the boundary. \underline{It follows from these definitions that a space $X$ is contractible if and only if the}
\underline{ identity map from $X$ to itself—which is always a homotopy equivalence—is null-homotopic.} \\

In algebraic topology, there are many interesting homotopic invariants. Namely, let $X, Y$ be topological spaces, then 

\begin{itemize}
    \item $X$ is path-connected if and only if Y is.
    \item $X$ is simply connected if and only if Y is.
    \item if $X$ and $Y$ are path-connected, then the fundamental groups of $X$ and $Y$ are isomorphic, and so are the higher homotopy groups. \\
\end{itemize}

\paragraph{\textbf{Hopf-Rinow theorem}}

Let $(M,g)$ be a connected Riemannian manifold. Then, the following statements are equivalent

\begin{itemize}
    \item The closed and bounded subsets of $M$ are compact,
    \item $M$ is a complete metric space, 
    \item $M$ is geodesically complete, that is, $\forall p \in M$, the exponential map $\mathfrak{e}\mathfrak{x}\mathfrak{p}_{p}$ is defined on the entire tangent space $T_p M$,
\end{itemize}

Furthermore, any one of the above implies that, given any two points $p, q \in M$, there exists a length-minimizing geodesic connecting these two points. \\

\section{Cuentitas}

Given a physical system, a density operator for it is a positive semi-definite, self-adjoint operator of trace one acting on the system's Hilbert space, denoted by $\mathds{H}$. The set of all density operators has the structure of a vector space $\mathcal{C}(\mathds{H})$,

$$
\mathcal{C}(\mathds{H}) = \{ \rho \in \textnormal{GL}(N, \mathds{C}) \blanky | \blanky \rho^\dagger = \rho, \blanky \rho \geq 0, \blanky \textnormal{Tr } \rho = 1 \},
$$

where $\textnormal{GL}(N, \mathds{C})$ is the general linear group over the complex number field, whose elements are squared matrices of $N \times N$-dimension. The following statements can then be proved:

\begin{enumerate}
    \item $\mathcal{C}(\mathds{H})$ is a topological space. This is, this space can be imbued with a topology $\mathcal{T}$ which satisfies a set of axioms. 
    \begin{itemize}
        \item In effect, the desired topology may be chosen to be the trivial topology $\mathcal{T} = \{\emptyset, \mathcal{C}(\mathds{H})\}$,
        \item or it may be chosen out to be the discrete topology, ie. any collection of $\tau$-sets, subsets of $\mathcal{C}(\mathds{H})$, so that that $\mathcal{T} = \bigcup \tau$ adheres to the topological space's axioms. 
        \item Another interesting election is to define a metric on this space, allowing for the construction of the metric topology. More on this later.   
    \end{itemize}
    \item $\topospace$ is a Hausdorff space, allowing for the distinction of elements via disjoint neighbourhoods, 
    \item $\topospace$ is a topological manifold. \item $\topospace$ is a differentiable manifold,
    \item and is a Riemannian non-convex manifold
\end{enumerate}

Let $\Lambda$ be a $d$-dimensional quantum spin system, with its lattice being defined as $\mathds{L} = \mathds{Z}^d$ so that $\Lambda \subset \mathds{Z}^d$. Its single-spin space is a probability space $(S, {\bm S}, \lambda)$ where $S = \{\pm 1\}$. A spin chain's regular crystalline structure can be viewed as a finite, non-oriented. Let $\mathfrak{H}\mathfrak{s}(\hilbert^{\otimes N})$ be the set of all trace-class operators, endomorphisms acting on the $N$-partite Hilbert space. By construction, this set is, in and on itself, a Hilbert space. In particular, consider the subset of all trace-class hermitian operators, labelled 

\begin{equation}
    \mathfrak{H}\mathfrak{s}^{\dagger}(\hilbert^{\otimes N}) \subset \mathfrak{H}\mathfrak{s}(\hilbert^{\otimes N}), \textnormal{ where } {\bf K} \in \mathfrak{H}\mathfrak{s}^{\dagger}(\hilbert^{\otimes N}) \textnormal{ if and only if } \begin{array}{c}
         {\bf K}: \hilbert^{\otimes N} \rightarrow \hilbert^{\otimes N} \\
         {\bf K} = \bf K^\dagger  \\
         \Tr \bf K^\dagger {\bf K} = \Tr \bf K^2 < \infty 
    \end{array}
\end{equation}

We claim the following

\begin{theo}
    Consider a finite dimensional system, eg. spin systems, or truncated boson/fermion systems. Then, let $\mathcal{M} = \bigg\{ \rho \in \mathcal{C}(\mathds{H}^{\otimes N}) \blanky | \blanky \exists \{\mu\}_{k=1}^{\ell} \subset \mathds{R} \land \exists {\bf K} \in \mathfrak{H}\mathfrak{s}^{\dagger}(\hilbert^{\otimes N}) \textnormal{ such that } \rho \propto e^{-{\bf K}} \bigg\}$. Then $M$ is a Riemannian manifold $(M, B)$, where $B$ is the Bures metric. 
\end{theo}

\begin{proof}

Special care must be taken when dealing with infinite-dimensional systems. We start our proof by concerning ourselves to finite-dimensional physical systems, eg. spins, truncated bosonic and fermionic systems. \\

\paragraph{\textbf{Finite dimensional case}}

It is clear that, for the finite dimensional case, $\mathcal{M}$ is a path-connected topological space. In effect, 

$$
    \forall \rho_1, \rho_2 \in \mathcal{M}, \blanky \exists f:\mathds{R}_{[0,1]} \rightarrow \mathcal{M} \textnormal{ such that } \begin{array}{c}
         f(0) = \rho_1, \\
         f(1) = \rho_2
    \end{array}
$$

This path always exists. In effect, consider the function 

\begin{align*}
    f&:\mathds{R}_{[0,1]} \rightarrow \mathcal{M} \\
    &f(x) = x \rho_1 + (1-x) \rho_2,
\end{align*}

which is continuous for arbitrary density matrices. 




In other words, the Hopf-Rinow theorem assures that, given any two arbitrary point $\rho, \sigma \in \mathcal{M}$, there always exists a length-minimizing geodesic connecting these two-points\footnote{Note that these conclusions only hold for finite dimensional manifolds. The theorem does not hold for infinite dimensional complete Hilbert manifolds. Note that, in our case, the Hilbert space $\hilbert^{\otimes N}$ can be thought of as a Hilbert manifold, with a single global chart given by the identity function $\mathds{1}_{\hilbert^{\otimes N}}$ on said Hilbert space. Moreover, since by definition a Hilber tspace is a vector space, the tangent space $T_{\rho} \mathcal{M}$ to $\mathcal{M}$ at the point $\rho \in \hilbert^{\otimes}$ is canonically isomorphic to the Hilbert space itself, and thus has a natural inner product, the same as the one defined on the Hilbert space. Thus $\hilbert^{\otimes N}$ can be given the structure of a Riemannian manifold with metric 

$$
    g(v, w) (\rho) = \langle v, w \rangle_{\hilbert^{\otimes N}}, \textnormal{ for } v, w \in T_{\rho} \mathcal{M}
$$

wherein $\langle \cdot, \cdot \rangle_{\hilbert^{\otimes N}}$ denotes the inner product in $\hilbert^{\otimes N}$
}. 

By construction, it is clear that the natural choice for the 
tangent space $T_\rho \mathcal{M}$ is $\mathfrak{H}\mathfrak{s}^{\dagger}(\hilbert^{\otimes N})$

$$
    T_\rho \mathcal{M} \simeq \mathfrak{H}\mathfrak{s}^{\dagger}(\hilbert^{\otimes N})
$$

, for all $\rho \in \mathcal{M}$. Therefore, the tangent bundle, a vector bundle made up of copies of $\mathfrak{H}\mathfrak{s}^{\dagger}(\hilbert^{\otimes N})$, can be written as 

$$
    {\displaystyle {\begin{aligned}T\mathcal{M}&=\bigsqcup _{\rho \in \mathcal{M}}T_{\rho}\mathcal{M}\\&=\bigcup _{\rho\in \mathcal{M}}\left\{\rho\right\}\times T_{\rho}\mathcal{M}\\&=\bigcup_{\rho\in \mathcal{M}}\left\{(\rho,{\bf O})\mid {\bf O}\in T_{\rho} \mathcal{M}\right\}\\&=\left\{(\rho,{\bf O})\mid \rho\in \mathcal{M},{\bf O}\in T_{\rho} \mathcal{M}\right\}\end{aligned}}}
$$

\end{proof}


\clearpage

Consider the set of all density operators, labelled $\mathcal{C}(\mathds{H}^{\otimes N})$, we are interested in two subsets of these, the Max-Ent manifolds




By definition, $\mathcal{C}(\mathds{H}^{\otimes N}) \subset \textnormal{GL}(N, \mathds{C})$. In particular, we claim that it has the structure of a Riemannian non-convex manifold. All of these density matrices can be written as the exponential of a hermitian ${\bf K}$-operator,

$$
    {\bf K} \in \textnormal{End}(\mathcal{C}(\mathds{H}^{\otimes N})) \textnormal{  and  } {\bf K}^\dagger = {\bf K}.
$$

If a basis $\mathcal{B}$ of trace-class operators is chosen, then

$$
    \mathcal{B} = \{{\bf O}_i\}_{i=1}^{\dim \mathcal{B}} \Rightarrow {\bf K} = \sum_{i} \alpha_i {\bf O}, \alpha_i \in \mathds{C}
 $$
 
so that there exists a mapping from the space of all trace-class endomorphisms to the space of all Max Ent-type density matrices, said mapping being the exponential mapping.

\begin{align*}
    &\mathfrak{e}\mathfrak{x}\mathfrak{p}: T_{\alpha} \mathcal{M} \rightarrow \mathcal{M} \\
    &\mathfrak{e}\mathfrak{x}\mathfrak{p}({\bf K}) \rightarrow e^{{\bf K}} \textnormal{ so long as } {\bf K} \in \textnormal{End}(\mathcal{C}(\mathds{H}^{\otimes N})) \textnormal{ and } {\bf K} = {\bf K}^{\dagger}
\end{align*}

The exponential mapping is well defined for all trace-class hermitian operators since the manifold is path-connected. In effect, 


\begin{align}
    \forall \rho_1, \rho_2 \in \mathcal{\bf S}_{\textnormal{ME2}}     
\end{align}

\clearpage 

For spin-systems, the Hilbert-Schmidt space, which in and of itself is a Hilbert space with respect to the Hilbert-Schmidt inner-product, is finite-dimensional. 

\clearpage

Density operators can either describe pure or mixed states, which are deffined as follows 

\begin{itemize}
    \item Pure states can be written as an outer product of a vector state with itself, this is 
    
    $$
    \rho \textnormal{ is a pure state if } \exists \ket{\psi} \in \hilbert \blanky | \blanky \rho \propto \ket{\psi} \bra{\psi}. 
    $$
    
    In other words, $\rho$ is a rank-one orthogonal projection. Equivalently, a density matrix is a pure state if there exists a unit vector in the Hilbert space such that $\rho$ is the orthogonal projection onto the span of $\psi$. \\
    
    Note as well that 
    
    $$
       \ket{\psi} \bra{\psi} \in \hilbert \otimes \hilbert^{\star}, \textnormal{ but } \hilbert \otimes \hilbert^{\star} \sim \textnormal{End}(\hilbert)
    $$
    
    ie. the tensor space $\hilbert \otimes \hilbert^{\star}$ is canonically isomorphic to the vector space of endormorphisms in $\hilbert$, ie. to the space of linear operators from $\mathds{H}$ to $\mathds{H}$.
    It's important to note that this isomorphism is only strictly valid in finite-dimensional Hilbert spaces, wherein for infinite-dimensional Hilbert spaces, the isomorphism holds as well provided the density operators are redefined as being trace-class.
    \item Mixed states do not adhere to the previous properties. 
\end{itemize}

Let $\mathcal{B}$ be the set of all operators which are endomorphisms on $\statespace$, ie. 

$$
\mathcal{B} = \{{\bf O} | \blanky {\bf O}: \statespace \rightarrow \statespace\}.
$$

Note that, by definition, $\statespace \subset \mathcal{B}$. Consider an $N$-partite physical system, then its associated Hilbert space will have $\mathcal{O}(2^N)$ dimension and its associated density operator space will have $\mathcal{O}(2^{2N})$ dimension. Then, all linear operators acting on $\statespace$ can be classified as $k$-body operators, with $k \leq N$. This is, in essence, operators whose action is non-trivial only for a total of $k$ particles. Therefore, the $N$-partite Hilbert space can be written as 

$$
    \hilbert = \bigotimes_{j=1}^{N} \mathfrak{H}_j,
$$ 

where $\mathfrak{H}_j$ is the $j$-th subsytem's Hilbert space. This definition thus allows for systems with different particles species (eg. fermions, bosons, spins etc.). Then, 

$$
\mathcal{B}_1(\mathds{H})=\{\hat{\bf O}| \hat{\bf O}:\mathfrak{H}_j \rightarrow \mathfrak{H}_j, \blanky \forall j \leq N \}
$$ 

is the space of all one-body operators. Then the
space of $k$-body operators can be recursively defined in terms of this set, 

\begin{align*}
\mathcal{B}_k(\mathds{H}) = \{\otimes_{i=1}^{k} {\bf O}_i | {\bf O}_i \in \mathcal{B}_1(\mathds{H}) \}, \textnormal{ where } \mathcal{B}(\mathds{H}) = \bigsqcup_{i=1}^{N} \mathcal{B}_i(\mathds{H}).
\end{align*}

%Let $(\mathcal{B}(\mathds{H}), ||\cdot||)$ denote the space of all linear operators acting on the Hilbert space, noting that the sub-space of all linear bounded operators is a Banach space. 

If $\hilbert$ is a Hilbert space and $A \in \mathcal{B}$ is a non-negative self-adjoint operator on $\hilbert$, then it can be shown that $A$ has a well-defined, but possible infinite, trace. Now, if ${\bf A}$ is a bounded operator, then ${\bf A}^\dagger {\bf A}$ is self-adjoint and non-negative. An operator ${\bf A}$ is said to be Hilbert-Schmidt if $\textnormal{Tr } \bf A^\dagger {\bf A} < \infty$. Naturally, the space of all Hilbert-Schmidt operators form a vector space, labelled by $\mathfrak{H}\mathfrak{s}(\hilbert)$. Then, the Hilbert Schmidt inner product can be defined as 

\begin{align*}
\langle \cdot , \cdot \rangle_{\textnormal{HS}}: \blanky \mathfrak{H}\mathfrak{s}(\hilbert) \times \mathfrak{H}\mathfrak{s}(\hilbert) \rightarrow \mathds{C}, & \textnormal{ where } &
    \begin{array}{c} 
         \langle {\bf A}, {\bf B} \rangle_{\textnormal{HS}} = \textnormal{Tr } \bf A^\dagger {\bf B} \\
         ||{\bf A}||_{\textnormal{HS}} = \sqrt{\textnormal{Tr } \bf A^\dagger {\bf A}}.
    \end{array}
\end{align*}

If the Hilbert space is finite-dimensional, the trace is well defined and if the Hilbert space is infinite-dimensional, then the trace can be proven to be absolutely convergent and independent of the orthonormal basis choice\footnote{In effect, given a non-negative, self-adjoint operator, its trace is always invariant under orthogonal change of basis. Should the trace be a finite number, then it is called a trace class. Any given operator ${\bf A} \in \mathcal{B}$ is trace-class if the non-negative self-adjoint operator $\sqrt{{\bf A^\dagger} {\bf A}}$ is trace class as well. Now, given two Hilbert-Schmidt operators ${\bf A}, {\bf B} \in \mathfrak{H}\mathfrak{s}(\hilbert)$, then the new operator $\bf A^\dagger{\bf B}$ is a trace-class operator, meaning that the sum 

$$
\textnormal{Tr } {\bf A^\dagger}{\bf B} = \sum_{\lambda \in \Lambda} \langle {\bf e}_{\lambda}, {\bf A}^\dagger {\bf B} {\bf e}_{\lambda} \rangle 
$$

is absolutely convergent and the value of the sum is independent of the choice of orthonormal basis $\{{\bf e}_{\lambda}\}_{\lambda \in \Lambda}$. 
}. \\

This inner product implies that $(\mathfrak{H}\mathfrak{s}(\hilbert), \langle \cdot , \cdot \rangle_{\textnormal{HS}})$ is a

\begin{itemize}
    \item inner product space since the norm is the square root of the inner product of a vector and itself ie.
    
    $$
    ||{\bf A}||_{\textnormal{HS}} = \langle {\bf A} , {\bf A} \rangle_{\textnormal{HS}} = \sqrt{\textnormal{Tr } \bf A^\dagger {\bf A}}
    $$
    
    \item and is a normed vector space since the norm is always well defined over $\mathfrak{h}\mathfrak{s}(\hilbert)$. 
\end{itemize}

\colorbox{red}{Acá va un comentario "importante": no tiene sentido que dos vectores estén infinitamente lejos, no?} 

\colorbox{red}{entónces tengo que definir esto producto interno y métrico solo en HS(H) y no sobre B(H)} \\

Now, every inner product space is a metric space. In effect, since the function 

\begin{align*}
    \begin{array}{c}
         {\bf A} \rightarrow \sqrt{\textnormal{Tr } {\bf A^\dagger}{\bf A}}  \\
         \textnormal{is a well-defined norm}  
    \end{array} & \textnormal{ then }  \begin{array}{c}
         {\bf A}, {\bf B} \overset{d}{\rightarrow} \sqrt{\textnormal{Tr } {\bf A}^{\dagger} {\bf B}}  \\
         \textnormal{is a well-defined distance}
    \end{array}
\end{align*}

\begin{equation*}
\begin{split}
   d_{\textnormal{HS}}(\cdot, \cdot): \blanky \mathfrak{H}\mathfrak{s}(\hilbert) \times \mathfrak{H}\mathfrak{s}(\hilbert) \rightarrow \mathds{R} \\
   d_{\textnormal{HS}}({\bf A}, {\bf B}) = \sqrt{\textnormal{Tr } {\bf A}^\dagger {\bf B}}
\end{split}
\end{equation*}

With this metric thus defined, then $(\mathfrak{H}\mathfrak{s}(\hilbert), d_{\textnormal{HS}})$ is a metric space. Every metric space can be modified, via the completions of its metric, in such a way that 
$(\mathfrak{H}\mathfrak{s}(\hilbert)^{\star}, d_{\textnormal{HS}}^{\star})$ is a complete metric space, in the sense of the convergence of Cauchy series, where $\mathfrak{H}\mathfrak{s}(\hilbert) \subset \mathfrak{H}\mathfrak{s}(\hilbert)^{\star}$. In this particular case, given that the metric over $\mathfrak{H}\mathfrak{s}(\hilbert)$ is always a finite number -having removed those elements with infinite trace-, then it is already complete $\mathfrak{H}\mathfrak{s}(\hilbert) \sim \mathfrak{H}\mathfrak{s}(\hilbert)^{\star}$. Therefore, $\mathfrak{H}\mathfrak{s}(\hilbert)$ is a Hilbert space with respect to the Hilbert-Schmidt inner product $\langle \cdot , \cdot \rangle_{\textnormal{HS}}$ (or with respect to the Hilbert-Schmidt distance $d_{\textnormal{HS}}$). \\

Thus defined, the Hilbert-Schmidt inner product is complex-valued, thus not immediately suited for our calculations.

\begin{theo}
Consider the modified Hilbert-Schmidt product given by 
 
 \begin{align*}
 \begin{array}{c}
    \langle \cdot , \cdot \rangle_{\textnormal{HS}}^{\rho_0}: \blanky \mathfrak{H}\mathfrak{s}(\hilbert) \times \mathfrak{H}\mathfrak{s}(\hilbert) \rightarrow \mathds{R} \\
    \\
    \langle {\bf A}, {\bf B} \rangle_{\textnormal{HS}}^{\rho_0} = \frac{1}{2}\textnormal{Tr } \rho_0 \bf \{A^\dagger, {\bf B}\}
 \end{array} & \textnormal{ where } \rho_0 \in \statespace.
\end{align*}

We claim this is a valid inner product over the space of all linear trace-class endomorphisms on $\hilbert$, $\mathfrak{H}\mathfrak{s}(\hilbert)$.
\end{theo}

\begin{proof}

In order to prove this is a well-defined inner product over the space of trace-class operators, we must prove that it linear in it second argument and sesquilinear in its first argument, hermitian and positive-defined. 

%First and foremost, it is clear that this inner product is real-valued. In effect, consider

%$$
%    \textnormal{Tr } \bigg(\sqrt{\rho}{\bf A} {\bf A}^{\dagger} + {\bf A}^\dagger {\bf A} \sqrt{\rho} \bigg) = \textnormal{Tr } \bigg({\bf A} {\bf A}^{\dagger} \sqrt{\rho} + \sqrt{\rho} {\bf A}^\dagger {\bf A} \bigg)^{\dagger} \textnormal{Tr } \bigg({\bf A} {\bf A}^{\dagger} \sqrt{\rho} + \sqrt{\rho} {\bf A}^\dagger {\bf A} \bigg)4
%   $$

In effect, 

\begin{enumerate}
    \item the linearity and sesquilinearity is self evident. \\
    \item Is it real? Yes
    
    \begin{equation}
    \begin{split}
        \langle {\bf A}, {\bf B} \rangle_{\textnormal{HS}}^{\rho_0} &= \frac{1}{2}\textnormal{Tr } \rho_0 \{{\bf A}^\dagger, {\bf B}\} = \frac{1}{2} \textnormal{Tr } \rho_0 \bigg({\bf A}^\dagger {\bf B} + {\bf B} {\bf A}^\dagger \bigg)  \\
        &= \frac{1}{2} \textnormal{Tr } \rho_0 \bigg( {\bf B}^\dagger {\bf A}  + {\bf A} {\bf B}^\dagger \bigg)^{\dagger} =  \frac{1}{2}\textnormal{Tr } \rho_0 \{{\bf B}^\dagger, {\bf A}\}^\dagger \\
        &= (\langle {\bf B}, {\bf A} \rangle_{\textnormal{HS}}^{\rho_0})^{*}
        \end{split}
    \end{equation} \\
    
    \item Is it positive-defined? Yes, in effect,
    
    \begin{equation}
        \begin{split}
            \langle {\bf A}, {\bf A} \rangle_{\textnormal{HS}}^{\rho_0}  &=  \frac{1}{2}\textnormal{Tr } \rho_0 \{{\bf A}^\dagger, {\bf B}\} = \textnormal{Tr } \sqrt{\rho} \frac{{\bf A} {\bf A}^{\dagger} + {\bf A}^{\dagger} {\bf A}}{2}\sqrt{\rho} \\
        \end{split}
    \end{equation}
\end{enumerate}
\end{proof}

In our context, since the calculations are to be computed via the (modified) Hilbert-Schmidt inner product and metric, only trace-class operators are allowed. Therefore, some redefinitions are needed. 
\begin{equation}
    \begin{array}{c}
        \mathfrak{H}\mathfrak{s}_1(\hilbert)=\{\hat{\bf O} | \hat{\bf O} \in \mathcal{B}_1 \land ||{\bf O}||_{\textnormal{HS}}^{\rho_0} < \infty , \blanky \forall j \leq N \} \\
        \\
        \mathfrak{H}\mathfrak{s}_k(\hilbert)=\{\otimes_{i=1}^k \hat{\bf O}_i | \hat{\bf O}_i \in \mathfrak{H}\mathfrak{s}_1(\hilbert) \blanky \forall i,k \leq N \} 
    \end{array} \begin{array}{c}
         \textnormal{is the space of all} \\
         \textnormal{one-body operators, } \\
         \\
         \textnormal{is the space of all} \\
         \textnormal{$k$-body operators } 
    \end{array} \begin{array}{c}
         \textnormal{and where} \\ 
         \\
         \mathfrak{H}\mathfrak{s}(\hilbert) = \bigsqcup_{k=1}^N \mathfrak{H}\mathfrak{s}_k(\hilbert)
    \end{array}
\end{equation} 

\blanky \\ 

Consider now an $N$-body quantum system, where correlations, entanglements, and interactions are present. Different particle species are allowed. The One-body and Two-body Max-Ent frameworks are defined as follows


\begin{tcolorbox}[title = N\"aive one-body Max-Ent]
    In the N\"aive one-body Max-Ent framework, there are $N$ sets of one-body operators, each one corresponding to one the $N$ subsystems, which are assumed to be the local basis. These operators must be local operators, acting non-trivially in only one Hilbert subspace, and must be trace-class. The framework thus allows for interactions between different particle species since the basis may have different dimension, this is
    
    \begin{equation*}
        \begin{array}{c}
            \{\mathbf{O}_i^{(1)}\}_{i=1}^{n_1}\in \mathfrak{H}\mathfrak{s}_1(\hilbert^{(1)})  \\
            \vdots \\
            \{\mathbf{O}_i^{(\ell)}\}_{i=1}^{n_{\ell}}\in \mathfrak{H}\mathfrak{s}_1(\hilbert^{(\ell)}) \\
            \vdots \\
            \{\mathbf{O}_i^{(N)}\}_{i=1}^{n_N}\in \mathfrak{H}\mathfrak{s}_1(\hilbert^{(N)})
        \end{array} \begin{array}{c}
             \textnormal{ The operators are then }\\
             \textnormal{redefined so that } \\
             \textnormal{they act on the global} \\
             \textnormal{Hilbert space $\hilbert^{\otimes N}$}
        \end{array} \begin{array}{c}
            \mathfrak{b}_1 = \{\mathbf{O}_i^{(1)} \otimes \bigotimes_{k=2}^{N} \mathds{1}^{(k)}\}_{i=1}^{n_1}\subset \mathfrak{H}\mathfrak{s}(\hilbert^{\otimes N}) \\
            \vdots \\
            \mathfrak{b}_\ell =  \{\otimes_{k=1}^{\ell} \mathds{1}^{(k)} \otimes \mathbf{O}_i^{(\ell)} \otimes_{k' = \ell + 1}^{N} \mathds{1}^{(k')}\}_{i=1}^{n_\ell}\subset \mathfrak{H}\mathfrak{s}(\hilbert^{\otimes N})\\
            \vdots \\
            \mathfrak{b}_N = \{ \bigotimes_{k=1}^{N-1} \mathds{1}^{(k)} \otimes  \mathbf{O}_i^{(N)}\}_{i=1}^{n_N}\subset \mathfrak{H}\mathfrak{s}(\hilbert^{\otimes N})
        \end{array}
    \end{equation*}
    
    Then, the one-body Max-Ent basis $\mathfrak{B}_{ME_1}$ is defined as the union of these sets 
    
    $$
        \mathfrak{B}_{ME_1} = \bigsqcup_{k=1}^{N} \mathfrak{b}_k,
    $$
    
    whose dimension is given by the sum of the $\mathfrak{b}$-basis dimensions ie. $\textnormal{dim}(\mathfrak{B}_{ME_1}) = \sum_{k=1}^N \textnormal{dim}(\mathfrak{b}_k) \sim \mathcal{N}$. Then, the one-body Max-Ent states are given by 
    
    \begin{equation}
         \bm{\mathcal{S}}_{ME,1}( \mathfrak{B}_{ME_1}) = \{ \rho \in \mathcal{C}(\mathds{H}) \blanky | \blanky \exists \{\lambda_k\}_{k=1}^{\dim \mathfrak{B}_{ME_1}} \subset \mathds{R}  \textnormal{ such that }  \rho \propto \exp(-\sum_{i}\lambda_i {\bf O}_i) \}.
    \end{equation}
\end{tcolorbox}

\begin{tcolorbox}[title = N\"aive two-body Max-Ent]

Here, similarly to the n\"aive one-body Max-Ent framework,
we have $N$ sets of local one-body operators at our disposal, which must be trace-class and must only non-trivially act in only one Hilbert subspace. If the one-body local operators are  

\begin{equation}
    \begin{array}{c}
            \{\mathbf{O}_i^{(1)}\}_{i=1}^{n_1}\in \mathfrak{H}\mathfrak{s}_1(\hilbert^{(1)})  \\
            \vdots \\
            \{\mathbf{O}_i^{(\ell)}\}_{i=1}^{n_{\ell}}\in \mathfrak{H}\mathfrak{s}_1(\hilbert^{(\ell)}) \\
            \vdots \\
            \{\mathbf{O}_i^{(N)}\}_{i=1}^{n_N}\in \mathfrak{H}\mathfrak{s}_1(\hilbert^{(N)})
        \end{array}, \begin{array}{c}
             \textnormal{ then the global}\\
             \textnormal{one-body operators are}
        \end{array}   \begin{array}{c}
            \mathfrak{b}_1 = \{\mathbf{O}_i^{(1)} \otimes \bigotimes_{k=2}^{N} \mathds{1}^{(k)}\}_{i=1}^{n_1}\subset \mathfrak{H}\mathfrak{s}(\hilbert^{\otimes N})\\
            \vdots \\
            \mathfrak{b}_\ell =  \{\otimes_{k=1}^{\ell} \mathds{1}^{(k)} \otimes \mathbf{O}_i^{(\ell)} \otimes_{k' = \ell + 1}^{N} \mathds{1}^{(k')}\}_{i=1}^{n_\ell}\subset \mathfrak{H}\mathfrak{s}(\hilbert^{\otimes N})\\
            \vdots \\
            \mathfrak{b}_N = \{ \bigotimes_{k=1}^{N-1} \mathds{1}^{(k)} \otimes  \mathbf{O}_i^{(N)}\}_{i=1}^{n_N}\subset \mathfrak{H}\mathfrak{s}(\hilbert^{\otimes N})
        \end{array} 
\end{equation}

\blanky\\

However, unlike the previous case, we now allow for two-body operators to be included. These new sets of two-body operators may be defined as follows

\begin{equation*}
\begin{array}{c}
             \mathfrak{c}_{11} = \{\mathbf{O}_i^{(1)} \mathbf{O}_j^{(1)} \bigotimes_{k=2}^{N} \mathds{1}^{(k)}\}_{\substack{i, j= 1}}^{\Gamma(n_1, n_1)} \blanky | \blanky {\bf O}_i^{(1)} \in \mathfrak{b}_{1}  \\
             \\
             \mathfrak{c}_{12} = \{\mathbf{O}_i^{(1)} \otimes \mathbf{Q}_j^{(2)} \bigotimes_{k=3}^{N} \mathds{1}^{(k)}\}_{\substack{i, j= 1}}^{\Gamma(n_1, n_2)} \blanky | \blanky {\bf O}_i^{(1)} \in \mathfrak{b}_{1}, {\bf Q}_j^{(2)} \in \mathfrak{b}_{2}\\
             \vdots \\
             \mathfrak{c}_{\ell \ell'} =  \{\mathbf{O}_i^{(\ell)} \otimes \mathbf{Q}_j^{(\ell)} \bigotimes_{\substack{k=1 \\
                                   k \neq \ell             \ell'}}^{N} 
            \mathds{1}^{(k)}\}_{\substack{i, j= 1}}^{\Gamma(n_\ell, n_{\ell'})} \blanky | \blanky {\bf O}_i^{(1)} \in \mathfrak{b}_{\ell}, {\bf Q}_j^{(2)} \in \mathfrak{b}_{\ell'},\\
            \vdots \\
            \mathfrak{c}_{NN} = \{\bigotimes_{k=1}^{N-1} \mathds{1}^{(k)} \otimes \mathbf{O}_i^{(N)} \mathbf{O}_j^{(N)}\}_{\substack{i, j= 1}}^{\Gamma(n_N, n_N)} \blanky | \blanky {\bf O}_i^{(N)} \in \mathfrak{b}_{N}  \\
        \end{array}
\end{equation*}

\blanky\\

where $\Gamma(n_a, n_b)$ counts all the possible, non-repeating, order notwithstanding, pair combinations of elements from a $n_a$-cardinality set with elements from a $n_b$-cardinality set. In other words, 

\begin{equation}
    \Gamma(n_a, n_b) = 
        n_a + n_b + \frac{n_a(n_b-1)}{2}
\end{equation} 

Then, the two-body Max-Ent basis is 

$$
    \mathfrak{B}_{ME_2} = \bigsqcup_{k=1}^{N} \mathfrak{b}_k \cup \bigsqcup_{k, k' = 1}^{N} \mathfrak{c}_{kk'},
$$

from which the two-body Max-Ent states are given by 

\begin{equation}
    \bm{\mathcal{S}}_{ME,2} (\mathfrak{B}_{ME_2}) = \{ \rho \in \mathcal{C}(\mathds{H}) \blanky | \blanky \exists \{\lambda_k\}_{k=1}^{\ell \leq N}, \{\gamma_{mn}\}_{\substack{m,n=1\\
                      {m \mbox{\textless} n}}}^{\ell \leq N} \subset \mathds{R} \textnormal{ such that }  \rho \propto \exp(-\sum_{i,j}\lambda_i {\bf O}_i - \gamma_{ij} {\bf O}_i {\bf O}_j ) \}. 
\end{equation}

Note that $\dim \mathfrak{B}_{ME_2} = \sum_{k=1}^N \dim(\mathfrak{b}_k) + \sum_{i, j=1}^N \dim(\mathfrak{c}_{ij}) \sim \mathcal{O}(N^2)$
\end{tcolorbox}

Both of these techniques require substantially fewer parameters than the exact dynamics, which requires $\mathcal{O}(2^{2N})$ complex-valued entries (or alternatively $\mathcal{O}(2^{2N + 1})$ real-valued parameters. 

%which requires $2^{2n} - 1 $

\clearpage

Consider a closed quantum many-body system described by a Hamiltonian ${\bf H}$ and with its initial state, $\rho_0$, given by

$$
\rho_0 = e^{-{\bf K}} \begin{array}{c}
     \textnormal{ where $\rho_0 \in \mathcal{C}(\hilbert^{\otimes N}))$ } \\
     \\
     \textnormal{ and with ${\bf K} \in \mathfrak{H}\mathfrak{s}(\hilbert^{\otimes N})$} \Longleftrightarrow {\bf K} = - \log \rho
\end{array}
$$

The system's time evolution of course governed by the Schr\"odinger equation. 

$$
i \frac{d\rho(t)}{dt} =  [{\bf H}, \rho]
$$

For the time being, consider $\rho_0$ as a one-body Max-Ent state, with respect to some general Max-Ent 1 basis composed of a collection of one-body local operators

\begin{equation}
    \rho \in \bm{\mathcal{S}}_{ME,1}( \mathfrak{B}_{ME_1}) \begin{array}{c}
         \textnormal{which in turn implies that the ${\bf K}$-operator } \\
         \textnormal{ can be uniquely decomposed,} \\
         \textnormal{upto phase factors, as } 
    \end{array}
    {\bf K} = \sum_{\mu}^{\ell} \phi^{\mu}(t) \bm{\mathcal{O}}_{\mu},
\end{equation}

where we have chosen the Schrodinger picture for the operators. Now we claim the following 

\begin{theo}
    Since $\rho = e^{-{\bf K}}$ is a well-defined density operator $\rho_0 \in \mathcal{C}(\hilbert^{\otimes N}))$, the ${\bf K}$-operator's time evolution is governed by a Schr\"odinger equation as well, this is 
    
    $$
        i \frac{d{\bf K}}{dt} =  [{\bf K}, \rho].
    $$
\end{theo}

\begin{proof}
    If $\rho = e^{-{\bf K}}$ then, by definition 
    
    \begin{equation}
    \begin{split}
            &\rho = \mathds{1} - {\bf K} + \frac{1}{2}{\bf K}^2 - \frac{1}{3!} {\bf K}^3 + \cdots \\
            &d\rho = 0 - d{\bf K} + \frac{1}{2} \bigg(d{\bf K} {\bf K} + {\bf K} d{\bf K} \bigg) - \frac{1}{3!} \bigg((d{\bf K}) {\bf K}^2 + {\bf K} (d{\bf K}) {\bf K} + {\bf K}^2 (d{\bf K}) \bigg) + \cdots 
    \end{split}
    \end{equation}
    
    which, if we are willing to assume that $[{\bf K},d{\bf K}] = 0$, yields 
    
    \begin{equation}
    \begin{split}
        d\rho &= 0 - d{\bf K} + \frac{1}{2} \bigg(d{\bf K} {\bf K} + {\bf K} d{\bf K} \bigg) - \frac{1}{3!} \bigg((d{\bf K}) {\bf K}^2 + {\bf K} (d{\bf K}) {\bf K} + {\bf K}^2 (d{\bf K}) \bigg) + \cdots \\
        &= - d{\bf K} + {\bf K}d{\bf K} - \frac{1}{2} {\bf K}^2 d{\bf K} + \cdots \\
        &= -\bigg( \mathds{1} - {\bf K} + \frac{1}{2} {\bf K}^2 + \cdots\bigg) d{\bf K} = -e^{-{\bf K}} d{\bf K} \\
        &\Rightarrow \frac{d\rho}{dt} = -e^{-{\bf K}} \frac{d{\bf K}}{dt},
    \end{split}
    \end{equation}
    
    and given that $\rho$'s time evolution is governed by the Schr\"odinger equation, this yields
    
    \begin{equation}
        \begin{split}
            &i\frac{d\rho}{dt} = [{\bf H}, \rho] \\
            & i e^{-{\bf K}} \frac{d{\bf K}}{dt}  = [{\bf H}, e^{-{\bf K}}]
        \end{split}
    \end{equation}
    
    \notaFTBP{Acá algunas cosas me hacen ruido. Primero, la identidad $\frac{d\rho}{dt} = -e^{-{\bf K}} \frac{d{\bf K}}{dt}$ se mantiene sí y solo sí asumo que $dK$ y $K$ conmutan, lo cual en general no es el caso. Por otro lado, si intento hacer la cuenta por el lado de la derivada del logaritmo tengo que $d \log \rho/dt = \rho^{-1} d\rho/dt $}
    
    \begin{align*}
        \frac{d}{dt} \log \rho(t) &= \lim_{\Delta t \rightarrow 0 } \frac{\log [ \rho + \rho' \Delta t ] - \log \rho}{\Delta t} \\
        &= \lim_{\Delta t \rightarrow 0 } \frac{\log [ \rho \rho^{-1} + \rho' \rho^{-1} \Delta t ]}{\Delta t} \\
        &=  \lim_{\Delta t \rightarrow 0 } \frac{\log [ \mathds{1} + \rho' \rho^{-1} \Delta t ]}{\Delta t} \\
        &= \lim_{\Delta t \rightarrow 0 } {\log \bigg([ \mathds{1} + \rho' \rho^{-1} \Delta t ]}\bigg)^{\frac{1}{\Delta t}} \\
        &= \lim_{\sigma \rightarrow 0 } {\log \bigg([ \mathds{1} + \sigma ]}\bigg)^{\rho' \rho^{-1}\sigma^{-1}} \textnormal{ where $\sigma = \rho' \rho^{-1} \Delta t$ } \\
        &= \rho' \rho^{-1} \lim_{\sigma \rightarrow 0 } {\log \bigg([ \mathds{1} + \sigma ]}\bigg)^{\sigma^{-1}} \\
        &= \rho' \rho^{-1} \lim_{\sigma \rightarrow 0 } \log e \\
        &=  \rho' \rho^{-1}
    \end{align*}
    \notaFTBP{ pero lo que no me gusta de acá es que es que usé que $\log AB^{-1} = \log A - \log B$, que vale solo si conmutan y si $\rho' \rho^{-1}$ conmutan dentro del límite. Me hace ruido}
\end{proof}

\blanky \\

\begin{theo}
     Consider an $N$-particle closed quantum system and consider an $\ell+\ell'$-dimensional basis of Hilbert-Schmidt operators, which includes upto two-body operators, ie. 
     
     $$
     \mathfrak{B} = \{{\bf O}_i\}_{i=1}^{\ell} \cup \{{\bf O}_i {\bf O}_j\}_{\substack{i=1\\
                                        j=1}}^{\ell'} \textnormal{ with } O_i \in \mathfrak{H}\mathfrak{s}_1(\hilbert^{(\ell)})  \blanky \forall i \textnormal{ and } \mathfrak{B} \subset \mathfrak{H}\mathfrak{s}_2(\hilbert^{(\ell)}).
     $$
     
     We then claim that the system's time evolution can then be approximated as 
     
     $$
     \rho(t) = e^{\bm{\mathfrak{K}}} \textnormal{ where } \bm{\mathfrak{K}} = \sum_{\mu} \phi^{\mu}(t) {\bf O}_{\mu} + \sum_{\mu \nu} \gamma^{\mu \nu}(t) {\bf O}_\mu {\bf O}_\nu \textnormal{ for some } \{\phi^{\mu}\}_{\mu}, \{\gamma^{\mu \nu}\}_{{\mu, \nu}} \subset C^\infty(\R) \textnormal{ such that } \rho \in \statespace, \forall t.
     $$
     
\end{theo}

\begin{proof}
    In effect, the closed evolution is governed by the Schrodinger equation on the density matrix, 
    
    $$
    \frac{d\rho}{dt} = \frac{[{\bf H}, \rho]}{i},
    $$
    
    If the density operator can be written as the exponential of a positive-defined operator $\bm{\mathfrak{K}}$, the previous equation naturally induces a Schrodinger equation on the $\bm{\mathfrak{K}}$-operator, as follows 
    
    $$
    \frac{d\bm{\mathfrak{K}}
    }{dt} = \frac{[{\bf H}, \bm{\mathfrak{K}}
    ]}{i}.
    $$
    
    Which entails,
    
    \begin{equation}
        \begin{split}
            \sum_{\mu} \frac{d\phi^{\mu}}{dt} {\bf O}_{\mu} + \sum_{\mu \nu} \frac{d\gamma^{\mu \nu}}{dt} {\bf O}_\mu {\bf O}_\nu &= \bigg[{\bf H}, \sum_{\mu} \phi^{\mu}(t) {\bf O}_{\mu} + \sum_{\mu \nu} \gamma^{\mu \nu}(t) {\bf O}_\mu {\bf O}_\nu\bigg] \\
            \sum_{\mu} \frac{d\phi^{\mu}}{dt} {\bf O}_{\mu} + \sum_{\mu \nu} \frac{d\gamma^{\mu \nu}}{dt} {\bf O}_\mu {\bf O}_\nu &=
            \sum_{\mu} \phi^\mu(t) [{\bf H}, {\bf O}_{\mu}] + \sum_{\mu \nu} \gamma^{\mu \nu}(t) [{\bf H}, {\bf O}_{\mu} {\bf O}_{\nu}] \\
        \end{split}
    \end{equation}
    \begin{equation}
        \begin{split}
            \Rightarrow \left({\bf O}_\alpha, \sum_{\mu} \frac{d\phi^{\mu}}{dt} {\bf O}_{\mu} + \sum_{\mu \nu} \frac{d\gamma^{\mu \nu}}{dt} {\bf O}_\mu {\bf O}_\nu \right) = \left({\bf O}_\alpha, \sum_{\mu} \phi^\mu(t) [{\bf H}, {\bf O}_{\mu}] + \sum_{\mu \nu} \gamma^{\mu \nu}(t) [{\bf H}, {\bf O}_{\mu} {\bf O}_{\nu}] \right) \\
            \sum_{\mu} \frac{d\phi^{\mu}}{dt} \cancelto{\delta_{\alpha \mu}}{({\bf O}_\alpha, {\bf O}_{\mu})} + \cancelto{0}{\sum_{\mu \nu} \frac{d\gamma^{\mu \nu}}{dt} ({\bf O}_\alpha, {\bf O}_\mu {\bf O}_\nu)} = \left(\sum_{\mu} \phi^\mu(t) ({\bf O}_\alpha, [{\bf H}, {\bf O}_{\mu}]) + \sum_{\mu \nu} \gamma^{\mu \nu}(t) ({\bf O}_\alpha, [{\bf H}, {\bf O}_{\mu} {\bf O}_{\nu}] )\right) \\
            {\frac{d\phi^{\alpha}}{dt}}{= \left(\sum_{\mu} \phi^\mu(t) ({\bf O}_\alpha, [{\bf H}, {\bf O}_{\mu}]) + \sum_{\mu \nu} \gamma^{\mu \nu}(t) ({\bf O}_\alpha, [{\bf H}, {\bf O}_{\mu} {\bf O}_{\nu}] )\right)} \\
            & \alignedbox{\frac{d\phi^{\alpha}}{d\alpha}}{= \sum_{\mu} \bm{\mathcal{H}} \phi^{\mu}},
            \end{split}
    \end{equation}

where 

\begin{align*}
    \bm{\mathcal{H}} : \mathfrak{H}\mathfrak{s}(\hilbert^{\otimes N}) & \times \mathfrak{H}\mathfrak{s}(\hilbert^{\otimes N}) \rightarrow \mathds{C}^{\dim \times \dim} \\
    & (\bm{\mathcal{H}})_{\mu \nu} = ({\bf O}_{\nu}, [{\bf H}, {\bf O}_{\nu}])
\end{align*}
    
\end{proof}


\end{document}

